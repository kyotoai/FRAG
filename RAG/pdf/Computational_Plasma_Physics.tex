\\documentclass[10pt]{article}
\\usepackage[utf8]{inputenc}
\\usepackage[T1]{fontenc}
\\usepackage{hyperref}
\\hypersetup{colorlinks=true, linkcolor=blue, filecolor=magenta, urlcolor=cyan,}
\\urlstyle{same}
\\usepackage{amsmath}
\\usepackage{amsfonts}
\\usepackage{amssymb}
\\usepackage[version=4]{mhchem}
\\usepackage{stmaryrd}
\\usepackage{graphicx}
\\usepackage[export]{adjustbox}
\\graphicspath{ {./images/} }
\\usepackage{mathrsfs}

\\title{Computational Plasma Science }


\\author{Shigeo Kawata\\\\
Emeritus, Utsunomiya University\\\\
Utsunomiya, Japan\\\\
Laboratory for Laser Plasmas\\\\
Shanghai Jiao Tong University\\\\
Shanghai, China}
\\date{}


%New command to display footnote whose markers will always be hidden
\\let\\svthefootnote\\thefootnote
\\newcommand\\blfootnotetext[1]{%
  \\let\\thefootnote\\relax\\footnote{#1}%
  \\addtocounter{footnote}{-1}%
  \\let\\thefootnote\\svthefootnote%
}

%Overriding the \\footnotetext command to hide the marker if its value is `0`
\\let\\svfootnotetext\\footnotetext
\\renewcommand\\footnotetext[2][?]{%
  \\if\\relax#1\\relax%
    \\ifnum\\value{footnote}=0\\blfootnotetext{#2}\\else\\svfootnotetext{#2}\\fi%
  \\else%
    \\if?#1\\ifnum\\value{footnote}=0\\blfootnotetext{#2}\\else\\svfootnotetext{#2}\\fi%
    \\else\\svfootnotetext[#1]{#2}\\fi%
  \\fi
}

\\def\\AA{\\mathring{\\mathrm{A}}}

\\begin{document}
\\maketitle
Shigeo Kawata

Physics and Selected Simulation

Examples

Springer

\\section*{Springer Series in Plasma Science and Technology }
Plasma Science and Technology covers all fundamental and applied aspects of what is referred to as the "fourth state of matter." Bringing together contributions from physics, the space sciences, engineering and the applied sciences, the topics covered range from the fundamental properties of plasma to its broad spectrum of applications in industry, energy technologies and healthcare.

Contributions to the book series on all aspects of plasma research and technology development are welcome. Particular emphasis in applications will be on high-temperature plasma phenomena, which are relevant to energy generation, and on low-temperature plasmas, which are used as a tool for industrial applications. This cross-disciplinary approach offers graduate-level readers as well as researchers and professionals in academia and industry vital new ideas and techniques for plasma applications.

Shigeo Kawata

\\section*{Computational Plasma Science}
Physics and Selected Simulation Examples

Springer

ISSN 2511-2007 ISSN 2511-2015 (electronic)

Springer Series in Plasma Science and Technology

ISBN 978-981-99-1136-3 ISBN 978-981-99-1137-0 (eBook)

\\href{https://doi.org/10.1007/978-981-99-1137-0}{https://doi.org/10.1007/978-981-99-1137-0}
\\footnotetext{(C) The Editor(s) (if applicable) and The Author(s), under exclusive license to Springer Nature Singapore Pte Ltd. 2023

This work is subject to copyright. All rights are solely and exclusively licensed by the Publisher, whether the whole or part of the material is concerned, specifically the rights of translation, reprinting, reuse of illustrations, recitation, broadcasting, reproduction on microfilms or in any other physical way, and transmission or information storage and retrieval, electronic adaptation, computer software, or by similar or dissimilar methodology now known or hereafter developed.

The use of general descriptive names, registered names, trademarks, service marks, etc. in this publication does not imply, even in the absence of a specific statement, that such names are exempt from the relevant protective laws and regulations and therefore free for general use.

The publisher, the authors, and the editors are safe to assume that the advice and information in this book are believed to be true and accurate at the date of publication. Neither the publisher nor the authors or the editors give a warranty, expressed or implied, with respect to the material contained herein or for any errors or omissions that may have been made. The publisher remains neutral with regard to jurisdictional claims in published maps and institutional affiliations.

This Springer imprint is published by the registered company Springer Nature Singapore Pte Ltd. The registered company address is: 152 Beach Road, \\#21-01/04 Gateway East, Singapore 189721, Singapore
}

\\section*{Preface}
The book of Computational Plasma Science-Physics and Selected Simulation Examples is prepared for beginners and students about plasma physics in graduate and undergraduate courses in universities and colleges. The essence of plasma physics is introduced in a concise manner together with computational techniques, which are powerful tools to analyze plasmas.

Collective behavior is the characteristic of plasma. Plasma is ionized and consists of ions and electrons. Plasma is electrically quasi-neutral. Each neutral particle in neutral gas moves independently. However, each charged particle produces the longrange Coulomb force. Many charged particles move together, responding to electric and magnetic forces. The collective motion of plasma is remarkable and is different from that of the neutral gas.

The book presents the essential physics in plasma and does not assume readers' background knowledge on plasmas. However, it is assumed that readers have learned basic physics on mechanics, electromagnetism, thermodynamics, statistical physics, relativity and fluid dynamics as well as mathematics related. Additional instructive texts are listed in Appendix A in each topics in plasmas for readers' convenience. Detailed references are listed at the end of each chapter, and they would help readers, who are interested in plasmas further beyond the book. Figure 1.7 serves a summary of the book structure and the relation among the chapters.

Scientific computing environment is one of the essential tools to perform researches in science, engineering and even in our society. Computer-assisted problem solving is one of the key methods to promote innovations in science and engineering including plasma physics. Computational power consists of computer hardware power, algorithm power and computer software power. In plasma physics, computer simulations and analyses are also essential tools to study plasma physics, as well as theoretical and experimental methods. This book focuses on introductions to plasma physics and mathematical methods in plasma physics and also on introduction to numerical simulation methods in plasma physics. Parallel computing is
widely used to simulate plasmas. A short introduction to a parallel computing based on OpenMP is summarized in Appendix E. ${ }^{1}$

Major mathematical models in plasma physics are fluid model and particle kinetic model. The fluid model is a kind of a macroscopic model, described by the fluid equations, in which it would be assumed that collisions between plasma particles are rich. The particle model is a microscopic kinetic model, in which individual charged particles are described by the equation of motion in electric and magnetic fields. The microscopic kinetic model is often based on distribution functions, which describe the motion of plasma particles. By averaging the distribution functions in the velocity space, the macroscopic fluid equations would be derived.

Corresponding to the mathematical models, numerical models have been intensively developed. Particle-in-Cell (PIC) methods among various particle methods are commonly used in plasma studies. The electric and magnetic fields are solved on spatial meshes, and charged articles are pushed by the fields, which are interpolated on particles from the field values on the meshes. The charged particle motion induces the net charge and the electric current, which change the original field values. In this book, the essence of a simple PIC method is introduced in Chap. 4, and the numerical results by the PIC simulations are used in Chaps. 1-4, 7 and 9. The fluid numerical models have been also developed well. The continuity equation, the equation of motion, that is, the Navier-Stokes equation and the energy equation would be solved with the equation of state. Frequently the radiation equation is also coupled with the fluid equations. For fluid simulations, a basic Euler fluid simulation method is first introduced in Chap. 5 and Appendix F. Then a Lagrange fluid method is also introduced in Chap. 5. A discretization method, called finite difference method (FDM), is introduced and used in this book in the PIC and fluid codes. In addition to the numerical techniques, numerical stability is also explained and discussed in this book. The numerical errors and instabilities do not come from physics but from the numerical methods artificially. They would have a significant impact on numerical results, and users must avoid the numerical errors and instability in order to obtain correct results. The numerical errors, numerical stability and uncertainty in numerical computations are also discussed in Subsection 1.7.1 and Chaps. 4 and 5.

The most remarkable characteristic of plasma is its collective behavior. Plasma is ionized. Plasma is a collection of charged particles and is electrically quasi-neutral. Each charged particle produces the long-range Coulomb force. Many charged particles move collectively, responding to electric and magnetic forces. The collective motion of plasma is remarkable and is different from that of neutral gas. Plasma or ionized gas is ubiquitous in space and on the sun. On earth, auroras, lightning and electric sparks are relating to plasmas. In nuclear fusion, which would be our future energy source, fusion fuel becomes a plasma state.

The author would like to extend his sincere gratitude to colleagues and friends, who have worked together with the author. Especially the late Professor Keishiro
\\footnotetext{${ }^{1}$ The OpenMP API supports multi-platform shared-memory parallel programming in C/C++ and Fortran. \\href{https://www.openmp.org/}{https://www.openmp.org/} Ref.: OpenMP Architecture Review Board. \\href{https://www.ope}{https://www.ope} \\href{http://nmp.org/.Cited28August}{nmp.org/.Cited28August}, 2021.
}

Niu at Tokyo Institute of Technology guided me properly as an academic advisor to work on inertial confinement fusion based on ion beam. At the first year of the graduate course Prof. Niu suggested the author to use computer simulation methods as well as theoretical and experimental methods. In addition, many international scientists and professors are also appreciated for their support and encouragements: especially the late Prof. Ronald C. Davidson at Princeton university for his continuous encouragement and support; Prof. Claude Deutsch at Universit ${ }^{\\prime}$ e Paris-Saclay for his warm support and collaborations; Dr. Alex Friedman for his friendly scientific support; Prof. Jiri Limpouch, Prof. Richard Liska, Prof. Ladislav Drska, Prof. Ondrej Klimo and Prof. Jan Psikal at Czech Technical University for their long warm collaborations; Prof. Dieter Hoffmann at Technische Universität Darmstadt for his friendly support; Prof. Jie Zhang and Prof. Zhengming Sheng at Shanghai Jiao Tong University for their strong support; Prof. Kazuhiko Horioka at Tokyo Institute of Technology for his intensive support; Prof. Thomas Mehlhorn, Prof. Hideaki Takabe, Prof. Shunjiro Shinohara, Prof. Hong Qing, Prof. Igor Kaganovich, Prof. Qing Kong, Prof. Ken Takayama, Dr. John J. Barnard, Dr. Peter A. Seidl, Dr. Masahiro Okamura, Dr. Takeshi Kanesue, Dr. Shunsuke Ikeda, Dr. Naeem Ahmad Tahir, Prof. Antonio Roberto Piriz, Prof. PingXiao Wang, Prof. Yongtao Zhao, Prof. Kuen Ting, Prof. Yintzer Jerry Shih, Prof. Wei-Min Wang and Dr. Rui Cheng for their warm friendly support; Dr. Sergei Bulanov, Dr. YanJun Gu, Prof. Alexander Ogoyski, Prof. Alexander Andreev, Prof. Yutong Li, Prof. YanYun Ma, Prof. Fuyuan Wu, Dr. Xiaofeng Li, Dr. Qin Yu and Dr. Zekun Xu for their friendly collaboration, as well as other warm friends. Many former graduate students and international visitors are also appreciated, and frequently they have asked the author to show numerical methods together with plasma physics. The communications with the young students and scientists through lectures and scientific collaborations also have encouraged the author to prepare the book which includes plasma physics and the relating numerical methods together.

This book is prepared partly based on my previously published book in Japanese on Introduction to Plasma Science (ISBN978-4-627-77592-3), published by Morikita Publisher in Japan. The part of plasma physics is based on this Japanese book, but it is rewritten largely. The part of plasma computation is newly prepared in this book. The book was prepared based on LATEX. ${ }^{2}$ The Particle-in-Cell (PIC) code of EPOCH was partly used to simulate plasma particle behavior. ${ }^{3}$ The visualization
\\footnotetext{${ }^{2}$ A document preparation system. \\href{https://www.latex-project.org}{https://www.latex-project.org}.

${ }^{3}$ A plasma physics simulation code which uses the Particle-in-Cell (PIC) method. Refs.: Arber, T. D., Bennett, K., et al.: Contemporary particle-in-cell approach to laser-plasma modelling, Arber, Plasma Phys. Control. Fusion 57, 113001(2015); Bennett, K.: Users Manual for the EPOCH PIC codes, EPOCH Version 4.3, (2014).
}
system of VisIt ${ }^{4}$ and MATLAB ${ }^{5}$ were partly used to visualize the numerical data. The mathematical software of Maple ${ }^{6}$ was used in part. Finally, the author would like to thank Dr. Akiyuki Tokuno, Ms. Kowsalya Raghunathan, Dr. Nobuko Kamikawa and Ms. Taeko Sato of the Springer editorial department for their helpful and great efforts in publishing the book.

The Author also would like to extend acknowledgments to authors and publishers, who gave permissions to reuse figures presented below. Figure captions and explanations related are also referred from the original works.

\\begin{itemize}
  \\item Figures 7.6, 7.7, 9.19-9.22: Reprinted figures with permission from Kawata, S., Karino, T. and Gu, Y. J.: Phase control of a -current-driven plasma column, Phys. Rev.E, 101 (2020) 041201. \\href{https://link.aps.org/doi/10.1103/PhysRevE}{https://link.aps.org/doi/10.1103/PhysRevE}. 101. 041201 Copyright 2020 by the American Physical Society.

  \\item Figures 9.13-9.15: Reprinted figures with permission from Kawata, S., Deutsch, C. and Gu, Y. J.: Peculiar behavior of Si cluster ions in a high-energy-density solid Al plasma, Phys. Rev. E, 99 (2019) 011201(R). \\href{https://link.aps.org/doi/10}{https://link.aps.org/doi/10}. 1103/PhysRevE.99.011201 Copyright 2019 by the American Physical Society.

  \\item Figures 7.16-7.18, 9.25-9.27: Reprinted figures with permission under Attribution 4.0 International (CC BY 4.0) from Gu, Y. J., Kawata, S. and Bulanov, S. V.: Dynamic mitigation of the tearing mode instability in a collisionless current sheet, Scientific Reports, 11 (2021) 11651. \\href{https://doi.org/10.1038/s41598-02191111-8}{https://doi.org/10.1038/s41598-02191111-8} Copyright (C) 2021, The Authors.

  \\item Figures 9.23 and 9.24: Reprinted figures with permission under Attribution 4.0 International (CC BY 4.0) from Kawata, S., Karino, T. and Gu, Y.: Dynamic stabilization of plasma instability, High Power Laser Sci. Eng., 7 (2019) e3. https:// \\href{http://doi.org/10.1017/hpl.2018.61}{doi.org/10.1017/hpl.2018.61} (C) The Authors 2019.

  \\item Figures 9.33-9.35: Reprinted figures with permission under Attribution 4.0 International (CC BY 4.0) from Kawata, S.: Direct-drive heavy ion beam inertial confinement fusion: a review, toward our future energy source, Advances in Physics X, 6 (2021) 1873860. \\href{https://doi.org/10.1080/23746149.2021}{https://doi.org/10.1080/23746149.2021}. 1873860 (C) 2021 The Author. Published by Informa UK Limited, trading as Taylor \\& Francis Group.
\\footnotetext{${ }^{4}$ An open source, interactive, scalable, visualization, animation and analysis tool. Refs.: bibitemVisIt Childs, H., Brugger, E., Whitlock, B., Meredith, J., Ahern, S., Pugmire, D., Biagas, K., Miller, M., Harrison, C., Weber, G. H., Krishnan, H., Fogal, T., Sanderson, A., Garth, C., Bethel, E. Wes, Camp, D., Rübel, O., Durant, M., Favre, J. M. and Navrátil, P.: VisIt: An End-User Tool For Visualizing and Analyzing Very Large Data, ed. by E. Wes Bethel, H. Childs and C. Hansen, High Performance Visualization-Enabling Extreme-Scale Scientific Insight, (Chapman and Hall/CRC, New York, 2012) pp. 357-372; VisIt homepage: \\href{https://visit-dav.github.io/visit-website/}{https://visit-dav.github.io/visit-website/}.

${ }^{5}$ A programming and numeric computing platform in science and engineering. Registered trademark of The MathWorks, Inc. \\href{https://www.mathworks.com/products/matlab.html}{https://www.mathworks.com/products/matlab.html}. Ref.: MATLAB: Registered trademark of The MathWorks, Inc. \\href{https://www.mathworks.com/products/matlab.html}{https://www.mathworks.com/products/matlab.html}. ${ }^{6}$ A computer algebra and visualization software. Registered trademark of Waterloo Maple Inc. \\href{https://www.maplesoft.com/}{https://www.maplesoft.com/}. Ref.: Maple, Computer algebra and visualization software, developed by Waterloo Maple Inc. \\href{https://www.maplesoft.com/}{https://www.maplesoft.com/}.
}

  \\item Figure 9.34: Reprinted figure with permission under Attribution 4.0 International (CC BY 4.0) from Sato, R., Kawata, S., Karino, T., Uchibori, K. and Ogoyski, A. I.: Non-uniformity smoothing of direct-driven fuel target implosion by phase control in heavy ion inertial fusion, Sci. Rep. 9 (2019) 6659. \\href{https://doi.org/10}{https://doi.org/10}. 1038/s41598-019-43221-7 Copyright (C) 2019, The Authors.

\\end{itemize}

\\section*{Contents}
1 Introduction to Plasma ..... 1
1.1 Fourth State of Matter: Plasma ..... 1
1.2 Brief Introduction to Plasma Physics World ..... 3
1.3 Collective Motion: The Debye Shielding ... ..... 7
1.4 Collective Motion: Plasma Oscillation ..... 10
1.5 Conditions for Plasma Collective Motion ..... 12
1.6 Introduction to Mathematical Models for Plasma .... ..... 14
1.7 Introduction to Computer Simulation for Plasma ..... 15
1.7.1 Computer Simulation Power and Uncertainty in Computation ..... 16
1.7.2 Example Computer Simulation for the Debye Shielding ..... 25
References ..... 26
2 Plasma in Equilibrium ..... 29
2.1 Distribution Function and Plasma ... ..... 29
2.2 The Maxwell Distribution ..... 31
2.3 Plasma Density ..... 35
2.4 The Coulomb Collision ..... 36
2.5 Plasma Temperature ..... 39
2.5.1 Simulation of Temperature Relaxation ..... 40
References ..... 41
3 Single Particle Motion ..... 43
3.1 Equation of Motion ..... 43
3.2 Cyclotron Motion ..... 44
3.3 Drift Motion ..... 45
3.4 Magnetic Moment ..... 50
3.5 Single Particle Simulations ..... 52
3.6 Simulation of Electron Motion in Laser Field ..... 55
References ..... 57
4 Equations for Electromagnetic Field ..... 59
4.1 The Poisson Equation ..... 59
4.2 The Maxwell Equations ..... 60
4.3 Potential ..... 63
4.4 Introduction to Kinetic Particle Simulation Model for Plasma ..... 64
4.4.1 Structure of Particle-in-Cell (PIC) Code ..... 65
4.4.2 Field Solver ..... 65
4.4.3 Interaction Between Field and Particles ..... 66
4.4.4 Simulation of Electron Cloud with Laser Field ..... 68
References ..... 69
5 Plasma by Fluid Model ..... 71
5.1 Basic Fluid Equations ..... 71
5.2 Introduction to Plasma Simulation by the Euler Fluid Model ..... 74
5.2.1 Summary of Finite Difference Method (FDM) ..... 75
5.2.2 Example 2D Simulation for Convection ..... 76
5.2.3 Numerical Instability and Time Step Control ..... 80
5.2.4 Numerical Instability and Its Analysis ..... 81
5.2.5 Example Simulation for Jet Injection ..... 82
5.2.6 Example Simulation for Diffusion: Heat Conduction ..... 85
5.3 Introduction to Plasma Simulation by the Lagrange Fluid Model ..... 88
5.4 Electron Plasma Wave ..... 93
5.5 Ion Acoustic Wave ..... 97
5.6 Electromagnetic Wave ..... 99
5.7 Magnetohydrodynamic Equation ..... 101
5.8 Frozen Magnetic Flux ..... 104
5.9 Waves in Magnetized Plasma ..... 105
References ..... 109
6 Plasma Treated by Distribution Function: Kinetic Model ..... 111
6.1 The Vlasov Equation ..... 111
6.1.1 The Klimontovich Equation ..... 111
6.1.2 The Liouville Equation ..... 112
6.1.3 The BBGKY Hierarchy and the Vlasov Equation ..... 115
6.2 Equilibrium Solution ..... 118
6.3 The Boltzmann Equation and Collision Effect ..... 119
6.4 Moment Equations and Fluid Model ..... 120
6.5 Dielectric Response Function for Unmagnetized Uniform Plasma ... ..... 122
6.6 Plasma Oscillation and the Debye Shielding .. ..... 126
6.7 Electron Plasma Wave and the Landau Damping ..... 128
6.8 Electron Wave Propagation in Equilibrium Plasmas ..... 129
6.9 Physical Meaning of the Landau Damping ..... 131
6.10 Dispersion Relation for Transverse Electromagnetic Waves ..... 133
6.11 Dispersion Relation for Magnetized Uniform Plasma ..... 135
6.12 Waves in Magnetized Uniform Plasma ..... 141
References ..... 144
7 Plasma Instability ..... 147
7.1 Two-Stream Instability ..... 147
7.1.1 Two-Stream Instability by Fluid Model ..... 147
7.1.2 Two-Stream Instability by Distribution Function ..... 150
7.1.3 Example Simulation for Two-Stream Instability ..... 152
7.2 Ion Acoustic Instability ..... 153
7.3 Instability of Magnetized Plasma Column ..... 154
7.3.1 Example Simulation for the Sausage and Kink Instabilities ..... 157
7.4 Interchange Instability—The Rayleigh-Taylor instability and an Example Simulation ..... 159
7.5 The Kelvin-Helmholtz Instability and an Example Simulation ..... 162
7.6 Parametric Instability ..... 165
7.7 The Weibel Instability ..... 166
7.8 Filamentation Instability and an Example Simulation ..... 167
7.9 Tearing Mode Instability and an Example Simulation ..... 170
7.10 Drift Instability ..... 173
References ..... 175
8 Short Introduction to Nonlinear Plasma Physics ... ..... 179
8.1 Solitary Wave: The Korteweg-de Vries (KdV) Equation ..... 179
8.1.1 The Korteweg-de Vries (KdV) Equation for Ion Acoustic Wave ..... 181
8.1.2 Property of the Korteweg-de Vries (KdV) Equation ..... 183
8.1.3 Inverse Scattering Transform for the Korteweg-de Vries (KdV) Equation ..... 185
8.2 The Burgers Equation and Shock Wave ............................ ..... 186
8.3 Plasma Echo ..... 188
8.4 A Glimpse at Turbulence ..... 189
References ..... 193
9 Applications of Plasmas ..... 197
9.1 Plasma Process ..... 197
9.2 Electron Temperature Measurement by Single-Probe Method ..... 197
9.3 Plasma Jet ..... 199
9.4 Nuclear Fusion ..... 200
9.4.1 Fusion Reaction ..... 201
9.4.2 The Lawson Criterion — To Sustain Fusion Reaction ..... 202
9.4.3 MCF: Magnetic Confinement Fusion ..... 203
9.4.4 ICF: Inertial Confinement Fusion ... ..... 205
9.5 Laser Particle Acceleration ..... 213
9.6 Cluster Ion Interaction with Plasma ..... 216
9.7 Control of Plasma: Dynamic Mitigation of Plasma Instabilities and Non-uniformities ..... 218
9.7.1 Control of Plasma: Theory ..... 220
9.7.2 Dynamic Control of Plasma Instabilities ..... 222
References ..... 242
Appendix A: Additional Readings ..... 251
Appendix B: Physical Constants and Mathematical Formulae ..... 255
Appendix C: Complex Analysis: Summary ..... 261
Appendix D: Derivation of Ponderomotive Force ..... 269
Appendix E: Parallel Computing by OpenMP ..... 271
Appendix F: Example 3D Pure Euler Fluid Code Structure ..... 275
Appendix G: Derivation of $\\epsilon_{x x}$ in Eq. (6.151) for Magnetized Uniform Plasma ..... 287
Index ..... 291

\\section*{Chapter 1 Introduction to Plasma }
\\begin{abstract}
Collective behavior is the characteristic of plasma. Plasma is ionized and consists of ions and electrons. Plasma is electrically quasi-neutral. Neutral gas is not ionized, and each neutral particle moves independently. However, each charged particle produces the long-range Coulomb force. Many charged particles move together, responding to electric and magnetic forces. The collective motion of plasma is remarkable and is different from that of the neutral gas. In this chapter, the collective behavior is introduced. In addition, computer simulation power and its uncertainty are also introduced to obtain correct physics.
\\end{abstract}

\\subsection*{Fourth State of Matter: Plasma}
Plasma is a collection of charged particles and is electrically quasi-neutral. Due to the long range of the electric force acting on charged particles, many plasma particles move together, responding to the field. This phenomenon shows the plasma collective motion, and the collective motion is the characteristic of plasma.

Plasma is often referred to as "the fourth state of matter". When solid is heated up, it becomes fluid. Fluid becomes gas after heating. When further heated, gas is ionized (see Fig. 1.1). Ionized gas, that is, "plasma" consists of ions and electrons.

When a hydrogen atom is ionized, the electron is ionized at the energy of about $13.6 \\mathrm{eV} .{ }^{1}$ Partially ionized gas is also plasma, in which neutral particles exist among charged particles.

On earth we can find plasma in fluorescent light tube, lightning, aurora and ionosphere. The fuels in the future nuclear fusion reactors also become plasma, and fusion energy is under intensive researches. On the other hand, it is considered that more than $99 \\%$ of matter is in the plasma state [1].
\\footnotetext{${ }^{1} \\mathrm{eV}$ is a unit of energy, and $1 \\mathrm{eV} \\sim 1.6022 \\times 10^{-19} \\mathrm{~J}$ (Joules). In plasma physics $\\mathrm{eV}$ is often used as a unit of energy. The temperature corresponding to $1 \\mathrm{eV}$ is about $11,604 \\mathrm{~K}$.

(C) The Author(s), under exclusive license to Springer Nature Singapore Pte Ltd. 2023
}

\\begin{center}
\\includegraphics[max width=\\textwidth]{2024_02_26_83e36543483eb7d284c1g-016(1)}
\\end{center}

a) Solid

\\begin{center}
\\includegraphics[max width=\\textwidth]{2024_02_26_83e36543483eb7d284c1g-016}
\\end{center}

Heating

\\begin{center}
\\includegraphics[max width=\\textwidth]{2024_02_26_83e36543483eb7d284c1g-016(2)}
\\end{center}

d) Plasma

Fig. 1.1 When a solid is heated up, it becomes b liquid, $\\mathbf{c}$ gas and then $\\mathbf{d}$ plasma. Each ball in $\\mathbf{a}-\\mathbf{c}$ shows atom schematically. Plasma is ionized and consists of ions and electrons. Larger balls present ions and smaller balls electrons in d. Arrows represent velocity. Plasma is electrically quasi-neutral. Charged particles interact with each other by the Coulomb force, that is, a long-range force

In plasma charged particles move together responding to the force created by other particles. The Coulomb force is a long-range force, which is proportional to $\\propto 1 / r^{2}$. Here $r$ shows the interparticle distance. In neutral gas, each particle interacts with each other, when one particle contacts physically to another particle. Neutral particles move individually. Plasma behaves collectively, and the collective behavior is the remarkable characteristic of plasma.

Figure 1.2 shows the plasma collective motion and the gas individual behavior schematically. The Coulomb force is a long-range force and extends to other charged particles. The neutral particle collision is an individual interaction with another neutral particle. Neutral gas does not show the collective motion. Plasmas show the collective behavior as well as the individual behavior.

When a small area in plasma becomes rich in positive charge, negative electrons come together to neutralize the excess positive charge. This is called shielding or the Debye shielding in plasma. In this situation, the electrons are accelerated during the shielding. When the electrons reach the ion excess part, they already have kinetic

\\begin{center}
\\includegraphics[max width=\\textwidth]{2024_02_26_83e36543483eb7d284c1g-017}
\\end{center}

Fig. 1.2 In plasma charged particles interact with each other by the Coulomb force, that is, a longrange force. Therefore, charged particles move collectively responding to the generated field by other charged particles. On the other hand, neutral particles move independently

energy and the electrons would pass through the positive area. The positive charge reappears, and then the electrons are pulled back again around the ion-rich area. Consequently the electrons would oscillate around the positive charge excess area. The phenomenon is called the plasma oscillation. The two phenomena show typical examples of the plasma collective motion.

Due to the plasma collective motion, interesting physical phenomena appear, and we will see them soon below: the Debye shielding, wave reflection and generation, particle acceleration, plasma instabilities, etc.

\\subsection*{Brief Introduction to Plasma Physics World}
Here let us see the difference between neutral gas and plasma briefly through example computer simulations. Now we have a spherical neutral hydrogen gas, which has no electric charge. The initial spherical gas density profile is shown in Fig. 1.3a. The gas density contours are shown in Fig. 1.3 for the gas hemisphere. The density is normalized by the initial density. The neutral gas expansion is characterized by $\\tau_{s} \\sim r_{0} / C_{s}$, where $r_{0}$ is the initial gas ball radius and $C_{s}$ shows the ion sound speed. Figure $1.3 \\mathrm{~b}$ and $\\mathrm{c}$ show that the neutral gas expands freely. The individual gas particles move independently (see also Fig. 1.2). Here we use the EPOCH Particle-in-Cell code to simulate gas and plasma behavior in Figs. 1.3 and $1.4[2,3]$.

$$
\\begin{array}{lll}
\\text { a) } t=0 & \\text { b) } t=\\tau_{s} / 2 & \\text { c) } t=\\tau_{s}
\\end{array}
$$

neutral gas expansion
\\includegraphics[max width=\\textwidth, center]{2024_02_26_83e36543483eb7d284c1g-018}

Fig. 1.3 Neutral spherical hydrogen gas expands freely. No collective motion is found in the results. Each hydrogen moves independently

On the other hand, Fig. 1.4 shows a collective motion of a spherical hydrogen plasma. Normalized electron density contours are shown for the hemispherical plasma ball. The outer part of the electrons first expands, and the electric field is created between the protons and the electron cloud. By the electric field, the electron cloud moves back to the ion cloud. The electric shielding and the collective oscillation are shown. The time scale $\\tau_{p}$ of the electron plasma oscillation is estimated by $\\sim 2 \\pi / \\omega_{p}$, where $\\omega_{p}$ is the plasma frequency for the electrons and is introduced soon below in Sect. 1.4 and in Eq. (1.13). The collective motion is found in the results. Electrons move collectively.

In Fig. 1.5, the normalized electric field is shown for the plasma collective motion shown in Fig. 1.4. The electric field vectors are shown by arrows for the hemispherical plasma ball. The radial electric field attracts electrons moving outward. The electron cloud moves back toward the ball center. Near the ball center, the oscillating electron cloud also generates the negative electric field to the ball center. The collective motion is again found in plasmas.

Figure 1.6 shows another example simulation results by the EPOCH code. A weak laser (electromagnetic wave) interacts with (a) a neutral gas ball and with (b) a dense plasma ball. The neutral gas has no electrons in this example, and the laser does not interact with the neutral gas. The laser penetrates the neutral gas freely as shown in Fig. 1.6a. On the other hand, Fig. 1.6b shows that the laser is reflected at the plasma surface. In this example case, the electron density is so high to fulfill the relation of $\\omega<\\omega_{p}$. Here $\\omega$ shows the laser wave frequency, and $\\omega_{p}$ the electron plasma oscillation frequency, which is introduced soon in Eq. (1.13). Later in Sect. 5.6 the relation of $\\omega<\\omega_{p}$ will be discussed. The physics for the laser reflection comes from the plasma collective motion, in which many electrons move collectively responding to the laser field and reflects the laser field. The detail physics will be presented in Sect. 5.6.

\\begin{center}
\\includegraphics[max width=\\textwidth]{2024_02_26_83e36543483eb7d284c1g-019(1)}
\\end{center}

\\section*{b) $t=0.3 \\tau_{p}$}
\\section*{Plasma behaves collectively by shielding \\& oscillation.}
$$
\\text { d) } t=\\tau_{p}
$$

\\begin{center}
\\includegraphics[max width=\\textwidth]{2024_02_26_83e36543483eb7d284c1g-019}
\\end{center}

Fig. 1.4 Spherical hydrogen plasma behaves collectively. Normalized electron density contours are shown for the hemispherical plasma ball. The outer part of the electrons first expands, and the electric field is created between the protons and the electron cloud. The electron cloud moves back to the ion cloud. The electric shielding and collective oscillation are shown. Collective motion is found in the results. Electrons move collectively

In the book, plasma physics is presented together with its numerical methods. Major mathematical models in plasma physics are fluid model and kinetic particle model [4-8]. The fluid model is a kind of a macroscopic model, described by the fluid equations, in which it is basically assumed that collisions between plasma particles are rich. The particle model is a microscopic model, in which individual charged particles are described by the equation of motion in electric and magnetic fields. The microscopic model would be based on distribution functions, which describe the motion of plasma particles. By averaging the distribution functions in the velocity space, the macroscopic fluid equations would be derived.

Figure 1.7 presents a summary of the book structure, which also serves a sketch of plasma physics world and its numerical methods.

Corresponding to the mathematical models, numerical methods have been intensively developed for plasma studies. Particle-in-Cell (PIC) methods among various particle methods are widely used in plasma studies [2,3,9-17]. The electric and magnetic fields are solved on spatial meshes, and charged particles are pushed by the fields, which are interpolated on particles from the field values on the spatial meshes. The charged particle motion induces the net charge and the electric current,

\\begin{center}
\\includegraphics[max width=\\textwidth]{2024_02_26_83e36543483eb7d284c1g-020}
\\end{center}

Fig. 1.5 Normalized electric field in a spherical hydrogen plasma. The electric field vectors are shown by arrows for the hemispherical plasma ball shown in Fig. 1.4. The radial electric field attracts electrons moving outward. The electron cloud moves back toward the ball center. Near the ball center, the oscillating electron cloud also generates the negative electric field to the ball center. Collective motion is found in plasmas

which change the original field values. The fluid numerical models have been also developed well. The continuity equation, the Navier-Stokes equation and the energy equation would be solved with the equation of state.

In the book, the essence of a simple PIC method is presented as a particle simulation method, and some example results are presented. For fluid simulations, a basic Euler fluid simulation method is first introduced. Then a Lagrange fluid method is also introduced. A discretization method, called finite difference method (FDM), is introduced and used in the book in the PIC and fluid codes.

In addition to the numerical techniques, numerical stability is also explained and discussed. The numerical errors and instabilities do not come from physics but from the numerical methods artificially. They would have a significant impact on numerical results, and users must avoid the numerical errors and instability in order to obtain correct results in physics. The numerical errors, numerical stability and uncertainty in numerical computations are also discussed.

\\begin{center}
\\includegraphics[max width=\\textwidth]{2024_02_26_83e36543483eb7d284c1g-021(1)}
\\end{center}

b) Laser is reflected.

\\begin{center}
\\includegraphics[max width=\\textwidth]{2024_02_26_83e36543483eb7d284c1g-021}
\\end{center}

Fig. 1.6 Weak laser (electromagnetic wave) interacts $\\mathbf{a}$ with a neutral gas ball and with $\\mathbf{b}$ a dense plasma ball. The neutral gas has no electrons in this example, and the laser does not interact with the neutral gas. The laser penetrates the neutral gas freely as shown in Fig. 1.6a. On the other hand, Fig. 1.6b shows that the laser is reflected at the plasma surface. In this example case, the electron density is so high to fulfill the relation of $\\omega<\\omega_{p}$. The electron collective motion contributes to the laser reflection at the plasma surface. The detail physics will be presented in Sect. 5.6

\\subsection*{Collective Motion: The Debye Shielding}
One of the collective behaviors is the Debye shielding, and that is roughly shown in Figs. 1.2, 1.4 and 1.5. Figure 1.8 shows the Debye shielding schematically. Plasma is quasi-neutral. When a plus-charged particle of $+q_{T}$ is introduced from the outside as a perturbation into plasma, electrons approach to $+q_{T}$ and ions move away together.

Here we analyze the Debye shielding. Electron is much lighter than heavy ion. ${ }^{2}$ So we assume that ions do not move and the ion number density $n_{0}$ is uniform in space. In the plasma a charge of $+q_{T}$ is introduced as shown in Fig. 1.8.

When electrons in plasma are in thermal equilibrium with the temperature of $T$, the electron distribution would be described below:


\\begin{equation*}
n_{e}=n_{0} \\exp \\left(+\\frac{e \\varphi}{T}\\right) \\tag{1.1}
\\end{equation*}


We assume that Eq. (1.1) is given, and later it will be derived (see Eqs. (2.18) and (2.19)). ${ }^{3}$ In Eq. (1.1), $\\varphi$ shows the Coulomb potential, and $-e$ the electron charge.
\\footnotetext{${ }^{2}$ For example, the proton mass $m_{p}$ is $\\sim 1836$ times the electron mass.

${ }^{3}$ In this book, temperature $T$ is treated in the unit of energy. The temperature $T$ means $k_{B} T$. Here $k_{B}$ is the Boltzmann constant (see Appendix B.1).
}

\\begin{center}
\\includegraphics[max width=\\textwidth]{2024_02_26_83e36543483eb7d284c1g-022}
\\end{center}

Fig. 1.7 Relation between mathematical models and numerical models in plasma physics. The book structure is also presented. Plasmas are described by microscopic mathematical models based on distribution functions $f(t, \\mathbf{v}, \\mathbf{v})$ and macroscopic fluid models. Corresponding to the mathematical models, numerical models are developed: particle simulation methods and fluid simulation methods

The self-consistent potential $\\varphi$ should be positive around $+q_{T}$, and electrons are attracted to $+q_{T}$. Around $+q_{T}$ the electron density becomes larger than $n_{0}$. Far from the extra charge of $+q_{T}$, the potential $\\varphi$ tends to 0 , and the electron density goes to $n_{0}$, as expected. Therefore, Eq. (1.1) would express the Debye shielding correctly. Here we consider the following case:


\\begin{equation*}
e \\varphi / T \\ll 1 \\tag{1.2}
\\end{equation*}


Then Eq. (1.1) is approximately expressed as follows:


\\begin{equation*}
n_{e} \\simeq n_{0}\\left(1+\\frac{e \\varphi}{T}\\right) \\tag{1.3}
\\end{equation*}


Here we employ the Poisson equation to obtain the potential $\\varphi$ explicitly. We assume that the phenomenon is spherically symmetric and depends only on the radius $r$. We assume the extra charge of $+q_{T}$ is located at the origin.

Fig. 1.8 In a quasi-neutral plasma one charged particle with a charge of $+q_{T}$ is introduced from the outside. Charged particles surrounding respond to the field by $+q_{T}$ and move together to neutralize the field

\\begin{center}
\\includegraphics[max width=\\textwidth]{2024_02_26_83e36543483eb7d284c1g-023}
\\end{center}


\\begin{align*}
\\nabla^{2} \\varphi & =\\frac{1}{r^{2}} \\frac{d}{d r}\\left(r^{2} \\frac{d \\varphi}{d r}\\right)=-\\frac{\\left(n_{i}-n_{e}\\right) e}{\\epsilon_{0}} \\\\
& \\simeq \\frac{n_{0} e^{2}}{\\epsilon_{0} T} \\varphi \\tag{1.4}
\\end{align*}


Here $n_{i}=n_{0}$, and Eq. (1.3) is used. ${ }^{4}$

Here we introduce the following parameter ${ }^{5}$ :


\\begin{equation*}
\\lambda_{D} \\equiv \\sqrt{\\frac{T \\epsilon_{0}}{n_{0} e^{2}}} \\sim 743 \\times \\sqrt{T(\\mathrm{eV}) / n\\left(\\mathrm{~cm}^{-3}\\right)}[\\mathrm{cm}] \\tag{1.5}
\\end{equation*}


The parameter $\\lambda_{D}$ is called the Debye length. Multiplying the both sides of Eq. (1.4) by $r$, the following Eq. (1.6) is obtained:


\\begin{equation*}
\\frac{1}{r} \\frac{d}{d r}\\left(r^{2} \\frac{d \\varphi}{d r}\\right)=\\frac{d^{2}(r \\varphi)}{d r^{2}}=\\frac{r \\varphi}{\\lambda_{D}^{2}} \\tag{1.6}
\\end{equation*}


Assuming that $r \\varphi$ is proportional to $\\exp (a r)$, and inserting the relation into Eq. (1.6), $a^{2}=1 / \\lambda_{D}^{2}$ is obtained. Then $a= \\pm 1 / \\lambda_{D}$.
\\footnotetext{${ }^{4}$ See Appendix B. 3 for the differential operator in the spherical coordinate.

${ }^{5}$ In plasma the SI unit is often used, but the CGS unit is also used.
}


\\begin{equation*}
\\varphi=\\frac{C_{ \\pm}}{r} \\exp \\left( \\pm \\frac{r}{\\lambda_{D}}\\right) \\tag{1.7}
\\end{equation*}


Here $C_{+}$and $C_{-}$show the integral constants. We need a solution, in which $\\varphi \\rightarrow 0$, when $r \\rightarrow \\infty$. Therefore, the sign of - is employed, and $C_{+}=0$. We assumed that $+q_{T}$ is located at $r=0$. In this case, we must have the bare Coulomb potential at $r \\sim 0: \\varphi \\rightarrow q_{T} /\\left(4 \\pi \\varepsilon_{0} r\\right)$, when $r \\rightarrow 0$. In Eq. (1.7), the integral constant of $C_{-}$ should be $q_{T} /\\left(4 \\pi \\varepsilon_{0}\\right)$.

Finally we obtain the following solution for the Debye shielding:


\\begin{equation*}
\\varphi=\\frac{q_{T}}{4 \\pi \\varepsilon_{0} r} \\exp \\left(-\\frac{r}{\\lambda_{D}}\\right) \\tag{1.8}
\\end{equation*}


The bare Coulomb potential $q_{T} /\\left(4 \\pi \\epsilon_{0} r\\right)$ in a vacuum by $q_{T}$ is shielded by the factor of $\\exp \\left(-r / \\lambda_{D}\\right)$ in plasma. The Debye length of $\\lambda_{D}$ is the length scale that the bare potential can reach in plasma. Beyond the Debye length the potential is shielded in plasma. The Debye length is defined by Eq. (1.5), and $k_{D} \\equiv 2 \\pi / \\lambda_{D}$ is called the Debye wavenumber.

The collective motion of the Debye shielding results from the collective motion of plasma to recover the charge neutrality.

Here we examine the Debye length scale by Eq. (1.5). In solar wind, the electron density is about $10^{1} / \\mathrm{cm}^{3}$ and the electron temperature is around $10 \\mathrm{keV}$. In the solar wind, $\\lambda_{D} \\sim 230 \\mathrm{~m}$. In the ionosphere, the electron number density is $\\sim 10^{6} / \\mathrm{cm}^{3}$, the electron temperature is about $0.1 \\mathrm{eV}$, and $\\lambda_{D} \\sim 0.23 \\mathrm{~cm}$. In nuclear fusion device, especially in inertial confinement fusion, the electron number density is more than $10^{25} / \\mathrm{cm}^{3}$, the electron temperature would be about $10 \\mathrm{keV}$, and $\\lambda_{D} \\sim 2.3 \\times 10^{-8} \\mathrm{~cm}$. The Debye length of $\\lambda_{D}$ changes largely depending on the number density and temperature.

\\subsection*{Collective Motion: Plasma Oscillation}
In the Sect. 1.4, plasma oscillation is introduced and was also roughly introduced in Figs. 1.2, 1.4 and 1.5. Plasma oscillation is also one of the collective phenomena in plasma.

As we see above, plasma is quasi-neutral and consists of ions and electrons. When the charge density is perturbed in plasma, the Coulomb force cancels the perturbed charge.

As shown in Fig. 1.9a, the perturbed electric charge may appear in plasma at $t=0$. In this case, the electric field is induced as indicated by an arrow with the sign of $E$, and the charged particles are pulled back to the original position. Here we focus on the light electrons. When the electrons come back to the original position and just overlap the ion cloud, the electrons are already accelerated and move beyond the ion

\\begin{center}
\\includegraphics[max width=\\textwidth]{2024_02_26_83e36543483eb7d284c1g-025}
\\end{center}

Fig. 1.9 Plasma oscillation. At $t=0$, a part of plasma is perturbed, and electric charge appears. The electric field, created by the charge, pulls back electrons and ions to compensate the charge separation for the charge inequality. Here we focus on the light electrons. When the electron cloud just overlaps the background ion cloud, the electrons are already accelerated and move further beyond the ion cloud

cloud. Then again the reverse electric field would appear in plasma. This collective electron motion is called the electron plasma oscillation.

Now we derive the electron plasma oscillation frequency. We assume that only electrons move and ions are stationary, as shown in Fig. 1.9. For simplicity, we also assume that the phenomenon is one dimensional in $x$, and the displacement is $\\delta$ in $x$ (see Fig. 1.9).

The electric field in Fig. 1.9 is estimated by the following Poisson equation:


\\begin{gather*}
\\frac{\\mathrm{d} E}{\\mathrm{~d} x}=\\frac{n e}{\\varepsilon_{0}}  \\tag{1.9}\\\\
E=\\frac{n e \\delta}{\\varepsilon_{0}} \\tag{1.10}
\\end{gather*}


The displacement of $\\delta$ is obtained by the equation motion for electrons.


\\begin{equation*}
m_{e} \\frac{\\mathrm{d}^{2} \\delta}{\\mathrm{d} t^{2}}=-e E=-\\frac{n e^{2} \\delta}{\\varepsilon_{0}} \\tag{1.11}
\\end{equation*}


It is rewritten as follows:


\\begin{equation*}
\\frac{\\mathrm{d}^{2} \\delta}{\\mathrm{d} t^{2}}+\\left(\\frac{n e^{2}}{m_{e} \\varepsilon_{0}}\\right) \\delta=0 \\tag{1.12}
\\end{equation*}


Equation (1.12) is the equation for a harmonic oscillator. The oscillation frequency is as follows:


\\begin{equation*}
\\omega_{p e}=\\sqrt{\\frac{n e^{2}}{m_{e} \\varepsilon_{0}}} \\sim 5.64 \\times 10^{4} \\sqrt{n_{e}\\left(\\mathrm{~cm}^{-3}\\right)}[\\mathrm{rad} / \\mathrm{s}] \\tag{1.13}
\\end{equation*}


This is the electron plasma frequency $\\omega_{p e}$. In Fig. 1.9b, $\\Delta t \\sim \\pi / \\omega_{p e}$.

By Eq. (1.13), let us see example values for the electron plasma oscillation frequency. In solar wind, the electron density is about $10^{1} / \\mathrm{cm}^{3}$ and $\\omega_{p e} \\sim 1.8 \\times 10^{5} / \\mathrm{s}$. In the ionosphere, the electron number density is $\\sim 10^{6} / \\mathrm{cm}^{3}$ and $\\omega_{p e} \\sim 5.6 \\times 10^{7} / \\mathrm{s}$. In nuclear fusion device, especially in inertial confinement fusion, the electron number density is more than $10^{25} / \\mathrm{cm}^{3}$ and $\\sim 1.8 \\times 10^{17} / \\mathrm{s}$. The electron plasma frequency of $\\omega_{p e}$ also changes largely depending on the number density.

\\subsection*{Conditions for Plasma Collective Motion}
As we see above, plasma particles have charges. Plasma behaves collectively and is different from neutral gas. What are the conditions, under which plasma shows the collective motion, including the Debye shielding and the plasma oscillation?

One condition comes from the spatial size $L$ of plasma. The plasma size $L$ should be larger than the Debye length $\\lambda_{D}$. Otherwise, the shielding cannot be realized.


\\begin{equation*}
\\lambda_{D} \\ll L \\tag{1.14}
\\end{equation*}


Another condition comes from the total particle number in plasma. Inside the radius of the Debye length around the extra charge of $+q_{T}$ in Fig. 1.8, there should be many charged particles to shield the extra charge collectively.


\\begin{equation*}
N_{D} \\equiv n \\frac{4 \\pi \\lambda_{D}^{3}}{3} \\gg 1 \\tag{1.15}
\\end{equation*}


If $N_{D} \\sim$ or $<1$, the plasma collective motion cannot be found. This is another condition for the plasma collective motion.

$N_{D}$ is rewritten as follows:


\\begin{equation*}
N_{D}=\\frac{1}{3 \\Gamma^{3 / 2}} \\tag{1.16}
\\end{equation*}


Here $\\Gamma$ is called the coupling parameter or the Coulomb coupling parameter:


\\begin{equation*}
\\Gamma \\equiv \\frac{\\left(\\frac{q^{2}}{4 \\pi \\varepsilon_{0} a}\\right)}{T} \\tag{1.17}
\\end{equation*}


Here $a$ is the averaged radius of volume, which one charged particle occupies in plasma. When $4 \\pi a^{3} n / 3 \\sim 1, a$ shows the averaged radius $a$.


\\begin{equation*}
a \\sim\\left(\\frac{3}{4 \\pi n}\\right)^{1 / 3} \\tag{1.18}
\\end{equation*}


Equation (1.17) presents the ratio of the Coulomb potential energy $q^{2} /\\left(4 \\pi \\varepsilon_{0} a\\right)$ to the plasma kinetic energy $T$. The condition of Eq. (1.15) means $\\ll 1$. In this case, the Coulomb force is weak compared with the particle kinetic motion, and the electrons would not be trapped by the nucleus Coulomb force.

When $\\geq 1, N_{D} \\leq 1$ and the normal plasma collective motion would not be seen. The plasma is called the strongly coupled plasma under the condition of $\\geq 1$. The strongly coupled plasma is not treated in the book. For the strongly coupled plasma, Refs. [4, 5] are recommended.

Here we examine the coupling parameter values by Eq. (1.17). In solar wind from the sun, the electron density is about $10^{1} / \\mathrm{cm}^{3}$ and the electron temperature is around $10 \\mathrm{KeV}$. In the solar wind, $\\sim 5 \\times 10^{-11}$. In the ionosphere, the electron number density is $\\sim 10^{6} / \\mathrm{cm}^{3}$, the electron temperature is about $0.1 \\mathrm{eV}$, and $\\sim 2.3 \\times 10^{-4}$. In inertial confinement fusion, the electron number density is more than $10^{25} / \\mathrm{cm}^{3}$, the electron temperature would be about $10 \\mathrm{KeV}$, and $\\sim 5.0 \\times 10^{-3} \\mathrm{~cm}$. In these examples the Coulomb coupling parameter value is rather small compared with 1 : $\\ll 1$.

\\subsection*{Introduction to Mathematical Models for Plasma}
There are two well-known mathematical models to describe plasma mathematically. One is based on each particle behavior described by particle distribution function of $f(t, \\mathbf{x}, \\mathbf{v})$. In this model, the independent variables are the time $t$, the spatial vector $\\mathbf{x}$ and the velocity $\\mathbf{v}$. The detail information for each plasma particle is represented by the distribution function $f(t, \\mathbf{x}, \\mathbf{v})$ in this microscopic kinetic model. The governing equation for $f(t, \\mathbf{x}, \\mathbf{v})$ would be the Vlasov equation and the Boltzmann equation $[4,5,8,18]$, which are introduced later in Chap. 6.

The other model is fluid model, in which physical quantities are treated in the space of $(t, \\mathbf{x})$ (see, for example, Chap. 7 in Ref. [6]). The fluid model is introduced in Chap. 5. The fluid model is a kind of averaged model, in which physical quantities are obtained by integrations of the distribution $f(t, \\mathbf{x}, \\mathbf{v})$ over the velocity $\\mathbf{v}$ or the momentum space $\\mathbf{p}$. For example, the fluid number density $n(t, \\mathbf{x})$ and the fluid velocity $\\langle\\mathbf{v}\\rangle(t, \\mathbf{x})$ are obtained as follows:


\\begin{gather*}
n(t, \\mathbf{x})=\\iint_{-\\infty}^{\\infty} \\int_{-\\infty}^{\\infty} \\mathrm{d} \\mathbf{v}(t, \\mathbf{x}, \\mathbf{v})=\\iiint_{-\\infty} \\mathrm{d} v_{x} \\mathrm{~d} v_{y} \\mathrm{~d} v_{z} f(t, \\mathbf{x}, \\mathbf{v})  \\tag{1.19}\\\\
n(t, \\mathbf{x})\\langle\\mathbf{v}\\rangle(t, \\mathbf{x})=\\iiint_{-\\infty}^{\\infty} \\mathrm{d} \\mathbf{v} \\mathbf{v} f(t, \\mathbf{x}, \\mathbf{v})=\\iiint_{-\\infty}^{\\infty} \\mathrm{d} v_{x} \\mathrm{~d} v_{y} \\mathrm{~d} v_{z} \\mathbf{v} f(t, \\mathbf{x}, \\mathbf{v}) \\tag{1.20}
\\end{gather*}


In the fluid model, the individuality of each particle is smeared out, and physical quantities are averaged over many particles at $(t, \\mathbf{x})$. In a small volume represented by the spatial length of $\\Delta l$ around $\\mathbf{x}$, each physical quantity is averaged. Therefore, $\\Delta l$ should be larger than the interparticle distance $a$ by Eq. (1.18). On the other hand, the spatial length of $\\Delta l$ should be small compared with the spatial size of fluid (plasma) $L$. In order to satisfy the relation of $a \\ll \\Delta l \\ll L$, the thermal equilibrium should be locally attained in the scale length of $\\Delta l$ in the fluid model. This requirement means that sufficient collisions occur in the scale length $\\Delta l$. At the same time, the thermal equilibrium should be attained in a short time scale of $\\Delta t$, which is small compared with the time scale of the fluid motion.

As shown above, the fluid model is a macroscopic mathematical model. However, in many cases the fluid model is rather powerful to describe the plasma behavior correctly. On the other hand, the microscopic method by the distribution function $f(t, \\mathbf{x}, \\mathbf{v})$ can describe plasmas precisely. However, for realistic plasmas we have to treat many particles, and sometimes too many particles are in plasmas. For example, the number density of the ionosphere is $\\sim 10^{6} / \\mathrm{cm}^{3}=10^{12} / \\mathrm{m}^{3}$. It would not be realistic to work with so many particles directly.

In addition to theory and experiment methods to treat plasmas, numerical methods are widely used. In the next Sect. 1.7, we briefly introduce the numerical methods.

\\subsection*{Introduction to Computer Simulation for Plasma}
Corresponding to the microscopic kinetic method based on the distribution function $f(t, \\mathbf{x}, \\mathbf{v})$, particle method $[2,3,9-11]$ is widely used in plasma studies. Especially, Particle-in-Cell (PIC) method is widely used in plasma simulations [2, 3, 9-17]. In PIC code numerical particles move through electromagnetic field. Charged particle motion is solved by the equation of motion or the relativistic equation of motion. The electric and magnetic fields are obtained on each particle for the particle pusher. The electric and magnetic fields are solved by the Maxwell equations (see, for example, Chap. 6 in Ref. [19]) on spatial mesh cells, which may be fixed to the spatial coordinate. PIC method would provide a very strong and useful tool to analyze plasmas numerically.

In PIC methods, normally each numerical particle, sometimes called superparticle, represents a collection of many real particles. Even in PIC method, normally it would be very difficult to simulate all real particles independently. Instead, each numerical superparticle carries many real particles, while keeping $q / m$ constant. This special prescription in PIC methods comes from the fact, that even supercomputers cannot simulate all the real plasma particles in realistic plasmas. Therefore, PIC methods has restrictions, when they are applied to plasma simulations. One numerical superparticle carries many real particles, and so detail real collisions among particles would not be realized in PIC methods. If smaller-size spatial meshes, which are much smaller than $\\lambda_{D}$, are used in PIC simulation, the bare Coulomb collisions appear among superparticles, which carry many real particles. The Coulomb collisions among superparticles would result in a large unphysical collional effect. Therefore, in PIC codes, usually mesh size should be the order of the Debye length $\\lambda_{D}$. Based on the PIC special feature, PIC methods would be appropriate to simulate collisionless plasmas. However, in order to extend the applicability of the PIC codes to collisional plasmas, there are several means to include the collision effect $[2,3$, 12-14, 20, 21]. For example, kinetic laser plasma simulation (KLAPS) PIC code includes the Coulomb collision, field ionization, radiation reaction, photon polarization and particle spin polarization in QED (quantum electrodynamics) regime [13, 14].

On the other hand, fluid model is appropriate to simulate collisional plasmas as discussed in Sect. 1.6. The fluid equations are prepared to analyze fluids, in which one small volume element in fluid does not penetrate the adjacent fluid element due to the collisions. When high-density plasmas are analyzed, the fluid model would be appropriate. In the plasma simulations based on fluid models, various numerical methods are applicable to simulate plasmas [22, 23].

\\subsubsection*{Computer Simulation Power and Uncertainty in Computation}
Computer simulation methods are powerful to analyze plasma and physics. However, we have to keep in our mind the simulation method restrictions and the non-physical features of numerical methods [24-26].

In this book we discuss on plasma physics and its computer simulation. In order to obtain correct physics, in Sects. 4.4 and 5.2 and this Sect. 1.7.1 we discuss on numerical errors or uncertainty, including numerical stability, reasonable size of time step $\\Delta t$ and spatial mesh size $\\Delta x$, and appropriate numerical algorithms.

In order to find correct physics from numerical results, we have to avoid all the numerical errors in all the computational processes. However, it would be quite difficult to check the correctness at every numerical step. Even if a few incorrect steps get mixed in computational processes, we may still obtain numerical results, which may seem correct.

Computer would provide unexpected results for some computations [24-26]. Below we briefly introduce the computational uncertainty. If readers are experts in computing science, you may already know the contents below. In that case, readers can skip the following.

In many cases in computer simulations for plasma or for physics [2, 10, 11, 22], the time $t$ and space $\\mathbf{x}$ are discretized in order to compute physical quantities on spatial meshes at time $t$ as shown in Fig. 1.10. Figure 1.10 shows an example for the spatial discretizations in a) two-dimensional (2D) space and b) time $t$. Along with the discretizations on $t$ and $\\mathbf{x}$, equations for physical quantities are also discretized. The discretizations for equations will be introduced later. Physical quantities would be defined on discrete points in $t$ and $\\mathbf{x}$ (see Fig. 1.11). Figure 1.11 shows an example for the definitions of physical quantities on the discretized $t$ and $\\mathbf{x}$.

Of course, physical quantities are continuous in $t$ and $\\mathbf{x}$. In order to solve physical equations on computers, usually the discretizations explained above are needed in many realistic cases in plasma and physics. If analytical solutions are found, the continuous solutions are obtained. However, for many cases, even supercomputers would not compute physical quantities on the continuous space of $t$ and $\\mathbf{x}$. The requirement of the discretizations for $t, \\mathbf{x}$ and equations does not come from physics but from requirement of computers. The discretization is introduced into computer analyses regardless of physics. Therefore, the discretization would introduce nonphysical phenomena in physical analyses. In this Sects. 1.7.1, 4.4 and 5.2, the nonphysical feature of the discretization and uncertainty of computation will be also introduced.

Non-physical feature may come from physical model error, mathematical model error, numerical model error, programming error, computation error, analysis error including visualization error, etc. [24-26]. Physics and mathematical model may not be always clear, and model errors may appear in plasma physics. Numerical algorithms or discretization methods are not always well developed, and the discretization may induce non-physical phenomena, which will be shown in Sects. 5.2 and 1.7.1.

\\includegraphics[max width=\\textwidth, center]{2024_02_26_83e36543483eb7d284c1g-031(1)}
b)

\\begin{center}
\\includegraphics[max width=\\textwidth]{2024_02_26_83e36543483eb7d284c1g-031}
\\end{center}

Fig. 1.10 In many cases in computer simulations time and space are discretized. In this example, 2-dimensional space is discretized in (a). Basically physical quantities are continuous in space and time. However, in order to solve them by computers, usually physical quantities are obtained approximately at discrete points in $\\mathbf{a}$ space and $\\mathbf{b}$ time
\\includegraphics[max width=\\textwidth, center]{2024_02_26_83e36543483eb7d284c1g-031(2)}

Fig. 1.11 Physical quantities are usually calculated and obtained at each definition point on each discrete meshes (cells) in time and space. In this example case in a, one spatial mesh (cell) is enlarged. Each spatial mesh (cell) is specified by the number of $(i, j, k)$ or so. In this example case, the $x$ and $y$ components of the electric field $\\mathbf{E}$ and the $z$ component of the magnetic field $\\mathbf{B}$ are obtained in different positions in one mesh. In $\\mathbf{b}$ the time $t$ is also discretized and physical quantities are obtained at $t=n \\times d t$ or $t=(n+1 / 2) \\times d t$ or so discretely. Here $n$ specifies the discrete time step

The discretization would impose numerical stability conditions. If the numerical stability conditions are violated in computations, numerical instability may appear, and correct physics may not be represented by the discretized equations. We have to also select appropriate numerical model or algorithm to express original physics. In addition, parallel computation may be needed to save computation time. As discussed in Appendix E, correct parallel algorithm must be selected to obtain correct results. If an inappropriate parallel algorithm is selected in computation, correct numerical values may be destroyed by other values. When plasmas are analyzed, sometimes
new computer programs may be required. In the cases, programming errors may be introduced in the new computer codes. Human errors, including programming errors, may not always be avoided. In this Sect. 1.7.1 some references for mechanical program generation are also introduced to show a new direction to assist program generation [27]. In computer numerical numbers are treated by a finite number of digits, and so rounding errors would be introduced to express physical quantities. Even in plasma computation, careful attention should be paid to numerical precision to obtain correct results. Numerical precision should be sufficient to avoid numerical truncation errors, as shown below. After obtaining numerical results, correct data analyses and visualizations are also required to extract correct physics.

In computer, numerical numbers are treated by a finite number of digits. Here, let us calculate some simple arithmetic by an equation manipulation software of MAPLE [28] with 10 digits. Set $a=7.0$, and add a small number $b$. In the following example, $b=10^{-9}$ or $b=10^{-10}$. In $c 1=7.0+10^{-9}$ the correct answer of 7.000000001 is obtained, and the $b=10^{-9}$ appears at the last digit as " 1 ". However, in the case of $c 2=7.0+10^{-10}, b=10^{-10}$ disappears. In this example, even if $b=10^{-10}$ is added to $a=7.0$ by $10^{10}$ times, the wrong result of 7.000000000 is kept. In the following we used the computer software of Maple [28]:

\\begin{center}
\\includegraphics[max width=\\textwidth]{2024_02_26_83e36543483eb7d284c1g-032}
\\end{center}

So we have to be aware of the accuracy of the computing. Let us assume that one rich group has 7 billion US\\$ in a bank and earns an interest of $1 \\$$ per second. Every second the group should get $1 \\$$ more. If the bank computer works as shown above with the 10 significant digits, the savings never increase. Where does the interest go? Perhaps someone else may operate the computer and may transfer the interest to his or her savings. He or she may collect the interest fractions from a number of accounts [29].

We may say that "computer has uncertainty, and computer is not perfect"

In February 1991 at Dhahran, Saudi Arabia, an accident happened by a numerical error. A missile defense system failed to intercept enemy missiles. As a result, 28
allies were killed. In this case numerical rounding errors provided a wrong time [30]. Consequently, a significant tracking error appeared and induced the tragedy.

The numerical errors would come from the following origins: physical models, mathematical models, numerical models, numerical errors including rounding errors, insufficient input data, boundary conditions and data analyses errors, as well as human errors.

In 1991 another accident happened during a construction of a marine platform for an oil well under the sea, and the platform under construction sank into the sea [31]. The platform structure was designed by numerical results, and its real strength was weak. In this case, the computer software and mathematical model were appropriate. However, the numerical accuracy was not sufficient.

Computer software and computer simulations have been powerful tools to support our society and our life. The accident examples shown above suggest that uncertainty in computations may cause serious accidents, human life losses and serious economical losses.

In order to avoid the accidents and serious social influences, intensive studies have been performed on the uncertainty in computations [24-26, 32-41]. If computers themselves tell us our incorrect usages of computers and software, it would be very welcome.

So far we have introduced briefly about errors in physical model, mathematical model, algorithms and numerical model, as well as human-relating errors and truncation errors. We may also meet difficulties to find correct initial conditions and boundary conditions. When we simulate some phenomena, we may meet difficulties to construct physical models or mathematical models. In addition to numerical simulations, we also need experiments or actual machine tests, as well as theoretical works.

In computers numerical values are represented by binary numbers. Here a decimal number of 1.2 is expressed by a binary number with the 20-digit accuracy.


\\begin{equation*}
1.2_{\\text {DecimalNumber }}=\\left.1.0011001100110011001\\right|_{\\text {BinaryNumber }} \\tag{1.21}
\\end{equation*}


In computers real numbers are expressed by floating-point numbers:


\\begin{equation*}
\\pm\\left(\\frac{1}{2}+\\frac{a_{2}}{2^{2}}+\\frac{a_{3}}{2^{3}}+\\cdots+\\frac{a_{n}}{2^{n}}\\right) \\times 2^{m} \\tag{1.22}
\\end{equation*}


Here the base is 2 in the binary expressions, $a_{n}$ is 0 or 1 , and $m$ is the exponent. For the single precision, the floating-point numbers are described by 32 bits. One bit is used for the sign, the exponent uses 8 bits, and the rest is used by the mantissa or the significand, that is, the significant digits showing in the bracket in Eq. (1.22) $[42,43]$. The real numbers are continuous. However, the floating-point numbers are discrete, and the distance between two consecutive floating-point numbers is called as the machine epsilon or $u l p$ (unit in the last place). For the single precision the ulp is about $10^{-7}$, for the double (64 bits) precision the ulp is about $10^{-16}$, and for
the quadruple precision the $u l p$ is about $10^{-34}$. Therefore, the floating-point number may be different from the real number by $\\pm 0.5$ ulp.

Due the limitation of the numerical value expression, we have to avoid underflows and overflows during computations. The normalizations of basic equations may help to avoid the limitation. Another method would be to employ the double or quadruple precisions in computations to avoid the computer restriction.

IEEE 754 defines the floating-point rounding method [42]: Round to Nearest (Even or away from zero): Rounding to the nearest value, Round Up, Round Down and Round toward 0: Truncation. By using the rounding methods, we can check the effect of the rounding error [44].

\\begin{center}
\\includegraphics[max width=\\textwidth]{2024_02_26_83e36543483eb7d284c1g-035}
\\end{center}

The results may suggest that we can estimate the required digit number for computations [25, 44-48].

Figure 1.12 shows an example, in which the simple function of $f_{1}(x)=x$ is computed. The result should be a linear line. The same simple function is modified as $f_{2}(x)=\\left(-\\ln \\left(\\exp \\left(-x^{-4}\\right)\\right)\\right)^{-1 / 4}$ [24]. Figure 1.12 shows the results in the single precision. The "Round to Nearest (Even)" is described as RN, "Round Upward" is shown by RP, "Round Downward" is described as RM, and "Round Toward 0" is

\\begin{center}
\\includegraphics[max width=\\textwidth]{2024_02_26_83e36543483eb7d284c1g-036(1)}
\\end{center}

Fig. 1.12 Numerical results for the mathematically same functions of $f_{1}(x)=x$ and $f_{2}(x)=$ $\\left(-\\ln \\left(\\exp \\left(-x^{-4}\\right)\\right)\\right)^{-1 / 4}$ in the single precision. The results suggest that the single precision of 32 bits is not sufficient to compute $f_{2}$

Fig. 1.13 Numerical results by the quad precision for the same functions of $f_{1}(x)=x$ and $f_{2}(x)=$ $\\left(-\\ln \\left(\\exp \\left(-x^{-4}\\right)\\right)\\right)^{-1 / 4}$ in Fig. 1.12. It seems that the quad precision is sufficient to obtain the correct results

\\section*{Quad precision}
\\begin{center}
\\includegraphics[max width=\\textwidth]{2024_02_26_83e36543483eb7d284c1g-036}
\\end{center}

shown by RZ. The results suggest that the single precision is not sufficient to calculate $f_{2}$.

For the function $f_{2}(x)=\\left(-\\ln \\left(\\exp \\left(-x^{-4}\\right)\\right)\\right)^{-1 / 4}$, the double precision was also insufficient. Figure 1.13 shows the results by the quad precision and 128 bits. The 5 results are overlapped. It seems that the quad precision is sufficient to obtain the correct results in this case.

As an example physical simulation, Fig. 1.14 shows a shock wave propagation in a plasma. The fluid code introduced later in Chap. 5 was used.

For the shock wave problem in Fig. 1.14, we measure the differences among computing results, and the results is summarized in Table 1.1, in which the computations

Fig. 1.14 Shock wave propagation. The fluid code introduced in Chap. 5 was used in this example physical simulation

\\begin{center}
\\includegraphics[max width=\\textwidth]{2024_02_26_83e36543483eb7d284c1g-037}
\\end{center}

Table 1.1 Results for a shock wave propagation by the single precision

\\begin{center}
\\begin{tabular}{l|l}
\\hline
Rounding method & Difference or error \\\\
\\hline
RN & $3.64 \\times 10^{-1} \\%$ \\\\
\\hline
RP & $1.87 \\times 10^{-1} \\%$ \\\\
\\hline
RM & $1.70 \\times 10^{-1} \\%$ \\\\
\\hline
RZ & $3.09 \\times 10^{-2} \\%$ \\\\
\\hline
\\end{tabular}
\\end{center}

The differences between the result by one rounding method and the averaged value over all the results by all the rounding methods. The results by the single precision provide relatively large differences among them

Table 1.2 Difference among the results for the shock wave propagation by the double precision

\\begin{center}
\\begin{tabular}{l|l}
\\hline
Rounding method & Difference or error \\\\
\\hline
RN & $2.50 \\times 10^{-5} \\%$ \\\\
\\hline
RP & $8.90 \\times 10^{-5} \\%$ \\\\
\\hline
RM & $3.20 \\times 10^{-5} \\%$ \\\\
\\hline
RZ & $3.20 \\times 10^{-5} \\%$ \\\\
\\hline
\\end{tabular}
\\end{center}

The differences among the results become small

are performed in the single precision. The errors in Table 1.1 are the difference between the result by one of the rounding methods and the averaged value among the results by all the rounding methods.

Table 1.2 shows that the differences among the results in the double precision. For this example shock wave problem, the results suggest that the double precision would be recommended to obtain the reasonable results.

These examples would suggest us one method to avoid the numerical rounding errors. For some scientific computations, if computers perform the computations with an automatic selection of the numerical precision or the sufficient bit number for each computation, it would be very helpful for users [49].

On the other hand, another interesting method, that is, the interval arithmetic, has been proposed to avoid the rounding errors $[33,34,50]$. In the floating-point
computations, one true value lies between two floating-point numbers. The interval, in which the true value stays, is computed in the interval arithmetic.

A simple example of the interval arithmetic is shown below. The MAPLE software is again used with a software of intpakX [51]. For the interval arithmetic in MAPLE, the arithmetic operators are modified by attaching $\\&$ to the operators: for example, $" a-b " \\rightarrow$ " $a \\&-b "$.

\\begin{center}
\\includegraphics[max width=\\textwidth]{2024_02_26_83e36543483eb7d284c1g-038(2)}
\\end{center}

Now we solve zero points of $\\sin (x)$ between $x=[-0.1,4.0]$. The solutions are $x=0$ and $x=\\pi$.

\\begin{center}
\\includegraphics[max width=\\textwidth]{2024_02_26_83e36543483eb7d284c1g-038}
\\end{center}

\\begin{center}
\\includegraphics[max width=\\textwidth]{2024_02_26_83e36543483eb7d284c1g-038(1)}
\\end{center}

Fig. 1.15 The Debye shielding is simulated by the PIC code of EPOCH [2, 3]. An extra charge of $+q_{T}$ is initially located at the origin of $r=0$. At the steady state, the electron distribution is shown by a solid line

\\begin{center}
\\includegraphics[max width=\\textwidth]{2024_02_26_83e36543483eb7d284c1g-039}
\\end{center}

If the interval range is confined in a reasonable one, it would be useful to measure the rounding errors $[34,52]$. It would be expected that in the near future computer software or compilers work to find the true values by the interval arithmetic.

In plasma simulations, reliable computer software are also available [2, 28, 53]. However, computer software are not always perfect [24, 35]. The mechanical production of computer software and the computer-assisted program generation have been also intensively studied [27, 36-41, 54-56]. In order to find software errors and malfunctions, interesting studies have been also performed [25-27, 57, 58].

In science including plasma physics, computers and computer software have been powerful tools to advance science forward. As introduced in this subsection, computers are not always perfect. Knowing the numerical errors and the computer uncertainties, let us use computers as a power tool for science, society and life.

\\subsubsection*{Example Computer Simulation for the Debye Shielding}
In the Sect. 1.7.2, one of the collective motions of plasma, that is, the Debye shielding in Fig. 1.8 is computed by using the EPOCH code [2, 3], which is one of well-known PIC computer codes $[2,3,9-11]$.

In Fig. 1.15, an extra positive charge is located and fixed at the origin $r=0$ in a plasma. Initially the electrons and ions of the background plasma are uniform in space. The ions are immobile in this example case to see the electron Debye shielding. The electric field is solved by the Maxwell equations, and the equation motion for the electrons is also solved in EPOCH. After starting the EPOCH simulation, the electrons are attracted around the extra charge of $+q_{T}$, and the electron density distribution becomes steady as shown in Fig. 1.15, in which the spatial length of the radius $r$ is normalized by the electron Debye length $\\lambda_{D e}$ and the electron number density $n_{e}$ is normalized by the initial number density of $n_{e 0}$.

Figure 1.15 would show a reasonable Debye shielding, that is, one of the plasma collective behaviors. It seems that the PIC method would express plasma physics correctly. Numerical methods, including PIC and fluid simulation methods, have
been used widely to study plasma physics together with experiments and theories. In the following chapters, useful numerical methods for plasma physics are introduced with plasma physics.

\\section*{References}
\\begin{enumerate}
  \\item National Academies of Sciences, Engineering, and Medicine 2021: Plasma Science: Enabling Technology, Sustainability, Security, and Exploration (The National Academies Press, Washington, DC). \\href{https://doi.org/10.17226/25802}{https://doi.org/10.17226/25802}

  \\item T.D. Arber, K. Bennett et al., Contemporary particle-in-cell approach to laser-plasma modelling. Arber. Plasma Phys. Control. Fusion 57, 113001 (2015)

  \\item K. Bennett, Users Manual for the EPOCH PIC codes, EPOCH Version 4.3 (2014)

  \\item S. Ichimaru, Statistical Plasma Physics, vol. 1 (CRC Press, Boca Raton, Basic Principles, 2004)

  \\item S. Ichimaru, Statistical Plasma Physics, vol. 2 (CRC Press, Boca Raton, Condensed Plasmas, 2004)

  \\item D.R. Nicholson, Introduction to Plasma Theory (John Wiley and Sons, New York, 1983)

  \\item F.F. Chen, Introduction to Plasma Physics and Controlled Fusion, 3rd ed. (Springer, 2015)

  \\item L.D. Landau, E.M. Lifshitz, Physical Kinetics (Pergamon Press Ltd., Oxford, 1981)

  \\item A.B. Langdon, B.F. Lasinski, Electromagnetic and relativistic plasma simulation models. Methods Comput. Phys. Adv. Res. Appl. 16, 327-366 (1976)

  \\item R.W. Hockney, J.W. Eastwood, Computer Simulation using Particles (Taylor \\& Francis, New York, 1988)

  \\item C.K. Birdsall, A.B. Langdon, Plasma Physics via Computer Simulation (Taylor \\& Francis, New York, 2005)

  \\item OSIRIS PIC Code: OSIRIS consortium. \\href{https://picksc.physics.ucla.edu/osiris.html}{https://picksc.physics.ucla.edu/osiris.html}

  \\item W.-M. Wang, P. Gibbon, Z.-M. Sheng, Y.-T. Li, Integrated simulation approach for laser-driven fast ignition. Phys. Rev. E 91, 013101 (2015)

  \\item S. Huai-Hang, W.-M. Wang, Y.-T. Li, Dense polarized positrons from laser-irradiated foil targets in the QED regime. Phys. Rev. Lett. 129, 035001 (2022)

  \\item A. Friedman, R.H. Cohen, D.P. Grote, S.M. Lund, W.M. Sharp, J.-L. Vay, I. Haber, R.A. Kishek, Computational methods in the Warp code framework for kinetic simulations of particle beams and plasmas. IEEE Trans. Plasma Sci., 42 (2014) pp. 1321-1334. \\href{https://doi.org/10.1109/TPS}{https://doi.org/10.1109/TPS}. 2014.2308546

  \\item Warp Wiki Home. \\href{https://warp.lbl.gov/home}{https://warp.lbl.gov/home}. Cited July 21, 2022

  \\item T. Tajima, Computational plasma physics with applications to fusion and astrophysics. Front. Phys. 72 (1989) . (Addison-Wesley Pub.)

  \\item S. Chapman, T.G. Cowling, The Mathematical Theory of Non-uniform Gases (Cambridge University Press, Cambridge, 1953)

  \\item J.D. Jackson, Classical Electrodynamics (John Wiley \\& Sons Inc, Hoboken, 1999)

  \\item H.H. Song, W.M. Wang, Y.T. Li, YUNIC: a multi-dimensional particle-in-cell code for laserplasma interaction. arXiv:2104.00642 (2021)

  \\item Y. Sentoku, A.J. Kemp, Numerical methods for particle simulations at extreme densities and temperatures: weighted particles, relativistic collisions and reduced currents. J. Comput. Phys. 227, 6846-6861 (2008)

  \\item P.J. Roache, Fundamentals of Computational Fluid Dynamics (Hermosa Pub, New Mexico, 2003)

  \\item R.D. Richtmeyer, K.W. Morton, Difference Methods for Initial-Value Problems (Interscience Pub, New York, 1967)

  \\item W. Kahan, Desparately needed remedies for the undebuggability of large floating-point computations in science and engineering, in IFIP Working Conference on Uncertainty Quantification in Scientific Computing. \\href{http://math.nist.gov/IFIP-UQSC-2011/slides/Kahan.pdf}{http://math.nist.gov/IFIP-UQSC-2011/slides/Kahan.pdf} (2011)

  \\item B. Einarsson, (eds.), Accuracy and Reliability in Scientific Computing, Siam (Society for Industrial and Applied Mathematics) (2005)

  \\item A. Dienstfrey, R. Boisvert (eds.), Uncertainty quantification in scientific computingframeworks, in Middleware and Environments, IFIP Advances in Information and Communication Technology, vol. 377 (Springer, New York, 2012)

  \\item E.N. Houstis, J.R. Rice, E. Gallopoulos, R. Bramley (eds.), Enabling Technologies for Computational Science, The Springer International Series in Engineering and Computer Science (Springer, New York, 2000). \\href{https://doi.org/10.1007/978-1-4615-4541-5}{https://doi.org/10.1007/978-1-4615-4541-5}

  \\item Maple, Computer algebra and visualization software, developed by Waterloo Maple Inc. https:// \\href{http://www.maplesoft.com/}{www.maplesoft.com/}

  \\item London firms reportedly offer amnesty to 'hacker thieves'. \\href{http://catless.ncl.ac.uk/Risks/8.85}{http://catless.ncl.ac.uk/Risks/8.85}. html\\# subj3. Cited 2 November, 2021

  \\item PATRIOT MISSILE DEFENSE, Software Problem Led to SystemFailure at Dhahran, SaudiArabia, GAO-IMTEC-92-96 (1992). \\href{https://www-users.cse.umn.edu/}{https://www-users.cse.umn.edu/} arnold/disasters/GAOIMTEC-92-96.pdf. Cited 2 November, 2021

  \\item M. Collins, F. Vecchio, R. Selby, P. Gupta, Failure of an offshore platform. Canad. Consult. Eng. Struct. 43-48 (2000)

  \\item W.M. Kahan, Significant discrepancies [between the computed and the true result] are very rare, too rare to worry about all the time, yet not rare enough to ignore; "The Regrettable Failure of Automated Error Analysis", A Mini Course prepared for the conference at MIT on Computers and Mathematics (1989)

  \\item \\href{http://www.cs.utep.edu/interval-comp/index.html}{http://www.cs.utep.edu/interval-comp/index.html}

  \\item U. Kulisch, Computer arithmetic and validity-theory, Implementation, and Applications, 33 in the series De Gruyter Studies in Mathematics, de Gruyter (2008)

  \\item Z. Drmac, Z. Bujanovic, On the failure of rank-revealing QR factorization software-a case study. ACM Trans. on Math. Softw. 35, 1-28 (2008)

  \\item Y. Umetani, C. Konno, T. Ohta, Visual PDEQSOL: a visual and interactive environment for numerical simulation, in Proceeding IFIP Conference on Problem Solving Environments for High Level-Scientific Problem Solving, IFIP Trans, vol. A-2, 259-269 (1991)

  \\item N. Sagawa, D. Finn, N.J. Hurley, An integrated problem solving environment for numerical simulation of engineering problems, in Proceeding IFIP Conference on Problem Solving Environments for High Level-Scientific Problem Solving, IFIP Trans, vol. A-2 (1991), pp. 191-201

  \\item S. Kawata, Review of PSE (Problem solving environment) study. J. Convergence Inf. Tech. 5, 204-215 (2010)

  \\item S. Kawata, K. Iijima, C. Boonmmee, Y. Manabe, Computer-Assisted ScientificComputation/Simulation-Software-Development System-Including a Visualization SystemIFIP Trans, vol. A-48 (1994), pp. 145-153

  \\item C. Boonmee, S. Kawata, Computer-assisted simulation environment for partial-differentialequation problem: 1. Data structure and steering of problem solving process, in Transactions of the Japan Society for Computational Engineering and Science, Paper 19980001 (1998)

  \\item C. Boonmee, S. Kawata, Computer-assisted simulation environment for partial-differentialequation problem: 2. Visualization and steering of problem solving process, in Transactions of the Japan Society for Computational Engineering and Science, Paper 19980002 (1998)

  \\item \\href{http://grouper.ieee.org/groups/754/index.html}{http://grouper.ieee.org/groups/754/index.html} \\href{http://babbage.cs.qc.edu/courses/cs341/IEEE754references.html}{http://babbage.cs.qc.edu/courses/cs341/IEEE754references.html}

  \\item W. Kahan, Lecture Notes on the Status of IEEE Standard 754 for Binary Floating-Point Arithmetic. \\href{http://www.cs.berkeley.edu/}{http://www.cs.berkeley.edu/} wkahan/ieee754status/ieee754.pdf (1996)

  \\item J.H. Wilkinson, Error analysis of direct methods of matrix inversion. J. ACM 8, 281-330 (1961)

  \\item J.H. Wilkinson, Rounding Errors in Algebraic Processes (Dover Publications, Reprint, 1994)

  \\item N.J. Higham, Accuracy and stability of numerical algorithms. Siam (Society for Industrial and Applied Mathematics) (1996)

  \\item F. Chaitin-Chatelin, V. Fraysse, Lectures on finite precision computations, Siam (1996)

  \\item LAPACK, \\href{http://www.netlib.org/lapack/index.html}{http://www.netlib.org/lapack/index.html}

  \\item S. Kawata, T. Ishihara, D. Barada, W. Zhang, J. Xie, H. Usami, T. Teramoto, M. Matsumoto, Y. Manabe, Y. Hayase, T.W. Kim, Uncertainty management in computer assisted problem solving environment (PSE) for scientific computing. J. Convergence Inf. Tech. 8, 449-456 (2013)

  \\item F.W.J. Olver, D.W. Lozier, R.F. Boisvert, C.W. Clark (ed.), NIST Handbook of Mathematical Functions (Cambridge University Press, Cambridge, 2010). \\href{https://dlmf.nist.gov/}{https://dlmf.nist.gov/}

  \\item \\href{https://www2.math.uni-wuppertal.de/org/WRST/xsc-frame/index.html}{https://www2.math.uni-wuppertal.de/org/WRST/xsc-frame/index.html}

  \\item S.M. Rump, Computer-assisted proofs and self-validating methods, in Accuracy and Reliability, in by B. ed. by S. Computing (Einarsson, Siam, 2005), pp.195-240

  \\item MATLAB: Registered trademark of The MathWorks, Inc. \\href{https://www.mathworks.com/}{https://www.mathworks.com/} products/matlab.html

  \\item E. Gallopoulos, E. Houstis, J.R. Rice, Computer as Thinker/doer: problem-solving environments for computational science. IEEE Comput. Sci. Eng. 1, 11-23 (1994)

  \\item S. Kawata, Y. Umetani, A review of computer-assisted problem-solving environment (PSE) in computational engineering and science. Tran. JSCES (Jpn. Soc. Comput. Eng. and Sci.), 20171001, 1-12 (2017)

  \\item IFIP WG2.5 (International Federation for Information Processing, Working Group 2.5), IFIP WG2.5 homepage retrieved on June 12, 2017. \\href{http://www.ifip.org/wg-2.5}{http://www.ifip.org/wg-2.5}

  \\item W.R. Weimer, Exceptional Situations and Program Reliability, PhD Thesis (Univ. California, Berkeley 2005)

  \\item C. Le Goues, T.V. Nguyen, S. Stephanie Forrest, W. Weimer, GenProg: a generic method for automatic software repair. IEEE Trans. Softw. Eng. 38, 54-72 (2012)

\\end{enumerate}

\\section*{Chapter 2 Plasma in Equilibrium }
\\begin{abstract}
Before going into the details of plasma physics, Chap. 2 introduces basic physical quantities associated to plasma, which is in thermal equilibrium. The basic physical quantities include distribution function, number density, temperature, the Coulomb collision, etc.
\\end{abstract}

\\subsection*{Distribution Function and Plasma}
As shown in Sect. 1.6, we have two mathematical models to describe plasma: the microscopic method by the distribution function $f(t, \\mathbf{x}, \\mathbf{v})$ and the macroscopic fluid model. In the macroscopic fluid model, macroscopic quantities, including number density, pressure, temperature, averaged velocity, etc., are treated.

In plasma, many particles are there and are described by the distribution function $f(t, \\mathbf{x}, \\mathbf{v})$, which shows a particle number located at the time $t$, the position of $\\mathbf{x}$ and the velocity of $\\mathbf{v}$. The macroscopic quantities of number density, pressure, temperature, averaged velocity, etc. would be obtained by averaging operations over many particles located at $t$ and $\\mathbf{x}$. The values of macroscopic quantities may be perturbed in time and space. However, when plasma contains sufficient number of particles, the macroscopic physical quantities would be obtained.

Equation (1.19) shows the number density $n(t, \\mathbf{x})$ at $t$ and $\\mathbf{x}$ by the integration of the distribution function of $f(t, \\mathbf{x}, \\mathbf{v})$ over the velocity space. The macroscopic quantities are obtained by the averaging procedure of the microscopic quantity of $f(t, \\mathbf{x}, \\mathbf{v})$. The macroscopic velocity is also obtained by Eq. (1.20).

When plasma is in thermal equilibrium, the distribution function $f(t, \\mathbf{x}, \\mathbf{v})$ becomes the Maxwell distribution (see Eq. (2.1)), which is also the normal distribution (the Gaussian distribution):


\\begin{equation*}
f\\left(v_{x}, v_{y}, v_{z}\\right) \\equiv f(\\mathbf{v})=n\\left(\\frac{m}{2 \\pi T}\\right)^{3 / 2} \\exp \\left\\{-\\frac{m\\left(v_{x}^{2}+v_{y}^{2}+v_{z}^{2}\\right)}{2 T}\\right\\} \\tag{2.1}
\\end{equation*}


Fig. 2.1 Distribution function $f\\left(v_{x}\\right)$. When plasma is in thermal equilibrium, the distribution becomes the Maxwell distribution. The width at $f\\left(v_{x}\\right)=f(0) / e$ introduces the plasma temperature $T$

\\begin{center}
\\includegraphics[max width=\\textwidth]{2024_02_26_83e36543483eb7d284c1g-044}
\\end{center}

Here $m$ shows the mass of plasma particle, $n$ the number density and $T$ the temperature in the unit of energy.

The Maxwell distribution is schematically shown in Fig. 2.1 along $v_{x}$.

The width of the distribution function defines the temperature of $T$. As shown in Fig. 2.1, the value of $v_{x}$ at $f\\left(v_{x}\\right)=f(0) / e$ introduces $\\sqrt{2 T / m}$, from which one can obtain the temperature $T$.

When distribution functions are not described by the Maxwell (or normal) distribution function, the temperature would not be defined. These plasmas are in non-equilibrium. If the non-equilibrium distribution function is not too far from the Maxwell distribution, it may be approximately treated by the Maxwell distribution with perturbations of physical quantities. Otherwise, we may rely on the numerical methods [1-5] or experiments.

The equilibrium distribution in Fig. 2.1 is symmetric with respect to the origin of $v_{x}=0$. Therefore, the macroscopic averaged velocity $\\langle\\mathbf{v}\\rangle$ obtained by Eq. (1.20) must be 0 .

Here we obtain the averaged kinetic energy $\\left\\langle m \\mathbf{v}^{2} / 2\\right\\rangle^{1}$


\\begin{equation*}
\\left\\langle\\frac{m \\mathbf{v}^{2}}{2}\\right\\rangle=\\frac{1}{n} \\iiint_{-\\infty}^{\\infty} \\frac{m \\mathbf{v}^{2}}{2} f(t, \\mathbf{x}, \\mathbf{v}) \\mathrm{d} v_{x} \\mathrm{~d} v_{y} \\mathrm{~d} v_{z} \\tag{2.2}
\\end{equation*}


The averaged kinetic energy becomes $\\left\\langle m \\mathbf{v}^{2} / 2\\right\\rangle=3 T / 2$ for the Maxwell distribution, and it is considered again in the next Sect.2.2.
\\footnotetext{${ }^{1}$ Equations (1.19), (1.20) and (2.2) would be called the zeroth, first and second moments of the distribution function $f(t, \\mathbf{x}, \\mathbf{v})$ with respect to the velocity $\\mathbf{v}$, respectively.
}

\\subsection*{The Maxwell Distribution}
The distribution function of $f(t, \\mathbf{x}, \\mathbf{v})$ shows the plasma state at $t, \\mathbf{x}$ and $\\mathbf{v}$. In quantum dynamics [6], the distribution function $f(E)$ is related to the particle energy $E$ as follows:


\\begin{equation*}
f(E)=\\frac{1}{A \\exp \\left(\\frac{E}{T}\\right) \\pm 1} \\tag{2.3}
\\end{equation*}


(see, for example, Chap. 9 in Ref. [6] and Chap. 5 in Ref. [7]) for quantum mechanics for details. ${ }^{2}$ ) In Eq. (2.3), the sign of + in the denominator is for the Fermi particles and shows the Fermi-Dirac distribution function. The - sign is for the Bose particles, including $H_{e}^{4}$ and photon, and shows the Bose-Einstein distribution function.

In many cases the following condition would be satisfied for plasmas:


\\begin{equation*}
A \\exp \\left(\\frac{E}{T}\\right) \\gg 1 \\tag{2.4}
\\end{equation*}


Then the following distribution function is obtained approximately:


\\begin{equation*}
f(E)=\\frac{1}{A} \\exp \\left(-\\frac{E}{T}\\right) \\tag{2.5}
\\end{equation*}


Hereafter the classical plasma is focused in this book. For freely moving particles, the particle energy $E$ is the kinetic energy:


\\begin{equation*}
E=\\frac{m}{2}\\left(v_{x}^{2}+v_{y}^{2}+v_{z}^{2}\\right) \\tag{2.6}
\\end{equation*}


When plasma particles are in an electromagnetic field, at the right side of Eq. (2.6), the corresponding potential is added.

The coefficient $1 / A$ in Eq. (2.5) is now calculated by using Eq. (1.19). After integrating the distribution function $f$ over the velocity space, the number density $n$ is obtained.


\\begin{equation*}
n=\\iiint_{-\\infty}^{\\infty} \\frac{1}{A} \\exp \\left\\{-\\frac{m\\left(v_{x}^{2}+v_{y}^{2}+v_{z}^{2}\\right)}{2 T}\\right\\} \\mathrm{d} v_{x} \\mathrm{~d} v_{y} \\mathrm{~d} v_{z} \\tag{2.7}
\\end{equation*}


In Eq. (2.7), each integration on $v_{x}, v_{y}$ and $v_{z}$ is independent with each other. Appendix B. 5 shows the following relation:
\\footnotetext{${ }^{2}$ In quantum dynamics, electron, ion and light have the duality of wave and particle. Energy is quantized. For the Fermi particles, including electron, one energy state is occupied by only one particle. However, any number of the Bose particles, including photon, occupy one state.
}


\\begin{equation*}
\\int_{-\\infty}^{\\infty} \\exp \\left(-\\frac{m V^{2}}{2 T}\\right) \\mathrm{d} V=\\sqrt{\\frac{2 \\pi T}{m}} \\tag{2.8}
\\end{equation*}


Equation (2.7) gives the following result:


\\begin{equation*}
\\frac{1}{A}\\left(\\frac{2 \\pi T}{m}\\right)^{3 / 2}=n \\tag{2.9}
\\end{equation*}


Then we obtain $A$, and Eq. (2.5) becomes Eq. (2.1), which is shown below again:


\\begin{equation*}
f\\left(v_{x}, v_{y}, v_{z}\\right) \\equiv f(\\mathbf{v})=n\\left(\\frac{m}{2 \\pi T}\\right)^{3 / 2} \\exp \\left\\{-\\frac{m\\left(v_{x}^{2}+v_{y}^{2}+v_{z}^{2}\\right)}{2 T}\\right\\} \\tag{2.1}
\\end{equation*}


The distribution function is called the Maxwell distribution for thermal-equilibrium plasma, which is uniform in space $\\mathbf{x}$ and temporally steady.

\\section*{Tips 2.1}
Here Eq. (2.8) is derived. Define $I=\\int_{-\\infty}^{\\infty} \\exp \\left(-a x^{2}\\right) \\mathrm{d} x$. Then, $I^{2}=$ $\\int_{-\\infty}^{\\infty} \\exp \\left(-a x^{2}\\right) \\mathrm{d} x \\times \\int_{-\\infty}^{\\infty} \\exp \\left(-a y^{2}\\right) \\mathrm{d} y=\\iint_{-\\infty}^{\\infty} \\exp \\left(-a\\left(x^{2}+y^{2}\\right)\\right) \\mathrm{d} x \\mathrm{~d} y$. Here $x^{2}+y^{2}=r^{2} \\quad$ and $\\quad \\mathrm{d} x \\mathrm{~d} y=2 \\pi r \\mathrm{~d} r . \\quad I^{2}=2 \\pi \\int_{0}^{\\infty} \\exp \\left(-a r^{2}\\right) r \\mathrm{~d} r=$ $2 \\pi\\left[-\\exp \\left(-a r^{2}\\right) / 2 a\\right]_{0}^{\\infty}=\\pi / a$. Now $I=\\sqrt{\\pi / a}$. In Eq. (2.8), $a=\\sqrt{m / 2 T}$. So we find now Eq. (2.8).

Here the averaged kinetic energy is obtained by Eq. (2.2) and the Maxwell distribution:


\\begin{equation*}
\\left\\langle\\frac{1}{2} m v^{2}\\right\\rangle=\\frac{1}{n} \\iiint_{-\\infty}^{\\infty} \\frac{1}{2} m\\left(v_{x}^{2}+v_{y}^{2}+v_{z}^{2}\\right) f\\left(v_{x}, v_{y}, v_{z}\\right) \\mathrm{d} v_{x} \\mathrm{~d} v_{y} \\mathrm{~d} v_{z} \\tag{2.2}
\\end{equation*}


At present, the velocity is isotropic in the velocity space. So, by setting $v^{2} \\equiv$ $v_{x}^{2}+v_{y}^{2}+v_{z}^{2}, \\mathrm{~d} v_{x} \\mathrm{~d} v_{y} \\mathrm{~d} v_{z}=4 \\pi v^{2} \\mathrm{~d} v$ :


\\begin{equation*}
\\left\\langle\\frac{1}{2} m \\mathbf{v}^{2}\\right\\rangle=\\frac{1}{n} \\int_{0}^{\\infty}\\left(\\frac{1}{2} m v^{2}\\right) f(v)\\left(4 \\pi v^{2}\\right) \\mathrm{d} v \\tag{2.10}
\\end{equation*}


After performing the integration, we obtained the following Eq. (2.11) (see Tips 2.1 and Appendix B.5):


\\begin{equation*}
\\left\\langle\\frac{m \\mathbf{v}^{2}}{2}\\right\\rangle=\\frac{3 T}{2} \\tag{2.11}
\\end{equation*}


\\section*{Tips 2.2}
Let us calculate Eq. (2.10). From the result in Tips 2.1, $I=$ $\\int_{-\\infty}^{\\infty} \\exp \\left(-a x^{2}\\right) \\mathrm{d} x=\\sqrt{\\pi / a}=\\sqrt{\\pi} a^{-1 / 2}$. Differentiate both the sides with $a$, and then we obtain $\\mathrm{d} I / \\mathrm{d} a=\\int_{-\\infty}^{\\infty}-x^{2} \\exp \\left(-a x^{2}\\right) \\mathrm{d} x=-(\\sqrt{\\pi} / 2) a^{-3 / 2}$. Differentiate once again both the sides with $a$, and $\\mathrm{d}^{2} I / \\mathrm{d} a^{2}=$ $\\int_{-\\infty}^{\\infty} x^{4} \\exp \\left(-a x^{2}\\right) \\mathrm{d} x=(3 \\sqrt{\\pi} / 4) a^{-5 / 2}$. Please note that the lower bound of the integration range in Eq. (2.10) is 0 and that the integrand $x^{4} \\exp \\left(-a x^{2}\\right)$ is an even function. Then, $\\int_{0}^{\\infty} x^{4} \\exp \\left(-a x^{2}\\right) \\mathrm{d} x=(3 \\sqrt{\\pi} / 8) a^{-5 / 2}$, and Eq. (2.11) is obtained: $\\left\\langle m \\mathbf{v}^{2} / 2\\right\\rangle=3 T / 2$.

Here we look back again the averaged velocity in Eq. (1.20).


\\begin{equation*}
\\left\\langle v_{i}\\right\\rangle=\\frac{1}{n} \\iiint_{-\\infty}^{\\infty} v_{i} f(\\mathbf{v}) \\mathrm{d} v_{x} \\mathrm{~d} v_{y} \\mathrm{~d} v_{z}, \\quad(i=x, y, z) \\tag{1.20}
\\end{equation*}


When the distribution function $f(\\mathbf{v})$ is the Maxwell one in Fig. 2.1, $f(\\mathbf{v})$ is even in $v_{i}$. Here $i=x, y$ and $z$. So the integrand of $v_{i} f(\\mathbf{v})$ is odd, and $\\int v_{i} f(\\mathbf{v}) \\mathrm{d} v_{i} / n=$ $\\left\\langle v_{i}\\right\\rangle=0$.

In Fig. 2.2, the Maxwell distribution is shown by a dashed line. Additionally, the drift Maxwell distribution by $V_{d}$ is presented by a sold line. The drift distribution function shows that the plasma moves with the averaged velocity of $V_{d}$ in the $x$ direction, in Fig. 2.2:


\\begin{equation*}
f\\left(v_{x}, v_{y}, v_{z}\\right)=n\\left(\\frac{m}{2 \\pi T}\\right)^{3 / 2} \\exp \\left\\{-\\frac{\\left(v_{x}-V_{d}\\right)^{2}+v_{y}^{2}+v_{z}^{2}}{2 T / m}\\right\\} \\tag{2.12}
\\end{equation*}


By Eq. (2.12), $\\left\\langle v_{x}\\right\\rangle$ is obtained:


\\begin{equation*}
\\left\\langle v_{x}\\right\\rangle=V_{d} \\tag{2.13}
\\end{equation*}


For example, a plasma jet may be represented by the drift distribution function.

As another special example, it may be instructive to see an equilibrium plasma with different temperatures in each spatial direction of $x, y$ and $z$. For example, strong magnetic field may restrict particle motion in space, and in this special case we may have different temperatures in each direction:

Fig. 2.2 Maxwell distribution (dashed line by Eq. (2.1)) and its drifted distribution (solid line). The drifting Maxwell distribution shows the Maxwell distribution with a drift velocity of $V_{d}$ by Eq. (2.12)

\\begin{center}
\\includegraphics[max width=\\textwidth]{2024_02_26_83e36543483eb7d284c1g-048(1)}
\\end{center}

Fig. 2.3 Contour map for distribution function with different temperatures in $x$ and $y$. In this example, $T_{x}>T_{y}$

\\begin{center}
\\includegraphics[max width=\\textwidth]{2024_02_26_83e36543483eb7d284c1g-048}
\\end{center}


\\begin{align*}
f\\left(v_{x}, v_{y}, v_{z}\\right)= & n\\left(\\frac{m}{2 \\pi T_{x}}\\right)^{1 / 2}\\left(\\frac{m}{2 \\pi T_{y}}\\right)^{1 / 2}\\left(\\frac{m}{2 \\pi T_{z}}\\right)^{1 / 2}  \\tag{2.14}\\\\
& \\times \\exp \\left(-\\frac{m v_{x}^{2}}{2 T_{x}}\\right) \\exp \\left(-\\frac{m v_{y}^{2}}{2 T_{y}}\\right) \\exp \\left(-\\frac{m v_{z}^{2}}{2 T_{z}}\\right)
\\end{align*}


Figure 2.3 shows a 2-dimensional (2D) contour map for an example distribution function, in which $T_{x}>T_{y}$.

\\subsection*{Plasma Density}
In the previous Sect. 2.2, the Maxwell distribution function was introduced. The number density $n$ was derived by Eqs. (1.19) or (2.7). Here we reconsider the number density $n$, together with charge density $\\rho_{e}$ and current density $\\mathbf{J}$.

The number density $n$ is related with the distribution function $f$ as follows:


\\begin{equation*}
n=\\iiint f\\left(v_{x}, v_{y}, v_{z}\\right) \\mathrm{d} v_{x} \\mathrm{~d} v_{y} \\mathrm{~d} v_{z} \\tag{2.15}
\\end{equation*}


Here let us remind that the distribution function $f$ is not only a function of velocity $\\mathbf{v}=\\left(v_{x}, v_{y}, v_{z}\\right)$ but also a function of time $t$ and space $\\mathbf{r} \\equiv(x, y, z)$ generally. In Eq. (2.5) $(f(E)=(1 / A) \\exp (-E / T))$, the energy $E$ was as follows:


\\begin{equation*}
E=\\frac{m}{2}\\left(v_{x}^{2}+v_{y}^{2}+v_{z}^{2}\\right) \\tag{2.6}
\\end{equation*}


In general, the energy $E$ includes the potential energy $E_{\\phi}$ as well as the kinetic energy Eq. (2.6):


\\begin{equation*}
E=\\frac{m \\mathbf{v}^{2}}{2}+E_{\\phi} \\tag{2.16}
\\end{equation*}


Therefore, the Maxwell distribution (2.1) is modified as follows:


\\begin{equation*}
f(\\mathbf{r}, \\mathbf{v})=n_{0}\\left(\\frac{m}{2 \\pi T}\\right)^{3 / 2} \\exp \\left(-\\frac{m \\mathbf{v}^{2} / 2+E_{\\phi}}{T}\\right) \\tag{2.17}
\\end{equation*}


The distribution function $f(\\mathbf{r}, \\mathbf{v})$ in Eq. (2.17) is substituted into Eq. (2.15) to obtain the number density $n$.


\\begin{equation*}
n=n_{0} \\exp \\left(-\\frac{E_{\\phi}}{T}\\right) \\tag{2.18}
\\end{equation*}


For example, the potential energy $E_{\\phi}$ is the Coulomb potential one and would be written as follows:


\\begin{equation*}
E_{\\phi}=q \\varphi \\tag{2.19}
\\end{equation*}


When the charged particles are electrons, $q=-e$ and Eq. (2.18) becomes Eq. (1.1). Equation (1.1) was introduced without explanations. However, now we understand the meaning of Eq. (1.1). Equation (1.1) represents that electrons are attracted to the higher potential area, as shown in Figs. 1.8 and 1.15.

The charge density $\\rho_{e}$ is also obtained straightforward. When plasma consists of several kinds of ions $i$ and electrons $e, f(\\mathbf{r}, \\mathbf{v})$ shows collections of ions and electrons. The charge density $\\rho_{e}$ is obtained:

Fig. 2.4 Coulomb scattering. A heavy immobile ion is fixed at the origin $O$. An ion with the mass $m$, the charge $Z^{\\prime}$, the velocity $v$ and the impact parameter of $b$ is injected from the left and is scattered to the direction of $\\theta$

\\begin{center}
\\includegraphics[max width=\\textwidth]{2024_02_26_83e36543483eb7d284c1g-050}
\\end{center}


\\begin{equation*}
\\rho_{e}=\\sum_{i} q_{i} n_{i}-e n_{e} \\tag{2.20}
\\end{equation*}


Here the summation $\\sum$ is taken over all the ion species. The charge of each ion species is described by $q_{i}$, and its number density is $n_{i}$. The electron number density is $n_{e}$.

The current density $\\mathbf{J}$ is derived by using $n\\langle\\mathbf{v}\\rangle$ in Eq. (1.20).


\\begin{equation*}
\\mathbf{J}=\\sum_{i} q_{i} n_{i}\\left\\langle\\mathbf{v}_{i}\\right\\rangle-e n_{e}\\left\\langle\\mathbf{v}_{e}\\right\\rangle \\tag{2.21}
\\end{equation*}


\\subsection*{The Coulomb Collision}
The Coulomb collision is considered in Sect. 2.4. The Coulomb collision is an individual interaction between 2 charged particles.

Figure 2.4 shows the Coulomb collision schematically. At the origin $O$ a heavy ion with its charge $z$ is fixed and does not move. From the left, another ion with the mass $m$ and the charge $z^{\\prime}$ comes in with the speed of $v$. The incoming ion is scattered by the heavy ion at the origin. Its direction changes by $\\theta$. The scattering angle $\\theta$ is below:


\\begin{equation*}
\\cot \\left(\\frac{\\theta}{2}\\right)=\\frac{4 \\pi \\varepsilon_{0} m b v^{2}}{z z^{\\prime} e^{2}} \\tag{2.22}
\\end{equation*}


(Its detail derivation can be found in, for example, Chap. 1 in Ref. [8] and Chap. 13 in [9] or other books on electromagnetics.) Here $b$ shows the impact parameter (see Fig. 2.4). In Fig. 2.4 the phenomenon is cylindrically symmetric. Ions, coming in through the small area of $2 \\pi b \\mathrm{~d} b$, are scattered to the direction $\\theta$ with a small area $\\mathrm{d} \\theta$. The solid angle $\\mathrm{d} \\Omega$ is as follows:


\\begin{equation*}
\\mathrm{d} \\Omega=2 \\pi \\sin \\theta \\mathrm{d} \\theta \\tag{2.23}
\\end{equation*}


From Eq. (2.22) the following is obtained:


\\begin{gather*}
2 \\pi b \\mathrm{~d} b=\\frac{\\left(z z^{\\prime} e^{2}\\right)^{2}}{4\\left(4 \\pi \\varepsilon_{0} m v^{2}\\right)^{2}} \\cdot \\frac{1}{\\sin ^{4} \\frac{\\theta}{2}} \\cdot \\mathrm{d} \\Omega=\\sigma(\\theta) \\mathrm{d} \\Omega  \\tag{2.24}\\\\
\\sigma(\\theta)=\\frac{1}{4}\\left(\\frac{z z^{\\prime} e^{2}}{4 \\pi \\varepsilon_{0} m v^{2}}\\right)^{2} \\frac{1}{\\sin ^{4} \\frac{\\theta}{2}} \\tag{2.25}
\\end{gather*}


Here $\\sigma(\\theta)$ is the differential cross section for the Coulomb collision (or the Rutherford differential cross section).

So far, the heavy ion is fixed at the origin $O$ in Fig. 2.4. In reality, it should move. In that case, the origin in Fig. 2.4 should be the center of mass for the two ions. The mass $m$ should be the reduced mass $m_{r}$ :


\\begin{equation*}
m_{r}=\\frac{m_{1} m_{2}}{m_{1}+m_{2}} \\tag{2.26}
\\end{equation*}


The mass $m$ should be replaced to the reduced mass $m_{r}$ in Eq. (2.25). Here $m_{1}$ and $m_{2}$ are the masses for the two ions.

Now let us think about the energy change during the Coulomb collision. The masses of the two particles are $m_{1}$ and $m_{2}$. The velocity is $\\mathbf{v}_{1}$ and $\\mathbf{v}_{2}$, respectively. From the momentum conservation, the following is obtained:


\\begin{equation*}
\\delta\\left(m_{1} \\mathbf{v}_{1}+m_{2} \\mathbf{v}_{2}\\right)=m_{1} \\delta \\mathbf{v}_{1}+m_{2} \\delta \\mathbf{v}_{2}=0 \\tag{2.27}
\\end{equation*}


The change in energy $\\delta E$ is follows:


\\begin{align*}
\\delta E_{1} & =\\delta\\left(\\frac{1}{2} m_{1} \\mathbf{v}_{1}^{2}\\right)=\\frac{1}{2} m_{1}\\left(\\mathbf{v}_{\\mathbf{1}}+\\delta \\mathbf{v}_{\\mathbf{1}}\\right)^{2}-\\frac{1}{2} m_{1} \\mathbf{v}_{\\mathbf{1}}{ }^{2} \\\\
& =m_{1} \\mathbf{v}_{1} \\cdot \\delta \\mathbf{v}_{1}+\\frac{m_{1}}{2}\\left(\\delta \\mathbf{v}_{1}\\right)^{2}=-\\delta E_{2} \\tag{2.28}
\\end{align*}


Here we set the relative velocity $\\mathbf{u}$ and the velocity of the center of mass $\\mathbf{v}$ :


\\begin{equation*}
\\mathbf{u}=\\mathbf{v}_{1}-\\mathbf{v}_{2}, \\quad \\mathbf{v}=\\frac{m_{1} \\mathbf{v}_{1}+m_{2} \\mathbf{v}_{2}}{m_{1}+m_{2}} \\tag{2.29}
\\end{equation*}


Then we obtain $\\delta \\mathbf{v}_{1}=\\left(m_{r} / m_{1}\\right) \\delta \\mathbf{u}$ and $\\delta \\mathbf{v}_{2}=-\\left(m_{r} / m_{2}\\right) \\delta \\mathbf{u}$. Equation (2.28) becomes as follows:


\\begin{equation*}
\\delta E_{1}=m_{1} \\mathbf{v} \\cdot \\delta \\mathbf{v}_{1}+m_{r} \\mathbf{u} \\cdot \\delta \\mathbf{v}_{\\mathbf{1}}+\\frac{m_{1}}{2}\\left(\\delta \\mathbf{v}_{1}\\right)^{2} \\tag{2.30}
\\end{equation*}



\\begin{align*}
\\delta E_{2} & =m_{2} \\mathbf{v} \\cdot \\delta \\mathbf{v}_{2}-m_{r} \\mathbf{u} \\cdot \\delta \\mathbf{v}_{2}+\\frac{m_{2}}{2}\\left(\\delta \\mathbf{v}_{2}\\right)^{2}  \\tag{2.31}\\\\
& =-\\delta E_{1}=-m_{1} \\mathbf{v} \\cdot \\delta \\mathbf{v}_{1}-m_{r} \\mathbf{u} \\cdot \\delta \\mathbf{v}_{\\mathbf{1}}-\\frac{m_{1}}{2}\\left(\\delta \\mathbf{v}_{1}\\right)^{2} \\tag{2.32}
\\end{align*}


Subtracting Eq. (2.32) from Eq. (2.31), the following is obtained:


\\begin{align*}
0 & =m_{r} \\mathbf{u} \\cdot \\delta \\mathbf{u}+\\frac{1}{2} m_{r}(\\delta \\mathbf{u})^{2} \\\\
& =\\frac{m_{1}}{m_{r}}\\left\\{m_{r} \\mathbf{u} \\cdot \\delta \\mathbf{v}_{1}+\\frac{1}{2} m_{1}\\left(\\delta \\mathbf{v}_{1}\\right)^{2}\\right\\} \\tag{2.33}
\\end{align*}


By using Eq. (2.33), the energy change $\\delta E_{1}$ in Eq. (2.30) becomes as follows:


\\begin{equation*}
\\delta E_{1}=m_{1} \\mathbf{v} \\cdot \\delta \\mathbf{v}_{1}=m_{r}(\\mathbf{v} \\cdot \\delta \\mathbf{u})=-\\delta E_{2} \\tag{2.34}
\\end{equation*}


By Eq. (2.34), the energy change for one collision was obtained. Now we find the averaged energy change $\\left\\langle\\delta E_{1}\\right\\rangle$ during a short time interval $\\delta t$ :


\\begin{equation*}
\\left\\langle\\frac{\\mathrm{d} E_{1}}{\\mathrm{~d} t}\\right\\rangle=n_{2} u \\int \\delta E_{1} \\sigma \\mathrm{d} \\Omega \\tag{2.35}
\\end{equation*}


Here $\\sigma$ is the cross section in Eq. (2.25). In order to calculate the energy change of the particle " 1 ", scattered by the particle " 2 ", which would be in the Maxwell distribution. For the purpose, after multiplying the Maxwell distribution $f_{2}$ for the particle "2" to the right side of Eq. (2.35) and performing the integration over $\\mathbf{v}_{2}$, we obtain the averaged energy change:


\\begin{equation*}
\\left\\langle\\frac{\\mathrm{d} E_{1}}{\\mathrm{~d} t}\\right\\rangle=\\iint f_{2} u \\delta E_{1} \\sigma \\mathrm{d} \\Omega \\mathrm{d} v_{2 x} \\mathrm{~d} v_{2 y} \\mathrm{~d} v_{2 z} \\tag{2.36}
\\end{equation*}


When the distribution function $f_{2}$ in thermal equilibrium, Eq. (2.36) becomes as follows:


\\begin{align*}
\\left\\langle\\frac{\\mathrm{d} E_{1}}{\\mathrm{~d} t}\\right\\rangle= & -\\frac{n_{2}\\left(z_{1} z_{2} e^{2}\\right)^{2}}{4 \\pi \\varepsilon_{0}^{2} v_{1} m_{2}} \\ln \\Lambda \\times \\\\
& \\left\\{\\Phi\\left(\\eta_{2} v_{1}\\right)-\\left(1+\\frac{m_{2}}{m_{1}}\\right) \\frac{2 \\eta_{2} v_{1}}{\\sqrt{\\pi}} \\exp \\left(-\\eta_{2}^{2} v_{1}^{2}\\right)\\right\\} \\tag{2.37}
\\end{align*}


Here $\\Phi(x)$ is the error function [10, 11]:


\\begin{equation*}
\\Phi(x)=\\frac{2}{\\sqrt{\\pi}} \\int_{0}^{x} \\exp \\left(-y^{2}\\right) \\mathrm{d} y, \\quad \\eta_{2} \\equiv \\sqrt{\\frac{m_{2}}{2 T_{2}}} \\tag{2.38}
\\end{equation*}


Here $\\ln \\Lambda$ is the Coulomb logarithm:


\\begin{equation*}
\\ln \\Lambda=\\int_{\\theta_{\\min }}^{\\pi} \\cot \\frac{\\theta}{2} \\mathrm{~d} \\theta \\tag{2.39}
\\end{equation*}


The minimum value of the scattering angle $\\theta_{\\min }$ corresponds to the largest impact parameter $b$ in Fig. 2.4. The largest impact parameter $b$ should be the Debye length $\\lambda_{D}$ in plasma. (See also Chap. 1 in Ref. [8] for $\\ln \\Lambda$. The detailed values of the Coulomb logarithm $\\ln \\Lambda$ can be found in Ref. [12].)

\\subsection*{Plasma Temperature}
As we see in Sect. 2.1, the temperature $T$ shows the width of the Maxwell distribution function Eq. (2.1) (see Fig. 2.1). Here let us think about two species of particles in plasma by Eq. (2.37). For example, when electrons have a different temperature $T_{e}$ from the ion temperature $T_{i}$, the two temperatures relax with time $t$. In Eq. (2.37), two species of particles are in thermal equilibrium. For one species, the temperature is $T_{1}$ and its distribution function is $f_{1}$. For the other second species, $T_{2}$ and $f_{2}$. Multiplying $f_{1} / n_{1}$ to Eq. (2.37) and integrating over $\\mathbf{v}_{1}$, then we obtain the following $^{3}$ :


\\begin{gather*}
\\left\\langle\\frac{\\mathrm{d} E_{1}}{\\mathrm{~d} t}\\right\\rangle=-\\frac{\\frac{3}{2}\\left(T_{1}-T_{2}\\right)}{\\tau}  \\tag{2.40}\\\\
\\tau=\\frac{3 \\sqrt{2 \\pi} \\pi \\varepsilon_{0}^{2} m_{1} m_{2}}{n_{2}\\left(z_{1} z_{2} e^{2}\\right)^{2} \\ln \\Lambda}\\left(\\frac{T_{1}}{m_{1}}+\\frac{T_{2}}{m_{2}}\\right)^{3 / 2} \\tag{2.41}
\\end{gather*}


By Eq. (2.41), one can obtain the relaxation time $\\tau^{e e}$ between electrons, $\\tau^{i i}$ between ions and $\\tau^{i e}$ between ions and electrons. By comparing among the relaxation times, the following relation is approximately found:


\\begin{equation*}
\\tau^{e e}: \\tau^{i i}: \\tau^{i e} \\simeq 1: \\frac{1}{z_{i}^{3}} \\sqrt{\\frac{m_{i}}{m_{e}}}: \\frac{1}{z_{i}}\\left(\\frac{m_{i}}{m_{e}}\\right) \\tag{2.42}
\\end{equation*}

\\footnotetext{3 For the detail derivations of Eqs. (2.40) and (2.41), see, for example, Ref. [13].
}

Here $z_{i}$ shows the ion charge and $n_{e}=Z_{i} n_{i}$. The relation (2.42) shows that the relaxation time between the same species is short compared with the relaxation time between the different species.

\\section*{Tips 2.3}
Let us calculate the particle speed $v_{*}$, at which the particle number density becomes $1 \\%$ of the peak number density at $\\mathbf{v}=0$ for oxygen with the temperature of $27^{\\circ} \\mathrm{C}$. The particle mass for one oxygen molecule $\\mathrm{O}_{2}$ is about $5.35 \\times 10^{-26} \\mathrm{~kg}$. The temperature $27^{\\circ} \\mathrm{C}$ is $300 \\mathrm{~K}$. In the Maxwell distribution in Eq. (2.1), the exponential factor is rewritten as $\\exp \\left(-v_{*}^{2} / v_{\\text {th }}^{2}\\right)$ with $v_{\\text {th }}=\\sqrt{2 T / m}$. For $300 \\mathrm{~K}, v_{\\text {th }} \\sim 394 \\mathrm{~m} / \\mathrm{s}$. At the speed $v_{*}, \\exp (-A)=0.01$. Then, $v_{*}^{2} / v_{\\text {th }}^{2}=A \\sim 4.60$, and $v_{*}$ becomes about $884 \\mathrm{~m} / \\mathrm{s}$, which is very fast.

\\subsubsection*{Simulation of Temperature Relaxation}
Here let us simulate a temperature relaxation between two electron groups. Two species of electrons are distributed in space uniformly. One part of electrons has a higher temperature of $10 \\mathrm{eV}$, and the temperature of the other electron group is 1

\\begin{center}
\\includegraphics[max width=\\textwidth]{2024_02_26_83e36543483eb7d284c1g-054}
\\end{center}

Fig. 2.5 Temperature relaxation. One kind of electrons (Electron 1) has a higher temperature of $10 \\mathrm{eV}$, and the other electron group (Electron 2) has $1 \\mathrm{eV}$
$\\mathrm{eV}$. Both the density is assumed to be $10^{18} / \\mathrm{m}^{3}$. In this case the electron collision time is about several $\\mu \\mathrm{s}$. In this simulation we used the 1-dimensional (1D) EPOCH code $[4,5]$ with collisions. The relaxed temperature should be $(10 \\mathrm{eV}+1 \\mathrm{eV}) / 2$ $\\sim\\left(1.16 \\times 10^{5}+1.16 \\times 10^{4}\\right) / 2 \\sim 6.38 \\times 10^{4} \\mathrm{~K}$. The simulation result in Fig. 2.5 shows a reasonable temperature relaxation.

\\section*{References}
\\begin{enumerate}
  \\item A.B. Langdon, B.F. Lasinski, Electromagnetic and relativistic plasma simulation models. Methods Comput. Phys. Adv. Res. Appl. 16, 327-366 (1976)

  \\item R.W. Hockney, J.W. Eastwood, Computer Simulation Using Particles (Taylor \\& Francis, New York, 1988)

  \\item C.K. Birdsall, A.B. Langdon, Plasma Physics via Computer Simulation (Taylor \\& Francis, New York, 2005)

  \\item T.D. Arber, K. Bennett et al., Contemporary particle-in-cell approach to laser-plasma modelling. Plasma Phys. Control. Fusion 57, 113001 (2015)

  \\item K. Bennett, Users Manual for the EPOCH PIC Codes, EPOCH Version 4.3 (2014)

  \\item L.D. Landau, E.M. Lifshitz, Quantum Mechanics (Pergamon Press Ltd., Oxford, 1977)

  \\item L.D. Landau, E.M. Lifshitz, Statistical Physics (Pergamon Press Ltd., Oxford, 1980)

  \\item S. Ichimaru, Statistical Plasma Physics. Vol. 1: Basic Principles (CRC Press, Boca Raton, 2004)

  \\item J.D. Jackson, Classical Electrodynamics (Wiley, Hoboken, 1999)

  \\item F.W. Olver, D.W. Lozier, R.F. Boivert, C.W. Clark (eds.), NIST Handbook of Mathematical Functions, chap. 7 (National Institute of Standards and Technology and Cambridge University Press, 2010). \\href{https://dlmf.nist.gov}{https://dlmf.nist.gov}

  \\item G.B. Arfken, H.J. Weber, Mathematical Methods for Physicists, chap. 8, 6th edn. (Elsevier Academic Press, Amsterdam, 2005)

  \\item A.S. Richardson, NRL Plasma Formulary (The Office of Naval Research, 2019)

  \\item D.V. Sivukhin, Coulomb collisions in a fully ionized plasma. Rev. Plasma Phys. 4 (1966), ed. by M.A. Leontovichi

\\end{enumerate}

\\section*{Chapter 3 Single Particle Motion }
\\begin{abstract}
Collective behavior is the characteristic of plasma. Plasma is also a collection of individual charged particles. In this chapter, we focus on the motion of the individual charged particles.
\\end{abstract}

\\subsection*{Equation of Motion}
The equation of motion for a charged particle is as follows:


\\begin{equation*}
\\frac{\\mathrm{d} \\mathbf{P}}{\\mathrm{d} t}=q(\\mathbf{E}+\\mathbf{v} \\times \\mathbf{B})=\\mathbf{F} \\tag{3.1}
\\end{equation*}


Here $\\mathbf{P}$ shows the momentum. The special relativity presents the following relativistic momentum:


\\begin{equation*}
\\mathbf{P}=\\frac{m \\mathbf{v}}{\\sqrt{1-\\frac{\\mathbf{v}^{2}}{c^{2}}}} \\equiv \\gamma m \\mathbf{v} \\tag{3.2}
\\end{equation*}


Here $m$ is the rest mass. In the case of $|\\mathbf{v}| \\ll c$, the momentum is approximately expressed as follows: $\\mathbf{P}=m \\mathbf{v}$. The equation of motion becomes the Newton equation of motion.

In the non-relativistic case of $|\\mathbf{v}| \\ll c$, when the force has only the $x$ component of $F_{x}$ in $x, v_{x}$ changes and $v_{y}=$ does not change. However, in the relativistic case, $d P_{y} / d t=0$ but $d v_{y} / d t$ is not zero in this case: $v_{y}$ is not constant, but $P_{y}$ is constant. Equation (3.2) shows $P_{y}=\\gamma m v_{y}$. Here $m$ is constant, and $\\gamma v_{y}$ is constant. Because of $\\gamma=1 / \\sqrt{1-\\left(v_{x}^{2}+v_{y}^{2}+v_{z}^{2}\\right) / c^{2}}$, even when only $F_{x}$ exists, $v_{y}$ is not constant. Through $\\gamma, v_{y}$ and $v_{z}$ in the transverse directions also change in this case. An example trajectory in the phase space $\\left(v_{x} / c, v_{y} / c\\right)$ is shown in Fig. 3.1, in which the $+q$ charged particle has the initial velocity of $\\left(v_{x}=0.2 c, v_{y}=0.2 c, v_{z}=0\\right)$, and it is accelerated in $x$ by $E_{x}$. During the acceleration, the transverse velocity of $v_{y}$ decreases.

Fig. 3.1 Charged particle $+q$ is accelerated by $E_{x}$. Initially $+q$ has the velocity of $(0.2 c, 0.2 c, 0)$

\\begin{center}
\\includegraphics[max width=\\textwidth]{2024_02_26_83e36543483eb7d284c1g-057}
\\end{center}

For non-relativistic cases, the following Newton equation of motion is used:


\\begin{equation*}
m \\frac{\\mathrm{d} \\mathbf{v}}{\\mathrm{d} t}=q(\\mathbf{E}+\\mathbf{v} \\times \\mathbf{B}) \\tag{3.3}
\\end{equation*}


\\subsection*{Cyclotron Motion}
First let us consider a charged particle motion in a uniform magnetic field in the $z$ direction. The equation of motion is as follows:

\\[
\\left.\\begin{array}{l}
\\text { (a) } m \\frac{\\mathrm{d} v_{x}}{\\mathrm{~d} t}=q v_{y} B_{z}  \\tag{3.4}\\\\
\\text { (b) } m \\frac{\\mathrm{d} v_{y}}{\\mathrm{~d} t}=-q v_{x} B_{z} \\\\
\\text { (c) } m \\frac{\\mathrm{d} v_{z}}{\\mathrm{~d} t}=0
\\end{array}\\right\\}
\\]

In $z$, the charged particle moves with the constant $v_{z}$. In the $x$ and $y$ directions, it is convenient to solve $v_{x}+i v_{y}$. The following equation is obtained by Eqs. (3.4) (a) and (b):


\\begin{equation*}
\\frac{\\mathrm{d}\\left(v_{x}+i v_{y}\\right)}{\\mathrm{d} t}=-i \\Omega\\left(v_{x}+i v_{y}\\right), \\quad \\Omega \\equiv \\frac{q B_{z}}{m} \\tag{3.5}
\\end{equation*}


Fig. 3.2 Cyclotron motion of a charged particle

\\begin{center}
\\includegraphics[max width=\\textwidth]{2024_02_26_83e36543483eb7d284c1g-058}
\\end{center}

Here $\\Omega$ is the cyclotron frequency. The solution is as follows:


\\begin{equation*}
v_{x}+i v_{y}=(\\text { constant }) \\exp (-i \\Omega t) \\tag{3.6}
\\end{equation*}


The cyclotron motion is found. In Fig. 3.2 the cyclotron motion is schematically shown. In Fig. 3.2 a positive charge has the initial speed in $x$ and $z$.

Here let us calculate the cyclotron frequency for an electron and a proton in a uniform magnetic field $B_{z}=1 \\mathrm{~T}$. For the cyclotron frequency of $\\Omega=\\frac{e B}{m}$, the electron mass $9.1094 \\times 10^{-31} \\mathrm{~kg}$, the proton mass $1.6726 \\times 10^{-27} \\mathrm{~kg}$ and the elementary charge $e=1.6022 \\times 10^{-19} \\mathrm{C}$ are used. The electron cyclotron frequency is $\\Omega_{e}=1.76 \\times 10^{11} 1 / \\mathrm{s}$, and the proton cyclotron frequency is $\\Omega_{i}=9.58 \\times 10^{7} 1 / \\mathrm{s}$.

\\subsection*{Drift Motion}
We have the magnetic field $B_{z}$ in the $z$ direction and the electric field $E_{y}$ in $y$. By the magnetic field of $B_{z}$, a charged particle makes the cyclotron motion shown in Fig. 3.2. In our case, there is $E_{y}$. the positive charge $+q$ is accelerated by $E_{y}$ in the $+y$ direction, and the rotation radius becomes large, when it moves in the $+y$ direction. When the positive charge moves in the $-y$ direction, it is decelerated and its rotation radius becomes small. Therefore, charged particles show the drift motion in addition to the cyclotron motion. Figure 3.3 shows the drift motion.

Here we focus on the $y$ component of the equation of motion:


\\begin{equation*}
m \\frac{\\mathrm{d} v_{y}}{\\mathrm{~d} t}=q\\left(E_{y}-v_{x} B_{z}\\right) \\tag{3.7}
\\end{equation*}


Fig. 3.3 $E \\times B$ drift

\\begin{center}
\\includegraphics[max width=\\textwidth]{2024_02_26_83e36543483eb7d284c1g-059}
\\end{center}

$B_{z}$

Here we assume that the drift motion does not depend on time, and the time dependence comes from the cyclotron motion:


\\begin{equation*}
E_{y}-v_{x} B_{z}=0 \\tag{3.8}
\\end{equation*}


Then we obtain the drift motion:


\\begin{equation*}
v_{x}=\\frac{E_{y}}{B_{z}} \\tag{3.9}
\\end{equation*}


In general, we can write the drift motion as follows:


\\begin{equation*}
\\mathbf{v}=\\frac{\\mathbf{E} \\times \\mathbf{B}}{\\mathbf{B}^{2}} \\tag{3.10}
\\end{equation*}


The drift is called the $E \\times B$ drift.

When the force $\\mathbf{F}$ is applied instead of the electric field $\\mathbf{E}$, we obtain the following $\\mathbf{F} \\times \\mathbf{B}$ drift. In Eq. (3.10), $\\mathbf{E}$ is replaced by $\\mathbf{F} / q$ :


\\begin{equation*}
\\mathbf{v}=\\frac{\\mathbf{F} \\times \\mathbf{B}}{q \\mathbf{B}^{2}} \\tag{3.11}
\\end{equation*}


When $\\mathbf{F}$ does not depend on $q$, the drift direction in Eq. (3.11) depends on the charge $q$. For example, the gravity $\\mathbf{F}=m \\mathbf{g}$ or the centrifugal force $\\mathbf{F}=m v^{2} \\mathbf{r} / \\mathbf{r}^{2}$ does not depend on the charge. Depending on the charge sign of $+q$ or $-q$, the drift direction changes, and the charge separation appears in these cases.

Fig. 3.4 Gradient $B$ drift $(\\nabla B$ drift $)$

\\begin{center}
\\includegraphics[max width=\\textwidth]{2024_02_26_83e36543483eb7d284c1g-060}
\\end{center}

Here we consider the gradient $B$ drift ( $\\nabla B$ drift). As shown in Fig. 3.4, only $B_{z}$ exists, and its strength becomes large along the $+y$ direction. When $B_{z}$ is stronger, the rotation radius $r_{L}$ becomes small. When $B_{z}$ is weak, the rotation radius $r_{L}$ becomes large. Therefore, charged particles make the drift motion in Fig. 3.4.

The $\\nabla B$ drift is expressed as follows:


\\begin{equation*}
\\mathbf{v}_{\\nabla B}=\\frac{1}{2 B_{0}^{2}} \\frac{v_{\\perp}^{2}}{\\Omega}\\left(\\mathbf{B}_{0} \\times \\nabla B_{0}\\right) \\tag{3.12}
\\end{equation*}


Here $v_{\\perp}$ shows the perpendicular velocity to the magnetic field, and $\\mathbf{B}_{0}$ is the averaged magnetic field, when $\\nabla B_{0}$ is small. The cyclotron frequency of $\\Omega=\\left(q B_{z}\\right) / m$ depends on the charge $q$. Therefore, in the $\\nabla B$ drift electrons and ions drift in opposite directions, resulting in charge separation. The detailed derivation of Eq. (3.12) can be found in Chap. 2 in Ref. [1] and Chap. 2 in Ref. [2].

When $\\mathbf{E}$ depends on time $t$ in the $E \\times B$ drift as shown in Fig.3.5, another drift appears in addition to the $E \\times B$ drift. In Fig. 3.5 $E_{y}$ becomes large with time. The plus $+q$ charge drifts to the upward direction, when $E_{y}$ becomes large.

Here the non-relativistic equation of motion for a charge $+q$ is used: $m \\frac{\\mathrm{d} \\mathbf{v}}{\\mathrm{d} t}=$ $q(\\mathbf{E}+\\mathbf{v} \\times \\mathbf{B})$. The motion of the charge $+q$ is divided into the cyclotron motion $\\mathbf{v}_{c}$, the $E \\times B$ drift $\\mathbf{v}_{E \\times B}$ and the polarization drift $\\mathbf{v}_{p}$. The cyclotron motion is described by $m \\frac{\\mathrm{d} \\mathbf{v}_{c}}{\\mathrm{~d} t}=q \\mathbf{v}_{c} \\times \\mathbf{B}$. The $E \\times B$ drift is derived from $0=q\\left(E_{y}-v_{x, E \\times B} \\times B_{z}\\right)$ : $v_{x, E \\times B}=E_{y} / B_{z}\\left(\\right.$ or $\\left.\\mathbf{v}_{x, E \\times B}=\\mathbf{E} \\times \\mathbf{B} / B^{2}\\right)$. Then the polarization drift is expressed as follows:


\\begin{equation*}
m \\frac{\\mathrm{d}}{\\mathrm{d} t} \\mathbf{v}_{x, E \\times B}=m \\frac{\\dot{\\mathbf{E}} \\times \\mathbf{B}}{B^{2}}=q \\mathbf{v}_{p} \\times \\mathbf{B} \\tag{3.13}
\\end{equation*}


Fig. 3.5 Polarization drift

\\begin{center}
\\includegraphics[max width=\\textwidth]{2024_02_26_83e36543483eb7d284c1g-061}
\\end{center}

Here $\\dot{\\mathbf{E}}=\\frac{\\mathrm{d} \\mathbf{E}}{\\mathrm{d} t}$. Then we obtain the polarization drift as follows, after using a vector formula in Appendix B.2:


\\begin{equation*}
\\mathbf{v}_{p}=\\frac{1}{\\Omega} \\frac{\\dot{\\mathbf{E}}}{B} \\tag{3.14}
\\end{equation*}


This is the polarization drift.

When $-\\nabla p$ exists in plasmas, the diamagnetic drift appears. Here $p$ shows the pressure. In this case we have a density gradient or a temperature gradient or both. If we assume the drift motion is steady in time and we have no electric field, the following approximation may hold:


\\begin{equation*}
0=-\\nabla p+n q \\mathbf{v} \\times \\mathbf{B} \\tag{3.15}
\\end{equation*}


The macroscopic equation of motion is introduced in Chap. 5 or Eq. (6.53). The diamagnetic drift velocity $\\mathbf{v}_{\\perp}$ transverse to $\\mathbf{B}$ is as follows:


\\begin{equation*}
\\mathbf{v}_{\\perp}=-\\frac{\\nabla p \\times \\mathbf{B}}{n q B^{2}} \\tag{3.16}
\\end{equation*}


In the diamagnetic drift, plasma particles themselves do not drift. The gradient of the physical quantity of $p$ produces a net current. When the plasma density or the temperature has the density gradient transverse to $\\mathbf{B}$, each plasma particle rotates around the magnetic field, and at the higher $p$ region the particle cyclotron motion induces the net current as shown in Fig.3.6.

Figure 3.7 shows the current distribution for the diamagnetic current. The initial conditions correspond to those shown in Fig. 3.6. In $+y$ the constant density gradient

\\begin{center}
\\includegraphics[max width=\\textwidth]{2024_02_26_83e36543483eb7d284c1g-062}
\\end{center}

Fig. 3.6 Diamagnetic drift

\\begin{center}
\\includegraphics[max width=\\textwidth]{2024_02_26_83e36543483eb7d284c1g-062(1)}
\\end{center}

Fig. 3.7 Current by the diamagnetic drift. In this case, the initial setup corresponds to Fig. 3.6. The diamagnetic current, shown by arrows, appears in the $-x$ direction in this case. The EPOCH 3D simulation shows the diamagnetic current clearly
is imposed. In the EPOCH3D simulation [3, 4], the boundary condition in $x$ and $z$ is the cyclic boundary condition. In $y$ the reflection boundary condition is imposed at the $+y$ boundary and the thermal heat bath is located at $-y$ boundary [4]. The magnetic field $B_{z}$ is in $+z$, and it strength is $3.0 \\mathrm{~T}$. The electron temperature is $1.0 \\mathrm{keV}$, and the ion temperature is $1.0 \\mathrm{eV}$. The fully ionized hydrogen plasma is employed in the EPOCH3D simulation. The averaged density is $10^{20} / \\mathrm{m}^{3}$ for both of electrons and protons. The density gradient of $10 \\%$ is imposed in $+y$ in $1 \\mathrm{~mm}$. The simulation box is a $1 \\mathrm{~mm}$-side cubic. The electron cyclotron radius is about $25.1 \\mu \\mathrm{m}$, and the electron cyclotron frequency is about $5.28 \\times 10^{11} / \\mathrm{s}$. The electron Debye length is about $23.5 \\mu \\mathrm{m}$, and the spatial mesh size is $5 \\mu \\mathrm{m}$.

\\subsection*{Magnetic Moment}
In Sect. 3.2, the cyclotron motion in a uniform magnetic field was considered. In the magnetic field, a charged particle $q$ rotates with the frequency of $\\Omega=q B / m$. The circular current $I$ by the rotating charge passes through one point of the circle by $\\Omega / 2 \\pi$ in a second.


\\begin{equation*}
I=\\frac{q \\Omega}{2 \\pi} \\tag{3.17}
\\end{equation*}


The circular current by a charged particle is equivalent to a magnetic dipole. The magnetic (dipole) moment $\\mu_{m}$ is the product of the circular current and its surface enclosed and is shown below:


\\begin{equation*}
\\mu_{m}=I \\pi r_{L}^{2}=\\frac{m v_{\\perp}^{2}}{2 B} \\tag{3.18}
\\end{equation*}


Here $r_{L}=v_{\\perp} / \\Omega$, and $v_{\\perp}$ is the transverse speed of the charge to the magnetic field.

As shown in Fig. 3.8, here we consider the magnetic field, which changes slowly in space. Let us obtain the force on the charge $q$. The magnetic field is cylindrically

Fig. 3.8 Magnetic moment is approximately conserved, when the magnetic field changes slowly in space

\\includegraphics[max width=\\textwidth, center]{2024_02_26_83e36543483eb7d284c1g-063}
symmetric and directs approximately in the $z$ direction (see Fig. 3.8). In this case $B_{\\theta}=0$. By $\\operatorname{div} \\mathbf{B} \\equiv \\nabla \\cdot \\mathbf{B}=0$, the following is obtained:


\\begin{equation*}
\\frac{1}{r} \\frac{\\partial}{\\partial r}\\left(r B_{r}\\right)+\\frac{\\partial B_{z}}{\\partial z}=0 \\tag{3.19}
\\end{equation*}


We consider that the magnetic field changes slowly:


\\begin{equation*}
\\frac{\\partial B_{z}}{\\partial z} \\sim \\text { constant } \\tag{3.20}
\\end{equation*}


Then we obtain the following:


\\begin{equation*}
B_{r} \\simeq-\\frac{1}{2} r \\frac{\\partial B_{z}}{\\partial z} \\tag{3.21}
\\end{equation*}


The force $f_{z}$ in the $z$ direction is obtained:


\\begin{equation*}
f_{z}=q v_{\\perp} B_{r} \\simeq q v_{\\perp}\\left(-\\frac{1}{2} r_{L} \\frac{\\partial B_{z}}{\\partial z}\\right)=-\\frac{\\mathrm{mv}_{\\perp}^{2} / 2}{B} \\frac{\\partial B}{\\partial z}=-\\mu_{m} \\frac{\\partial B}{\\partial z} \\tag{3.22}
\\end{equation*}


Here we will check that the magnetic moment $\\mu_{m}$ is conserved. The change in energy for the charge $q$ is considered:


\\begin{equation*}
\\Delta\\left(\\frac{\\mathrm{mv}_{z}^{2}}{2}\\right)=f_{z} \\Delta z=-\\mu_{m} \\frac{\\partial B}{\\partial z} \\Delta z \\simeq-\\mu_{m} \\Delta B \\tag{3.23}
\\end{equation*}


Now we investigate the change in $\\mu_{m}$ of $\\Delta \\mu_{m}$ :


\\begin{equation*}
\\Delta \\mu_{m}=\\Delta\\left(\\frac{\\mathrm{mv}_{\\perp}^{2} / 2}{B}\\right)=\\frac{\\Delta\\left(\\mathrm{mv}_{\\perp}^{2} / 2\\right)}{B}-\\frac{\\mathrm{mv}_{\\perp}^{2}}{2} \\frac{\\Delta B}{B^{2}} \\tag{3.24}
\\end{equation*}


On the other hand, the magnetic field does not work on charge, and the particle energy should be conserved:


\\begin{equation*}
\\Delta\\left(\\frac{\\mathrm{mv}_{\\perp}^{2}}{2}\\right)+\\Delta\\left(\\frac{\\mathrm{mv}_{z}^{2}}{2}\\right)=0 \\tag{3.25}
\\end{equation*}


From Eq. (3.23), the following is found:


\\begin{equation*}
\\Delta\\left(\\frac{\\mathrm{mv}_{\\perp}^{2}}{2}\\right)=-\\Delta\\left(\\frac{\\mathrm{mv}_{z}^{2}}{2}\\right)=\\mu_{m} \\Delta B \\tag{3.26}
\\end{equation*}


Therefore, Eq. (3.24) becomes as follows:


\\begin{align*}
\\Delta \\mu_{m} & =\\frac{\\Delta\\left(\\mathrm{mv}_{\\perp}^{2} / 2\\right)}{B}-\\frac{\\mathrm{mv}_{\\perp}^{2}}{2} \\frac{\\Delta B}{B^{2}} \\\\
& =\\mu_{m} \\frac{\\Delta B}{B}-\\mu_{m} \\frac{\\Delta B}{B}=0 \\tag{3.27}
\\end{align*}


Approximately $\\mu_{m}$ is conserved. The physical quantity conserved, like $\\mu_{m}$, is called an adiabatic invariant.

Because of the conservation of the magnetic moment, the following interest phenomenon is found. In the slowly varying magnetic field the following magnetic moment and the kinetic energy are conserved:


\\begin{align*}
& \\mu_{m}=\\frac{m v_{\\perp}^{2}}{2 B}=\\text { constant }  \\tag{3.28}\\\\
& \\frac{m v_{z}^{2}}{2}+\\frac{m v_{\\perp}^{2}}{2}=\\text { constant } \\tag{3.29}
\\end{align*}


When the magnetic field becomes strong, the magnetic moment conservations show that the transverse kinetic energy $m v_{\\perp}^{2} / 2$ should increase with the increase in $B$. In this case, the energy conservation shows that the longitudinal kinetic energy should decrease. When the total longitudinal kinetic energy is converted to the transverse kinetic energy, the charged particle has no longitudinal velocity and bounces off. As shown in Fig. 3.8, at the place where the magnetic field is strong, charged particles are reflected. When we have the mirror magnetic field shown in Fig. 3.9, charged particles are reflected near both the ends of the mirror field.

When we have the mirror magnetic field in space and at the same time one end of the mirror field moves fast to the mirror field center, charged particles trapped by the moving mirror field would be strongly accelerated.

\\subsection*{Single Particle Simulations}
Here we calculate again an acceleration of electrons by a static electric field $E_{x}$ in $x$. The electron has the initial velocity of $(0.2 c, 0.8 c, 0)$ with a small distrobution. Here $c$ shows the speed of light, that is, $\\sim 2.9979 \\times 10^{8} \\mathrm{~m} / \\mathrm{s}$. Figure 3.10 a shows the initial electron distribution in the phase space of $\\left(v_{x}, v_{y}\\right)$. In this example case, we have

\\begin{center}
\\includegraphics[max width=\\textwidth]{2024_02_26_83e36543483eb7d284c1g-066}
\\end{center}

Fig. 3.9 Mirror magnetic field, which changes slowly in space, confines charged particles. This result comes from the conservation of the magnetic moment. The mirror field can be found even on earth. When we have the mirror magnetic field in space and at the same time one end of the mirror field moves fast to the mirror field center, particles trapped by the moving mirror field would be strongly accelerated
\\includegraphics[max width=\\textwidth, center]{2024_02_26_83e36543483eb7d284c1g-066(1)}

Fig. 3.10 When an electron is accelerated by a strong electric field $E_{x}$ in $x$, the electron approaches the speed of light. Initially the electron has a velocity of $(0.2 c, 0.8 c, 0)$ with a small distribution in this example

only the electric force in $x$, and initially $v_{y} \\sim 0.8 c$ is larger than $v_{x} \\sim 0.2 c$. After the acceleration in Fig. 3.10b, $v_{y}$ reduces and $v_{x}$ approaches to $c$. In the relativistic case, the velocity components coupled with each other through the relativistic factor of $\\gamma=1 / \\sqrt{1-\\left(v_{x}^{2}+v_{y}^{2}+v_{z}^{2}\\right) / c^{2}}$ in Eq. (3.2).

Here we used again the EPOCH PIC code [3, 4]. In this computation, the relativistic equation of motion Eq. (3.1) is solved without the magnetic field.


\\begin{equation*}
\\frac{\\mathrm{d} \\mathbf{P}}{\\mathrm{d} t}=q(\\mathbf{E}+\\mathbf{v} \\times \\mathbf{B}) \\tag{3.1}
\\end{equation*}


Here we consider the algorithm to solve the equation of motion [5-9]. In many PIC codes, the Boris scheme [9] or the Buneman scheme [8] is employed for the

Fig. 3.11 The Buneman (Boris) scheme for charged particle rotation in magnetic field

\\begin{center}
\\includegraphics[max width=\\textwidth]{2024_02_26_83e36543483eb7d284c1g-067}
\\end{center}

particle pusher. In the particle pusher algorithm, first the particle velocity is advanced by the electric field in time for $\\Delta t / 2$. The new velocity is used to rotate the particle velocity by the magnetic field. Then the particle is again pushed by the electric field for $\\Delta t / 2$ :

\\begin{enumerate}
  \\item Push particle by $\\mathbf{E}$ for $\\Delta t / 2$
\\end{enumerate}

$\\mathbf{P}^{*}=\\mathbf{P}^{n-1}+q \\mathbf{E}^{n} \\times \\frac{\\Delta t}{2 m}$.

\\begin{enumerate}
  \\setcounter{enumi}{1}
  \\item Rotate particle by $\\mathbf{B}$
a. $\\mathbf{v}^{*^{\\prime}}=\\mathbf{v}^{*}+\\mathbf{v}^{*} \\times \\mathbf{t}$. Here $\\mathbf{v}^{*}=\\mathbf{P}^{*} / \\gamma^{*}$ and $\\mathbf{t}=\\frac{q \\mathbf{B} \\delta t}{2 m \\gamma^{*}}$. It corresponds to $\\tan (\\theta / 2)$ in Fig. 3.11. The angle $\\theta$ is indicated also in Fig. 3.11.
b. $\\mathbf{v}^{* *}=\\mathbf{v}^{*}+\\mathbf{v}^{*^{\\prime}} \\times \\mathbf{s} . \\quad$ Here $\\mathbf{s}=\\frac{2 \\mathbf{t}}{1+\\mathbf{t}^{2}}$.

  \\item Push particle by $\\mathbf{E}$ for $\\Delta t / 2$

\\end{enumerate}

$\\mathbf{P}^{n+1 / 2}=\\mathbf{P}^{* *}+q \\mathbf{E}^{n} \\times \\frac{\\Delta t}{2 m}$.

Here we assumed that the fields of $\\mathbf{E}$ and $\\mathbf{B}$ are obtained at the time index of $n$ in Fig. 1.10b. As usual in numerical schemes, the equation of motion is discretized as shown above. The approximation provides reasonable results, when the time step $\\Delta t$ is sufficiently small.

The results in Figs. 3.1, 3.2, 3.3, 3.4, 3.5 and 3.10 were also obtained by the Buneman scheme.
\\includegraphics[max width=\\textwidth, center]{2024_02_26_83e36543483eb7d284c1g-068}

Fig. 3.12 Charged particles are confined in a mirror magnetic field in Fig. 3.9

Here we calculate a single particle motion in the mirror magnetic field in Fig. 3.9 by the Buneman scheme shown above. Figure 3.12 shows that charged particles are confined in the mirror magnetic field.

\\subsection*{Simulation of Electron Motion in Laser Field}
Here we see an electron motion in a laser field. In the Sect. 3.6, the laser field is given by an analytical solution of the Maxwell equations.

Figure 3.13 shows an electron trajectory in a plane laser field, which propagates in the $z$ direction and has $E_{x}$ and $B_{y}$ components. They are proportional to $\\sin (k z-\\omega t)$, and $\\omega / k=c$. Here $c$ is the speed of light.

The electron is initially stationary and is accelerated to $-x$ by $E_{x}$. Its trajectory is bent by $B_{y}$. When the laser field phase is reversed, the electron comes back to the initial $x$ position. The zigzag trajectory in Fig. 3.13 is a typical one.

When a laser has a Gauss ( or normal ) spatial distribution in its transverse, electron motion is modifed as shown in Fig. 3.14. In this case $E_{x}$ and $B_{y}$ have the Gaussian spacial profile, which is proportional to $\\exp \\left(-r^{2} / \\sigma^{2}\\right)$. Here $r$ shows the distance from the laser axis, which is shown by a dotted line in Fig. 3.14.

Figure 3.14 shows a typical electron trajectory in the Gaussian laser beam. The electron is accelerated by $E_{x}$ and bent by $B_{y}$. The electron receives the weak field

Fig. 3.13 Electron motion in plane laser wave. Electron is accelerated by the laser electric component and is bent by the magnetic field. Laser field components oscillate in time, and correspondingly the electron also oscillates
Fig. 3.14 Electron motion in a laser wave with the Gauss distribution in transverse. The laser propagates in the $z$ direction, and the laser center is denoted by a dotted line. Electron is accelerated by the laser electric component $E_{x}$ and is bent by the magnetic field $B_{y}$. The electron receives the weak field strength at the turning point compared with the peak field value. The ponderomotive force pushes the electron outward from the laser axis
\\includegraphics[max width=\\textwidth, center]{2024_02_26_83e36543483eb7d284c1g-069}

Fig. 3.15 Electron motion in a longitudinal oscillating electric field $E_{x}$ with the Gauss distribution in $x$. Electron oscillates by the longitudinal $E_{x}$. The electron is expelled from the central area by the ponderomotive force

\\begin{center}
\\includegraphics[max width=\\textwidth]{2024_02_26_83e36543483eb7d284c1g-070}
\\end{center}

strength at the turning point compared with the peak laser field value at the laser center. Consequently the electron moves outward from the laser central axis. The force acting on the electron is called the ponderomotive force. In Appendix D and Chap. 2 in Ref. [1] one can find the explicit expression for the ponderomotive force.

The ponderomotive force appears, when the electric field oscillates locally with a high frequency. In Fig. 3.14 the transverse electric field produces the ponderomotive force. In Fig. 3.15, a longitudinal oscillating electric field $E_{x}$ creates the ponderomotive force, and $E_{x}$ has the Gauss distribution in $x$. The electron is again pushed away from the central area by the ponderomotive force.

\\section*{References}
\\begin{enumerate}
  \\item D.R. Nicholson, Introduction to Plasma Theory (John Wiley \\& Sons, New York, 1983)

  \\item F.F. Chen, Introduction to Plasma Physics and Controlled Fusion, 3rd ed., (Springer, 2015)

  \\item T.D. Arber, K. Bennett et al., Contemporary particle-in-cell approach to laser-plasma modelling. Arber. Plasma Phys. Control. Fusion 57, 113001 (2015)

  \\item K. Bennett, Users Manual for the EPOCH PIC codes, EPOCH Version 4.3 (2014)

  \\item A.B. Langdon, B.F. Lasinski, Electromagnetic and relativistic plasma simulation models. Methods Comput. Phys. Adv. Res. Appl. 16, 327-366 (1976)

  \\item R.W. Hockney, J.W. Hockney, Computer Simulation Using Particles (Taylor \\& Francis, New York, 1988)

  \\item C.K. Birdsall, A.B. Langdon, Plasma Physics via Computer Simulation (Taylor \\& Francis, New York, 2005)

  \\item O. Buneman, Time-reversible difference procedures. J. Comput. Phys. 1, 515-535 (1967)

  \\item Boris, J.P.: The acceleration calculation from a scalar potential, Plasma Physics Laboratory, Princeton Univeristy, MATT-152 (1970)

\\end{enumerate}

\\section*{Chapter 4 
 Equations for Electromagnetic Field }
\\begin{abstract}
In Chap. 3, we examined one particle behavior in electromagnetic field, which is given. In this chapter, we summarize the equations for electromagnetic field (for details in electromagnetic fields, see, for example, Jackson in Classical Electrodynamics. John Wiley \\& Sons Inc, Hoboken (1999) [1] and Landau and Lifshitz in The classical theory of fields. Butterworth-Heinemann, Amsterdam (1980) [2]).
\\end{abstract}

\\subsection*{The Poisson Equation}
In electrostatic field, the electrostatic potential $\\phi$ gives the electric field:


\\begin{equation*}
\\mathbf{E}=-\\nabla \\phi \\tag{4.1}
\\end{equation*}


On the other hand, the Gauss theorem is as follows:


\\begin{equation*}
\\operatorname{div}\\left(\\varepsilon_{0} \\mathbf{E}\\right)=\\rho \\equiv n_{i} q_{i}-n_{e} q_{e} \\tag{4.2}
\\end{equation*}


Then we obtain the Poisson equation:


\\begin{equation*}
\\Delta \\phi=-\\frac{\\rho}{\\varepsilon_{0}} \\tag{4.3}
\\end{equation*}


Here we consider one simple example. We assume the phenomenon depends on just $x$. At $x=0$, a cathode is located with $\\phi=0$, and at $x=1$, an anode is located with $\\phi=\\phi_{a}$. There is no charge between $0 \\leq x \\leq 1$, that is, between the electrodes.


\\begin{equation*}
\\frac{\\mathrm{d}^{2} \\phi}{\\mathrm{d} x^{2}}=0 \\tag{4.4}
\\end{equation*}


We obtain the solution of $\\phi=\\phi_{a} x$.

\\subsection*{The Maxwell Equations}
When electromagnetic field changes rapidly, the Maxwell equations should be used. In the Maxwell equations, terms including time derivatives appear. The electric current density is denoted by $\\mathbf{J}$.

\\[
\\left.\\begin{array}{l}
\\text { (1) } \\nabla \\times \\mathbf{E}=-\\frac{\\partial \\mathbf{B}}{\\partial t}, \\quad \\mathbf{B}=\\mu_{0} \\mathbf{H} \\\\
\\text { (2) } \\nabla \\times \\mathbf{H}=\\frac{\\partial \\mathbf{D}}{\\partial t}+\\mathbf{J}, \\mathbf{D}=\\varepsilon_{0} \\mathbf{E}  \\tag{4.5}\\\\
\\text { (3) } \\nabla \\cdot \\mathbf{E}=\\frac{\\rho}{\\varepsilon_{0}} \\\\
\\text { (4) } \\nabla \\cdot \\mathbf{B}=0
\\end{array}\\right\\}
\\]

The first equation (1) in Eq. (4.5) shows the electromagnetic induction. Here we take one closed path shown in Fig.4.1. We set $\\Phi$ to the total sum of B, which penetrates the area $S$ surrounded by the path in Fig. 4.1.

Along the closed path in Fig. 4.1 the voltage $V$ appears:


\\begin{equation*}
V \\equiv \\int_{l} \\mathbf{E} \\cdot d \\mathbf{l}=-\\frac{\\mathrm{d} \\Phi}{\\mathrm{d} t}=-\\frac{\\mathrm{d}}{\\mathrm{d} t} \\int_{S} \\mathbf{B} \\cdot d \\mathbf{S} \\tag{4.6}
\\end{equation*}


This is the electromagnetic induction. The second term in Eq. (4.6) is written by using the following equation:


\\begin{equation*}
\\int_{S} \\mathbf{E} \\cdot d \\mathbf{l}=\\int_{l}(\\nabla \\times \\mathbf{E}) \\cdot d \\mathbf{S} \\tag{4.7}
\\end{equation*}


Then we obtain the first equation (1) in Eq. (4.5) ( see Appendix B.2).

Fig. 4.1 Electromagnetic induction

\\begin{center}
\\includegraphics[max width=\\textwidth]{2024_02_26_83e36543483eb7d284c1g-072}
\\end{center}

\\begin{center}
\\includegraphics[max width=\\textwidth]{2024_02_26_83e36543483eb7d284c1g-073}
\\end{center}

Fig. 4.2 Derivation of equation of continuity

In the second equation (2) in Eq. (4.5), the first term on the right hand is the displacement current. In order to find the mening of the displacement curret, let us take the divergence ( $\\operatorname{div} \\equiv \\nabla \\cdot$ ) of the equation:


\\begin{align*}
\\nabla \\cdot(\\nabla \\times \\mathbf{H}) & =0=\\frac{\\partial \\nabla \\cdot\\left(\\varepsilon_{0} \\mathbf{E}\\right)}{\\partial t}+\\nabla \\cdot \\mathbf{J}  \\tag{4.8}\\\\
\\therefore 0 & =\\frac{\\partial \\rho}{\\partial t}+\\nabla \\cdot \\mathbf{J}
\\end{align*}


Therefore, the equation of continuity for electric charge cannot be obtained without the displacement current.

Here we derive the equation of continuity Eq. (4.8). As shown in Fig.4.2, we consider a one-dimensional case depending on $x$, and $J_{x}=\\rho_{e} v_{x}$. Now we find the change in the charge density $\\rho_{e}$ during the time interval of $\\Delta t$ between $x$ and $x+\\Delta x$ :


\\begin{align*}
\\Delta \\rho_{e} \\Delta x & =\\Delta t\\left\\{\\left.\\rho_{e} v_{x}\\right|_{x}-\\left.\\rho_{e} v_{x}\\right|_{x+d x}\\right\\} \\\\
& =\\Delta t\\left\\{\\left.\\rho_{e} v_{x}\\right|_{x}-\\left(\\left.\\rho_{e} v_{x}\\right|_{x}+\\left.\\Delta x \\frac{\\partial \\rho_{e} v_{x}}{\\partial x}\\right|_{x}\\right)\\right\\} \\tag{4.9}
\\end{align*}


The left hand side of Eq. (4.9) shows the change in $\\rho_{e}$, and the right hand side shows the flux of $J_{x}=\\rho_{e} v_{x}$ in and out of $\\Delta x$.


\\begin{equation*}
\\frac{\\partial \\rho_{e}}{\\partial t}=\\lim _{\\Delta t \\rightarrow 0} \\frac{\\Delta \\rho_{e}}{\\Delta t}=-\\frac{\\partial \\rho_{e} v_{x}}{\\partial x}=-\\frac{\\partial J_{x}}{\\partial x} \\tag{4.10}
\\end{equation*}


The equation of continuity Eq. (4.8) was derived.

Here we see propagation of electromagnetic wave in vacuum. In vacuum, we have no charge $\\rho_{e}=0$ and no current $J=0$. The first equation (1) in the Maxwell equation Eq. (4.5) is differentiated by time $t$. Then we use the second equation (2) of Eq. (4.5):


\\begin{align*}
-\\frac{\\partial^{2} \\mathbf{B}}{\\partial t^{2}}=-\\mu_{0} \\frac{\\partial^{2} \\mathbf{H}}{\\partial t^{2}} & =\\nabla \\times\\left(\\frac{1}{\\varepsilon_{0}} \\nabla \\times \\mathbf{H}\\right)  \\tag{4.11}\\\\
& =\\frac{1}{\\varepsilon_{0}}\\{\\nabla(\\nabla \\cdot \\mathbf{H})-\\Delta \\mathbf{H}\\} .
\\end{align*}


Here we use $\\nabla \\cdot \\mathbf{H}=0$, and we obtain the following wave equation:


\\begin{equation*}
\\varepsilon_{0} \\mu_{0} \\frac{\\partial^{2} \\mathbf{H}}{\\partial t^{2}}-\\Delta \\mathbf{H}=0 \\tag{4.12}
\\end{equation*}


For $\\mathbf{E}$, the following is obtained:


\\begin{equation*}
\\varepsilon_{0} \\mu_{0} \\frac{\\partial^{2} \\mathbf{E}}{\\partial t^{2}}-\\Delta \\mathbf{E}=0 \\tag{4.13}
\\end{equation*}


These are the wave equations. For example, in Eq. (4.13), we assume that $\\mathbf{E}$ is a plane wave propagating in $x: \\mathbf{E}=\\mathbf{E}_{0} \\exp \\{i(k x-\\omega t)\\}$. Equation (4.13) gives the following relation:


\\begin{align*}
& -\\varepsilon_{0} \\mu_{0} \\omega^{2}+k^{2}=0 \\\\
& \\therefore \\frac{\\omega}{k}=\\frac{ \\pm 1}{\\sqrt{\\varepsilon_{0} \\mu_{0}}}= \\pm c \\tag{4.14}
\\end{align*}


The result shows that the wave propagates at the speed of light $c$ in a vacuum.

Here we consider about the energy flow of the electromagnetic field. After multiplying $\\mathbf{H}$ to the first equation (1) in Eq. (4.5), also multiplying $\\mathbf{E}$ to the second equation (2) in Eq. (4.5) and subtracting each other, then we obtain the following:


\\begin{equation*}
\\mathbf{H} \\cdot \\nabla \\times \\mathbf{E}-\\mathbf{E} \\cdot \\nabla \\times \\mathbf{H}=-\\mathbf{H} \\cdot \\frac{\\partial B}{\\partial t}-\\mathbf{E} \\cdot \\frac{\\partial \\mathbf{D}}{\\partial t}-\\mathbf{J} \\cdot \\mathbf{E} \\tag{4.15}
\\end{equation*}


This equation is summarized as follows (see also Appending B.2):


\\begin{equation*}
\\nabla \\cdot(\\mathbf{E} \\times \\mathbf{H})+\\frac{\\partial}{\\partial t}\\left(\\frac{\\varepsilon_{0} \\mathbf{E}^{2}+\\mu_{0} \\mathbf{H}^{2}}{2}\\right)=-\\mathbf{J} \\cdot \\mathbf{E} \\tag{4.16}
\\end{equation*}


The second term at the left hand in Eq. (4.16) shows the change in time of the energy density $\\epsilon_{e}$ of the electromagnetic field. Here we define the poynting vector $\\mathbf{S}_{e}$ :


\\begin{equation*}
\\mathbf{S}_{e} \\equiv \\mathbf{E} \\times \\mathbf{H} \\tag{4.17}
\\end{equation*}


Equation (4.16) becomes as follows:


\\begin{equation*}
\\frac{\\partial \\epsilon_{e}}{\\partial t}+\\nabla \\cdot \\mathbf{S}_{e}=-\\mathbf{J} \\cdot \\mathbf{E} \\tag{4.18}
\\end{equation*}


Equation (4.16) presents the energy conservation. The poynting vector $\\mathbf{S}_{e}$ shows the energy flux.

\\subsection*{Potential}
So far the electrostatic potential $\\phi$ was introduced in Eq. (4.1). For the magnetic field B, the vector potential A is introduced. From the fourth equation (4) of the Maxwell equation Eq. (4.5), that is, $\\nabla \\cdot \\mathbf{B}=0$ and the following mathematical identity (see also Appendix B.2.),


\\begin{equation*}
\\nabla \\cdot(\\nabla \\times \\mathbf{A})=0 \\tag{4.19}
\\end{equation*}


We introduce the vector potential $\\mathbf{A}$ :


\\begin{equation*}
\\mathbf{B}=\\nabla \\times \\mathbf{A} \\tag{4.20}
\\end{equation*}


Now we substitute $\\mathbf{A}$ into the first equation (1) in Eq. (4.5).


\\begin{align*}
& \\nabla \\times \\mathbf{E}=-\\frac{\\partial \\mathbf{B}}{\\partial t}=-\\nabla \\times \\frac{\\partial \\mathbf{A}}{\\partial t} \\\\
& \\therefore \\quad \\nabla \\times\\left(\\mathbf{E}+\\frac{\\partial \\mathbf{A}}{\\partial t}\\right)=0 \\tag{4.21}
\\end{align*}


Then we obtain the following:


\\begin{equation*}
\\mathbf{E}=-\\frac{\\partial \\mathbf{A}}{\\partial t} \\tag{4.22}
\\end{equation*}


In addition, we also use the following relation (see also Appendix B.2):


\\begin{equation*}
\\nabla \\times(\\nabla \\phi)=0 \\tag{4.23}
\\end{equation*}


Therefore, Eq. (4.22) can be modified as follows:


\\begin{equation*}
\\mathbf{E}=-\\frac{\\partial \\mathbf{A}}{\\partial t}-\\nabla \\phi \\tag{4.24}
\\end{equation*}


Equation (4.1) is also included in Eq. (4.24) as a special case.

\\[
\\left\\{\\begin{array}{l}
\\mathbf{E}=-\\nabla \\phi-\\frac{\\partial \\mathbf{A}}{\\partial t}  \\tag{4.25}\\\\
\\mathbf{B}=\\nabla \\times \\mathbf{A}
\\end{array}\\right.
\\]

Similar to the above discussion, we can further introduce another function of $\\xi$, which does not change the values of $\\mathbf{E}$ and $\\mathbf{B}$.


\\begin{align*}
\\phi & \\rightarrow \\phi-\\frac{\\partial \\xi}{\\partial t}  \\tag{4.26}\\\\
\\mathbf{A} & \\rightarrow \\mathbf{A}+\\nabla \\xi \\tag{4.27}
\\end{align*}


\\subsection*{Introduction to Kinetic Particle Simulation Model for Plasma}
When we have a plasma, in which many charged particles are contained, the electromagnetic field is also affected by the electric current, which is formed by the charged particle motion. In plasmas, normally we have to include the interaction between the charged particles and the electromagnetic field in plasmas. The charged particles create an electric current and charge, and they affect the electromagnetic field. In Sect. 4.4 we briefly outline the Particle-in-Cell (PIC) simulation method for plasmas.

Section 1.7 shows discretizations of space $\\mathbf{x}$ and time $t$ in Figs. 1.10 and 1.11. In order to obtain the electric and magnetic fields in $\\mathbf{x}$ and $t$, the Maxwell equations are also discretized [3-7].

In PIC codes, electric and magnetic fields are solved, for example, by the FiniteDifference Time-Domain (FDTD) method [5-8]. The charged particle motion is solved by the relativistic equation of motion in Eq. (3.1). One of the numerical algorithms to solve the equation of motion was introduced in Sect. 3.5. The particle behavior changes the electromagnetic field.

The PIC method is one of the kinetic particle methods for plasmas. In the PIC code, each numerical particle carries many real particles. The numerical particles are called superparticles. When we simulate a gas plasma, for example, with a density of $10^{18} / \\mathrm{m}^{3}$ with its size in a cubic meter, the total particle number is $10^{18}$. If we simulate a high-density plasma with its density of $10^{28} / \\mathrm{m}^{3}$ in a cubic centimeter, the total particle number is $10^{22}$.

For everyday-use computers, it would be preferable to have a smaller number of superparticles in particle simulations to obtain computation results in a reasonable CPU time: For example, it may be preferable to have less than $10^{9} \\sim 10^{15}$ particles or hopefully much less than this value in practical simulations. Recent exascale supercomputers [9] may accommodate $\\sim 10^{15}$ particles.

Depending on simulation scale size and plasma density, the total real charged particle number in plasmas changes largely. However, the computer resources required by the actual plasma particles exceed the available computer resources. Therefore, the superparticle, that is a kind of numerical macroparticle, was introduced.

When a charged superparticle collides with another one, the Coulomb collision must be very different from the real collision. Consequently, when the individual behaviors, including the Coulomb collision, are dominant in plasmas, it would be difficult to simulate the plasma individual behavior reasonably by the PIC method. However, we can expect that the collective behaviors are correctly simulated by the PIC code. The characteristics of the PIC method would impose some limitations on plasma PIC simulations. For example, the spacial mesh size would be the order of the Debye length $\\lambda_{D}$, as already discussed in Sect. 1.7. In addition, many superparticles may be needed in each spatial mesh to reduce numerical noises. It would be better to keep the properties of the PIC method in mind, though there are several methods to include the collision effect $[3,4,10-14]$ in order to extend the applicability of the PIC codes to collisional plasmas, as mentioned in Sect. 1.7. The PIC methods are widely used and very useful to analyze the plasma phenomena $[3-7,10,14]$.

\\subsubsection*{Structure of Particle-in-Cell (PIC) Code}
Figure 4.3 shows a rough sketch for Particle-in-Cell (PIC) code. The electromagnetic field is solved on meshes. Charged particles move among the meshes. Therefore, from the field values on the meshes, the field values on each particle are interpolated. The particle behavior provides the electric current density $\\mathbf{J}$ and the charge density $\\rho$, which change the electromagnetic field through the Maxwell equations.

\\subsubsection*{Field Solver}
For the electrostatic phenomena, the Poisson equation of Eq. (3) in Eq. (4.5) is solved. Here we show a field solving method based on Eqs. (1) and (2) in Eq. (4.5) by the Finite-Difference Time-Domain (FDTD) method in 2D.

\\[
\\left.\\begin{array}{l}
\\text { (1) } \\frac{\\partial \\mathbf{B}}{\\partial t}=-\\nabla \\times \\mathbf{E}  \\tag{4.5}\\\\
\\text { (2) } \\varepsilon_{0} \\frac{\\partial \\mathbf{E}}{\\partial t}=\\frac{1}{\\mu_{0}} \\nabla \\times \\mathbf{B}+\\mathbf{J}
\\end{array}\\right\\}
\\]

The definition points for $E_{x}, E_{y}$ and $B_{z}$ in the spatial mesh are shown in Figs. 1.11 and 4.3a.


\\begin{equation*}
\\frac{B_{z, i, j}^{n+\\frac{1}{2}}-B_{z, i, j}^{n-\\frac{1}{2}}}{\\Delta t}=-\\frac{E_{y, i+1, j}^{n}-E_{y, i, j}^{n}}{\\Delta x}+\\frac{E_{x, i, j+1}^{n}-E_{x, i, j+1}^{n}}{\\Delta y} \\tag{4.28}
\\end{equation*}


\\begin{center}
\\includegraphics[max width=\\textwidth]{2024_02_26_83e36543483eb7d284c1g-078}
\\end{center}

Fig. 4.3 Example structure of Particle-in-Cell (PIC) code. a The electromagnetic field is solved on meshes. Charged particles move among the meshes. b From the field values on the meshes, the field values on each particle are interpolated. $\\mathbf{c}$ The particle behavior provides the electric current density $\\mathbf{J}$ and the charge density $\\rho$, which change the electromagnetic field through the Maxwell equations


\\begin{align*}
\\varepsilon_{0} \\frac{E_{x, i, j}^{n+1}-E_{x, i, j}^{n}}{\\Delta t} & =\\frac{1}{\\mu_{0}} \\frac{B_{z, i, j}^{n+\\frac{1}{2}}-B_{z, i, j-1}^{n+\\frac{1}{2}}}{\\Delta y}-J_{x, i, j}^{n+\\frac{1}{2}}  \\tag{4.29}\\\\
\\varepsilon_{0} \\frac{E_{y, i, j}^{n+1}-E_{y, i, j}^{n}}{\\Delta t} & =-\\frac{1}{\\mu_{0}} \\frac{B_{z, i, j}^{n+\\frac{1}{2}}-B_{z, i-1, j}^{n+\\frac{1}{2}}}{\\Delta x}-J_{y, i, j}^{n+\\frac{1}{2}} \\tag{4.30}
\\end{align*}


Here the superscripts show the time index shown in Figs. 1.10b and 1.11b. For example, $n+1 / 2$ means the time at $\\Delta t \\times(n+1 / 2)$.

Various algorithms for the field solver have been proposed. Some for the field solvers in the PIC codes can be found, for example, in Refs. [5, 7, 8, 15, 16].

\\subsubsection*{Interaction Between Field and Particles}
After obtaining the field values on each spatial mesh, each particle needs the field values on the particle, as shown in Fig. 4.3b. The field values of $\\mathbf{E}$ and $\\mathbf{B}$ on particles are interpolated from the field values on spatial meshes by various interpolation methods [5-7].

Fig. 4.4 Example area weighting method for the interpolation of field values on meshes to those on particles. The particle position is shown by $\\mathrm{a} \\times$ sign

\\begin{center}
\\includegraphics[max width=\\textwidth]{2024_02_26_83e36543483eb7d284c1g-079}
\\end{center}

Figure 4.4 shows an example interpolation method, called the area weighting method. In this example interpolation method, each particle has a finite size, which is the same as the mesh size. The field value of $B_{z}$ in Fig. 4.4 is interpolated by the area weights: $B_{z}($ on particle $)=A 1 \\times B_{z}(i, j)+A 2 \\times B_{z}(i-1, j)+A 3 \\times$ $B_{z}(i-1, j-1)+A 4 \\times B_{z}(i, j-1) /(A 1+A 2+A 3+A 4)$.

Each particle is pushed by the interpolated field values by the particle pusher algorithm presented in Sect. 3.5. On the other hand, the particle behavior introduces electric charge and current. In order to obtain the electric current density $\\mathbf{J}$ and the charge density $\\rho$, the current and the charge carried by each particle are collected on each mesh in turn. In the collection of $\\mathbf{J}$ and $\\rho$ on meshes, the interpolation from particles to meshes is required. The interpolation is performed by the similar way shown above. Figure 4.3c shows the particle motion among meshes and the particle contribution to $\\mathbf{J}$ on the meshes schematically.

\\section*{Tips 4.1}
So far we have assumed that the spatial mesh size is uniform or does not change in space. If the spatial mesh size changes in space as shown in Fig. 4.5 in PIC codes, superparticles may be artificially compressed, together with the spatial mesh shrinking. Each superparticle carries many real charged particles. In order to avoid this numerical error, in PIC codes it would be better to use a sufficient number of superparticles in each mesh. In this example case in Fig. 4.5, we assumed the superparticle size changes with the mesh size. If the superparticle size is fixed to a reasonable certain size and does not change with the spatial mesh size, we can also avoid the numerical error.

Fig. 4.5 One kind of computational uncertainties, coming from an algorithm. When a superparticle size changes along with the mesh size as shown in Fig. 4.5, the particle may be artificially compressed

\\begin{center}
\\includegraphics[max width=\\textwidth]{2024_02_26_83e36543483eb7d284c1g-080}
\\end{center}

Fig. 4.6 Short pulse laser propagates to the right. From the left boundary the laser is injected

\\begin{center}
\\includegraphics[max width=\\textwidth]{2024_02_26_83e36543483eb7d284c1g-080(1)}
\\end{center}

\\subsubsection*{Simulation of Electron Cloud with Laser Field}
First we show a short pulse laser propagation in Fig. 4.6 by solving the Maxwell equations. A short pulse laser comes into the simulation box from the left boundary and propagates to the right. The EPOCH code was used to compute the laser propagation[3, 4], and the result in Fig. 4.6 was visualized by the visualization software of Visit [17, 18].

In Fig. 4.6, the laser wavelength is $0.8 \\mu \\mathrm{m}$, and the contour lines of $E_{y}$ are displayed. The laser has the Gauss distribution in longitudinal and transverse. The full width at half maximum is $25.0 \\mathrm{fs}$ in this example. The field solver solves the electromagnetic field fairly well.

\\begin{center}
\\includegraphics[max width=\\textwidth]{2024_02_26_83e36543483eb7d284c1g-081}
\\end{center}

Fig. 4.7 a Short pulse laser hits a thin-disk overdense plasma, in which the electron density is $5 \\times n_{c}$. Here $n_{c}$ shows the critical electron density, at which the electron plasma frequency $\\omega_{p e}$ is equal to the laser frequency $\\omega$. b The laser is reflected at the plasma surface and leaves from the left boundary of the computation domain

In Figs. 4.7, the short pulse laser with its intensity of $10^{10} \\mathrm{~W} / \\mathrm{cm}^{2}$ propagates in the $+x$ direction and illuminates a plasma thin disk. The plasma electron density is $5 \\times n_{c}$. Here $n_{c}$ shows the critical electron density, at which the electron plasma frequency $\\omega_{p e}$ is equal to the laser frequency $\\omega$. As presented later in Sect. 5.6, electromagnetic waves do not propagate inside dense plasmas, when $\\omega_{p e}>\\omega$. The laser is reflected at the plasma surface and leaves from the left boundary of the computation domain. Figures $4.7 \\mathrm{a}$ and $\\mathrm{b}$ reproduce the reasonable physics well. The example simulation results show that the PIC method, which is a combination of the field solver (Sect. 4.4.2) and the particle pusher (Sect. 3.5), would work well for full plasma simulations.

\\section*{References}
\\begin{enumerate}
  \\item J.D. Jackson, Classical Electrodynamics (John Wiley \\& Sons Inc, Hoboken, 1999)

  \\item L.D. Landau, E.M. Lifshitz, The Classical Theory of Fields (Butterworth-Heinemann, Amsterdam, 1980)

  \\item T.D. Arber, K. Bennett et al., Contemporary particle-in-cell approach to laser-plasma modelling. Arber. Plasma Phys. Control. Fusion 57, 113001 (2015)

  \\item K. Bennett, Users Manual for the EPOCH PIC codes, EPOCH Version 4.3 (2014)

  \\item A.B. Langdon, B.F. Lasinski, Electromagnetic and relativistic plasma simulation models. Methods Comput. Phys. Adv. Res. Appl. 16, 327-366 (1976)

  \\item R.W. Hockney, J.W. Eastwood, Computer Simulation using Particles (Taylor \\& Francis, New York, 1988)

  \\item C.K. Birdsall, A.B. Langdon, Plasma Physics via Computer Simulation (Taylor \\& Francis, New York, 2005)

  \\item Y. Hao, R. Mittra, FDTD Modeling of Materials Theory and Applications (Artech House, Boston, 2009)

  \\item List for Top 500 computers. \\href{https://top500.org/lists/top500}{https://top500.org/lists/top500}. Cited 14 October 2021

  \\item OSIRIS PIC Code: OSIRIS consortium. \\href{https://picksc.physics.ucla.edu/osiris.html}{https://picksc.physics.ucla.edu/osiris.html}

  \\item W.-M. Wang, P. Gibbon, Z.-M. Sheng, Y.-T. Li, Integrated simulation approach for laser-driven fast ignition. Phys. Rev. E 91, 013101 (2015)

  \\item H.-H. Song, W.-M. Wang, Y.-T. Li, Dense polarized positrons from laser-irradiated foil targets in the QED Regime. Phys. Rev. Lett. 129, 035001 (2022)

  \\item H.H. Song, W.M. Wang, Y.T. Li, YUNIC: a multi-dimensional particle-in-cell code for laserplasma interaction. arXiv:2104.00642 (2021)

  \\item Y. Sentoku, A.J. Kemp, Numerical methods for particle simulations at extreme densities and temperatures: weighted particles, relativistic collisions and reduced currents. J. Comput. Phys. 227, 6846-6861 (2008)

  \\item B.M. Cowan, D.L. Bruhwiler, J.R. Cary, E. Cormier-Michel, C.G.R. Geddes, Generalized algorithm for control of numerical dispersion in explicit time-domain electromagnetic simulations. Phys. Rev. Accel. Beams 16, 041303 (2013)

  \\item A. Pukhov, Three-dimensional electromagnetic relativistic particle-in-cell code VLPL (Virtual Laser Plasma Lab). J. Plasma Phys. 61, 425-433 (1999)

  \\item H. Childs, E. Brugger, B. Whitlock, J. Meredith, S. Ahern, D. Pugmire, K. Biagas, M. Miller, C. Harrison, G.H. Weber, H. Krishnan, T. Fogal, A. Sanderson, C. Garth, E.W. Bethel, D. Camp, O. Rübel, M. Durant, J.M. Favre, P. Navrátil, VisIt: an end-user tool for visualizing and analyzing very large data, in High Performance Visualization-Enabling Extreme-Scale Scientific Insight. ed. by E. Wes Bethel, H. Childs, C. Hansen (Chapman and Hall/CRC, New York, 2012), pp.357-372

  \\item VisIt homepage: \\href{https://visit-dav.github.io/visit-website/}{https://visit-dav.github.io/visit-website/}

\\end{enumerate}

\\section*{Chapter 5 Plasma by Fluid Model }
\\begin{abstract}
So far, we considered single particle motion and electromagnetic field. Now we enter the details of plasma physics. In the Chap. 5, the macroscopic model, that is, the fluid model, is introduced for plasma.
\\end{abstract}

\\subsection*{Basic Fluid Equations}
In fluid model, we need equations for mass density $\\rho$ (or number density $n$ ), velocity $\\mathbf{v}$ and energy density $e$ (or temperature $T$ ).

The equation of continuity Eq. (4.8) was derived for the charge density. Here we rewrite the equation of continuity for the mass density $\\rho=m n$, where $m$ is the mass. When there is a sink or a source of fluid, the corresponding term of $S$ is added to the right side of Eq. (5.1). In Eq. (5.1), we set $S$ to 0 at present.


\\begin{equation*}
\\frac{\\partial \\rho}{\\partial t}+\\nabla \\cdot \\rho \\mathbf{v}=0 \\tag{5.1}
\\end{equation*}


The equation of motion would be the Newton equation of motion:


\\begin{equation*}
\\rho \\frac{\\mathrm{d} \\mathbf{v}}{\\mathrm{d} t}=\\mathbf{F} \\tag{5.2}
\\end{equation*}


In fluid dynamics, the force at the right side hand of Eq. (5.2) includes the pressure gradient force, that is, $-\\nabla p$. The form of Eq. (5.2) is called the Lagrangian form in fluid dynamics. In the Lagrange method, a small piece of fluid is specified and followed, just like a fluid "particle" drop. In this form, the independent variable is just the time $t$. The left hand side of Eq. (5.2) is the total derivative.

On the other hand, in the equation of continuity Eq. (5.1), the independent variables are $t$ and $\\mathbf{x}$. The first term of the left side in Eq. (5.1) is the partial derivative of the mass density $\\rho$ by $t$, and it means the temporal change in $\\rho$ at the fixed $\\mathbf{x}$. So in Eq. (5.1) the spatial position $\\mathbf{x}$ is fixed to measure the temporal change in the mass density $\\rho$. This measuring method is called the Euler method.

\\begin{center}
\\includegraphics[max width=\\textwidth]{2024_02_26_83e36543483eb7d284c1g-084}
\\end{center}

Fig. 5.1 Convection term. At $t=t$, a physical quantity $g(t, x)$ moves to the right in this figure with the velocity of $v_{x}$. The spatial profile of $g$ does not change, but there is a flow with $v_{x}$. We stand at $x=x$, and measure the physical quantity of $g$. After $\\Delta t$, the profile of $g$ move to $g(t+\\Delta t)$. We find that $g$ increases by $g(t+\\Delta t, x)-g(t, x)$ at $x$

Equation (5.2) is here rewritten in the Euler form. In the Euler method the independent variables are time $t$ and space $\\mathbf{x}$. So in the right hand side of Eq. (5.2), the force $\\mathbf{F}(\\mathbf{x}, t)$ also depends on $\\mathbf{x}$ and $t$ :


\\begin{equation*}
\\rho \\frac{\\mathrm{d} \\mathbf{v}}{\\mathrm{d} t}=\\rho\\left\\{\\frac{\\partial \\mathbf{v}}{\\partial t}+(\\mathbf{v} \\cdot \\nabla) \\mathbf{v}\\right\\}=\\mathbf{F}(\\mathbf{x}, t) \\tag{5.3}
\\end{equation*}


The term expressed in the form of $(\\mathbf{v} \\cdot \\nabla) g$, which appears in the second term of Eq. (5.3), is called the convection term or the advection term. The convection term shows the effect of fluid flow. Here $g$ is a physical quantity, and in Eq. (5.3) $g$ is $\\mathbf{v}$.

Let us think about $\\mathbf{F}(\\mathbf{x}, t)$. The force of $\\mathbf{F}(\\mathbf{x}, t)$ includes the pressure gradient, ${ }^{1}$ and may include the Lorentz force by $\\mathbf{E}$ and $\\mathbf{B}$, depending on physics concerned. The following is an example equation of motion, in which $\\mathbf{F}(\\mathbf{x}, t)=-\\nabla P$ :


\\begin{equation*}
\\rho\\left\\{\\frac{\\partial \\mathbf{v}}{\\partial t}+(\\mathbf{v} \\cdot \\nabla) \\mathbf{v}\\right\\}=-\\nabla P \\tag{5.4}
\\end{equation*}


Here we consider about the convection term $(\\mathbf{v} \\cdot \\nabla) g$. We assume that no force is applied $(\\mathbf{F}=0)$, and that the fluid flows with a constant velocity $v_{x}=$ constant in $x$ as shown in Fig. 5.1. At $t=t, g=g(t, x)$ is measured at $x$. We are now standing at $x$, and measure the physical quantity of $g$, which may be the mass density $\\rho$, the velocity $\\mathbf{v}$ or the temperature $T$ (or energy). In our example, $g=g(t, x)$ has the profile shown in Fig.5.1. At $t=t+\\Delta t$, we measure the larger value of $g(t+\\Delta t, x)$ at the same place of $x$ as follows:
\\footnotetext{${ }^{1}$ The pressure may come from collisions among fluid particles. The fluid model is an averaged macroscopic model and a collision-rich model.
}


\\begin{align*}
& \\Delta g=g(t+\\Delta t, x)-g(t, x) \\sim-\\Delta t v_{x} \\frac{\\partial g}{\\partial x}  \\tag{5.5}\\\\
& \\therefore \\lim _{\\Delta t \\rightarrow 0} \\frac{\\Delta g}{\\Delta t}=\\frac{\\partial g}{\\partial t}=-v_{x} \\frac{\\partial g}{\\partial x}=-(\\mathbf{v} \\cdot \\nabla) g
\\end{align*}


On the other hand, if we move with the fluid with the velocity $v_{x}$, we always measure the same fluid part and get the same value of $g$. This measuring method is the Lagrange method. This is described by $\\mathrm{d} g / \\mathrm{d} t=0$. To give a stress on the total derivative in the Lagrange method, we would frequently use the following mathematical symbol for the Lagrange derivative:


\\begin{equation*}
\\frac{D g}{D t} \\equiv \\frac{\\partial g}{\\partial t}+(\\mathbf{v} \\cdot \\nabla) g=\\frac{\\mathrm{d} g}{\\mathrm{~d} t} \\tag{5.6}
\\end{equation*}


For the fluid energy, we need an energy equation for energy or temperature. From the energy conservation, we find the following energy equation (see, for example, [1], Chap. 2 in Ref. [2] and Chap. 1 in Ref. [3]):


\\begin{equation*}
\\rho \\frac{D e}{D t}=\\rho\\left\\{\\frac{\\partial e}{\\partial t}+(\\mathbf{v} \\cdot \\nabla) e\\right\\}=-P \\nabla \\cdot \\mathbf{v}+\\nabla \\cdot \\kappa(\\nabla T) \\tag{5.7}
\\end{equation*}


Here $e$ is the specific internal energy, the second term of the right side is the heat conduction, and $\\kappa$ shows the heat conduction coefficient. For ideal fluid, $e=3 n T /(2 \\rho)=3 T /(2 m)$. Equation (5.7) is transformed by using the specific heat $C_{v}=\\left.(\\partial e / \\partial T)\\right|_{\\rho}$ and the compressibility $B=\\left.(\\partial e / \\partial \\rho)\\right|_{T}$ as follows:


\\begin{equation*}
C_{v} \\frac{D T}{D t}+B \\frac{D \\rho}{D t}=-\\frac{P}{\\rho} \\nabla \\cdot \\mathbf{v}+\\frac{1}{\\rho} \\nabla \\cdot \\kappa(\\nabla T) \\tag{5.8}
\\end{equation*}


For ideal gas, the compressibility $B$ would be $B=0$, and $C_{v}=3 k /(2 m)$. Together with Eqs. (5.1), (5.4) and (5.8), the pressure $p$, the specific heat $C_{v}$ and the compressibility $B$ should be obtained from the mass density $\\rho$ and the temperature $T$ through the equation of state.

For ideal plasma, $P=n T, B=0$, and $C_{v}=3 k /(2 m)$. For an adiabatic fluid or plasma, the following relation would be used:


\\begin{equation*}
P / \\rho^{\\gamma}=\\text { constant } \\tag{5.9}
\\end{equation*}


For ideal monoatomic molecule, $\\gamma=5 / 3$. If the temperature is always constant, we can just set to the following:


\\begin{equation*}
T=\\text { constant } \\tag{5.10}
\\end{equation*}


In this case, we do not need to solve the energy equation. For non-ideal plasmas, we need to prepare the equation of state. For example, References [4-6] may introduce the realistic equation of state.

\\section*{Tips 5.1}
Here let us think about the second term at the left hand side of Eq. (5.4). When we consider one-dimensional phenomenon approximately without the pressure term, it becomes

$$
\\frac{\\partial v}{\\partial t}=-v \\frac{\\partial v}{\\partial x}
$$

The right had side has 2 " $v$ " s, and is so nonlinear with respect to $v$. When perturbations appear with $v>0$, the perturbations grow at $\\partial v / \\partial x<0$ : $\\partial v / \\partial t>$ 0 . The perturbed wave front becomes faster and stands, as seen in an Ukiyo-e of great ocean waves.

\\subsection*{Introduction to Plasma Simulation by the Euler Fluid Model}
Here a short introduction is presented for fluid simulation for plasma. In the Sect. 5.2, a first step of the finite difference method is introduced. Then simple fluid simulation examples are presented. In addition, a short summary for parallel computings is also available in Appendix E, in which parallel computings based on shared-memory architectures are focused. In scientific computings frequently parallel computings are required to save computing time.

In scientific computer simulation or computing, the following processes would be included:

\\begin{enumerate}
  \\item Find physical model: Understanding physics or problems involved.

  \\item Find mathematical model.

  \\item Normalization of physical quantities: dimensionless quantities would have benefit to reduce numerical errors, and similarity may be found.

  \\item Discretization of equations for computations.

  \\item Programming and debugging.

  \\item Computation

  \\item Data analyses and visualization

\\end{enumerate}

\\subsubsection*{Summary of Finite Difference Method (FDM)}
In order to solve differential equations including fluid equations, differentials are approximately solved by finite difference method (FDM). The FDM is based on the definition of the differential itself:


\\begin{equation*}
\\frac{\\mathrm{d} f(x)}{\\mathrm{d} x} \\equiv \\lim _{\\Delta x \\rightarrow 0} \\frac{f(x+\\Delta x)-f(x)}{\\Delta x} \\tag{5.11}
\\end{equation*}


In FDM this Eq. (5.11) is approximated as follows:


\\begin{equation*}
\\frac{\\mathrm{d} f(x)}{\\mathrm{d} x} \\sim \\frac{f(x+\\Delta x)-f(x)}{\\Delta x} \\tag{5.12}
\\end{equation*}


In FDM the difference of $\\Delta x$ is assumed to be sufficiently small, compared with the representative scale length of phenomena concerned, as shown in Fig. 5.2.

In Fig. 5.2, $f(x \\pm \\Delta x)$ is expanded around $x$ by the Taylor expansion:


\\begin{equation*}
f(x \\pm \\Delta x) \\sim f(x) \\pm \\Delta x \\frac{f(x)}{\\mathrm{d} x}+\\frac{\\Delta x^{2}}{2} \\frac{\\mathrm{d}^{2} f(x)}{\\mathrm{d} x^{2}}+\\cdots \\tag{5.13}
\\end{equation*}


From Eq. (5.13), $f(x+\\Delta x)-f(x-\\Delta x)$ is obtained:


\\begin{gather*}
f(x+\\Delta x)-f(x-\\Delta x) \\sim 2 \\Delta x \\frac{f(x)}{\\mathrm{d} x}+O\\left(\\Delta x^{3}\\right) \\\\
\\left.\\frac{f(x)}{\\mathrm{d} x}\\right|_{x} \\sim \\frac{f(x+\\Delta x)-f(x-\\Delta x)}{2 \\Delta x}+O\\left(\\Delta x^{2}\\right) \\tag{5.14}
\\end{gather*}


\\begin{center}
\\includegraphics[max width=\\textwidth]{2024_02_26_83e36543483eb7d284c1g-087}
\\end{center}

Fig. 5.2 In FDM (finite difference method) derivatives are approximated by differences

Here $O\\left(\\Delta x^{2}\\right)$ shows the second order of error. The finite difference method in Eq. (5.14) is the central difference with the second-order accuracy. If we look at the profile of $f(x)$ in Fig. 5.2, it is found that Eq. (5.14) does not include $f(x)$ at $x=x$. In Eq. (5.14), we found $\\mathrm{d} f(x) / \\mathrm{d} x$ at $x$ or at $i$. We can also estimate $\\mathrm{d} f(x) /\\left.\\mathrm{d} x\\right|_{i}$ at $x \\equiv i$ by the following equations:


\\begin{equation*}
\\left.\\left.\\frac{f(x)}{\\mathrm{d} x}\\right|_{x} \\equiv \\frac{\\mathrm{d} f}{\\mathrm{~d} x}\\right|_{i} \\sim \\frac{f(x+\\Delta x)-f(x)}{\\Delta x}+O\\left(\\Delta x^{1}\\right) \\tag{5.15}
\\end{equation*}


This is called the forward difference method.


\\begin{equation*}
\\left.\\left.\\frac{f(x)}{\\mathrm{d} x}\\right|_{x} \\equiv \\frac{\\mathrm{d} f}{\\mathrm{~d} x}\\right|_{i} \\sim \\frac{f(x)-f(x-\\Delta x)}{\\Delta x}+O\\left(\\Delta x^{1}\\right) \\tag{5.16}
\\end{equation*}


The method in Eq. (5.16) is the backward difference method. In these Eqs. (5.15) and (5.16) the difference equations are the one-sided difference ones.

Frequently we may also need the derivative at $x+\\Delta x / 2 \\equiv i+1 / 2$ :


\\begin{equation*}
\\left.\\left.\\left.\\frac{f(x)}{\\mathrm{d} x}\\right|_{x+\\Delta x / 2} \\equiv \\frac{\\mathrm{d} f}{\\mathrm{~d} x}\\right|_{i+1 / 2} \\sim \\frac{f(x+\\Delta x)-f(x)}{\\Delta x}\\right|_{i+1 / 2}+O\\left(\\Delta x^{2}\\right) \\tag{5.17}
\\end{equation*}


Here the index $i$ indicates the spatial position at $x=x$ as shown in Figs. 1.10, 1.11 and 5.2.

If finite differences of for example, $\\Delta x, \\Delta y, \\ldots, \\Delta t$ are sufficiently small, the finite difference method (FDM) works well to describe the corresponding differentials. The detailed FDM description can be found in Refs. [7, 8].

\\subsubsection*{Example 2D Simulation for Convection}
In the Sect. 5.2.2 an example fluid convection is simulated. Figure 5.3 shows the concept. Two stratified fluids flow in the $x$ direction with $v_{x}$. The surface between two fluids is perturbed a little bit. Let us simulate the simple example. Here we summarize the 2D Euler equations:

\\[
\\begin{array}{r}
\\frac{\\partial \\rho}{\\partial t}=-\\left(v_{x} \\frac{\\partial}{\\partial x}+v_{y} \\frac{\\partial}{\\partial y}\\right) \\rho+\\rho\\left(\\frac{\\partial v_{x}}{\\partial x}+\\frac{\\partial v_{y}}{\\partial y}\\right) \\\\
\\frac{\\partial v_{x}}{\\partial t}=-\\left(v_{x} \\frac{\\partial}{\\partial x}+v_{y} \\frac{\\partial}{\\partial y}\\right) v_{x}-\\frac{1}{\\rho} \\frac{\\partial P}{\\partial x} \\\\
\\frac{\\partial v_{y}}{\\partial t}=-\\left(v_{x} \\frac{\\partial}{\\partial x}+v_{y} \\frac{\\partial}{\\partial y}\\right) v_{y}-\\frac{1}{\\rho} \\frac{\\partial P}{\\partial y} \\\\
\\frac{\\partial T}{\\partial t}=-\\left(v_{x} \\frac{\\partial}{\\partial x}+v_{y} \\frac{\\partial}{\\partial y}\\right) T-\\frac{P}{C_{v} \\rho}\\left(\\frac{\\partial v_{x}}{\\partial x}+\\frac{\\partial v_{y}}{\\partial y}\\right) \\tag{5.18}
\\end{array}
\\]

Fig. 5.3 An example 2D (dimensional) simulation setup for fluid convection by FDM (finite difference method)

\\begin{center}
\\includegraphics[max width=\\textwidth]{2024_02_26_83e36543483eb7d284c1g-089}
\\end{center}

First the equations are normalized to make each physical quantity dimensionless. For example, the time $t$ has the dimension of sec, and $t$ is divided by $t_{0}$, which is the typical time scale and has the dimension of time. So $\\bar{t}=t / t_{0}$ has no dimension. For $x$, it can be normalized by the typical length of $L_{0}: \\bar{x}=x / L_{0}$. By the normalization, values of physical quantities are normalized by the typical values concerned, and the normalized values would be the order of 1 . The normalization is one effective method to avoid numerical errors. Another merit of the normalization would be to introduce similarity among physical quantities. For example, $\\bar{x} / \\bar{t}$ does not change, even when both $L_{0}$ and $t_{0}$ are doubled. One numerical solution may use for different scale as long as the normalized dimensionless values do not change.


\\begin{align*}
t & =\\bar{t} \\times t_{0}, x=\\bar{x} \\times x_{0}, m=\\bar{m} \\times m_{0}, v=\\bar{v} \\times v_{0} \\\\
T & =\\bar{T} \\times T_{0}, P=\\bar{P} \\times P_{0}, \\rho=\\bar{\\rho} \\times \\rho_{0}, C_{v}=\\bar{C}_{v} \\times C_{0} \\tag{5.19}
\\end{align*}


For the ideal equation of state $P=n k_{B} T$ and $C_{v}=3 k_{B} / 2 m$. Here the Boltzmann constant $k_{B}$ is explicitly shown. For convenience, the following relations are used:


\\begin{align*}
& v_{0}=L_{0} / t_{0}, \\rho_{0}=m_{0} n_{0}, m_{0}=m  \\tag{5.20}\\\\
& P_{0}=n_{0} k_{B} T_{0}=n_{0} m_{0} v_{0}^{2}=n_{0} m_{0}\\left(L_{0} / t_{0}\\right)^{2}, C_{0}=k_{B} / m_{0}
\\end{align*}


Therefore, $\\bar{m}=1$ and $\\bar{C}_{v}=3 / 2$. After the normalization, we obtain the dimensionless equations:

\\[
\\begin{array}{r}
\\frac{\\partial \\bar{\\rho}}{\\partial \\bar{t}}=-\\left(\\bar{v}_{x} \\frac{\\partial}{\\partial \\bar{x}}+\\bar{v}_{y} \\frac{\\partial}{\\partial \\bar{y}}\\right) \\bar{\\rho}+\\bar{\\rho}\\left(\\frac{\\partial \\bar{v}_{x}}{\\partial \\bar{x}}+\\frac{\\partial \\bar{v}_{y}}{\\partial \\bar{y}}\\right) \\\\
\\frac{\\partial \\bar{v}_{x}}{\\partial \\bar{t}}=-\\left(\\bar{v}_{x} \\frac{\\partial}{\\partial \\bar{x}}+\\bar{v}_{y} \\frac{\\partial}{\\partial \\bar{y}}\\right) \\bar{v}_{x}-\\frac{1}{\\bar{\\rho}} \\frac{\\partial \\bar{P}}{\\partial \\bar{x}} \\\\
\\frac{\\partial \\bar{v}_{y}}{\\partial \\bar{t}}=-\\left(\\bar{v}_{x} \\frac{\\partial}{\\partial \\bar{x}}+\\bar{v}_{y} \\frac{\\partial}{\\partial \\bar{y}}\\right) \\bar{v}_{y}-\\frac{1}{\\bar{\\rho}} \\frac{\\partial \\bar{P}}{\\partial \\bar{y}} \\\\
\\frac{\\partial \\bar{T}}{\\partial \\bar{t}}=-\\left(\\bar{v}_{x} \\frac{\\partial}{\\partial \\bar{x}}+\\bar{v}_{y} \\frac{\\partial}{\\partial \\bar{y}}\\right) \\bar{T}-\\frac{\\bar{P}}{\\bar{C}_{v} \\bar{\\rho}}\\left(\\frac{\\partial \\bar{v}_{x}}{\\partial \\bar{x}}+\\frac{\\partial \\bar{v}_{y}}{\\partial \\bar{y}}\\right) \\tag{5.24}
\\end{array}
\\]

All the normalized factors, including $v_{0} t_{0} / L_{0}$ and $\\left(P_{0} t_{0}^{2}\\right) /\\left(\\rho_{0} L_{0}^{2}\\right)$ become 1 . In our case the final dimensionless basic equations have the same form with the original ones in Eq. (5.18).

Here we assume the fluid or plasma is hydrogen, and so $m_{0}$ is the hydrogen mass $\\sim 1.6726 \\times 10^{-27} \\mathrm{~kg}$. We also set $L_{0}=1 \\mathrm{~m}$ and $t_{0}=1 \\mathrm{~s}$. These values give us the followings: $v_{0}=L_{0} / t_{0}=1 \\mathrm{~m} / \\mathrm{s}$ and $T_{0}=m_{0} L_{0}^{2} /\\left(k_{B} t_{0}^{2}\\right)=1.211414 \\times 10^{-4} \\mathrm{~K}$.

First we try to simulate just a convection term without the plasma/fluid compression or expansion $(\\nabla \\cdot \\mathbf{v}=0)$ with a constant speed in $x\\left(v_{x}=\\right.$ constant and $\\left.v_{y}=0\\right)$ :

$$
\\frac{\\partial \\bar{\\rho}}{\\partial \\bar{t}}=-\\left(\\bar{v}_{x} \\frac{\\partial}{\\partial \\bar{x}}+\\bar{v}_{y} \\frac{\\partial}{\\partial \\bar{y}}\\right) \\bar{\\rho}=-\\bar{v}_{x} \\frac{\\partial \\bar{\\rho}}{\\partial \\bar{x}}
$$

In this example, Eq. (5.21) is solved by $\\frac{\\partial \\bar{\\rho}}{\\partial \\bar{t}}=-\\bar{v}_{x} \\frac{\\partial \\bar{\\rho}}{\\partial \\bar{x}}$. It is discretized as follows:

\\[
\\frac{\\rho_{i, j}^{n+1}-\\rho_{i, j}^{n}}{\\Delta t}=-\\left\\{\\begin{array}{l}
v_{x} \\frac{\\rho_{i, j}^{n}-\\rho_{i-1, j}^{n}}{\\Delta x}\\left(v_{x} \\geq 0\\right)  \\tag{5.25}\\\\
v_{x} \\frac{\\rho_{i+1, j}^{n}-\\rho_{i, j}^{n}}{\\Delta x}\\left(v_{x}<0\\right)
\\end{array}\\right.
\\]

Here we omit the top bar symbol for each quantity, and Eq. (5.25) shows an example discretization method in FDM. Here $v_{x}=$ constant at present. The mass density $\\rho$ is defined at the cell (or mesh) center (see Fig. 5.4) in this example. When $v_{x} \\geq 0$, the plasma/fluid moves from left to right. In Eq. (5.25) the upwind difference method is used, though its accuracy is the first order. In the discretization method in Eq. (5.25), the new value of $\\rho^{n+1}$ is straightway and explicitly obtained by the algebraic equation of Eq. (5.25) from the old value of $\\rho^{n}$. This kind of difference method is called the explicit method.

In Fig. 5.5 just the density boundary line is shown, and the total mesh number is 100 in $x$ and also 100 in $y$. The spatial mesh size is $\\mathrm{d} x=\\mathrm{d} y=0.01$, and the time step of $\\Delta t$ was set to $\\mathrm{d} x / v_{x} \\times 0.02$. The factor 0.02 means that a fluid piece moves one mesh size $\\mathrm{d} x$ in $50(=1 / 0.02)$ time steps of $\\Delta t$. The numerical factor of 0.02 is called the CFL (Courant-Friedrichs-Lewy) number or the Courant number [7, 8]. The Courant number is usually set to a small number less than 1 for the explicit method. The "CFL number $=1$ " means that in one $\\Delta t$ step a piece of fluid moves one spatial mesh.

The initial sinusoidal perturbation is applied at the surface of two stratified fluids. $\\bar{\\rho}=1.0$ for $y<\\mathrm{ymax} / 2+\\delta \\times \\sin (2 \\times 2 \\pi x / \\mathrm{xmax})$ and $\\bar{\\rho}=2.0$ for $y \\geq \\mathrm{ymax} / 2+$ $\\delta \\times \\sin (2 \\times 2 \\pi x / \\mathrm{xmax})$. Here $\\delta$ shows the small amplitude of the perturbation at the boundary as shown in Fig. 5.5. In our case, the mesh size of $\\mathrm{d} x$ and $\\mathrm{d} y$ are very small compared with the perturbation wavelength of $x \\max / 2$. Even in this case, the density jump is set in one mesh in the $y$ direction. During the propagation of the perturbed density profile, the density jump becomes smooth.
a)

\\includegraphics[max width=\\textwidth, center]{2024_02_26_83e36543483eb7d284c1g-091(1)}
b)

\\begin{center}
\\includegraphics[max width=\\textwidth]{2024_02_26_83e36543483eb7d284c1g-091}
\\end{center}

Fig. 5.4 Example definition points for physical quantities in an example 2D (dimensional) simulation setup for fluid convection by FDM in a the spatial mesh and in $\\mathbf{b}$ the time domain

\\begin{center}
\\includegraphics[max width=\\textwidth]{2024_02_26_83e36543483eb7d284c1g-091(2)}
\\end{center}

Fig. 5.5 Fluid convection results with a constant speed $v_{x}$. The boundary surface between two stratified fluids is presented. Initially the boundary surface is perturbed sinusoidally

Fig. 5.6 Density contour lines are displayed. The sinusoidal perturbation has a long scale length in $x$. However, in $y$ the density jump is set in one mesh initially. During the fluid convection in $+x$ the density jump in $y$ becomes smoother

\\section*{$\\bar{t}=0.35$}
\\begin{center}
\\includegraphics[max width=\\textwidth]{2024_02_26_83e36543483eb7d284c1g-092}
\\end{center}

Figure 5.6 shows the density contour lines between $\\rho=1.0$ and $\\rho=2.0$ for the same results. The density jump smoothing is found over several spatial meshes. This is quite natural, and as we discussed above already, the mesh size should be smaller than the characteristic scale length. In addition, the time step $\\Delta t$ should be also smaller than the characteristic time scale.

\\subsubsection*{Numerical Instability and Time Step Control}
Here we try to compute the same result with the "CFL number $=1$ " in our example case in Eq. (5.25). At present $v_{x}>0$, and $\\left(\\rho_{i, j}^{n+1}-\\rho_{i, j}^{n}\\right) / \\Delta t=v_{x}\\left(\\rho_{i, j}^{n}-\\rho_{i-1, j}^{n}\\right) / \\mathrm{d} x$, which is rewritten as follows:


\\begin{equation*}
\\rho_{i, j}^{n+1}=\\rho_{i, j}^{n}-\\frac{\\Delta t \\times v_{x}}{d x}\\left(\\rho_{i, j}^{n}-\\rho_{i-1, j}^{n}\\right) \\tag{5.26}
\\end{equation*}


In this example in Figs. 5.5 and 5.6, $\\Delta t$ is usually set to CFL $\\times \\mathrm{d} x / v_{x}$, and CFL was 0.02. If we set $\\mathrm{CFL}=1$, Eq. (5.26) becomes as follows:


\\begin{equation*}
\\rho_{i, j}^{n+1}=\\rho_{i, j}^{n}-1 \\times\\left(\\rho_{i, j}^{n}-\\rho_{i-1, j}^{n}\\right)=\\rho_{i-1, j}^{n} \\tag{5.27}
\\end{equation*}


In this specific case, in which $v_{x}=$ constant, $\\mathrm{CFL}=1$ is ideal and gives an exact solution for the fluid convection. A fluid piece located at $i-1$ moves perfectly to the right at $i$ after $\\Delta t$. This special case show that the fluid moves along the fluid characteristic line. However, in general plasmas or fluids do not move uniformly in space and the velocity depends on time. Therefore, usually we employ a small CFL number less than 1 .

Now we try to simulate with a larger CFL number than 1: CFL $>1$. In this case we cannot get a reasonable for our example case in Sect. 5.2.2. A numerical instability appears during the computation for CFL $>$ 1. In the next Sect. 5.2.4 we see the numerical instability briefly.

\\subsubsection*{Numerical Instability and Its Analysis}
For the example case of the fluid convection with a constant velocity in Fig.5.5, the numerical instability can be understood as follows physically. If we insert $\\Delta t=$ $\\mathrm{CFL} \\times \\mathrm{d} t / v_{x}$ into Eq. (5.26), it becomes $\\rho_{i, j}^{n+1}=\\rho_{i, j}^{n}-\\mathrm{CFL} \\times\\left(\\rho_{i, j}^{n}-\\rho_{i-1, j}^{n}\\right)$. When $\\mathrm{CFL}>1.0$, the fluid located at $x<\\mathrm{d} x \\times(i-1)$ should contribute to $\\rho_{i, j}^{n+1}$. However, the information on $\\rho_{i-2, j}^{n}, \\rho_{i-3, j}^{n}, \\ldots$ is not included in the discretized equations of Eqs. (5.25) and (5.26). Therefore, physically and mathematically Eq. (5.25) does not work for CFL $>1.0$, and it induces a numerical instability.

Here we analyze the numerical instability mathematically [7, 8]. During $\\Delta t \\rho^{n+1}$ grows by a factor of $g$ from $\\rho^{n}$.


\\begin{equation*}
\\rho^{n+1}=g \\times \\rho^{n} \\tag{5.28}
\\end{equation*}


In this discussion, the condition of $|g|<1$ is required for the numerical stability. In addition, we assume that $\\rho_{\\iota}$ behaves as $\\rho_{i} \\propto \\exp (-i k \\Delta x \\times \\iota)$. Then the equation of $\\rho_{i, j}^{n+1}=\\rho_{i, j}^{n}-\\mathrm{CFL} \\times\\left(\\rho_{i, j}^{n}-\\rho_{i-1, j}^{n}\\right)$ becomes as follows:


\\begin{equation*}
g=1-\\mathrm{CFL} \\times(1-\\exp (-i k \\Delta x)) \\tag{5.29}
\\end{equation*}


The stability condition of $|g|<1$ shows that the numerical stability is realized, when CFL $<1$. This stability analysis was proposed by Von Neumann and would be applied mechanically to find the numerical stability (see, for example, Chap. 3 in Ref. [7] and Chap. 1 in Ref. [8] ). However, in many cases the direct application of the stability analysis would not be easy. Practically multiple stability conditions are employed to realize stable computations with CFL $<1$.

Here we check the stability condition for the following difference equation for the convection equation Eq. (5.30), which is different from Eq. (5.26): The time index of $n$ is replaced by $n+1$ for the convection term in Eq. (5.30).


\\begin{equation*}
\\rho_{i, j}^{n+1}=\\rho_{i, j}^{n}-\\frac{\\Delta t \\times v_{x}}{\\mathrm{~d} x}\\left(\\rho_{i, j}^{n+1}-\\rho_{i-1, j}^{n+1}\\right) \\tag{5.30}
\\end{equation*}


In Eq. (5.30) the $g$ factor becomes as follows:


\\begin{align*}
g & =1-\\mathrm{CFL} \\times(g-g \\exp (-i k \\Delta x)) \\\\
& =\\frac{1}{1+\\mathrm{CFL}(1-\\exp (-i k \\Delta x))} \\tag{5.31}
\\end{align*}


The result shows that the discretization method of Eq. (5.30) is stable for all $\\Delta t$. The discretization method in Eq. (5.30) is one of implicit methods. The implicit method is robust against a larger $\\Delta t$. However, Eq. (5.30) is not a simple algebraic equation, and it should be solved by a matrix solver or an iterative method [7-9]. On the other hand, Eq. (5.26) is easily solved and is one of the explicit methods.

\\subsubsection*{Example Simulation for Jet Injection}
Now the full 2D fluid equations of Eqs. (5.21)-(5.24) are used to simulate a plasma motion.

First a jet injection into a plasma is simulated. Figure 5.7 shows the density profiles at a) $\\bar{t}=0.15$ and b) $\\bar{t}=0.31$. In this example simulation, the computing area is $\\left(L_{x}=5\\right) \\times\\left(L_{y}=5\\right)$ in $\\bar{x}$ and $\\bar{y}$. A thin higher density $(\\bar{\\rho}=1.2)$ jet is injected into a lower density ( $\\bar{\\rho}=1.0$ ) plasma from the left to the right, and the jet speed is $\\bar{v}_{x}=50.0$. The initial temperature is $\\bar{T}=2.0 \\times 10^{4}$, and the sound speed $\\bar{C}_{s}$ is larger than $\\bar{v}_{x}:\\left|\\bar{v}_{x}\\right|<\\bar{C}_{s}$, which is called subsonic. Initially a small perturbation is imposed on a jet center position with sinusoidal functions of $\\sin \\left(10 x / L_{x}\\right)+\\sin \\left(x / L_{x}\\right)$. The thickness of the jet in $\\bar{y}$ is 0.4 .

In this example, the Kelvin-Helmholtz instability (see Ref. [10] and Chap. 11 in Ref. [11]) and vortices may appear in Fig. 5.7.

Figure 5.8 shows a density profile, in which a supersonic $\\left(\\left|\\bar{v}_{x}\\right|>\\bar{C}_{s}\\right)$ plasma bullet is injected at the left boundary. The jet speed is $\\bar{v}_{x}=200$, and larger than the sound speed. In plasma and fluid, perturbation or "information" propagates with the sound speed $C_{s}$. When a plasma or an object moves faster than $C_{s}$, a shock wave appears. The shock speed is larger than the sound speed $C_{s}$. In Fig. 5.8, the periodic boundary conditions are used at the boundaries at $x=0$ and at the right edge of $x=x_{\\max }$. Therefore, from the right boundary a rarefaction wave comes in.

In this example, the initial density is set to 1, including the plasma bullett, which is localized at the left boundary center. The size of the plasma bullet is $50 \\times \\Delta x$ in the length in $x$ and $15 \\times \\Delta y$ in the thickness in $y$. In this specific simulation the total mesh number used is 500 in each of $x$ and $y$.

Here we focus on the shock wave briefly. The shock wave is created, for example, by collisions, by which the plasma- or fluid-ordered motion is converted to the disordered thermal motion. In real plasmas and fluids the thickness of the shock wave is

\\begin{center}
\\includegraphics[max width=\\textwidth]{2024_02_26_83e36543483eb7d284c1g-095(1)}
\\end{center}

Fig. 5.7 A plasma jet is injected into a plasma. The 2D Euler fluid simulation is performed. The density profiles are presented. The Kelvin-Helmholtz instability may be induced by the injected jet

Fig. 5.8 A supersonic plasma bullet is injected into a plasma near the left boundary. The density profile is presented. A shock wave propagates. In Fig. 5.8, the periodic boundary conditions are used at the boundaries at $x=0$ and at the right edge of $x=x_{\\max }$. Therefore, from the right boundary a rarefaction wave comes in

\\begin{center}
\\includegraphics[max width=\\textwidth]{2024_02_26_83e36543483eb7d284c1g-095}
\\end{center}

$\\bar{x}$

very narrow and the order of the collision mean free path (see, for example, Chaps. 1 and 7 in Ref. [12]). For example, in air the collisional mean free path is less than $\\mu \\mathrm{m}$. Therefore, in many cases it would be quite difficult to simulate the real shock wave structure including the real shock wave thickness. The spatial mesh size of $\\Delta x$ or $\\Delta y$ or so is usually much larger than the real shock wave thickness.

In order to describe shock waves in plasma and fluid, an artificial viscosity would be required (see, for example, Chap. 5 in Ref. [7] and Chap. 12 in Ref. [8]). The artificial viscosity is numerically introduced to widen the shock thickness artificially over several spatial meshes. Fortunately the Rankine-Hugoniot relation in fluid dynamics shows that the values of the physical quantities on both sides of the shock wave front are not influenced by the shock wave thickness [7, 8, 12].

In Fig. 5.8, we used the following well-known simple artificial viscosity $q$ :

\\[
q=\\rho \\begin{cases}0 & \\text { for } \\Delta V \\geq 0  \\tag{5.32}\\\\ C_{1}(\\Delta V)^{2}+C_{2}|\\Delta V| C_{s} & \\text { for } \\Delta V<0\\end{cases}
\\]

In Fig. 5.8 $C_{1}=C_{2}=1.0$. The physical meaning of the artificial viscosity is simple. When one spatial mesh (cell) is compressed strongly by a shock wave, $\\Delta V<$ 0. In Eqs. (5.4) and (5.8), the pressure $P$ should be replaced with $P+q$ to ensure the energy conservation:


\\begin{gather*}
\\rho\\left\\{\\frac{\\partial \\mathbf{v}}{\\partial t}+(\\mathbf{v} \\cdot \\nabla) \\mathbf{v}\\right\\}=-\\nabla(P+q)  \\tag{5.33}\\\\
C_{v} \\frac{D T}{D t}+B \\frac{D \\rho}{D t}=-\\frac{P+q}{\\rho} \\nabla \\cdot \\mathbf{v}+\\frac{1}{\\rho} \\nabla \\cdot \\kappa(\\nabla T) \\tag{5.34}
\\end{gather*}


In Fig. 5.8, the boundary condition in $x$ was cyclic, and from the right boundary a rarefaction wave enters to the left. After the time in Fig. 5.8, the shock wave interacts with the rarefaction wave as shown in Fig. 5.9. The shock wave shape is deformed by the non-uniformity introduced by the rarefaction wave. The deformation would be similar to the beginning of the Richtmyer-Meshkov instability [13, 14], though here further discussions are not performed on it.

In Sect. 5.2.2 and in Sect. 5.2.5, a simple 2D Euler fluid simulation code was introduced and used. In Appendix F, a 3D Euler fluid code is introduced with a part

Fig. 5.9 After the time in Fig. 5.8, the shock wave interacts with the rarefaction wave coming from the right boundary. In the 2D Euler fluid simulation, the cyclic boundary conditions are employed in $x$ and $y$

\\section*{$\\bar{t}=0.0126$}
\\begin{center}
\\includegraphics[max width=\\textwidth]{2024_02_26_83e36543483eb7d284c1g-096}
\\end{center}

\\begin{center}
\\includegraphics[max width=\\textwidth]{2024_02_26_83e36543483eb7d284c1g-097}
\\end{center}

Fig. 5.10 Discretizations of $x$ and $t$

of its source code. One example result is also shown in Fig. F.1, in which a 3D shock wave is generated by a supersonic plasma bullet.

In 2D and 3D simulations, parallel computations may be needed to save CPU time. An introduction of parallel computation based on OpenMP [15] is also presented in Appendix E.

\\subsubsection*{Example Simulation for Diffusion: Heat Conduction}
In the energy equation of Eq. (5.8), a heat conduction term is included. Here the heat conduction is simulated.


\\begin{equation*}
\\frac{\\partial T(t, x)}{\\partial t}=\\frac{\\partial^{2} T(t, x)}{\\partial x^{2}} \\tag{5.35}
\\end{equation*}


In Eq. (5.35), the conduction coefficient $\\kappa$ is set to $\\kappa=1$. As presented in the previous sections, the time $t$ is discretized by $\\Delta t$, and the space $x$ is discretized by $\\Delta x$ (see Fig. 5.10). Equation (5.35) is discretized by the explicit central finite difference method as follows:


\\begin{equation*}
\\frac{T_{i}^{n+1}-T_{i}^{n}}{\\Delta t}=\\frac{T_{i+1}^{n}-2 T_{i}^{n}+T_{i-1}^{n}}{\\Delta x^{2}} \\tag{5.36}
\\end{equation*}


Fig. 5.11 Diffusion of $T(t, x)$ solved by Eq. (5.37). In this example case, $\\Delta t=1.0 \\times 10^{-4}$ and $\\Delta x=0.02$. In $x$ the total mesh number is 50

\\begin{center}
\\includegraphics[max width=\\textwidth]{2024_02_26_83e36543483eb7d284c1g-098}
\\end{center}

In the explicit equation of Eq. (5.36), the new value of $T_{i}^{n+1}$ is obtained easily from the old values of $T_{i+1}^{n}, T_{i}^{n}$ and $T_{i-1}^{n}$ :


\\begin{equation*}
T_{i}^{n+1}=T_{i}^{n}+\\frac{\\Delta t}{\\Delta x^{2}}\\left(T_{i+1}^{n}-2 T_{i}^{n}+T_{i-1}^{n}\\right) \\tag{5.37}
\\end{equation*}


The heat conduction is simulated by Eq. (5.37), and Fig. 5.11 shows the results. In this example case, $\\Delta t=1.0 \\times 10^{-4}$ and $\\Delta x=0.02$. In $x$ the total mesh number is 50 .

Now in Fig. 5.12 we change $\\Delta t$ to $\\Delta t=2.2 \\times 10^{-4}$, which is larger by 2.2 than that in Fig. 5.11. The other parameters have the same values, including $\\Delta x=0.02$. At $t=120 \\Delta t$, the temperature profile becomes unstable. This instability does NOT come from the physical reason but from the numerical instability.

In Fig. 5.12, the difference equation and the parameter values, except $\\Delta t$, are the same with those in Fig. 5.11.

In order find the stability condition for the explicit method for Eq. (5.37), the stability analysis, introduced in Sect. 5.2.3, is performed below:


\\begin{align*}
f_{i}^{n+1} & =f_{i}^{n}+\\frac{\\Delta t}{\\Delta x^{2}}\\left(f_{i+1}^{n}-2 f_{i}^{n}+f_{i-1}^{n}\\right) \\\\
g \\times f_{i}^{n} & =f_{i}^{n}+\\frac{\\Delta t}{\\Delta x^{2}}\\left(e^{k \\Delta x}-2+e^{-k \\Delta x}\\right) f_{i}^{n}  \\tag{5.38}\\\\
\\therefore g & =1+\\frac{2 \\Delta t}{\\Delta x^{2}}(\\cos k \\Delta x-1)
\\end{align*}


Fig. 5.12 Diffusion results of $T(t, x)$. At $t=120 \\Delta t$, the temperature profile becomes unstable. In this example, $\\Delta t=2.2 \\times 10^{-4}$, which is larger by 2.2 than that in Fig. 5.11. The other parameters have the same values, including $\\Delta x=0.02$

\\begin{center}
\\includegraphics[max width=\\textwidth]{2024_02_26_83e36543483eb7d284c1g-099}
\\end{center}

For the numerical stability $|g|<1$.


\\begin{align*}
|g| & <1 \\\\
\\therefore \\frac{\\Delta t}{(\\Delta x)^{2}} & <\\frac{1}{2} \\tag{5.39}
\\end{align*}


In Fig. 5.11, $\\Delta t=1.0 \\times 10^{-4}$ and $\\Delta x=0.02$. So $\\frac{\\Delta t}{(\\Delta x)^{2}}=1 / 4$, which satisfies the stability condition Eq. (5.39). However, in Fig. 5.12, $\\Delta t=2.2 \\times 10^{-4}$ and $\\Delta x=$ 0.02. $\\frac{\\Delta t}{(\\Delta x)^{2}}=1.1 / 2$, and the numerical stability condition of Eq. (5.39) was not satisfied. Therefore, in Fig. 5.12 small numerical noises grow and show the unphysical result.

The numerical instability comes from the discretizations in space, time and equations. It has been studied well (see, for example, Chap. 3 in Ref. [7] and Chap. 1 in Ref. [8]).

In Eq. (5.37), the spatial difference is central and the time difference is forward. When we make the time difference also central, the following implicit equation is obtained:


\\begin{equation*}
\\frac{T_{i}^{n+1}-T_{i}^{n}}{\\Delta t}=\\frac{1}{2}\\left(\\frac{T_{i+1}^{n+1}-2 T_{i}^{n+1}+T_{i-1}^{n+1}}{\\Delta x^{2}}+\\frac{T_{i+1}^{n}-2 T_{i}^{n}+T_{i-1}^{n}}{\\Delta x^{2}}\\right) \\tag{5.40}
\\end{equation*}


Fig. 5.13 Diffusion result of $T(t, x)$ by Eq. (5.40) (an implicit method). The results are different from those in Fig. 5.12, and it seems that the results here are physically reasonable. The parameter values employed are the same as those in Fig. 5.12: $\\Delta t=2.2 \\times 10^{-4}$ and $\\Delta x=0.02$

\\begin{center}
\\includegraphics[max width=\\textwidth]{2024_02_26_83e36543483eb7d284c1g-100}
\\end{center}

In Eq. (5.40), the right hand side includes the values of $T$ at $n+1$, that is, $t+\\Delta t$. Equation (5.40), that is one of implicit methods, is solved by an iterative method or by constricting and solving a matrix. The numerical results by (5.40) are shown in Fig. 5.13.

Here the numerical stability condition is obtained for Eq. (5.40).


\\begin{align*}
g & =\\frac{1-\\frac{\\Delta t}{\\Delta x^{2}}(1-\\cos k \\Delta x)}{1+\\frac{\\Delta t}{\\Delta x^{2}}(1-\\cos k \\Delta x)}  \\tag{5.41}\\\\
\\therefore|g| & \\leq 1
\\end{align*}


Equation (5.41) always satisfies the stability condition of $|g| \\leq 1$. Therefore, the implicit scheme is always stable numerically.

\\subsection*{Introduction to Plasma Simulation by the Lagrange Fluid Model}
By Eqs. (5.1), (5.4) and (5.7) with the equation of state, the ideal fluid behavior is described. So far the simulation examples are performed based on the Euler method, in which the convection terms (see Eq. 5.6) are explicitly solved: $\\frac{D}{D t}=\\frac{\\partial}{\\partial t}+(\\mathbf{v} \\cdot \\nabla)$.

In the Euler method, the numerical spatial meshes are fixed to the space, and do not move with the fluid motion. Therefore, if we have multiple layers consisted of different materials in computational area, the boundary tracking between two materials would be necessary $[16,17]$.

The numerical spatial mesh points are the measuring points to observe the plasma and fluid behavior. Here we introduce the measuring points moving with the mesh velocity of $\\mathbf{U}_{\\mathbf{M}}$ :

$$
\\frac{D g}{D t}=\\frac{\\partial g}{\\partial t}+(\\mathbf{v} \\cdot \\nabla) g=\\frac{\\partial g}{\\partial t}+\\left(\\mathbf{U}_{M} \\cdot \\nabla\\right) g+\\left\\{\\left(\\mathbf{v}-\\mathbf{U}_{M}\\right) \\cdot \\nabla\\right\\} g=\\left.\\frac{D g}{D t}\\right|_{M}+
$$

$\\left\\{\\left(\\mathbf{v}-\\mathbf{U}_{M}\\right) \\cdot \\nabla\\right\\} g$. Here $g$ shows a physical quantity, and $\\left.\\frac{D g}{D t}\\right|_{M}=\\frac{\\partial g}{\\partial t}+\\left(\\mathbf{U}_{M} \\cdot \\nabla\\right) g$.

When $\\mathbf{U}_{M}=0$, the mesh does not move and the Euler method is recovered. When $\\mathbf{U}_{M}=\\mathbf{v}$, the mesh moves together with the plasma and fluid motion, and the Lagrange method is realized [18-24]. When $\\mathbf{U}_{M}$ is an arbitrary velocity, the arbitrary Lagrangian-Eulerian (ALE) method is also possible [24-28]. In the Lagrange method the convection terms disappear from the basic equations, and the mesh points follow the fluid motion and are computed. In the Lagrange method plasma and fluid do not pass through the mesh boundary, and the mesh follows the plasma motion. Therefore, in the Lagrange method we can represent the material boundaries naturally, and the mass would be conserved well. However, the Lagrange meshes move and the mesh shape would be deformed. The large deformation of mesh shape may crash computations.

Here we briefly introduce the Lagrange method in 2D [18]. The total mass of each mesh is conserved, because the meshes move together with plasmas. Therefore, the mass conservation is used to obtain the new plasma density:


\\begin{equation*}
\\rho^{n+1}=\\rho^{n} V^{n} / V^{n+1} \\tag{5.42}
\\end{equation*}


Here $V$ shows the volume or surface of each mesh, and $n$ and $n+1$ show the indices for the discretized time step in Fig. 1.11b). The equation of motion and the energy equation are shown below:


\\begin{gather*}
\\rho \\frac{\\mathrm{d} \\mathbf{v}}{\\mathrm{d} t}=-\\nabla(P+q)  \\tag{5.43}\\\\
c_{v} \\frac{\\mathrm{d} T}{\\mathrm{~d} t}+B \\frac{\\mathrm{d} \\rho}{\\mathrm{d} t}=-\\frac{1}{\\rho}(P+q) \\nabla \\cdot \\mathbf{v} \\tag{5.44}
\\end{gather*}


In our example cases, the compressibility $B$ is set to $B=0$. For the ideal plasma, the equation of state gives $P=n k_{B} T, B=0$ and $C_{v}=3 k_{B} /(2 m)$. The artificial viscosity of $q$ would be introduced to describe shock waves.

Here a 2D Lagrange code is considered to simulate plasma behaviors in the cylindrical coordinate $(R, Z)$. In this case $V$ in Eq. (5.42) is the volume and will be soon discussed below. In the Lagrange mesh, the orthogonal curved coordinate is appropriate to describe the mesh position in the logical space of $(\\xi, \\eta)$ as shown in Fig.5.14.

In the 2D Lagrangian fluid equations, the derivatives of $\\frac{\\partial}{\\partial R}$ and $\\frac{\\partial}{\\partial Z}$ should be transformed to them in the $(\\xi, \\eta)$ space:


\\begin{align*}
\\frac{\\partial}{\\partial R} & =\\frac{\\partial \\xi}{\\partial R} \\frac{\\partial}{\\partial \\xi}+\\frac{\\partial \\eta}{\\partial R} \\frac{\\partial}{\\partial \\eta}  \\tag{5.45}\\\\
\\frac{\\partial}{\\partial Z} & =\\frac{\\partial \\xi}{\\partial Z} \\frac{\\partial}{\\partial \\xi}+\\frac{\\partial \\eta}{\\partial Z} \\frac{\\partial}{\\partial \\eta} \\tag{5.46}
\\end{align*}


Here we introduce the area $j$, that is, the Jacobian [29]:


\\begin{equation*}
j=\\frac{\\partial R}{\\partial \\xi} \\frac{\\partial Z}{\\partial \\eta}-\\frac{\\partial R}{\\partial \\eta} \\frac{\\partial Z}{\\partial \\xi} \\tag{5.47}
\\end{equation*}


Then the following relations are obtained:


\\begin{align*}
& \\frac{\\partial \\xi}{\\partial R}=\\frac{1}{j} \\frac{\\partial Z}{\\partial \\eta}  \\tag{5.48}\\\\
& \\frac{\\partial \\eta}{\\partial R}=-\\frac{1}{j} \\frac{\\partial Z}{\\partial \\xi}  \\tag{5.49}\\\\
& \\frac{\\partial \\xi}{\\partial Z}=-\\frac{1}{j} \\frac{\\partial R}{\\partial \\eta}  \\tag{5.50}\\\\
& \\frac{\\partial \\eta}{\\partial Z}=\\frac{1}{j} \\frac{\\partial R}{\\partial \\xi} \\tag{5.51}
\\end{align*}


The right sides of Eqs. (5.48)-(5.51) are easy to be computed. For example, $\\left.\\frac{\\partial R}{\\partial \\xi}\\right|_{i+1 / 2, j}$ is easily estimated by $R_{i+1, j}-R_{i, j}$ in Fig. 5.14. The mesh volume $V$ in Eq. (5.42) is obtained by $j R \\theta$, when the thickness in the $\\theta$ direction is $\\theta$ radian in the cylindrical coordinate. Hereafter in the Sect.5.3, $\\theta=1$ radian. Please remind our Lagrange code is in $2 \\mathrm{D}(R, Z)$ and in the $\\theta$ direction the phenomena is uniform.

In addition, the meshes move with fluid, and the mesh new positions are computed by the fluid velocity.

After transforming the equations to ones in $(\\xi, \\eta)$, the rest of the procedures are same with those in the Euler fluid code. The normalization and the discretization of the basic equations transformed are needed. The equation of motion is transformed as follows:


\\begin{align*}
\\frac{\\mathrm{d} v_{R}}{\\mathrm{~d} t} & =-\\frac{1}{j \\rho}\\left\\{\\frac{\\partial Z}{\\partial \\eta} \\frac{\\partial(P+q)}{\\partial \\xi}-\\frac{\\partial Z}{\\partial \\xi} \\frac{\\partial(P+q)}{\\partial \\eta}\\right\\}  \\tag{5.52}\\\\
\\frac{\\mathrm{d} v_{Z}}{\\mathrm{~d} t} & =-\\frac{1}{j \\rho}\\left\\{\\left(-\\frac{\\partial R}{\\partial \\eta}\\right) \\frac{\\partial(P+q)}{\\partial \\xi}-\\left(-\\frac{\\partial R}{\\partial \\xi}\\right) \\frac{\\partial(P+q)}{\\partial \\eta}\\right\\} \\tag{5.53}
\\end{align*}


\\begin{center}
\\includegraphics[max width=\\textwidth]{2024_02_26_83e36543483eb7d284c1g-103}
\\end{center}

Fig. 5.14 Lagrange mesh in 2D. The mesh movement is solved in the logical space $(\\xi, \\eta)$

The new mesh point coordinate $\\left(R_{M}, Z_{M}\\right)$ is obtained:


\\begin{align*}
& \\frac{\\mathrm{d} R_{M}}{\\mathrm{~d} t}=v_{R}  \\tag{5.54}\\\\
& \\frac{\\mathrm{d} Z_{M}}{\\mathrm{~d} t}=v_{Z} \\tag{5.55}
\\end{align*}


The new density is obtained by Eq. (5.42). The energy equation is transformed as follows:


\\begin{equation*}
C_{v} \\frac{\\mathrm{d} T}{\\mathrm{~d} t}=-\\frac{1}{\\rho}(P+q) \\nabla \\cdot \\mathbf{v}=-\\frac{(P+q)}{\\rho V} \\frac{D V}{D t} \\simeq-(P+q) \\frac{1}{\\rho V} \\frac{V^{n+1}-V^{n}}{\\Delta t} \\tag{5.56}
\\end{equation*}


Here we neglected the heat conduction and used the relation of $\\nabla \\cdot \\mathbf{v}=-\\frac{1}{\\rho} \\frac{D \\rho}{D t}=$ $\\frac{1}{V} \\frac{D V}{D t}$ by the equation of continuity and by Eq. (5.42).

In our example Lagrange code, the simple artificial viscosity is introduced to describe shock waves:

\\[
q=C 1 \\rho \\begin{cases}0 & \\text { for } \\Delta \\mathbf{v} \\geq 0  \\tag{5.57}\\\\ (\\Delta \\mathbf{v})^{2} & \\text { for } \\Delta \\mathbf{v}<0\\end{cases}
\\]

Here $C 1$ is a constant and $\\Delta \\mathbf{v}=\\frac{\\partial \\mathbf{v}}{\\partial \\xi}+\\frac{\\partial \\mathbf{v}}{\\partial \\eta}=\\left.\\Delta \\mathbf{v}\\right|_{\\xi}+\\left.\\Delta \\mathbf{v}\\right|_{\\eta}$. The prescription is one simple method for the artificial viscosity [18]:


\\begin{align*}
\\left.\\Delta \\mathbf{v}\\right|_{\\xi} & =\\left(\\frac{\\partial Z}{\\partial \\eta},-\\frac{\\partial R}{\\partial \\eta}\\right) \\cdot\\left(\\frac{\\partial v_{R}}{\\partial \\xi}, \\frac{\\partial v_{Z}}{\\partial \\xi}\\right) /\\left|\\left(\\frac{\\partial Z}{\\partial \\eta},-\\frac{\\partial R}{\\partial \\eta}\\right)\\right| \\\\
\\left.\\Delta \\mathbf{v}\\right|_{\\eta} & =-\\left(\\frac{\\partial Z}{\\partial \\xi},-\\frac{\\partial R}{\\partial \\xi}\\right) \\cdot\\left(\\frac{\\partial v_{R}}{\\partial \\eta}, \\frac{\\partial v_{Z}}{\\partial \\eta}\\right) /\\left|\\left(\\frac{\\partial Z}{\\partial \\xi},-\\frac{\\partial R}{\\partial \\xi}\\right)\\right| \\tag{5.58}
\\end{align*}


Now we apply the simple Lagrange code to simulate a spherical shell plasma motion. Initially the spherical shell outer radius is $3 \\mathrm{~mm}$, and the inner radius is $2.5 \\mathrm{~mm}$. The plasma is the hydrogen plasma, and its initial density is uniform. The temperature has a distribution. The inner quarter of the spherical shell is in a low temperature of $0.01 \\mathrm{eV}$. From the outside of the cold part, the temperature rises linearly from 100 to $300 \\mathrm{eV}$ at the most outside of the shell. Figure $5.15 \\mathrm{a}$ shows an example simulation result, in which the density is shown, and at $12.26 \\mathrm{~ns}$ the density reaches to about 300 times the initial density near the shell center. Initially the Lagrange spatial meshes are distributed uniformly. The total mesh number is 100 in the radius direction and in the lateral direction, respectively. Figure $5.15 \\mathrm{~b}$ shows a part of the Lagrange spatial mesh positions, in which near the shell center the distance especially in the radial direction (the $\\xi$ direction) expands compared with that in the outer region. The spatial resolution changes along with the plasma movement.
\\includegraphics[max width=\\textwidth, center]{2024_02_26_83e36543483eb7d284c1g-104}

Fig. 5.15 An example simulation result for a spherical shell implosion by the 2D Lagrangian code. a The density is presented, and at $12.26 \\mathrm{~ns}$ the density is compressed to about 300 times the initial density in this example. $\\mathbf{b}$ The Lagrange mesh points at $12.26 \\mathrm{~ns}$. Near the shell center the distance between two meshes, especially in the radial direction (the $\\xi$ direction), expands compared with that in the outer region

\\includegraphics[max width=\\textwidth, center]{2024_02_26_83e36543483eb7d284c1g-105(1)}
b)

\\begin{center}
\\includegraphics[max width=\\textwidth]{2024_02_26_83e36543483eb7d284c1g-105}
\\end{center}

Fig. 5.16 A spherical shell non-uniform implosion by the 2D Lagrangian code. In the lateral direction we impose a sinusoidal temperature non-uniformity initially: $0.03 \\times \\sin \\left(2 \\pi \\cdot 8 \\eta / \\eta_{\\max }\\right)$. The amplitude of the non-uniformity is $3 \\%$. a The density is presented at $10.29 \\mathrm{~ns}$ the density is compressed to about 2 times the initial density in this example. b The Lagrange mesh points at $10.29 \\mathrm{~ns}$. Near the shell center the mesh deformation becomes serious

The next example simulation is another similar spherical shell implosion with the initial temperature non-uniformity. In the lateral direction we impose a sinusoidal temperature non-uniformity: $0.03 \\times \\sin \\left(2 \\pi \\cdot 8 \\eta / \\eta_{\\max }\\right)$. The amplitude of the nonuniformity is $3 \\%$. The simulation box in the lateral $\\eta$ direction is $\\pi / 2$.

Figure 5.16a shows the density profile near the shell center at $10.29 \\mathrm{~ns}$. The initial non-uniformity is enhanced during the shell implosion. Near the spherical shell center the Lagrange mesh deformation becomes serious as shown in Fig. 5.16b. Some time after, the computation is crashed.

\\subsection*{Electron Plasma Wave}
Now we consider waves in plasmas. In Chap. 1, one of plasma collective behaviors, plasma oscillation was introduced. Here we examine the electron plasma wave.

We set the following assumptions for simplicity:

\\begin{enumerate}
  \\item Ions are immobile. Ions are heavy compared with electrons. Ions are distributed uniformly with the number density of $n_{0}$.

  \\item One-dimensional (1D) phenomenon, which depends on $x$.

  \\item No magnetic field $\\mathbf{B}=0$, and electrostatic phenomenon.

  \\item Adiabatic phenomenon. When the wavelength is longer than the electron traveling distance, the phenomenon would be adiabatic.

  \\item The averaged electron velocity is zero: $v_{0}=0$.

\\end{enumerate}

Then the basic equations are as follows:


\\begin{align*}
\\frac{\\partial n_{e}}{\\partial t}+\\frac{\\partial n_{e} v_{\\mathrm{ex}}}{\\partial x} & =0  \\tag{5.59}\\\\
m_{e} n_{e}\\left(\\frac{\\partial v_{\\mathrm{ex}}}{\\partial t}+v_{\\mathrm{ex}} \\frac{\\partial v_{\\mathrm{ex}}}{\\partial x}\\right) & =-\\frac{\\partial P_{e}}{\\partial x}-e n_{e} E  \\tag{5.60}\\\\
\\frac{\\partial E}{\\partial x} & =\\frac{1}{\\varepsilon_{0}} e\\left(n_{0}-n_{e}\\right) \\tag{5.61}
\\end{align*}


Here $m_{e}$ is the electron mass, $n_{e}$ the electron number density, $v_{e}$ the electron velocity and $P_{e}$ the electron pressure. We assume that each physical quantity is a combination of the zeroth-order quantity and the smaller first-order one as follows:

\\[
\\left\\{\\begin{array}{l}
n_{e}=n_{0}+n_{1}  \\tag{5.62}\\\\
v_{\\mathrm{ex}}=0+v_{1} \\\\
E=0+E_{1}
\\end{array}\\right.
\\]

The first-order quantity shows the perturbation or the wave with a small amplitude. For example, the term of $n_{1} \\times v_{1}$ shows a multiplication of two small quantities, and is called the second-order term, which is ignored in the following analysis. This mathematical procedure is called linearization. So we retain terms, containing just one of $n_{1}, v_{1}$ and $E_{1}$ in each term.

Then we obtain the following linearized equations:

\\[
\\left\\{\\begin{array}{l}
\\frac{\\partial n_{1}}{\\partial t}+n_{0} \\frac{\\partial v_{1}}{\\partial x}=0  \\tag{5.63}\\\\
m_{e} n_{0} \\frac{\\partial v_{1}}{\\partial t}=-e n_{0} E_{1}-\\gamma T_{0} \\frac{\\partial n_{1}}{\\partial x} \\\\
\\frac{\\partial E_{1}}{\\partial x}=-\\frac{1}{\\varepsilon_{0}} e n_{1}
\\end{array}\\right.
\\]

We also assume that the first-order quantity depends on $\\exp (-i \\omega t+i k x)$ :


\\begin{equation*}
n_{1}, v_{1}, E_{1} \\propto \\exp (-i \\omega t+i k x) \\tag{5.64}
\\end{equation*}


So we pick up and consider one Fourier component, because Eq. (5.63) are linearized.

The amplitude of each component is denoted by the top bar symbol: for example, $\\bar{n}_{1}$, etc. Equations (5.63) becomes as follows:

\\[
\\begin{array}{r}
-i \\omega \\bar{n}_{1}+i k n_{0} \\bar{v}_{1}=0 \\\\
-i \\omega m_{e} n_{0} \\bar{v}_{1}=-e n_{0} \\bar{E}_{1}-i k \\gamma T_{0} \\bar{n}_{1} \\\\
i k \\bar{E}_{1}=-\\frac{1}{\\varepsilon_{0}} e \\bar{n}_{1} \\tag{5.67}
\\end{array}
\\]

From Eq. (5.65), the following is obtained:


\\begin{equation*}
\\bar{n}_{1}=\\frac{k n_{0}}{\\omega} \\bar{v}_{1} \\tag{5.68}
\\end{equation*}


Substituting Eq. (5.68) into Eq. (5.67), $\\bar{v}_{1}$ is obtained:


\\begin{equation*}
\\bar{v}_{1}=-\\frac{\\epsilon_{0}}{e n_{0}}(i \\omega) \\bar{E}_{1} \\tag{5.69}
\\end{equation*}


Then, substituting Eqs. (5.69) and (5.67) into Eq. (5.66), we obtained the following relation:


\\begin{equation*}
\\left(1-\\frac{\\omega_{\\mathrm{pe}}^{2}}{\\omega^{2}}-\\frac{\\gamma k^{2} T_{0}}{m_{e} \\omega^{2}}\\right) \\bar{E}_{1} \\equiv \\varepsilon(k, \\omega) \\bar{E}_{1}=0 \\tag{5.70}
\\end{equation*}


Here $\\omega_{\\mathrm{pe}}$ is the electron plasma frequency: $\\sqrt{n_{0} e^{2} /\\left(m_{e} \\epsilon_{0}\\right)}$.

Now the solutions of Eq. (5.70) are $\\bar{E}_{1}=0$ and $/$ or $\\epsilon(k, \\omega)=0$.


\\begin{equation*}
\\epsilon(k, \\omega)=0 \\Rightarrow \\omega^{2}-\\omega_{\\mathrm{pe}}^{2}-\\gamma k^{2} T_{0} / m_{e}=0 \\tag{5.71}
\\end{equation*}


The solution of $\\bar{E}_{1}=0$ shows that no electric field (no wave or no perturbation) is there, and is not interesting for us. The other solution of Eq. (5.71) would show the plasma properties. Under this solution, $\\bar{E}_{1} \\neq 0$. It means that $\\epsilon(k, \\omega)=0$ contains the information of waves generated in the plasma. In the relation of $\\bar{E}_{1} \\neq 0, \\omega$ and $k$ would depend on each other. Therefore, it is called the dispersion relation, ${ }^{2}$ The solution of (5.71) is as follows:


\\begin{equation*}
\\omega^{2}=\\omega_{\\mathrm{pe}}^{2}+\\gamma k^{2} T_{0} / m_{e} \\tag{5.72}
\\end{equation*}

\\footnotetext{${ }^{2}$ When the wavenumber $k$ and the frequency $\\omega$ are given, the phase velocity is $\\omega / k$ and the group velocity is $\\mathrm{d} \\omega / \\mathrm{d} k$. If $\\omega$ depends on $k$ linearly, every wave propagates with the same speed. However, in general waves propagate with the different speeds, depending on $k$ and $\\omega$. It means that after some time, a wave packet, consisting of waves, would spread. Therefore, the relations of (5.71) and (5.72) are frequently called the dispersion relation.
}

When $T_{0}=0$, that is, a cold plasma, $\\omega=\\omega_{\\text {pe }}$, which is the plasma frequency shown above and in Sect. 1.4. When $T_{0} \\neq 0$ and the following conditions are fulfilled,


\\begin{equation*}
\\frac{\\omega}{k} \\sim \\frac{\\omega_{\\mathrm{pe}}}{k} \\gg \\sqrt{\\frac{T_{0}}{m_{e}}} \\quad \\therefore k \\lambda_{e} \\ll 1 \\tag{5.73}
\\end{equation*}


the following solution Eq. (5.74) is obtained:


\\begin{align*}
& \\omega= \\pm \\sqrt{\\omega_{\\mathrm{pe}}^{2}+\\gamma k^{2} \\frac{T_{0}}{m_{e}}}  \\tag{5.74}\\\\
& \\simeq \\pm \\omega_{\\mathrm{pe}}\\left(1+\\frac{\\gamma}{2} k^{2} \\lambda_{e}^{2}\\right), \\quad \\lambda_{e}^{2} \\equiv \\frac{T \\epsilon_{0}}{n_{0} e^{2}}
\\end{align*}


The adiabatic condition of Eq. (5.73), that is, a long wavelength limit $\\lambda=2 \\pi / k \\gg$ $\\lambda_{e}$, is employed here. The adiabatic condition was assumed in this Section, and under the condition the electron movement is slow compared with the wave propagation speed. Figure 5.17 shows the dispersion relation for the electron plasma wave, which is also called the Langmuir wave.

Fig. 5.17 Dispersion relation for the electron plasma wave (the Langmuir wave)

\\begin{center}
\\includegraphics[max width=\\textwidth]{2024_02_26_83e36543483eb7d284c1g-108}
\\end{center}

\\subsection*{Ion Acoustic Wave}
In the last Sect. 5.4, the electron plasma wave was introduced. The electron plasma frequency $\\omega \\sim \\omega_{\\mathrm{pe}}$ is relatively high. Here we focus on a lower frequency mode for ion plasma wave.

We set the following assumptions for simplicity:

\\begin{enumerate}
  \\item One-dimensional (1D) phenomenon, which depends just on $x$.

  \\item No magnetic field $\\mathbf{B}=0$, and electrostatic phenomenon.

  \\item Adiabatic phenomenon.

  \\item Because of $m_{e} \\ll m_{i}$, the electron mass is ignored: $m_{e}=0$.

\\end{enumerate}

Here $m_{i}$ shows the ion mass, $n_{i}$ the ion number density, $v_{i}$ the ion velocity and $T_{i}$ the ion temperature. The ion charge is set to be 1 here. The basic equations are as follows:

\\[
\\left\\{\\begin{array}{l}
\\frac{\\partial n_{e}}{\\partial t}+\\frac{\\partial\\left(n_{e} v_{\\mathrm{ex}}\\right)}{\\partial x}=0  \\tag{5.75}\\\\
0=-\\gamma T_{e} \\frac{\\partial n_{e}}{\\partial x}-e n_{e} E \\\\
\\frac{\\partial n_{i}}{\\partial t}+\\frac{\\partial n_{i} v_{i x}}{\\partial x}=0 \\\\
m_{i} n_{i}\\left(\\frac{\\partial v_{i x}}{\\partial t}+v_{i x} \\frac{\\partial v_{i x}}{\\partial x}\\right)=-\\gamma T_{i} \\frac{\\partial n_{i}}{\\partial x}+e n_{i} E \\\\
\\frac{\\partial E}{\\partial x}=\\frac{1}{\\epsilon_{0}} e\\left(n_{i}-n_{e}\\right)
\\end{array}\\right.
\\]

The linearization gives the followings:


\\begin{align*}
& \\frac{\\partial n_{e 1}}{\\partial t}+n_{0} \\frac{\\partial v_{e 1}}{\\partial x}=0  \\tag{5.76}\\\\
& 0=-\\gamma T_{e} \\frac{\\partial n_{e 1}}{\\partial x}-e n_{0} E_{1}  \\tag{5.77}\\\\
& \\frac{\\partial n_{i 1}}{\\partial t}+n_{0} \\frac{\\partial v_{i 1}}{\\partial x}=0  \\tag{5.78}\\\\
& m_{i} n_{0} \\frac{\\partial v_{i 1}}{\\partial t}=-\\gamma T_{i} \\frac{\\partial n_{i 1}}{\\partial x}+e n_{0} E_{1}  \\tag{5.79}\\\\
& \\frac{\\partial E_{1}}{\\partial x}=\\frac{1}{\\epsilon_{0}} e\\left(n_{i 1}-n_{e 1}\\right) \\tag{5.80}
\\end{align*}


As usual, one Fourier component is extracted: $n_{e 1}, v_{e 1}, n_{i 1}, v_{i 1}, E_{1} \\propto \\exp (-i \\omega t+$ $i k x$ ). We assumed $m_{e}=0$, and Eq. (5.76) is not needed. After inserting the Fourier component into the basic equations and manipulating the algebraic equations, we obtain the following:


\\begin{equation*}
\\epsilon(k, \\omega) \\bar{E}_{1}=\\left(1-\\frac{\\omega_{p i}^{2}}{\\omega^{2}-k^{2} \\gamma T_{i} / m_{i}}+\\frac{\\omega_{\\mathrm{pe}}^{2}}{k^{2} \\gamma T_{e} / m_{e}}\\right) \\bar{E}_{1}=0 \\tag{5.81}
\\end{equation*}


Here $\\omega_{p i}$ is the ion plasma frequency. The dispersion relation for the ion wave is $\\epsilon(k, \\omega)=0$, which gives the following:


\\begin{equation*}
\\omega^{2}=\\frac{k^{2} \\gamma T_{i}}{m_{i}}+\\frac{k^{2} \\gamma T_{e}}{m_{i}\\left(1+\\gamma k^{2} \\lambda_{e}^{2}\\right)} \\tag{5.82}
\\end{equation*}


Here $\\lambda_{e}$ shows the electron Debye length (see Eq. (1.5)). For the long wavelength of $k^{2} \\lambda_{e}^{2} \\ll 1$, in which the electron shielding takes place well, the following relation is obtained:


\\begin{equation*}
\\omega^{2} \\simeq k^{2}\\left\\{\\frac{\\gamma\\left(T_{i}+T_{e}\\right)}{m_{i}}\\right\\} \\equiv k^{2} c_{s}^{2} \\tag{5.83}
\\end{equation*}


Equation (5.83) shows the ion acoustic wave. Figure 5.18 shows the dispersion relation for the ion wave at $T_{i}=0$. When $T_{i}=0$ and $k^{2} \\lambda_{e}^{2} \\gg 1, \\omega^{2} \\simeq \\omega_{p i}^{2}=\\frac{n q_{i}^{2}}{m_{i} \\varepsilon_{0}}$, which shows the ion plasma oscillation. In the Sect. $5.5 q_{i}=1$.

\\begin{center}
\\includegraphics[max width=\\textwidth]{2024_02_26_83e36543483eb7d284c1g-110}
\\end{center}

Fig. 5.18 Dispersion relation for the ion plasma wave with the electron plasma wave

\\subsection*{Electromagnetic Wave}
In the preceding Sections, static waves in plasma are considered. Here electromagnetic wave is focused. Electrostatic waves and sound wave are sometimes called longitudinal waves, in which the medium oscillation direction is parallel to the wave direction. The electromagnetic waves would be called as transverse waves, in which the electric and magnetic fields oscillate transversely to the wave propagation direction. The longitudinal wave cannot propagate in a space without the medium or in vacuum. However, the transverse wave can propagate in the vacuum. Therefore, the electromagnetic waves take energy away from plasmas.

We introduce the following assumptions for simplicity:

\\begin{enumerate}
  \\item Ions are immobile.

  \\item One-dimensional $(x)$ phenomenon.

  \\item No averaged motion.

  \\item Zero temperature, and so the pressure is also zero: $P=0$.

\\end{enumerate}

The basic equations are as follows:


\\begin{align*}
m_{e}\\left\\{\\frac{\\partial \\mathbf{v}_{e}}{\\partial t}+\\left(\\mathbf{v}_{e} \\cdot \\nabla\\right) \\mathbf{v}_{e}\\right\\} & =-e\\left(\\mathbf{E}+\\mathbf{v}_{e} \\times \\mathbf{B}\\right)  \\tag{5.84}\\\\
\\nabla \\times \\mathbf{E} & =-\\frac{\\partial \\mathbf{B}}{\\partial t}  \\tag{5.85}\\\\
\\nabla \\times \\mathbf{H} & =\\mathbf{J}+\\frac{\\partial \\mathbf{D}}{\\partial t} \\tag{5.86}
\\end{align*}


The linearization gives the followings:


\\begin{align*}
m_{e} \\frac{\\partial \\mathbf{v}_{1}}{\\partial t} & =-e \\mathbf{E}_{1}  \\tag{5.87}\\\\
\\nabla \\times \\mathbf{E}_{1} & =-\\frac{\\partial \\mu_{0} \\mathbf{H}_{1}}{\\partial t}  \\tag{5.88}\\\\
\\nabla \\times \\mathbf{H}_{1} & =-e n_{0} \\mathbf{v}_{1}+\\frac{\\partial \\epsilon_{0} \\mathbf{E}_{1}}{\\partial t} \\tag{5.89}
\\end{align*}


The time derivative of Eq. (5.89) gives the following:


\\begin{equation*}
\\nabla \\times \\frac{\\partial \\mathbf{H}_{1}}{\\partial t}=-e n_{0} \\frac{\\partial \\mathbf{v}_{1}}{\\partial t}+\\frac{\\partial^{2} \\epsilon_{0} \\mathbf{E}_{1}}{\\partial t^{2}} \\tag{5.90}
\\end{equation*}


Substituting Eqs. (5.87) and (5.88) into Eq. (5.90), the following is obtained:


\\begin{equation*}
-\\frac{1}{\\mu_{0}} \\nabla \\times \\nabla \\times \\mathbf{E}_{1}=\\frac{n_{0} e^{2}}{m_{e}} \\mathbf{E}_{1}+\\epsilon_{0} \\frac{\\partial^{2} \\mathbf{E}_{1}}{\\partial t^{2}} \\tag{5.91}
\\end{equation*}


Fig. 5.19 Dispersion relation for electromagnetic wave in plasma

\\begin{center}
\\includegraphics[max width=\\textwidth]{2024_02_26_83e36543483eb7d284c1g-112}
\\end{center}

Here again we focus on one Fourier component. Equation (5.91) becomes as follows:


\\begin{align*}
& \\frac{-1}{\\mu_{0} \\epsilon_{0}} k^{2} \\mathbf{E}_{1}=\\omega_{\\mathrm{pe}}^{2} \\mathbf{E}_{1}-\\omega^{2} \\mathbf{E}_{1} \\\\
& \\therefore\\left(\\omega^{2}-\\omega_{\\mathrm{pe}}^{2}-\\frac{k^{2}}{\\mu_{0} \\epsilon_{0}}\\right) \\mathbf{E}_{1}=0 \\tag{5.92}
\\end{align*}


Here $\\omega_{\\mathrm{pe}}$ is the electron plasma frequency: $\\sqrt{n_{0} e^{2} /\\left(m_{e} \\varepsilon_{0}\\right)}$. By using the relation of $1 / \\mu_{0} \\varepsilon_{0}=c^{2}$, the dispersion relation for the electromagnetic wave is as follows:


\\begin{equation*}
\\omega^{2}=\\omega_{\\mathrm{pe}}^{2}+k^{2} c^{2} \\tag{5.93}
\\end{equation*}


When no plasma exists, $\\omega_{e}^{2}=0$ and $\\omega^{2}=k^{2} c^{2}$, which indicates the electromagnetic wave propagation in vacuum. For the long wavelength limit of $k \\rightarrow 0, \\omega^{2} \\simeq \\omega_{\\mathrm{pe}}^{2}$. Figure 5.19.

Now, from Eq. (5.93), let us obtain the wave phase speed of $\\omega / k$ :


\\begin{equation*}
\\frac{\\omega^{2}}{k^{2}}=\\frac{\\omega_{\\mathrm{pe}}^{2}}{k^{2}}+c^{2}=\\frac{c^{2}}{1-\\omega_{\\mathrm{pe}}^{2} / \\omega^{2}} \\tag{5.94}
\\end{equation*}


Equation (5.94) shows that in plasmas electromagnetic wave propagates with a speed larger than the speed of light in vacuum. When $\\omega_{\\mathrm{pe}}^{2}>\\omega^{2}$ in the third term of Eq. (5.94), the following relation is found:


\\begin{equation*}
\\frac{\\omega^{2}}{k^{2}}<0 \\tag{5.95}
\\end{equation*}


If $\\omega$ is real and the wavenumber $k$ is imaginary, $k=i K$. The electromagnetic wave propagates in the form of $\\exp (-i \\omega t+i k x)$. For $k=i K$, the following is obtained:


\\begin{equation*}
\\left|E_{1}\\right|,\\left|B_{1}\\right| \\propto \\exp (-K x) \\text {, when } \\omega_{\\mathrm{pe}}>\\omega \\text {. } \\tag{5.96}
\\end{equation*}


For areas of $\\omega_{\\mathrm{pe}}>\\omega$ in plasma, electromagnetic wave does not propagate and is reflected from the plasma. The frequency $\\omega$ at the plasma frequency of $\\omega_{\\mathrm{pe}}$ is sometimes called as the cutoff frequency.

In Figs. 1.6b and 4.7, a laser interacts with the high-density plasma $\\left(\\omega_{\\mathrm{pe}}>\\omega\\right)$ and is reflected by the dense plasma surface. The physics for the electromagnetic wave (laser) reflection comes from the physics shown above. The laser field interacts with plasma electrons, and the electrons absorb the laser energy. The oscillating electrons emit radiation. The electron density is so high to reflect the incoming laser field, when $\\omega_{\\mathrm{pe}}>\\omega$. The high-density plasma under the condition of $\\omega_{\\mathrm{pe}}>\\omega$ is called the overdense plasma.

\\subsection*{Magnetohydrodynamic Equation}
In the preceding Sects.5.4-5.6, ions and electrons are treated as different two fluids. In the Sects. 5.1-5.3 simple basic fluid equations are introduced and employed for neutral plasmas. In the Sect. 5.7 the magnetohydrodynamic (MHD) equation is introduced, and plasmas are also treated as one fluid.

First, continuity equations for ions and electrons are shown again below:


\\begin{align*}
& \\frac{\\partial n_{e}}{\\partial t}+\\nabla \\cdot\\left(n_{e} \\mathbf{v}_{e}\\right)=0  \\tag{5.97}\\\\
& \\frac{\\partial n_{i}}{\\partial t}+\\nabla \\cdot\\left(n_{i} \\mathbf{v}_{i}\\right)=0 \\tag{5.98}
\\end{align*}


The simple operation of ( $m_{e} \\times$ Eq. (5.97) $+m_{i} \\times$ Eq. (5.98) ) provides the following:


\\begin{equation*}
\\frac{\\partial}{\\partial t}\\left(m_{e} n_{e}+m_{i} n_{i}\\right)+\\nabla \\cdot\\left(m_{e} n_{e} \\mathbf{v}_{e}+m_{i} n_{i} \\mathbf{v}_{i}\\right)=0 \\tag{5.99}
\\end{equation*}


Here we introduce the followings:


\\begin{align*}
\\rho_{m} & \\equiv m_{e} n_{e}+m_{i} n_{i}  \\tag{5.100}\\\\
\\rho_{m} \\mathbf{v} & \\equiv m_{e} n_{e} \\mathbf{v}_{e}+m_{i} n_{i} \\mathbf{v}_{i} \\tag{5.101}
\\end{align*}


Then we obtain the equation for $\\rho_{m}$ :


\\begin{equation*}
\\frac{\\partial \\rho_{m}}{\\partial t}+\\nabla \\cdot\\left(\\rho_{m} \\mathbf{v}\\right)=0 \\tag{5.102}
\\end{equation*}


The continuity equation for the electric charge is obtained. It is obtained by $q_{i} \\times$ Eq. (5.98) $-e \\times$ Eq. (5.97):


\\begin{equation*}
\\frac{\\partial}{\\partial t}\\left(q_{i} n_{i}-e n_{e}\\right)+\\nabla \\cdot\\left(n_{i} q_{i} \\mathbf{v}_{i}-n_{e} e \\mathbf{v}_{e}\\right)=0 \\tag{5.103}
\\end{equation*}


The charge density $\\rho_{c}$ and the current density $\\mathbf{J}$ are as follows:


\\begin{align*}
\\rho_{c} & \\equiv n_{i} q_{i}-n_{e} e  \\tag{5.104}\\\\
\\mathbf{J} & \\equiv n_{i} q_{i} \\mathbf{v}_{i}-n_{e} e \\mathbf{v}_{e} \\tag{5.105}
\\end{align*}


Then the continuity equation for electric charge is obtained:


\\begin{equation*}
\\frac{\\partial \\rho_{c}}{\\partial t}+\\nabla \\cdot \\mathbf{J}=0 \\tag{5.106}
\\end{equation*}


Next, the equation of motion is considered. The equations of motion for ions and electrons are as follows:


\\begin{align*}
m_{i} n_{i}\\left\\{\\frac{\\partial \\mathbf{v}_{i}}{\\partial t}+\\left(\\mathbf{v}_{i} \\cdot \\nabla\\right) \\mathbf{v}_{i}\\right\\} & =-\\nabla P_{i}+n_{i} q_{i}\\left(\\mathbf{E}+\\mathbf{v}_{i} \\times \\mathbf{B}\\right)+\\mathbf{F}_{i e}  \\tag{5.107}\\\\
m_{e} n_{e}\\left\\{\\frac{\\partial \\mathbf{v}_{e}}{\\partial t}+\\left(\\mathbf{v}_{e} \\cdot \\nabla\\right) \\mathbf{v}_{e}\\right\\} & =-\\nabla P_{e}-n_{e} e\\left(\\mathbf{E}+\\mathbf{v}_{e} \\times \\mathbf{B}\\right)-\\mathbf{F}_{i e} \\tag{5.108}
\\end{align*}


Here we introduce the interaction force between ions and electrons: $\\mathbf{F}_{i e}$. The operation of Eq. (5.107) + Eq. (5.108) gives the following:


\\begin{align*}
m_{i} n_{i}\\left\\{\\frac{\\partial \\mathbf{v}_{i}}{\\partial t}+\\left(\\mathbf{v}_{i} \\cdot \\nabla\\right) \\mathbf{v}_{i}\\right\\} & +m_{e} n_{e}\\left\\{\\frac{\\partial \\mathbf{v}_{e}}{\\partial t}+\\left(\\mathbf{v}_{e} \\cdot \\nabla\\right) \\mathbf{v}_{e}\\right\\}  \\tag{5.109}\\\\
& =-\\nabla P+\\rho_{c} \\mathbf{E}+\\mathbf{J} \\times \\mathbf{B} .
\\end{align*}


Here $P=P_{i}+P_{e}$.

(Left Hand Side of Eq. (5.109)) + Eq. $5.97 \\times m_{e} \\mathbf{v}_{e}+$ Eq. $5.98 \\times m_{i} \\mathbf{v}_{i}$


\\begin{align*}
& =\\frac{\\partial \\rho_{m} \\mathbf{v}}{\\partial t}+\\nabla \\cdot\\left(m_{e} n_{e} \\mathbf{v}_{e} \\mathbf{v}_{e}+m_{i} n_{i} \\mathbf{v}_{i} \\mathbf{v}_{i}\\right)  \\tag{5.110}\\\\
& \\simeq \\frac{\\partial \\rho_{m} \\mathbf{v}}{\\partial t}+\\nabla \\cdot\\left(\\rho_{m} \\mathbf{v} \\mathbf{v}\\right)
\\end{align*}


Here we assumed that the relative velocity between ions and electrons is not large, and the following approximation is employed:


\\begin{equation*}
m_{e} n_{e} \\mathbf{v}_{e} \\mathbf{v}_{e}+m_{i} n_{i} \\mathbf{v}_{i} \\mathbf{v}_{i} \\simeq\\left(m_{e} n_{e} \\mathbf{v}_{e}+m_{i} n_{i} \\mathbf{v}_{i}\\right) \\mathbf{v}=\\rho_{m} \\mathbf{v} \\mathbf{v} \\tag{5.111}
\\end{equation*}


The operation of Eq. (5.110)-v.Eq. (5.102) gives the following equation of motion:


\\begin{equation*}
\\rho_{m}\\left[\\frac{\\partial \\mathbf{v}}{\\partial t}+(\\mathbf{v} \\cdot \\nabla) \\mathbf{v}\\right]=-\\nabla P+\\rho_{c} \\mathbf{E}+\\mathbf{J} \\times \\mathbf{B} \\tag{5.112}
\\end{equation*}


The dependent variables are $\\rho_{m}, \\rho_{c}, \\mathbf{v}, \\mathbf{E}, \\mathbf{B}, P$ and $\\mathbf{J}$. The variables of $\\rho_{m}, \\rho_{c}$ and $\\mathbf{v}$ are solved by the equations above. The electric $\\mathbf{E}$ and magnetic $\\mathbf{B}$ fields are obtained by the Maxwell Equations. The pressure $P$ is obtained by the equation of state or the energy equation. The current density $\\mathbf{J}$ is obtained by $q_{i} / m_{i} \\times$ Eq. (5.107) $-e / m_{e} \\times$ Eq. (5.108):


\\begin{align*}
& q_{i} n_{i}\\left\\{\\frac{\\partial \\mathbf{v}_{i}}{\\partial t}+\\left(\\mathbf{v}_{i} \\cdot \\nabla\\right) \\mathbf{v}_{i}\\right\\}-e n_{e}\\left\\{\\frac{\\partial \\mathbf{v}_{e}}{\\partial t}+\\left(\\mathbf{v}_{e} \\cdot \\nabla\\right) \\mathbf{v}_{e}\\right\\} \\\\
= & -\\frac{q_{i}}{m_{i}} \\nabla P_{i}+\\frac{e}{m_{e}} \\nabla P_{e}+\\frac{n_{i} q_{i}^{2}}{m_{i}}\\left(\\mathbf{E}+\\mathbf{v}_{i} \\times \\mathbf{B}\\right)+\\frac{n_{e} e^{2}}{m_{e}}\\left(\\mathbf{E}+\\mathbf{v}_{e} \\times \\mathbf{B}\\right)  \\tag{5.113}\\\\
& +\\frac{q_{i}}{m_{i}} \\mathbf{F}_{i e}+\\frac{e}{m_{e}} \\mathbf{F}_{i e} .
\\end{align*}


Here, by using $q_{i} \\times$ Eq. (5.98) $-e \\times$ Eq. (5.97), the following is obtained:


\\begin{align*}
& \\frac{\\partial \\mathbf{J}}{\\partial t}+\\nabla \\cdot(\\mathbf{J v}) \\\\
& \\simeq \\frac{e}{m_{e}} \\nabla P_{e}+\\frac{n_{e} e^{2}}{m_{e}}\\left(\\mathbf{E}+\\mathbf{v}_{e} \\times \\mathbf{B}\\right)+\\frac{e}{m_{e}} \\mathbf{F}_{i e} \\tag{5.114}
\\end{align*}


Here we used the relation of $m_{i} \\gg m_{e}$. When $P_{e} \\sim P / 2$, the further manipulation is done:


\\begin{align*}
\\frac{\\partial \\mathbf{J}}{\\partial t} \\simeq & \\frac{e}{2 m_{e}} \\nabla P+\\frac{n_{e} e^{2}}{m_{e}}(\\mathbf{E}+\\mathbf{v} \\times \\mathbf{B})-\\frac{e}{m_{e}} \\mathbf{J} \\times \\mathbf{B}-\\frac{n_{e} e^{2}}{m_{e}} \\frac{\\mathbf{J}}{\\sigma} \\\\
& \\simeq \\frac{e}{m_{e}}\\left\\{\\frac{1}{2} \\nabla P+\\frac{\\rho_{m} e}{m_{i}}(\\mathbf{E}+\\mathbf{v} \\times \\mathbf{B})-\\mathbf{J} \\times \\mathbf{B}-\\frac{\\rho_{m} e}{m_{i}} \\frac{\\mathbf{J}}{\\sigma}\\right\\} . \\tag{5.115}
\\end{align*}


Here we also assume that $\\mathbf{v}_{i} \\sim \\mathbf{v}_{e} \\sim \\mathbf{v}$ and $\\nabla \\cdot(\\mathbf{J v}) \\sim 0$. The approximation of $n_{e} \\sim \\rho_{m} / m_{i}$ was also used. Further, we assumed $\\mathbf{F}_{i e} \\sim-v m n\\left(\\mathbf{v}_{i}-\\mathbf{v}_{e}\\right)$. Here $v$ is the collision frequency. The electric conductivity of $\\sigma$ is $n e^{2} /\\left(m_{e} v\\right)$, and $\\mathbf{F}_{i e} \\sim$ $-\\rho_{m} e \\mathbf{J} /\\left(\\sigma m_{i}\\right)$.

Here we consider a case, in which the density gradient of $\\nabla P \\sim 0, \\partial \\mathbf{J} / \\partial t \\sim 0$ and $\\mathbf{J} \\sim 0$. Then we obtain the following:


\\begin{equation*}
\\mathbf{J}=\\sigma(\\mathbf{E}+\\mathbf{v} \\times \\mathbf{B}) \\tag{5.116}
\\end{equation*}


Further, when $\\sigma \\rightarrow \\infty$, the following is obtained:


\\begin{equation*}
\\mathbf{E}=-\\mathbf{v} \\times \\mathbf{B} \\tag{5.117}
\\end{equation*}


When $\\sigma \\rightarrow \\infty$, no net charge $\\rho_{c}$ is there: $\\rho_{c}=0$.

The details for the MHD model are found, for example, in Chaps. 7 and 8 in Ref. $[30]$.

\\subsection*{Frozen Magnetic Flux}
In the Sect. 5.8 it is shown that the magnetic flux is frozen to plasma, when the plasma is a perfect conductor.

As shown above in Eq. (5.117), when the plasma is a perfect conductor $(\\sigma \\rightarrow \\infty)$, the following holds:

$$
\\mathbf{E}=-\\mathbf{v} \\times \\mathbf{B} \\quad 5.117
$$

Therefore, when $\\sigma \\rightarrow \\infty$, the following is obtained:


\\begin{equation*}
\\frac{\\partial \\mathbf{B}}{\\partial t}=-\\nabla \\times \\mathbf{E}=\\nabla \\times(\\mathbf{v} \\times \\mathbf{B}) \\tag{5.118}
\\end{equation*}


Here we consider the magnetic flux of $\\Phi$ through a closed surface of $\\mathbf{S}$ :


\\begin{equation*}
\\Phi=\\int_{S} \\mathbf{B} \\cdot d \\mathbf{S} \\tag{5.119}
\\end{equation*}


The time derivative of $\\Phi$ is obtained for the moving surface $\\mathbf{S}$ :


\\begin{equation*}
\\frac{\\mathrm{d} \\Phi}{\\mathrm{d} t}=\\int_{S} \\frac{\\partial \\mathbf{B}}{\\partial t} \\cdot d \\mathbf{S}+\\int_{C} \\mathbf{B} \\cdot(\\mathbf{v} \\times d \\mathbf{l}) \\tag{5.120}
\\end{equation*}


The closed surface $\\mathbf{S}$ moves with $\\mathbf{v}$, and $C$ shows the closed line surrounding the surface $\\mathbf{S}$. To Eqs. (5.120), (5.118) is applied:


\\begin{align*}
\\frac{\\mathrm{d} \\Phi}{\\mathrm{d} t} & =\\int_{S} \\nabla \\times(\\mathbf{v} \\times \\mathbf{B}) \\cdot d \\mathbf{S}+\\int_{C} \\mathbf{B} \\cdot(\\mathbf{v} \\times d \\mathbf{l}) \\\\
& =\\int_{C}(\\mathbf{v} \\times \\mathbf{B}) \\cdot d \\mathbf{l}+\\int_{C} \\mathbf{B} \\cdot(\\mathbf{v} \\times d \\mathbf{l}) \\\\
& =-\\int_{C} \\mathbf{B} \\cdot(\\mathbf{v} \\times d \\mathbf{l})+\\int_{C} \\mathbf{B} \\cdot(\\mathbf{v} \\times d \\mathbf{l}) \\\\
& =0 \\tag{5.121}
\\end{align*}


Here the relations in Appendix B. 2 are used. The result in Eq. (5.121) shows that the magnetic flux is conserved, when the perfect-conductive plasma $(\\sigma \\rightarrow \\infty)$ moves. If the perfect conductor does not move together with the magnetic field, the electric field of $\\mathbf{E}=\\mathbf{v} \\times \\mathbf{B}$ would appear. In this case, because of the infinite conductivity $\\sigma=\\infty$, infinite current would flow. Therefore, plasmas with infinite conductivity should move together with the magnetic field.

\\subsection*{Waves in Magnetized Plasma}
In Sects. 5.4, 5.5 and 5.6 the waves were introduced in plasmas without a background magnetic field $B_{0}$. In the Sect. 5.9, some waves in a uniform magnetized plasma are introduced. When a constant magnetic field $B_{0}$ is applied to plasmas, interesting phenomena appear in plasmas. The polarized waves, the Alfvén wave, the ordinary wave (the $O$ wave) and the extraordinary wave (the $X$ wave) are example waves of them. In the Sect. 5.9 we assume that the magnetic field $\\mathbf{B}_{0}$ is parallel to $z$ : $\\mathbf{B}_{\\mathbf{0}}\\left(=\\left(0,0, B_{0}\\right)\\right)$.

First, electron electromagnetic waves propagating along the magnetic field $B_{0}$ are considered: $\\mathbf{k} \\| \\mathbf{B}_{\\mathbf{0}}\\left(=\\left(0,0, B_{0}\\right)\\right)$ and $\\mathbf{k}=\\left(0,0, k_{\\|}\\right)$. The electron equation of motion and the Maxwell equations are employed, as we did in a similar way in Sec.

5.6. The linearized equations transformed for one component of $\\exp (-i \\omega+i \\mathbf{k} \\cdot \\mathbf{x})$ are as follows:


\\begin{align*}
-i \\omega m_{e} v_{1 x} & =-e E_{1 x}-e v_{1 y} B_{0}  \\tag{5.122}\\\\
-i \\omega m_{e} v_{1 y} & =-e E_{1 y}+e v_{1 x} B_{0}  \\tag{5.123}\\\\
-i k_{\\|} E_{1 y} & =i \\omega B_{1 x}  \\tag{5.124}\\\\
i k_{\\|} E_{1 x} & =i \\omega B_{1 y}  \\tag{5.125}\\\\
-i k_{\\|} \\frac{1}{\\mu_{0}} B_{1 y} & =-i \\omega \\epsilon_{0} E_{1 x}-n_{0} e v_{1 x}  \\tag{5.126}\\\\
i k_{\\|} \\frac{1}{\\mu_{0}} B_{1 x} & =-i \\omega \\epsilon_{0} E_{1 y}-n_{0} e v_{1 y} \\tag{5.127}
\\end{align*}


The dispersion relation is obtained as follows:


\\begin{equation*}
\\left(\\frac{k_{\\|} c}{\\omega}\\right)^{2}=1-\\frac{\\omega_{\\mathrm{pe}}^{2} / \\omega^{2}}{1 \\pm \\Omega_{e} / \\omega} \\tag{5.128}
\\end{equation*}


Here $\\omega_{\\mathrm{pe}}$ shows the electron plasma frequency and $\\Omega_{e}=-e B_{0} / m_{e}$ presents the electron cyclotron frequency. The solution with the + sign shows the right hand polarization (the $R$ wave) and the minus sign shows the left hand polarization (the $L$ wave). The electron rotation direction around the magnetic field $B_{0}$ is the same with the $R$ wave rotation direction. Electrons resonate with the $R$ wave. At the resonance frequency $\\omega=\\left|\\Omega_{e}\\right|, k_{\\|} \\rightarrow \\infty$ and the $R$ wave is resonantly absorbed by the electrons. Equation (5.128) shows that the $R$ wave has a large phase speed compared with that of the $L$ wave at a high frequency.

Next we focus on the the Alfvén wave. The plasma ions play a main role in the Alfvén wave. The magnetic field $B_{0}$ has a tension. When a perturbation is applied in the transverse direction, the magnetic field tension contributes to pull back the plasma to the original position. Then the electromagnetic Alfvén wave propagates along the magnetic field.

Based on the magnetohydrodynamic equations, the Alfvén wave is solved here. We assume that the plasma is adiabatic and neutral. The plasma conductivity is infinite for simplicity. The basic equations may be as follows:


\\begin{align*}
\\frac{\\partial \\rho_{m}}{\\partial t}+\\nabla \\cdot\\left(\\rho_{m} \\mathbf{v}\\right) & =0  \\tag{5.129}\\\\
\\rho_{m}\\left\\{\\frac{\\partial \\mathbf{v}}{\\partial t}+(\\mathrm{v} \\cdot \\nabla) \\mathbf{v}\\right\\} & =-\\nabla p+\\mathbf{J} \\times \\mathbf{B}  \\tag{5.130}\\\\
-\\frac{\\partial \\mathbf{B}}{\\partial t} & =\\nabla \\times \\mathbf{E}  \\tag{5.131}\\\\
\\mu_{0} \\mathbf{J} & =\\nabla \\times \\mathbf{B}  \\tag{5.132}\\\\
\\mathbf{E}+\\mathbf{v} \\times \\mathbf{B} & =0  \\tag{5.133}\\\\
P \\rho_{m}^{\\gamma} & =\\text { constant } \\tag{5.134}
\\end{align*}


Here $\\rho_{m}$ is the plasma mass density, $\\gamma$ is the specific heat ratio, and $\\gamma=5 / 3$ for the ideal monoatomic gas. The linearized basic equations are shown below:


\\begin{align*}
\\frac{\\partial \\rho_{m 1}}{\\partial t} & =-\\rho_{m 0} \\nabla \\cdot \\mathbf{v}_{1}  \\tag{5.135}\\\\
\\rho_{m 0} \\frac{\\partial \\mathbf{v}_{1}}{\\partial t} & =-\\left(\\frac{\\gamma P_{0}}{\\rho_{m 0}}\\right) \\nabla \\rho_{1}+\\frac{1}{\\mu_{0}}\\left(\\nabla \\times \\mathbf{B}_{1}\\right) \\times \\mathbf{B}_{0}  \\tag{5.136}\\\\
\\frac{\\partial \\mathbf{B}_{1}}{\\partial t} & =\\nabla \\times\\left(\\mathbf{v}_{1} \\times \\mathbf{B}_{0}\\right) \\tag{5.137}
\\end{align*}


When the perturbations are proportional to the form $\\exp (-\\omega t+i \\mathbf{k} \\cdot \\mathbf{x})$, the dispersion relation is obtained, though its explicit form is not shown here. The explicit form of the dispersion relation can be found, for example, in Chap. 7 in Ref. [31].

Now the sound speed $C_{s}$ is introduced, and $C_{s}=\\sqrt{\\gamma P_{0} / \\rho_{m 0}}$. We also introduced $v_{A}$, that is called the Alfvén speed: $v_{A}=\\frac{B_{0}}{\\sqrt{\\mu_{0} \\rho_{m 0}}}$, and its direction is parallel to $\\mathbf{B}_{0}\\left(\\mathbf{v}_{\\mathbf{A}} \\| \\mathbf{B}_{\\mathbf{0}}\\right)$. For the Alfvén wave, $\\mathbf{k} \\| \\mathbf{B}_{0}$ and the perturbation is transverse to $B_{0}$ : $\\mathbf{v}_{1}, \\mathbf{E}_{1}, \\mathbf{B}_{1} \\perp \\mathbf{B}_{0}$. Then we obtain the following dispersion relation for the Alfvén wave:


\\begin{equation*}
\\frac{\\omega^{2}}{k^{2}}=v_{A}^{2}=\\frac{B_{0}^{2}}{\\mu_{0} \\rho_{m 0}} \\tag{5.138}
\\end{equation*}


When $\\mathbf{k} \\perp \\mathbf{B}_{\\mathbf{0}}, \\mathbf{k} \\cdot \\mathbf{v}_{\\mathbf{A}}=0$ and the following dispersion relation is obtained for $\\mathbf{k} \\| \\mathbf{v}_{1}$, which means a longitudinal wave:


\\begin{equation*}
\\frac{\\omega^{2}}{k^{2}}=C_{s}^{2}+v_{A}^{2} \\tag{5.139}
\\end{equation*}


This is the magnetosonic wave in the magnetized plasma.

The Alfvén wave is a transverse wave, and propagates along the magnetic field $\\mathbf{B}_{0}$, which vibrates like music strings. On the other hand, the magnetosonic wave is a longitudinal wave, propagating perpendicular to $\\mathbf{B}_{0}$, which is compressed in transverse.

Now we move to high-frequency waves and ignore the ion motion. First, we consider the case of $\\mathbf{k}=\\left(k_{\\perp}\\left(=k_{x}\\right), 0, k_{\\|}=0\\right)$ and $\\mathbf{E}_{1} \\| z$. Here for simplicity the electron temperature is set to zero: $T_{e}=0$. The linearized basic equations may provide the following equations for one component of $\\exp (-i \\omega t+i \\mathbf{k} \\cdot \\mathbf{x})$ :


\\begin{align*}
-i \\omega m_{e} \\mathbf{v}_{1} & =-e \\mathbf{E}_{1}-e \\mathbf{v}_{1} \\times \\mathbf{B}_{0}  \\tag{5.140}\\\\
i \\mathbf{k} \\times \\mathbf{E}_{1} & =i \\omega \\mathbf{B}_{1}  \\tag{5.141}\\\\
\\frac{1}{\\mu_{0}} i \\mathbf{k} \\times \\mathbf{B}_{1} & =-e n_{0} \\mathbf{v}_{1}-i \\omega \\epsilon_{0} \\mathbf{E}_{1} \\tag{5.142}
\\end{align*}


In this wave, $\\mathbf{E}_{1}$, which is parallel to $z$, contributes to $\\mathbf{v}_{1}$. However, the second term of the right side of Eq. (5.140) does not contribute to $\\mathbf{v}_{1}$. Therefore, the wave in this case should be the ordinary electromagnetic wave:


\\begin{equation*}
\\omega^{2}=\\omega_{\\mathrm{pe}}^{2}+k_{\\perp}^{2} c^{2} \\tag{5.143}
\\end{equation*}


Here $n_{0}$ is the electron number density, $e$ is the elementary charge, $m_{e}$ shows the electron mass, and $\\omega_{\\mathrm{pe}}$ is the electron plasma frequency: $\\omega_{\\mathrm{pe}}=\\sqrt{\\frac{n_{0} e^{2}}{m_{e} \\epsilon}}$. The wave is called the ordinary wave (the $O$ wave), which propagates in plasmas transversely to $B_{0}$, as if it is unaffected by $B_{0}$.

When $\\mathbf{E}_{1}$ is located in the $x-y$ plane for the electron wave $\\left(\\mathbf{E}_{1}=\\left(E_{1 x}, E_{1 y}, 0\\right)\\right)$, the basic equations are as follows for one component of $\\exp (-i \\omega t+i \\mathbf{k} \\cdot \\mathbf{x})$. Here we assume again that the wave vector is parallel to $x: \\mathbf{k}=\\left(k_{\\perp}\\left(=k_{x}\\right), 0, k_{\\|}=0\\right)$.


\\begin{align*}
-i \\omega m_{e} v_{1 x} & =-e E_{1 x}-e v_{1 y} B_{0}  \\tag{5.144}\\\\
-i \\omega m_{e} v_{1 y} & =-e E_{1 y}+e v_{1 x} B_{0}  \\tag{5.145}\\\\
i k_{\\perp} E_{1 y} & =i \\omega B_{1 z}  \\tag{5.146}\\\\
0 & =-i \\omega \\epsilon_{0} E_{1 x}-n_{0} e v_{1 x}  \\tag{5.147}\\\\
-i k_{\\perp} \\frac{1}{\\mu_{0}} B_{1 z} & =-i \\omega \\epsilon_{0} E_{1 y}-n_{0} e v_{1 y} \\tag{5.148}
\\end{align*}


The dispersion relation is as follows:


\\begin{equation*}
\\frac{k_{\\perp}^{2} c^{2}}{\\omega^{2}}=-\\frac{\\left(1-\\omega_{\\mathrm{pe}}^{2} / \\omega^{2}\\right) \\omega_{\\mathrm{pe}}^{2}}{\\omega^{2}-\\left(\\omega_{\\mathrm{pe}}^{2}+\\Omega_{e}^{2}\\right)}=-\\frac{\\left(1-\\omega_{\\mathrm{pe}}^{2} / \\omega^{2}\\right) \\omega_{\\mathrm{pe}}^{2}}{\\omega^{2}-\\omega_{U P}^{2}} \\tag{5.149}
\\end{equation*}


This wave is called the extraordinary wave (the $X$ wave). In the $X$ wave, $E_{1 x}$ is parallel to $\\mathbf{k}$. The extraordinary wave, propagating transversely to $B_{0}$, is a combination of the longitudinal wave and the electromagnetic wave. Here $\\omega_{U H}^{2}=\\omega_{\\mathrm{pe}}^{2}+\\Omega_{e}^{2}$, and $\\omega_{U H}$ shows the upper hybrid frequency. The extraordinary wave has the resonance frequency at $\\omega=\\omega_{U H}$. At the resonance frequency, the wave is resonantly absorbed. Its detailed discussions can be found in Chap. 7 in Ref. [30] and Refs.[32, 33].

The further detailed discussions on the waves in magnetized plasmas can be found in Chap. 7 in Ref. [30] and Refs. [32, 33].

\\section*{References}
\\begin{enumerate}
  \\item N.A. Tahir, K.A. Long, E.W. Laing, Method of solution of a three-temperature plasma model and its application to inertial confinement fusion target design studies. J. Appl. Phys., 60, 898-903 (1986). \\href{https://doi.org/10.1063/1.337330}{https://doi.org/10.1063/1.337330}

  \\item H. Lamb, Hydrodynamics (Cambridge University Press, Cambridge, 1945)

  \\item L.D. Landau, E.M. Lifshitz, Fluid Dynamics (Pergamon Press Ltd., Oxford, 1987)

  \\item S.P. Lyon, J.D. Johnson, Group T-1: Sesame: The los alamos national laboratory equation of state database, LA-UR-92-3407 (1992)

  \\item A.R. Bell, New equations of state for MEDUSA, Rutherford Lab. Report, RU-80-091 (1981)

  \\item R. More, K.H. Warren, D.A. Young, G.B. Zimmerman, A new quotidian equation of state (QEOS) for hot dense matter. Phys. Fluids 31, 3059-3078 (1988). \\href{https://doi.org/10.1063/1}{https://doi.org/10.1063/1}. 866963

  \\item P.J. Roache, Fundamentals of Computational Fluid Dynamics (Hermosa Pub, New Mexico, 2003)

  \\item R.D. Richtmeyer, K.W. Morton, Difference Methods for Initial-Value Problems (Interscience Pub, New York, 1967)

  \\item W.H. Press, S.A. Teukolsky, W.T. Vetterling, B.P. Flannery, Numerical Recipes in C (2nd ed.): The Art of Scientific Computing (Cambridge University Press, New York, 1992). ISBN: 0521431085

  \\item The summary on the Kelvin-Helmholtz can be found below (2021). \\href{https://www.sciencedirect}{https://www.sciencedirect}. com/topics/earth-and-planetary-sciences/kelvin-helmholtz-instability. Cited 15 August 2021

  \\item S. Chandrasekhar, Hydrodynamics and Hydromagnetic Stability (Dover pub, New York, 1970)

  \\item Ya.B. Zel'dovich, Yu.P. Raizer, Physics of Shock Waves and High-Temperature Hydrodynamic Phenomena (Dover, New York, 2002)

  \\item R.D. Richtmyer, Taylor instability in shock acceleration of compressible fluids. Comm. Pure Appl. Math. 13, 297-319 (1960)

  \\item E.E. Meshkov, Instability of the interface of two gasses accelerated by a shock wave. Fluid Dyn. 4, 101-104 (1969)

  \\item OpenMP Architecture Review Board \\href{https://www.openmp.org/}{https://www.openmp.org/}. Cited 28 August, 2021

  \\item R. Sato, S. Kawata, T. Karino, K. Uchibori, T. Innuma, H. Katoh, A.I. Ogoyski, Code O-SUKI: simulation of direct-drive fuel target implosion in heavy ion inertial fusion. Comput. Phys. Comm. 40, 83-100 (2019). \\href{https://doi.org/10.1016/j.cpc.2019.03.003}{https://doi.org/10.1016/j.cpc.2019.03.003}

  \\item H. Nakamura, K. Uchibori, S. Kawata, T. Karino, R. Sato, A.I. Ogoyski, Code O-SUKI-N 3D: upgraded direct-drive fuel target 3D implosion code in heavy ion inertial fusion. Comput. Phys. Comm., 272, 108223 (2022). \\href{https://doi.org/10.1016/j.cpc.2021.108223}{https://doi.org/10.1016/j.cpc.2021.108223} \\href{https://arxiv.org/abs/}{https://arxiv.org/abs/} 2105.00092

  \\item W.D. Schulz, Two-dimensional Lagrangian hydrodynamic difference equations. Methods Computational Physics, in Fundamental Methods in Hydrodynamics, vol. 3, ed. by B. Alder, S. Fernbach, M. Rotenberg (Academic Press, New York, 1964), pp.1-45

  \\item J.P. Christiansen, D.E.T.F. Ashby, K.V. Roberts, MEDUSA - A one-dimensional laser fusion code. Comp. Phys. Comm. 7, 271-287 (1974)

  \\item N.A. Tahir, K.A. Long, MEDUSA-KA: a one-dimensional computer code for lnertial confinement fusion target design, KfK-3454 (1983)

  \\item R. Ramis, J. Meyer-ter-Vehn, MULTI-IFE-A one-dimensional computer code for inertial fusion energy (IFE) target simulations. Comp. Phys. Comm. 203, 226-237 (2016)

  \\item R. Ramis, J. Meyer-ter-Vehn, J. Ramírez, MULTI2D-a computer code for two-dimensional radiation hydrodynamics. Comp. Phys. Comm. 180, 977-994 (2009)

  \\item R. Ramis, B. Canaud, M. Temporal, W.J. Garbett, F. Philippe, Analysis of three-dimensional effects in laser driven thin-shell capsule implosions. Matter Radiat. Extremes 4, 055402 (2019)

  \\item J.A. Harte, W.E. Alley, D.S. Bailey, J.L., Eddleman, G.B. Zimmerman, 1996 ICF Annual Report, Lawrence Livermore National Laboratory, Livermore, CA, UCRL-LR-105821-96 (1997), pp. 150-164

  \\item C.W. Hirt, A.A. Asden, J.L. Cook, An arbitrary Lagrangian-Eulerian computing method for all flow speeds. J. Comput. Phys. 14, 227-253 (1974)

  \\item R.M. Frank, R.B. Lazarus, Mixed Eulerian-Lagrangian method, methods in Computational Physics, in Fundamental Methods in Hydrodynamics, vol. 3, ed. by B. Alder, S. Fernbach, M. Rotenberg (Academic Press, New York, 1964), pp.47-67

  \\item M. Kucharik, R. Liska, R. Loubere, M. Shashkov, Arbitrary lagrangian-eulerian (ALE) method in cylindrical coordinates for laser plasma simulations, in Hyperbolic Problems: Theory, Numerics, Applications ed. by S. Benzoni-Gavage, D. Serre (Springer, Berlin, Heidelberg, 2008), pp. 687-694. \\href{https://doi.org/10.1007/978-3-540-75712-2_69}{https://doi.org/10.1007/978-3-540-75712-2\\_69}

  \\item M.M. Marinak, S.W. Haan, R.E. Tipton, Three-Dimensional Simulations of National Ignition Facility 143 Capsule Implosions with HYDRA: 1996 ICF Annual Report, UCRL-LR-10582196, pp. 143-149

  \\item Chapter 1: F.W.J. Olver, D.W. Lozier, R.F. Boisvert, C.W. Clark (Ed.) NIST Handbook of Mathematical Functions (Cambridge University Press, Cambridge, 2010). \\href{https://dlmf.nist}{https://dlmf.nist}. gov/

  \\item D.R. Nicholson, Introduction to Plasma Theory (John Wiley \\& Sons, New York, 1983)

  \\item J.D. Jackson, Classical Electrodynamics (John Wiley \\& Sons Inc, Hoboken, 1999)

  \\item Chapter 10: K. Miyamoto, Plasma Physics and Controlled Nuclear Fusion (Springer, Springer, Berlin, Heidelberg, 2013)

  \\item Chapter 2: T.H. Stix, The Theory of Plasma Waves (MacGraw Hill, New York, 1962)

\\end{enumerate}

\\section*{Chapter 6 Plasma Treated by Distribution Function: Kinetic Model }
\\begin{abstract}
In the last Chap. 5, plasma is treated macroscopically by fluid model. In the fluid model, the distribution function is averaged in the plasma particle velocity space, and the fluid model is described by density, temperature and averaged velocity. The microscopic behavior is treated by distribution function. In the Chap. 6 we focus on the distribution function to describe plasmas. The details of the plasma treatment by the distribution function can be found in Refs. [1-4], and the model is also called kinetic model. The general distribution function in physics can be found in books on statistical physics [5].
\\end{abstract}

\\subsection*{The Vlasov Equation}
\\subsubsection*{The Klimontovich Equation}
Now an $n$-particle plasma system is considered. When the space of the position $\\mathbf{x}$ and the velocity $\\mathbf{v}$ is employed, one plasma particle is described by one point in the six-dimensional phase space of $(\\mathbf{x}, \\mathbf{v})$. The density $N_{i}(t, \\mathbf{x}, \\mathbf{v})$ for one particle $i$ would be expressed by the following:


\\begin{equation*}
N_{i}(t, \\mathbf{x}, \\mathbf{v})=\\delta\\left[\\mathbf{x}-\\mathbf{X}_{i}(t)\\right] \\delta\\left[\\mathbf{v}-\\mathbf{V}_{i}(t)\\right] \\tag{6.1}
\\end{equation*}


Here $\\mathbf{X}_{i}(t)$ shows the particle trajectory in the $\\mathbf{x}$ space, and $\\mathbf{V}_{i}(t)$ shows the trajectory in the velocity $\\mathbf{v}$ space. For the $n$-particle plasma system, the density in the six-dimensional phase space is written as follows:


\\begin{equation*}
N(t, \\mathbf{x}, \\mathbf{v})=\\sum_{i=1}^{n} \\delta\\left[\\mathbf{x}-\\mathbf{X}_{i}(t)\\right] \\delta\\left[\\mathbf{v}-\\mathbf{V}_{i}(t)\\right]=\\sum_{i=1}^{n} N_{i}(t, \\mathbf{x}, \\mathbf{v}) \\tag{6.2}
\\end{equation*}


We take the time derivative of $N$ :


\\begin{align*}
\\frac{\\partial N(t, \\mathbf{x}, \\mathbf{v})}{\\partial t} & =\\sum_{i=1}^{n} \\frac{\\partial \\mathbf{X}_{i}(t)}{\\partial t} \\cdot \\frac{\\partial N_{i}(t, \\mathbf{x}, \\mathbf{v})}{\\partial \\mathbf{X}_{i}(t)}+\\sum_{i=1}^{n} \\frac{\\partial \\mathbf{V}_{i}(t)}{\\partial t} \\cdot \\frac{\\partial N_{i}(t, \\mathbf{x}, \\mathbf{v})}{\\partial \\mathbf{V}_{i}(t)} \\\\
& =-\\sum_{i=1}^{n} \\mathbf{v} \\cdot \\frac{\\partial N_{i}}{\\partial \\mathbf{x}}-\\sum_{i=1}^{n} \\frac{\\partial \\mathbf{v}}{\\partial t} \\cdot \\frac{\\partial N_{i}}{\\partial \\mathbf{v}} \\\\
& =-\\mathbf{v} \\cdot \\frac{\\partial N}{\\partial \\mathbf{x}}-\\mathbf{a} \\cdot \\frac{\\partial N}{\\partial \\mathbf{v}} \\tag{6.3}
\\end{align*}


Here a shows the acceleration, and the relations of $A \\delta(A-\\alpha)=\\alpha \\delta(A-\\alpha)$ and $\\frac{\\partial G(a-b)}{\\partial a}=-\\frac{\\partial G(a-b)}{\\partial b}$ are used.

From Eq. (6.3), we find that the density is constant along the particle trajectory in the six-dimensional space:


\\begin{equation*}
\\frac{\\mathrm{d} N(t, \\mathbf{x}, \\mathbf{v})}{\\mathrm{d} t}=\\frac{\\partial N}{\\partial t}+\\mathbf{v} \\cdot \\frac{\\partial N}{\\partial \\mathbf{x}}+\\mathbf{a} \\cdot \\frac{\\partial N}{\\partial \\mathbf{v}}=0 \\tag{6.4}
\\end{equation*}


In plasmas, the acceleration a would be obtained by the Maxwell equations in Eq. (4.5) for the electromagnetic field: $\\mathbf{a}=\\frac{q}{m}(\\mathbf{E}+\\mathbf{v} \\times \\mathbf{B})$. The the charge density $\\rho$ and electric current density $\\mathbf{J}$ are obtained by the density $N(t, \\mathbf{x}, \\mathbf{v})$ :


\\begin{align*}
\\rho & =\\int \\mathrm{d} \\mathbf{v} q N(t, \\mathbf{x}, \\mathbf{v})  \\tag{6.5}\\\\
\\mathbf{J} & =\\int \\mathrm{d} \\mathbf{v} q N(t, \\mathbf{x}, \\mathbf{v}) \\mathbf{v} \\tag{6.6}
\\end{align*}


Together with the Maxwell equations and a, Eq. (6.4) serves the Klimontovich equation to describe plasmas microscopically (see, for example, Chap. 2 in Ref. [1], Chap. 3 in Refs. [3] and [4]).

\\subsubsection*{The Liouville Equation}
Here we consider again about a particle system, which consists of $n$ particles. The positions are specified by $\\mathbf{X}_{i}(t)(i=1 \\sim n)$ and their velocities are denoted by $\\mathbf{V}_{i}(t)(i=1 \\sim n)$. In the phase space of $\\left(\\mathbf{x}_{i}, \\mathbf{v}_{i} ; i=1 \\sim n\\right)$, the particle number density $N$ is described as follows:


\\begin{equation*}
N\\left(\\mathbf{x}_{1}, \\ldots, \\mathbf{x}_{n} ; \\mathbf{v}_{1}, \\ldots, \\mathbf{v}_{n}\\right)=\\prod_{i=1}^{n} \\delta\\left(\\mathbf{x}_{i}-\\mathbf{X}_{i}\\right) \\delta\\left(\\mathbf{v}_{i}-\\mathbf{V}_{i}\\right) \\tag{6.7}
\\end{equation*}


Here $\\prod_{i=1}^{n} z_{i}=z_{1} \\cdot z_{2} \\cdot \\ldots \\cdot z_{n}$. In the $6 n$-dimensional space $\\left(\\mathbf{x}_{i}, \\mathbf{v}_{i} ; i=1 \\sim n\\right)$, it shows the position, at which the $n$ particle system locates. In the Klimontovich equation in the last Sect. 6.1.1, the plasma particle system was considered in the six-dimensional phase space $(\\mathbf{x}, \\mathbf{v})$.

Here the delta function of $\\delta\\left(\\mathbf{x}_{i}-\\mathbf{X}_{i}\\right)$ is again shown and has an infinite value at the $i$ th particle position of $\\mathbf{X}_{i}$ and otherwise zero:

\\[
\\delta\\left(\\mathbf{x}_{i}-\\mathbf{X}_{i}\\right)=\\left\\{\\begin{array}{cc}
\\infty, & \\mathbf{x}_{i}=\\mathbf{X}_{i}  \\tag{6.8}\\\\
0, & \\mathbf{x}_{i} \\neq \\mathbf{X}_{i}
\\end{array}\\right\\}=\\delta\\left(\\mathbf{X}_{i}-\\mathbf{x}_{i}\\right)
\\]

The integration of the delta function gives 1:


\\begin{equation*}
\\int_{-\\infty}^{\\infty} \\delta\\left(\\mathbf{x}_{i}-\\mathbf{X}_{i}\\right) \\mathrm{d} \\mathbf{x}_{i}=1 \\tag{6.9}
\\end{equation*}


In the Appendix B. 4 the additional information is provided for the delta function.

Here we take the time derivative of Eq. (6.7). The independent variables are $\\left(t, \\mathbf{x}_{i}, \\mathbf{v}_{i} ; i=1 \\sim n\\right)$, and the particle position $\\mathbf{X}_{i}(t)(i=1 \\sim n)$ and the particle velocity $\\mathbf{V}_{i}(t)(i=1 \\sim n)$ depend on time.


\\begin{align*}
\\frac{\\partial N}{\\partial t}= & \\sum_{i=1}^{n} \\frac{\\partial \\mathbf{X}_{i}}{\\partial t} \\cdot \\frac{\\partial}{\\partial \\mathbf{X}_{i}}\\left\\{\\prod_{i=1}^{n} \\delta\\left(\\mathbf{x}_{i}-\\mathbf{X}_{i}\\right) \\delta\\left(\\mathbf{v}_{i}-\\mathbf{V}_{i}\\right)\\right\\} \\\\
& +\\sum_{i=1}^{n} \\frac{\\partial \\mathbf{V}_{i}}{\\partial t} \\cdot \\frac{\\partial}{\\partial \\mathbf{V}_{i}}\\left\\{\\prod_{i=1}^{n} \\delta\\left(\\mathbf{x}_{i}-\\mathbf{X}_{i}\\right) \\delta\\left(\\mathbf{v}_{i}-\\mathbf{V}_{i}\\right)\\right\\} \\tag{6.10}
\\end{align*}


At the same time, the followings are employed below.


\\begin{gather*}
\\frac{\\partial \\mathbf{X}_{i}}{\\partial t}=\\mathbf{V}_{i}  \\tag{6.11}\\\\
\\mathbf{V}_{i} \\delta\\left(\\mathbf{v}_{i}-\\mathbf{V}_{i}\\right)=\\mathbf{v}_{i} \\delta\\left(\\mathbf{V}_{i}-\\mathbf{v}_{i}\\right)  \\tag{6.12}\\\\
\\frac{\\partial}{\\partial \\mathbf{X}_{i}} \\delta\\left(\\mathbf{x}_{i}-\\mathbf{X}_{i}\\right)=-\\frac{\\partial}{\\partial \\mathbf{x}_{i}} \\delta\\left(\\mathbf{x}_{i}-\\mathbf{X}_{i}\\right) \\tag{6.13}
\\end{gather*}


Then we obtain the following:


\\begin{equation*}
\\frac{\\partial N}{\\partial t}=\\sum_{i=1}^{n}\\left(-\\mathbf{v}_{i}\\right) \\cdot \\frac{\\partial N}{\\partial \\mathbf{x}_{i}}+\\sum_{i=1}^{n}\\left(-\\frac{\\partial \\mathbf{V}_{i}}{\\partial t}\\right) \\cdot \\frac{\\partial N}{\\partial \\mathbf{v}_{i}} \\tag{6.14}
\\end{equation*}


Here $\\partial \\mathbf{V}_{i} / \\partial t$ is replaced by the acceleration:


\\begin{equation*}
\\frac{\\partial \\mathbf{V}_{i}}{\\partial t}=\\frac{q}{m}\\left[\\mathbf{E}\\left(\\mathbf{X}_{i}, t\\right)+\\mathbf{V}_{i} \\times \\mathbf{B}\\left(\\mathbf{X}_{i}, t\\right)\\right] \\tag{6.15}
\\end{equation*}


In Eq. (6.15) the position $\\mathbf{X}_{i}$, where $\\mathbf{E}$ and $\\mathbf{B}$ are obtained, is replaced to $\\mathbf{x}_{i}$ by Eq. (6.12).


\\begin{equation*}
\\frac{\\partial N}{\\partial t}+\\sum_{i=1}^{n} \\mathbf{v}_{i} \\cdot \\frac{\\partial N}{\\partial \\mathbf{x}_{i}}+\\sum_{i=1}^{n} \\frac{\\partial \\mathbf{v}_{i}}{\\partial t} \\cdot \\frac{\\partial N}{\\partial \\mathbf{v}_{i}}=0 \\tag{6.16}
\\end{equation*}


Equation (6.16) is called as the Liouville equation. If Eq. (6.16) is solved with Eq. (6.15) and the Maxwell equations for $\\mathbf{E}$ and $\\mathbf{B}$, the plasma behavior is exactly obtained.

Equation (6.16) can be rewritten in a total derivative as follows:


\\begin{equation*}
\\frac{\\mathrm{d}}{\\mathrm{d} t} N\\left(\\mathbf{x}_{1}, \\ldots, \\mathbf{x}_{n} ; \\mathbf{v}_{1}, \\ldots, \\mathbf{v}_{n}\\right)=0 \\tag{6.17}
\\end{equation*}


Equation (6.17) presents that the point, describing the particle system in the phase space, moves along the trajectory in the $6 n$-dimensional phase space $\\left(\\mathbf{x}_{1}, \\ldots, \\mathbf{x}_{n} ; \\mathbf{v}_{1}\\right.$, $\\ldots, \\mathbf{v}_{n}$ ).

The particle density $N$ is described by the delta function and is not smooth in space. Instead of $N$, a statistically averaged smooth distribution function $f_{n}$ is welcome for our purpose.


\\begin{equation*}
f_{n}\\left(\\mathbf{x}_{1}, \\ldots, \\mathbf{x}_{n} ; \\mathbf{v}_{1}, \\ldots, \\mathbf{v}_{n}\\right)=\\left\\langle N\\left(\\mathbf{x}_{1}, \\ldots, \\mathbf{x}_{n} ; \\mathbf{v}_{1}, \\ldots, \\mathbf{v}_{n}\\right)\\right\\rangle \\tag{6.18}
\\end{equation*}


The distribution function of $f_{n}$ shows a probability that the system concerned exists in the phase space of $\\left(\\mathbf{x}_{1}, \\ldots, \\mathbf{x}_{n} ; \\mathbf{v}_{1}, \\ldots, \\mathbf{v}_{n}\\right)$.

\\subsubsection*{The BBGKY Hierarchy and the Vlasov Equation}
The Liouville equation expresses the plasma system exactly. From the Liouville equation, the averaged distribution function Eq. (6.18) is introduced. Here, based on Eq. (6.18) for $f_{n}\\left(\\mathbf{x}_{1}, \\ldots, \\mathbf{x}_{n} ; \\mathbf{v}_{1}, \\ldots, \\mathbf{v}_{n}\\right)$, we define $f_{1}, f_{2}, \\ldots f_{k} \\ldots$ for the (1) particle, the (2) particles, ..., $(k)$ particles, ..., respectively:


\\begin{align*}
& f_{k}\\left(\\mathbf{x}_{1}, \\ldots, \\mathbf{x}_{k} ; \\mathbf{v}_{1}, \\ldots, \\mathbf{v}_{k}\\right)= \\\\
& V^{k} \\int f\\left(\\mathbf{x}_{1}, \\ldots, \\mathbf{x}_{n} ; \\mathbf{v}_{1}, \\ldots, \\mathbf{v}_{n}\\right) \\mathrm{d} \\mathbf{x}_{k+1} \\ldots \\mathrm{d} \\mathbf{x}_{n} \\mathrm{~d} \\mathbf{v}_{k+1} \\ldots \\mathrm{d} \\mathbf{v}_{n} \\tag{6.19}
\\end{align*}


Here $V^{k}$ is the normalization factor, and $V$ shows the volume, in which each particle exists in the space of $\\mathbf{x}$. The equation for the distribution function $f_{k}$ is not closed.

The Liouville equation of Eq. (6.17) is now integrated over the subspace of $\\left(\\mathbf{x}_{k+1} \\ldots \\mathbf{x}_{n} ; \\mathbf{v}_{k+1} \\ldots \\mathbf{v}_{n}\\right)$, after averaging statistically by Eq. (6.18):


\\begin{align*}
\\int \\cdots & \\int \\mathrm{d} \\mathbf{x}_{k+1} \\ldots \\mathrm{d} \\mathbf{x}_{n} \\mathrm{~d} \\mathbf{v}_{k+1} \\ldots \\mathrm{d} \\mathbf{v}_{n} \\\\
& {\\left[\\frac{\\partial f_{n}}{\\partial t}+\\sum_{i=1}^{n} \\mathbf{v}_{i} \\cdot \\frac{\\partial f_{n}}{\\partial \\mathbf{x}_{i}}+\\sum_{i=1}^{n} \\frac{\\partial \\mathbf{v}_{i}}{\\partial t} \\cdot \\frac{\\partial f_{n}}{\\partial \\mathbf{v}_{i}}\\right]=0 } \\tag{6.20}
\\end{align*}


In the first term in Eq. (6.20), the integrals and the time derivative can be exchanged:


\\begin{equation*}
\\int \\cdots \\int \\mathrm{d} \\mathbf{x}_{k+1} \\ldots \\mathrm{d} \\mathbf{x}_{n} \\mathrm{~d} \\mathbf{v}_{k+1} \\ldots \\mathrm{d} \\mathbf{v}_{n} \\frac{\\partial f_{n}}{\\partial t}=V^{-k} \\frac{\\partial f_{k}}{\\partial t} \\tag{6.21}
\\end{equation*}


The second term in Eq. (6.20) becomes as follows for $i \\leq k$ :


\\begin{align*}
\\int \\ldots \\int & \\mathrm{d} \\mathbf{x}_{k+1} \\ldots \\mathrm{d} \\mathbf{x}_{n} \\mathrm{~d} \\mathbf{v}_{k+1} \\ldots \\mathrm{d} \\mathbf{v}_{n} \\sum_{i=1}^{k} \\mathbf{v}_{i} \\cdot \\frac{\\partial f_{n}}{\\partial \\mathbf{x}_{i}} \\\\
= & V^{-k} \\sum_{i=1}^{k} \\mathbf{v}_{i} \\cdot \\frac{\\partial f_{k}}{\\partial \\mathbf{x}_{i}} \\tag{6.22}
\\end{align*}


The rest of the second term in Eq. (6.20) becomes as follows for $i>k$ :


\\begin{align*}
\\int \\cdots \\int= & \\mathrm{d} \\mathbf{x}_{k+1} \\ldots \\mathrm{d} \\mathbf{x}_{n} \\mathrm{~d} \\mathbf{v}_{k+1} \\ldots \\mathrm{d} \\mathbf{v}_{n} \\mathbf{v}_{i} \\cdot \\frac{\\partial f_{n}}{\\partial \\mathbf{x}_{i}} \\\\
= & V^{-k-1} \\int \\mathrm{d} \\mathbf{v}_{i} \\int \\mathrm{d} \\mathbf{x}_{i} \\mathbf{v}_{i} \\cdot \\frac{\\partial f_{k}}{\\partial \\mathbf{x}_{i}} \\\\
= & V^{-k-1} \\int \\mathrm{d} \\mathbf{v}_{i} \\mathbf{v}_{i} \\cdot \\int_{S_{i}} \\mathrm{~d} \\mathbf{S}_{i} f_{k} \\rightarrow 0\\left(\\mathbf{x}_{i} \\rightarrow \\pm \\infty\\right) \\tag{6.23}
\\end{align*}


Here $S_{i}$ shows the closed surface in the space of $\\mathbf{x}_{i}$, and $f_{k}$ should go to zero, when the surface $S_{i}$ is taken to be far enough, that is, $\\mathbf{x}_{i} \\rightarrow \\pm \\infty$. The third term in Eq. (6.20) becomes also zero for $i>k$, because plasma energy should be finite and $f_{k} \\rightarrow 0$ for $\\mathbf{v}_{i} \\rightarrow \\pm \\infty$. In the rest $(i \\leq k)$ of the third term in Eq. (6.20), the force (or the acceleration $\\frac{\\partial \\mathbf{v}_{i}}{\\partial t}$ ) is divided into the two parts: $\\frac{\\partial \\mathbf{v}_{i}}{\\partial t}=\\frac{\\partial \\mathbf{v}_{i}{ }^{e}}{\\partial t}+\\sum_{r=1}^{n} \\frac{\\partial \\mathbf{v}_{i, r}}{\\partial t}$. Here the first term of $\\frac{\\partial \\mathbf{v}_{i}{ }^{e}}{\\partial t}$ is the acceleration of the external force, and $\\frac{\\partial \\mathbf{v}_{i, i}}{\\partial t}=0$. Then the third term of Eq. (6.20) becomes as follows:


\\begin{align*}
\\int \\cdots \\int & \\mathrm{d} \\mathbf{x}_{k+1} \\ldots \\mathrm{d} \\mathbf{x}_{n} \\mathrm{~d} \\mathbf{v}_{k+1} \\ldots \\mathrm{d} \\mathbf{v}_{n}\\left[\\sum_{i=1}^{n} \\frac{\\partial \\mathbf{v}_{i}}{\\partial t} \\cdot \\frac{\\partial f_{n}}{\\partial \\mathbf{v}_{i}}\\right] \\\\
= & V^{-k} \\sum_{i=1}^{k}\\left\\{\\frac{\\partial \\mathbf{v}_{i}{ }^{e}}{\\partial t}+\\sum_{r=1}^{k} \\frac{\\partial \\mathbf{v}_{i, r}}{\\partial t}\\right\\} \\cdot \\frac{\\partial f_{k}}{\\partial \\mathbf{v}_{i}} \\\\
+ & V^{-k-1} \\sum_{i=1}^{k} \\int \\mathrm{d} \\mathbf{x}_{k+1} \\mathrm{~d} \\mathbf{v}_{k+1} \\sum_{r=k+1}^{n} \\frac{\\partial \\mathbf{v}_{i, r}}{\\partial t} \\cdot \\frac{\\partial f_{k+1}}{\\partial \\mathbf{v}_{i}} \\tag{6.24}
\\end{align*}


All the particles are equivalent with each other. Then Eq. (6.24) is rewritten as follows:


\\begin{align*}
\\int \\cdots \\int & \\mathrm{d} \\mathbf{x}_{k+1} \\ldots \\mathrm{d} \\mathbf{x}_{n} \\mathrm{~d} \\mathbf{v}_{k+1} \\ldots \\mathrm{d} \\mathbf{v}_{n}\\left[\\sum_{i=1}^{n} \\frac{\\partial \\mathbf{v}_{i}}{\\partial t} \\cdot \\frac{\\partial f_{n}}{\\partial \\mathbf{v}_{i}}\\right] \\\\
= & V^{-k} \\sum_{i=1}^{k}\\left\\{\\frac{\\partial \\mathbf{v}_{i}^{e}}{\\partial t}+\\sum_{r=1}^{k} \\frac{\\partial \\mathbf{v}_{i, r}}{\\partial t}\\right\\} \\cdot \\frac{\\partial f_{k}}{\\partial \\mathbf{v}_{i}} \\\\
+ & V^{-k-1} \\sum_{i=1}^{k}(n-k) \\int \\mathrm{d} \\mathbf{x}_{k+1} \\mathrm{~d} \\mathbf{v}_{k+1} \\frac{\\partial \\mathbf{v}_{i, k+1}}{\\partial t} \\cdot \\frac{\\partial f_{k+1}}{\\partial \\mathbf{v}_{i}} \\tag{6.25}
\\end{align*}


Then, Eq. (6.20) becomes as follows:


\\begin{align*}
\\frac{\\partial f_{k}}{\\partial t} & +\\sum_{i=1}^{k} \\mathbf{v}_{i} \\cdot \\frac{\\partial f_{k}}{\\partial \\mathbf{x}_{i}}+\\sum_{i=1}^{k}\\left\\{\\frac{\\partial \\mathbf{v}_{i}^{e}}{\\partial t}+\\sum_{r=1}^{k} \\frac{\\partial \\mathbf{v}_{i, r}}{\\partial t}\\right\\} \\cdot \\frac{\\partial f_{k}}{\\partial \\mathbf{v}_{i}} \\\\
& =-\\frac{n-k}{V} \\sum_{i=1}^{k} \\int \\mathrm{d} \\mathbf{x}_{k+1} \\mathrm{~d} \\mathbf{v}_{k+1} \\frac{\\partial \\mathbf{v}_{i, k+1}}{\\partial t} \\cdot \\frac{\\partial f_{k+1}}{\\partial \\mathbf{v}_{i}} \\tag{6.26}
\\end{align*}


This is the BBGKY hierarchy [6-9]. For the distribution function of $f_{1}$ for the one-particle system, $f_{2}$ for the two-particles system is needed. For $f_{2}, f_{3}$ is needed.

For example, $f_{2}$ would be written as follows:


\\begin{equation*}
f_{2}(a, b)=f_{1}(a) f_{1}(b)+g(a, b) \\tag{6.27}
\\end{equation*}


Here the function $g$ is the correlation function between two particles. If no correlation exists, $g(a, b)=0$, and then $f_{k}$ is described by $f_{1}$. This approximation may hold for dilute gas plasmas. Then we obtain the following Vlasov equation:


\\begin{equation*}
\\frac{\\partial f_{1}}{\\partial t}+(\\mathbf{v} \\cdot \\nabla) f_{1}+\\mathbf{a} \\cdot \\frac{\\partial f_{1}}{\\partial \\mathbf{v}}=0 \\tag{6.28}
\\end{equation*}


The Vlasov equation of Eq. (6.28) is also called as the collisionless Boltzmann equation. Here a shows the acceleration and $\\mathbf{F}_{\\text {ext }}$ the external force.


\\begin{equation*}
\\mathbf{a}=\\frac{q}{m}(\\mathbf{E}+\\mathbf{v} \\times \\mathbf{B})+\\frac{1}{m} \\mathbf{F}_{\\mathrm{ext}} . \\tag{6.29}
\\end{equation*}


By solving the Vlasov equation (6.28) with the Maxwell equations for $\\mathbf{E}$ and $\\mathbf{B}$, the plasma motion is described.

The Vlasov equation of Eq. (6.28) can be again written by the total differentiation:


\\begin{equation*}
\\frac{\\mathrm{d} f_{1}}{\\mathrm{~d} t}=0 \\tag{6.30}
\\end{equation*}


Along with the trajectory, $f_{1}=$ constant.

Here we assume that $f_{1}$ is a function of $\\left(A_{1}, \\ldots, A_{j}\\right)$. Equation (6.30) is written as follows:


\\begin{equation*}
\\sum_{i=1}^{j} \\frac{\\partial f_{1}}{\\partial A_{i}} \\frac{\\mathrm{d} A_{i}}{\\mathrm{~d} t}=0 \\tag{6.31}
\\end{equation*}


The derivative $\\mathrm{d} / \\mathrm{d} t$ is that along the trajectory. If $A_{i}$ is the constant of the motion, for example, the energy $H$, the following holds:


\\begin{equation*}
\\frac{\\mathrm{d} A_{i}}{\\mathrm{~d} t}=0 \\tag{6.32}
\\end{equation*}


Therefore, Eq. (6.31) shows that $f_{1}$ is a function of the constants of the motion.

\\subsection*{Equilibrium Solution}
In Sect. 6.2, example equilibrium solutions are obtained by the Vlasov equation of Eq. (6.28):


\\begin{equation*}
(\\mathbf{v} \\cdot \\nabla) f+\\mathbf{a} \\cdot \\frac{\\partial f}{\\partial \\mathbf{v}}=0 \\tag{6.33}
\\end{equation*}


When $\\mathbf{E}=0, \\mathbf{B}=0$ and the external force $\\mathbf{F}_{\\mathrm{ext}}=0$, the Vlasov equation becomes as follows:


\\begin{equation*}
(\\mathbf{v} \\cdot \\nabla) f=0 \\tag{6.34}
\\end{equation*}


The solution of Eq. (6.34) is that $f$ depends on just $\\mathbf{v}$ :


\\begin{equation*}
f=f(\\mathbf{v}) \\tag{6.35}
\\end{equation*}


For example, the solutions include the Maxwell distribution function of Eq. (2.1) or (2.12).

As shown in Sect. 6.1.3, the distribution function $f$ is a function of the constants of the motion. As an example, let us think about a cylindrically symmetric system.


\\begin{equation*}
\\frac{\\partial}{\\partial \\theta}=0 \\tag{6.36}
\\end{equation*}


The angular momentum $P_{\\theta}$ is conserved, and the energy $H$ is also conserved.


\\begin{equation*}
f=f\\left(H, P_{\\theta}\\right) \\tag{6.37}
\\end{equation*}


If the system is infinitely long in $z, \\partial / \\partial z=0$.


\\begin{equation*}
P_{z}=\\text { constant } \\tag{6.38}
\\end{equation*}


Therefore, the following is obtained:


\\begin{equation*}
f=f\\left(H, P_{\\theta}, P_{z}\\right) \\tag{6.39}
\\end{equation*}


This is the equilibrium solution for this special case. ${ }^{1}$

Another example is shown below. In this case $H$ and $P_{\\theta}$ are the constants of the motion.

$$
\\left\\{\\begin{array}{l}
f\\left(H, P_{\\theta}\\right)=f\\left(H-\\omega P_{\\theta}\\right) \\\\
H=\\frac{1}{2 m}\\left(p_{r}^{2}+p_{\\theta}^{2}+p_{z}^{2}\\right) \\\\
P_{\\theta}=r p_{\\theta}
\\end{array}\\right.
$$

Here $\\omega$ is a constant.


\\begin{equation*}
H-\\omega P_{\\theta}=\\frac{1}{2 m}\\left\\{p_{r}^{2}+\\left(p_{\\theta}-m r \\omega\\right)^{2}+p_{z}^{2}\\right\\}-\\frac{m}{2} r^{2} \\omega^{2} \\tag{6.40}
\\end{equation*}


From Eq. (6.40), the angular momentum of $P_{\\theta}$ is obtained:


\\begin{align*}
m V_{\\theta}(r) & \\equiv \\frac{\\int p_{\\theta} f\\left(H-\\omega P_{\\theta}\\right) \\mathrm{d} p_{r} \\mathrm{~d} p_{\\theta} \\mathrm{d} p_{z}}{\\int f\\left(H-\\omega P_{\\theta}\\right) \\mathrm{d} p_{r} \\mathrm{~d} p_{\\theta} \\mathrm{d} p_{z}} \\\\
& =m r \\omega \\tag{6.41}
\\end{align*}


The result shows that the plasma rotates in the $\\theta$ direction with $V_{\\theta}=r \\omega$. Here $\\omega$ is constant. The plasma rotates as a rigid rotator.

\\subsection*{The Boltzmann Equation and Collision Effect}
When collisions between particles cannot be ignored, the collision effect should be included (see, for example, Chaps. 1, 2 and 4 in Ref. [10]):


\\begin{equation*}
\\frac{\\partial f}{\\partial t}+(\\mathbf{v} \\cdot \\nabla) f+\\mathbf{a} \\cdot \\frac{\\partial f}{\\partial \\mathbf{v}}=\\left(\\frac{\\partial f}{\\partial t}\\right)_{c} \\tag{6.42}
\\end{equation*}

\\footnotetext{${ }^{1}$ Here let us derive $P_{\\theta}=$ constant, when $\\partial / \\partial \\theta=0$. For the Lagrangian $L\\left(q_{i}, \\dot{q}_{i}, t\\right)=T\\left(q_{i}, \\dot{q}_{i}, t\\right)-$ $U\\left(q_{i}, \\dot{q}_{i}, t\\right)$, the Lagrange equation of motion is $\\frac{\\mathrm{d}}{\\mathrm{d} t}\\left(\\frac{\\partial L}{\\partial \\dot{q}_{i}}\\right)-\\frac{\\partial L}{\\partial q_{i}}=0$. Here $T$ is the kinetic energy and $U$ the potential energy. In the case of $q_{i}=\\theta$ and $\\partial / \\partial \\theta=0, \\frac{\\mathrm{d}}{\\mathrm{d} t}\\left(\\frac{\\partial L}{\\partial \\dot{\\theta}_{i}}\\right)=0$. Therefore, $\\partial L / \\partial \\dot{\\theta}_{i}=$ $P_{\\theta}=$ constant.
}

The right side shows the collisional effect. Equation (6.42) would be called the Boltzmann equation [10]. For the long-range force of the Coulomb collision, the Fokker-Planck equation was derived [11-13]. The original explicit forms of the collision effect are complicated and are not shown here. One can find them in, for example, Chap. 8 in Ref. [1], Chap. 5 in Ref. [3] and Chap. 2 in Ref. [10]. Simplified equations were also proposed, and one of them is the Krook model [14]:


\\begin{equation*}
\\frac{\\partial f}{\\partial t}+(\\mathbf{v} \\cdot \\nabla) f+\\frac{q}{m}(\\mathbf{E}+\\mathbf{v} \\times \\mathbf{B}) \\cdot \\frac{\\partial f}{\\partial \\mathbf{v}}=-v\\left(f-f_{0}\\right) \\tag{6.43}
\\end{equation*}


Here $f_{0}$ shows the equilibrium distribution function, and $v$ is the collision frequency. In our case $f_{0}$ would be the Maxwell distribution.

\\subsection*{Moment Equations and Fluid Model}
In the Boltzmann equation and the Vlasov equation, the independent variables are the time $t$, the space $\\mathbf{x}$ and the velocity $\\mathbf{v}$. In the fluid equations introduced in Chap. 5 , the independent variables are the time $t$ and the space $\\mathbf{x}$. Here from the Boltzmann equation, the fluid equations are approximately derived. By the average in the velocity space $\\mathbf{v}$, the fluid equations are obtained. The detailed discussions and derivations of the fluid equations can be found in Chap. 7 in Ref. [3], Chap. 1 in Ref. [15] and Chap. 3 in Ref. [16].

One physical quantity of $\\phi$ is averaged as follows:


\\begin{align*}
& \\int \\mathrm{d} \\mathbf{v} \\phi\\left\\{\\frac{\\partial f}{\\partial t}+(\\mathbf{v} \\cdot \\nabla) f+\\mathbf{a} \\cdot \\frac{\\partial f}{\\partial \\mathbf{v}}\\right\\}=\\int \\mathrm{d} \\mathbf{v} \\phi\\left(\\frac{\\partial f}{\\partial t}\\right)_{c},  \\tag{6.44}\\\\
& \\int \\mathrm{d} \\mathbf{v} \\phi f=n \\bar{\\phi}  \\tag{6.45}\\\\
& \\int \\mathrm{d} \\mathbf{v} f=n  \\tag{6.46}\\\\
& \\int \\mathrm{d} \\mathbf{v} \\phi\\left(\\frac{\\partial f}{\\partial t}\\right)_{c}=n \\overline{\\Delta \\phi} . \\tag{6.47}
\\end{align*}


Here $\\mathbf{a}$ is the acceleration, $n$ is the number density, $\\bar{\\phi}$ shows the averaged value of $\\phi$ averaged over the velocity space $\\mathbf{v}$, and $\\overline{\\Delta \\phi}$ is the change in $\\phi$ by collision among particles. The integration range is $(-\\infty,+\\infty)$ in the velocity space. When $\\phi=\\mathbf{v}^{m}$, Eq. (6.44) is called the $m$-th moment equation.

In the 0-th moment equation, $\\phi=1$. Then, $\\int \\mathrm{d} \\mathbf{v} 1 \\times \\frac{\\partial f}{\\partial t}=\\frac{\\partial n}{\\partial t}, \\int \\mathrm{d} \\mathbf{v} \\mathbf{v} \\cdot \\nabla f=$ $\\nabla \\cdot \\int \\mathrm{d} \\mathbf{v}(\\mathbf{v} f)=\\nabla \\cdot(n \\overline{\\mathbf{v}})$, and the rest terms become zero for the 0 -th moment equation. Then we obtain the equation of continuity:


\\begin{equation*}
\\frac{\\partial n}{\\partial t}+\\nabla \\cdot(n \\overline{\\mathbf{v}})=0 \\tag{6.48}
\\end{equation*}


In the Lagrange form in Sect. 5.3, it becomes $\\frac{\\mathrm{D} n}{\\mathrm{D} t}+n(\\nabla \\cdot \\overline{\\mathbf{v}})=0$. Here $\\frac{\\mathrm{D}}{\\mathrm{D} t}=$ $\\frac{\\partial}{\\partial t}+(\\overline{\\mathbf{v}} \\cdot \\nabla)$.

For the 1-st moment, $\\phi=m \\mathbf{v}$, and the equation motion is obtained. Here $m$ shows the particle mass. The first term becomes as follows:


\\begin{equation*}
\\int \\mathrm{d} \\mathbf{v} m \\mathbf{v} \\frac{\\partial f}{\\partial t}=\\frac{\\partial}{\\partial t} \\int \\mathrm{d} \\mathbf{v} m \\mathbf{v} f=\\frac{\\partial m n \\overline{\\mathbf{v}}}{\\partial t} \\tag{6.49}
\\end{equation*}


The second term in Eq. (6.44) becomes as follows:


\\begin{equation*}
\\int \\mathrm{d} \\mathbf{v} m \\mathbf{v}(\\mathbf{v} \\cdot \\nabla) f=\\int \\mathrm{d} \\mathbf{v} \\nabla \\cdot(m \\mathbf{v} \\mathbf{v}) f=\\nabla \\cdot m n\\langle\\mathbf{v} \\mathbf{v}\\rangle \\tag{6.50}
\\end{equation*}


Here $\\langle\\mathbf{v v}\\rangle$ should be evaluated, and $\\langle\\mathbf{v}\\rangle$ may be expressed by the averaged fluid velocity $\\overline{\\mathbf{v}}$ and the thermal velocity of $\\delta \\mathbf{v}$. The thermal motion may be expressed by the fluid pressure of $P$. In general, $P$ would be pressure tensor [15]. Here for simplicity the scalar pressure $P$ is used. Therefore, $m n\\langle\\mathbf{v v}\\rangle$ may become $m n \\bar{v}+m n \\delta \\mathbf{v}^{2}$. The term of $m n \\delta \\mathbf{v}^{2}$ would be the pressure term of $P$. The third term becomes as follows:


\\begin{equation*}
\\int \\mathrm{d} \\mathbf{v} m \\mathbf{v} \\mathbf{a} \\cdot \\frac{\\partial f}{\\partial \\mathbf{v}}=-\\int \\mathrm{d} \\mathbf{v} m \\mathbf{a} f \\tag{6.51}
\\end{equation*}


Here we assumed as usual that $f$ goes to zero, when $\\mathbf{v} \\rightarrow \\pm \\infty$. By integration by parts, Eq. (6.51), that is the third term of Eq. (6.44), becomes $-m n \\mathbf{a}$, where the acceleration a may be $q(\\mathbf{E}+\\overline{\\mathbf{v}} \\times \\mathbf{B}) / m$. Here $q$ is the electric charge. The momentum change by the collisions should be zero, because the collisions are among the particles. The equation of motion is obtained:


\\begin{equation*}
\\frac{\\partial \\rho \\overline{\\mathbf{v}}}{\\partial t}+\\nabla \\cdot \\rho \\overline{\\mathbf{v}} \\mathbf{v}=-\\nabla P+\\rho \\mathbf{a} \\tag{6.52}
\\end{equation*}


Here $\\rho=m n$. By using Eq. (6.48), the equation of motion becomes as follows:


\\begin{equation*}
\\rho\\left\\{\\frac{\\partial \\overline{\\mathbf{v}}}{\\partial t}+(\\overline{\\mathbf{v}} \\cdot \\nabla) \\overline{\\mathbf{v}}\\right\\}=\\rho \\frac{\\mathrm{D} \\overline{\\mathbf{v}}}{\\mathrm{D} t}=-\\nabla P+\\rho \\mathbf{a} \\tag{6.53}
\\end{equation*}


For the second moment, $\\phi=m \\mathbf{v}^{2} / 2$, and the energy equation is obtained. The first term would become as follows:


\\begin{align*}
\\int \\mathrm{d} \\mathbf{v} \\frac{m \\mathbf{v}^{2}}{2} \\frac{\\partial f}{\\partial t} & =\\int \\mathrm{d} \\mathbf{v} \\frac{m}{2}\\left(\\overline{\\mathbf{v}}^{2}+2 \\overline{\\mathbf{v}} \\delta \\mathbf{v}+\\delta \\mathbf{v}^{2}\\right) \\frac{\\partial f}{\\partial t} \\\\
& =\\frac{\\partial}{\\partial t}\\left(\\frac{\\rho \\overline{\\mathbf{v}}^{2}}{2}\\right)+\\frac{\\partial \\rho e}{\\partial t} \\tag{6.54}
\\end{align*}


Here $\\rho e=\\int \\mathrm{d} \\mathbf{v} \\frac{m}{2} \\delta \\mathbf{v}^{2} f$, and $e$ is the specific internal energy. The second term may be expressed as follows:


\\begin{align*}
\\int \\mathrm{d} \\mathbf{v} \\frac{m \\mathbf{v}^{2}}{2} \\mathbf{v} \\cdot \\nabla f & =\\nabla \\cdot \\int \\mathrm{d} \\mathbf{v}\\left(\\frac{m}{2}\\left(\\overline{\\mathbf{v}}^{2}+2 \\overline{\\mathbf{v}} \\delta \\mathbf{v}+\\delta \\mathbf{v}^{2}\\right) \\mathbf{v} f\\right) \\\\
& =\\nabla \\cdot\\left(\\frac{\\rho \\overline{\\mathbf{v}}^{2} \\overline{\\mathbf{v}}}{2}\\right)+\\nabla \\cdot(P \\overline{\\mathbf{v}})+\\nabla \\cdot(\\rho e \\overline{\\mathbf{v}})+\\nabla \\cdot \\mathbf{q} \\tag{6.55}
\\end{align*}


Here $\\mathbf{q}$ is introduced to express the heat flux. The third term of Eq. (6.44) becomes as follows:


\\begin{align*}
\\int \\mathrm{d} \\mathbf{v} \\frac{m \\mathbf{v}^{2}}{2} \\mathbf{a} \\cdot \\frac{\\partial f}{\\partial \\mathbf{v}} & =-\\int \\mathrm{d} \\mathbf{v} m \\mathbf{v} \\cdot \\mathbf{a} f \\\\
& =-\\rho \\mathbf{a} \\cdot \\mathbf{v}=-n \\mathbf{F} \\cdot \\mathbf{v} \\tag{6.56}
\\end{align*}


We assumed again that $f$ goes to zero, when $\\mathbf{v} \\rightarrow \\pm \\infty$. Here $\\mathbf{F}=m \\mathbf{a}$ is the force, and the contribution from the collisions may be again zero.

By using Eqs. (6.48) and (6.53), the energy equation would be obtained as follows:


\\begin{equation*}
\\rho\\left\\{\\frac{\\partial e}{\\partial t}+(\\mathbf{v} \\cdot \\nabla) e\\right\\}=\\rho \\frac{\\mathrm{D} e}{\\mathrm{D} t}=-P \\nabla \\cdot \\mathbf{v}-\\nabla \\cdot \\mathbf{q}+\\rho \\mathbf{a} \\cdot \\mathbf{v} \\tag{6.57}
\\end{equation*}


\\subsection*{Dielectric Response Function for Unmagnetized Uniform Plasma}
The dispersion functions $\\epsilon(k, \\omega)$ introduced in Sects. 5.4, 5.5 and 5.6 are obtained by the distribution function. Plasma is a kind of dielectric, and $\\epsilon(k, \\omega)$ represents the plasma response to the external fields. Therefore, $\\epsilon(k, \\omega)$ is also called the dielectric response function.

Here we consider plasma in a static electric field without magnetic field $\\mathbf{B}=0$. The distribution function of $f$ may be written as follows:


\\begin{equation*}
f=f_{0}+\\delta f \\tag{6.58}
\\end{equation*}


The equilibrium is described by $f_{0}$, and $\\delta f$ shows the perturbation. Inserting this equation into the Vlasov equation and linearizing it, we obtain the linearized Vlasov equation:


\\begin{equation*}
\\frac{\\partial \\delta f}{\\partial t}+(\\mathbf{v} \\cdot \\nabla) \\delta f+\\frac{q}{m} \\mathbf{E} \\cdot \\frac{\\partial f_{0}}{\\partial \\mathbf{v}}=0 \\tag{6.59}
\\end{equation*}


The Fourier transform is applied to the space $\\mathbf{x}$, and we focus on one component of $\\exp (i \\mathbf{k} \\cdot \\mathbf{x})$. The Laplace transform is used to the time space. Then Eq. (6.59) becomes as follows:


\\begin{equation*}
-i \\omega \\delta f+i \\mathbf{k} \\cdot \\mathbf{v} \\delta f+\\frac{q}{m} \\mathbf{E} \\cdot \\frac{\\partial f_{0}}{\\partial \\mathbf{v}}=\\delta f(\\mathbf{k}, \\mathbf{v}, t=0) \\tag{6.60}
\\end{equation*}


At the left side of Eq. (6.60), $\\delta f$ means $\\delta f(\\mathbf{k}, \\mathbf{v}, \\omega)$. The right side of Eq. (6.60) of $\\delta f(\\mathbf{k}, \\mathbf{v}, t=0)$ shows the initial condition.


\\begin{equation*}
\\delta f=\\frac{1}{i(\\omega-\\mathbf{k} \\cdot \\mathbf{v})}\\left\\{\\frac{q}{m} \\mathbf{E} \\cdot \\frac{\\partial f_{0}}{\\partial \\mathbf{v}}-\\delta f(\\mathbf{k}, \\mathbf{v}, t=0)\\right\\} \\tag{6.61}
\\end{equation*}


Here we use the Poisson equation:


\\begin{equation*}
i \\mathbf{k} \\cdot \\mathbf{E}=\\frac{q}{\\varepsilon_{0}} n_{1}=\\frac{n_{0} q}{\\varepsilon_{0}} \\int \\delta f \\mathrm{~d} \\mathbf{v} \\tag{6.62}
\\end{equation*}


Substituting Eq. (6.61) into Eq. (6.62), the following is obtained:


\\begin{equation*}
\\mathbf{E} \\cdot\\left\\{i \\mathbf{k}-\\omega_{p}^{2} \\int \\frac{\\frac{\\partial f_{0}}{\\partial \\mathbf{v}}}{i(\\omega-\\mathbf{k} \\cdot \\mathbf{v})} \\mathrm{d} \\mathbf{v}\\right\\}=-\\frac{n_{0} q}{\\varepsilon_{0}} \\int \\frac{\\delta f(\\mathbf{k}, \\mathbf{v}, t=0)}{i(\\omega-\\mathbf{k} \\cdot \\mathbf{v})} \\mathrm{d} \\mathbf{v} \\tag{6.63}
\\end{equation*}


Here $\\omega_{p}^{2}=n_{0} q^{2} /\\left(m \\varepsilon_{0}\\right)$ is the plasma oscillation frequency. At present we consider about a static mode, that is, the longitudinal mode: $\\mathbf{E} \\| \\mathbf{k}$. Therefore, Eq. (6.63) becomes as follows:


\\begin{equation*}
\\left(1+\\frac{\\omega_{p}^{2}}{\\mathbf{k}^{2}} \\int \\frac{\\mathbf{k} \\cdot \\frac{\\partial f_{0}}{\\partial \\mathbf{v}}}{\\omega-\\mathbf{k} \\cdot \\mathbf{v}} \\mathrm{d} \\mathbf{v}\\right) \\mathbf{E}=\\frac{n_{0} q \\mathbf{k}}{\\varepsilon_{0} \\mathbf{k}^{2}} \\int \\frac{\\delta f(\\mathbf{k}, \\mathbf{v}, t=0)}{\\omega-\\mathbf{k} \\cdot \\mathbf{v}} \\mathrm{d} \\mathbf{v} \\tag{6.64}
\\end{equation*}


When the right side of Eq. (6.64) is zero, the initial condition $\\delta f(\\mathbf{k}, \\mathbf{v}, t=0)=0$. Even in this case, when terms in the parentheses at the left side of Eq. (6.64) vanish, $\\mathbf{E} \\neq 0$ and the plasma is perturbed. The dielectric function $\\epsilon(\\mathbf{k}, \\omega)$, expressing the terms in the parentheses at the left side of Eq. (6.64), is as follows for our present case:


\\begin{equation*}
\\epsilon(\\mathbf{k}, \\omega)=1+\\frac{\\omega_{p}^{2}}{\\mathbf{k}^{2}} \\int \\frac{\\mathbf{k} \\cdot \\frac{\\partial f_{0}}{\\partial \\mathbf{v}}}{\\omega-\\mathbf{k} \\cdot \\mathbf{v}} \\mathrm{d} \\mathbf{v} \\tag{6.65}
\\end{equation*}


Here we can assume $\\mathbf{k}$ is parallel to the $z$ axis: $\\mathbf{k}=k \\hat{z}$


\\begin{align*}
\\epsilon(k, \\omega) & =1+\\frac{\\omega_{p}^{2}}{k^{2}} \\int \\frac{k \\frac{\\partial f_{0}}{\\partial v_{z}}}{\\omega-k v_{z}} \\mathrm{~d} v_{x} \\mathrm{~d} v_{y} \\mathrm{~d} v_{z} \\\\
& =1+\\frac{\\omega_{p}^{2}}{k^{2}} \\int \\frac{k \\frac{\\partial f_{0}}{\\partial v_{z}}}{\\omega-k v_{z}} \\mathrm{~d} v_{z} \\tag{6.66}
\\end{align*}


Here $f_{0}$ is normalized as $\\int f_{0} \\mathrm{~d} v_{x} \\mathrm{~d} v_{y} \\mathrm{~d} v_{z}=1$. Here we use the Maxwell distribution of Eq. (2.1): $f_{0}=(m / 2 \\pi T)^{3 / 2} \\exp \\left(-m \\mathbf{v}^{2} / 2 T\\right)$. (Note $f_{0}$ is different from Eq. (2.1) by the factor of $n$.)


\\begin{equation*}
\\epsilon(k, \\omega)=1+\\frac{1}{k^{2} \\lambda_{D}^{2}} \\sqrt{\\frac{m}{2 \\pi T}} \\int \\frac{k v_{z}}{k v_{z}-\\omega} \\exp \\left(-\\frac{m v_{z}^{2}}{2 T}\\right) \\mathrm{d} v_{z} \\tag{6.67}
\\end{equation*}


We introduce the variable of $\\xi \\equiv v_{z} / \\sqrt{T / m}$ :


\\begin{equation*}
\\epsilon(k, \\omega)=1+\\frac{1}{k^{2} \\lambda_{D}^{2}} \\frac{1}{\\sqrt{2 \\pi}} \\int \\frac{\\xi}{\\xi-(\\omega / k) \\sqrt{m / T}} \\exp \\left(-\\frac{\\xi^{2}}{2}\\right) \\mathrm{d} \\xi \\tag{6.68}
\\end{equation*}


Here we also introduce the $W$ function following (see Chap. 4 in Ref. [1]):


\\begin{equation*}
W(Z)=\\frac{1}{\\sqrt{2 \\pi}} \\int \\frac{\\xi}{\\xi-Z} \\exp \\left(-\\frac{\\xi^{2}}{2}\\right) \\mathrm{d} \\xi \\tag{6.69}
\\end{equation*}


Relating to the $W(z)$ function, the plasma dispersion function $\\mathscr{Z}(Z)$ is well studied in Ref. [17]:


\\begin{equation*}
\\mathscr{Z}(Z)=\\frac{1}{\\sqrt{\\pi}} \\int \\frac{1}{\\xi-Z} \\exp \\left(-\\xi^{2}\\right) \\mathrm{d} \\xi \\tag{6.70}
\\end{equation*}


The $\\mathscr{Z}$ function is related with the $W$ function as follows:


\\begin{equation*}
W(Z)=1+\\frac{Z}{\\sqrt{2}} \\mathscr{Z}\\left(\\frac{Z}{\\sqrt{2}}\\right) \\tag{6.71}
\\end{equation*}


Fig. 6.1 Integral path for the inverse Laplace transform. All the poles are located below the path $L$ in $\\mathrm{Eq}$. (6.74)

\\begin{center}
\\includegraphics[max width=\\textwidth]{2024_02_26_83e36543483eb7d284c1g-137}
\\end{center}

By using $W(Z)$, the dielectric function is rewritten as follows:


\\begin{equation*}
\\epsilon(k, \\omega)=1+\\frac{1}{k^{2} \\lambda_{D}^{2}} W\\left(\\frac{\\omega}{k} \\sqrt{\\frac{m}{T}}\\right)=0 \\tag{6.72}
\\end{equation*}


The upper and lower limits of the $W$ and $\\mathscr{Z}$ functions are $(-\\infty, \\infty)$. When we obtain Eq. (6.60), the Laplace transform is employed. For example, $\\mathbf{E}(\\mathbf{k}, \\omega)$ is obtained by Eq. (6.64):


\\begin{equation*}
\\mathbf{E}(\\mathbf{k}, \\omega)=\\frac{1}{\\epsilon(\\mathbf{k}, \\omega)} \\frac{n_{0} q \\mathbf{k}}{\\varepsilon_{0} \\mathbf{k}^{2}} \\int \\frac{\\delta f(\\mathbf{k}, \\mathbf{v}, t=0)}{\\omega-\\mathbf{k} \\cdot \\mathbf{v}} \\mathrm{d} \\mathbf{v} \\tag{6.73}
\\end{equation*}


Then $\\mathbf{E}(\\mathbf{k}, t)$ is obtained as follows:


\\begin{equation*}
\\mathbf{E}(\\mathbf{k}, t)=\\frac{1}{2 \\pi} \\int_{L} \\mathbf{E}(\\mathbf{k}, \\omega) \\exp (-i \\omega t) \\mathrm{d} \\omega \\tag{6.74}
\\end{equation*}


The integral path $L$ is taken as shown in Fig. 6.1. All the poles of $\\mathbf{E}(\\mathbf{k}, \\omega)$ are located below the path. When $t<0$, the integral path in Fig. 6.1 can be closed in the upper half surface as shown in the dotted line in Fig. 6.1, and the following is realized:


\\begin{equation*}
\\delta f(t<0)=0 \\tag{6.75}
\\end{equation*}


In this case, we have no poles inside the integral path, and about the dotted-line path in Fig. 6.1 the following holds:

\\begin{center}
\\includegraphics[max width=\\textwidth]{2024_02_26_83e36543483eb7d284c1g-138}
\\end{center}

Fig. 6.2 Integral path for the dielectric response function of $\\epsilon(\\mathbf{k}, \\omega)$


\\begin{equation*}
\\lim _{\\omega_{i} \\rightarrow \\infty} \\exp (-i \\omega t) \\propto \\lim _{\\omega_{i} \\rightarrow \\infty} \\exp \\left(-\\omega_{i}|t|\\right)=0 \\tag{6.76}
\\end{equation*}


When the imaginary part $\\omega_{i}$ of $\\omega$ is positive $\\left(\\omega_{i}>0\\right)$, the integral range in Eq. (6.68) is $(-\\infty, \\infty)$. For $\\omega_{i} \\leq 0, \\epsilon(\\mathbf{k}, \\omega)$ is continuated analytically. Then we take the integral path shown in Figs. 6.2 for Eq. (6.68) or (6.69).

Here we introduce the approximate expression for the $W$ function:


\\begin{gather*}
W(z)=\\sqrt{\\frac{\\pi}{2}} i z \\exp \\left(-\\frac{z^{2}}{2}\\right)+1-z^{2}+\\frac{z^{4}}{3} \\cdots, \\text { for }|z|<1  \\tag{6.77}\\\\
W(z)=\\sqrt{\\frac{\\pi}{2}} i z \\exp \\left(-\\frac{z^{2}}{2}\\right)-\\frac{1}{z^{2}}-\\frac{3}{z^{4}} \\cdots, \\text { for }|z| \\gg 1 \\tag{6.78}
\\end{gather*}


\\subsection*{Plasma Oscillation and the Debye Shielding}
First we derive the plasma oscillation for a zero-temperature plasma:


\\begin{equation*}
f_{0}=\\delta\\left(v_{z}\\right) \\tag{6.79}
\\end{equation*}


Inserting $f_{0}$ into Eq. (6.66), then we obtain the following:


\\begin{align*}
\\epsilon(k, \\omega) & =1+\\frac{\\omega_{p}^{2}}{k^{2}} \\int \\frac{k \\frac{\\partial \\delta\\left(v_{z}\\right)}{\\partial v_{z}}}{\\omega-k v_{z}} \\mathrm{~d} v_{z} \\\\
& =1+\\frac{\\omega_{p}^{2}}{k}\\left(\\left[\\frac{\\delta\\left(v_{z}\\right)}{\\omega-k v_{z}}\\right]_{-\\infty}^{\\infty}-\\int \\frac{k \\delta\\left(v_{z}\\right)}{\\left(\\omega-k v_{z}\\right)^{2}} \\mathrm{~d} v_{z}\\right) \\\\
& =1-\\frac{\\omega_{p}^{2}}{\\omega^{2}}=0 \\tag{6.80}
\\end{align*}


The plasma oscillation is obtained:


\\begin{equation*}
\\omega=\\omega_{p} \\tag{6.81}
\\end{equation*}


Now the Debye shielding is considered by the dielectric response function $\\epsilon(k, \\omega)$. The potential of $\\varphi_{\\mathrm{ext}}$, created by a charge $q \\delta(\\mathbf{r})$ introduced from outside, is shielded by the plasma electrons. Equation (6.64) suggests that the net potential in the plasma $\\varphi(k, \\omega)\\left(=\\varphi_{\\text {ext }}(k, \\omega)+\\varphi_{\\text {ind }}(k, \\omega)\\right)$ would be expressed as follows:


\\begin{equation*}
\\varphi(k, \\omega)=\\frac{\\varphi_{\\mathrm{ext}}(k, \\omega)}{\\epsilon(k, \\omega)} \\tag{6.82}
\\end{equation*}


We consider the static Debye shielding, that means $\\omega=0 .^{2}$ We obtain the static potential as follows:


\\begin{equation*}
\\varphi(\\mathbf{x})=\\int \\frac{\\varphi_{\\text {ext }}(k)}{\\epsilon(k, 0)} \\exp (-i k x) \\mathrm{d} k \\tag{6.83}
\\end{equation*}


In order to obtain $\\varphi_{\\mathrm{ext}}(k)$, the potential $\\varphi_{\\mathrm{ext}}$ created by a charge $q \\delta(\\mathbf{r})$ is used:


\\begin{equation*}
\\nabla \\varphi_{\\mathrm{ext}}^{2}(\\mathbf{r})=-\\frac{q}{\\varepsilon_{0}} \\delta(\\mathbf{r}) \\tag{6.84}
\\end{equation*}


By the Fourier transform, we obtain the following:


\\begin{gather*}
\\int \\nabla^{2} \\varphi_{\\mathrm{ext}}(\\mathbf{r}) \\mathrm{e}^{i \\mathbf{k} \\cdot \\mathbf{r}} \\mathrm{d} \\mathbf{r}=-\\int \\frac{q}{\\varepsilon_{0}} \\delta(\\mathbf{r}) \\mathrm{e}^{i \\mathbf{k} \\cdot \\mathbf{r}} \\mathrm{d} \\mathbf{r}  \\tag{6.85}\\\\
-k^{2} \\varphi_{\\mathrm{ext}}(\\mathbf{k})=-\\frac{q}{\\varepsilon_{0}}  \\tag{6.86}\\\\
\\therefore \\quad \\varphi_{\\mathrm{ext}}(\\mathbf{k})=\\frac{q}{k^{2} \\varepsilon_{0}} . \\tag{6.87}
\\end{gather*}


From Eq. (6.83), we obtain the following:


\\begin{equation*}
\\varphi(\\mathbf{r})=\\frac{1}{(2 \\pi)^{3}} \\int \\frac{\\varphi_{\\mathrm{ext}}(\\mathbf{k})}{\\epsilon(\\mathbf{k}, 0)} \\mathrm{e}^{-i \\mathbf{k} \\cdot \\mathbf{r}} \\mathrm{d} \\mathbf{k} \\tag{6.88}
\\end{equation*}

\\footnotetext{${ }^{2}$ For example, for a static vale, the Fourier transform gives $\\int_{-\\infty}^{\\infty} 1 \\cdot \\exp (-i \\omega t) \\mathrm{d} t=2 \\pi \\delta(\\omega)$. The result shows that static phenomena mean $\\omega=0$.
}

Here $\\epsilon(\\mathbf{k}, 0)$ is obtained by Eqs. (6.72) and (6.77):


\\begin{equation*}
\\epsilon(\\mathbf{k}, 0)=1+\\frac{1}{k^{2} \\lambda_{D}^{2}} \\tag{6.89}
\\end{equation*}


After inserting Eq. (6.89) into Eq. (6.88), the complex integration gives the following:


\\begin{equation*}
\\varphi=\\frac{q}{4 \\pi \\varepsilon_{0} r} \\exp \\left(-\\frac{r}{\\lambda_{D}}\\right) \\tag{6.90}
\\end{equation*}


The Debye shielding is again obtained by the dielectric response function. (The derivation of Eq. (6.90) is shown in Appendix C.3.)

\\subsection*{Electron Plasma Wave and the Landau Damping}
In the last Sect. 6.6, the plasma temperature was zero, and we obtained the plasma frequency of $\\omega=\\omega_{p}$. It means the electrons oscillate locally.

In the Sect. 6.7, the electron temperature is introduced. The ions are immobile again. For this purpose, the dispersion relation of Eq. (6.72) is solved for electrons.


\\begin{equation*}
\\epsilon(k, \\omega)=1+\\frac{1}{k^{2} \\lambda_{D}^{2}} W\\left(\\frac{\\omega}{k} \\sqrt{\\frac{m_{e}}{T_{e}}}\\right)=0 \\tag{6.91}
\\end{equation*}


Here we consider the long wave length limit:


\\begin{equation*}
\\left|\\frac{\\omega}{k} \\sqrt{\\frac{m_{e}}{T_{e}}}\\right| \\gg 1 \\tag{6.92}
\\end{equation*}


By using Eq. (6.78), the following is obtained:


\\begin{align*}
\\epsilon & =\\epsilon_{r}+i \\epsilon_{i} \\\\
& =1-\\frac{\\omega_{p e}^{2}}{\\omega^{2}}-\\frac{3 k^{2} \\omega_{p e}^{2} T_{e}}{\\omega^{4} m_{e}}+\\cdots \\\\
& +\\sqrt{\\frac{\\pi}{2}} i \\frac{\\omega}{k^{3} \\lambda_{D}^{2}} \\sqrt{\\frac{m_{e}}{T_{e}}} \\exp \\left(-\\frac{\\omega^{2} m_{e}}{2 k^{2} T_{e}}\\right)=0 \\tag{6.93}
\\end{align*}


Here we set as follows:


\\begin{equation*}
\\omega=\\omega_{r}+i \\omega_{i} \\tag{6.94}
\\end{equation*}


We also assume the following:


\\begin{equation*}
\\left|\\omega_{r}\\right|>\\left|\\omega_{i}\\right| \\tag{6.95}
\\end{equation*}


Then we obtain the following:


\\begin{gather*}
\\epsilon=\\epsilon_{r}+i \\epsilon_{i} \\simeq \\epsilon_{r}\\left|\\omega_{r}+i \\omega_{i} \\frac{\\partial \\epsilon_{r}}{\\partial \\omega}\\right|_{\\omega_{r}}+\\left.i \\epsilon_{i}\\right|_{\\omega_{r}}=0  \\tag{6.96}\\\\
\\therefore \\epsilon_{r}\\left(\\omega_{r}\\right)=0  \\tag{6.97}\\\\
\\omega_{i}=-\\left.\\left(\\epsilon_{i} / \\frac{\\partial \\epsilon_{r}}{\\partial \\omega}\\right)\\right|_{\\omega_{r}} \\tag{6.98}
\\end{gather*}


By Eqs. (6.92), (6.93) and (6.97), we get $\\omega_{r}$ :


\\begin{equation*}
\\omega_{r}^{2} \\simeq \\omega_{p e}^{2}+\\frac{3 k^{2} T_{e}}{m_{e}} \\tag{6.99}
\\end{equation*}


From Eq. (6.98), $\\omega_{i}$ is also obtained:


\\begin{equation*}
\\omega_{i}=-\\sqrt{\\frac{\\pi}{8}} \\frac{\\omega_{r}}{k^{3} \\lambda_{D}^{3}} \\exp \\left(-\\frac{\\omega_{r}^{2} m_{e}}{2 k^{2} T_{e}}\\right) \\tag{6.100}
\\end{equation*}


Now we find that Eq. (6.99) is same with Eq. (5.72), when $\\gamma=3$.

In the macroscopic fluid model, Eq. (6.100) did not appear. Note that we assumed the following for a physical quantity $\\psi$ :


\\begin{equation*}
\\psi \\propto \\exp (-i \\omega t)=\\exp \\left(-i \\omega_{r} t+\\omega_{i} t\\right) \\tag{6.101}
\\end{equation*}


It means that the plasma wave damps, when $\\omega_{i}$ exists and negative. In the Sect. 6.7 we employ the Maxwell distribution, and in this case the plasma wave damps. This is called the Landau damping, which does not appear in the fluid model. The physical meaning of the Landau damping is shown in Sect. 6.9.

\\subsection*{Electron Wave Propagation in Equilibrium Plasmas}
In the last Sect. 6.7, electron plasma wave was considered and the dispersion relation was solved. Now let us consider the equilibrium-plasma response to an impulse $\\varphi_{\\text {ext }}$ applied externally at $t=0: \\varphi_{\\mathrm{ext}}(x, t)=\\varphi_{0} \\delta\\left(\\omega_{\\mathrm{pe}} t\\right) \\exp \\left(i k_{0} x\\right)=\\varphi_{0}\\left(\\delta(t) / \\omega_{\\mathrm{pe}}\\right)$
$\\exp \\left(i k_{0} x\\right)$. Here $\\omega_{\\mathrm{pe}}(>0)$ is the electron plasma frequency. After performing the Fourier and Laplace transformations, the electrical response becomes as follows:


\\begin{equation*}
\\varphi_{\\mathrm{ext}}(k, \\omega)=2 \\pi \\frac{\\varphi_{0}}{\\omega_{\\mathrm{pe}}} \\delta\\left(k-k_{0}\\right) \\tag{6.102}
\\end{equation*}


The net potential $\\varphi_{\\text {ind }}+\\varphi_{\\text {ext }}$ is obtained by Eq. (6.82):


\\begin{equation*}
\\varphi_{\\mathrm{ind}}(k, \\omega)+\\varphi_{\\mathrm{ext}}(k, \\omega)=\\frac{1}{\\epsilon(k, \\omega)} \\varphi_{\\mathrm{ext}}(k, \\omega) \\tag{6.103}
\\end{equation*}


Here the dielectric function $\\epsilon(k, \\omega)$ was obtained in Eq. (6.91) or (6.93) for the long wavelength limit in Eq. (6.92) for the Maxwell plasma.

Now the potential $\\varphi_{\\text {ind }}(x, t)$ in the real space $(x, t)$ is obtained:


\\begin{equation*}
\\varphi_{\\text {ind }}(x, t)+\\varphi_{\\text {ext }}(x, t)=\\frac{\\varphi_{0}}{2 \\pi} \\int \\mathrm{d} \\omega \\frac{1}{\\epsilon(k, \\omega)} \\exp \\left(-i\\left(k_{0} x-\\omega t\\right)\\right) \\tag{6.104}
\\end{equation*}


The poles of $\\epsilon(k, \\omega)=0$ contribute to the integral, and the net potential should develop in time as follows from Eqs. (6.99) and (6.100): $\\phi_{\\text {ind }}+\\phi_{\\text {ext }} \\propto \\exp \\left(-i \\omega_{r} t-\\right.$ $\\left.\\left|\\omega_{i}\\right| t\\right)$. Here $\\exp \\left(-\\left|\\omega_{i}\\right| t\\right)$ shows the Landau damping, and the potential induced damps exponentially.

Now the perturbation $\\delta f$ of the distribution function is calculated, and $\\delta f$ is obtained by Eq. (6.61) without the initial perturbation of $\\delta f(\\mathbf{k}, \\mathbf{v}, t=0)=0$ for our $1 \\mathrm{D}$ case.


\\begin{equation*}
\\delta f(k, \\omega)=\\frac{1}{i(\\omega-k v)}\\left\\{\\frac{q}{m} E \\cdot \\frac{\\partial f_{0}}{\\partial v}\\right\\} \\tag{6.105}
\\end{equation*}


Here $E=-i k\\left(\\phi_{\\text {ind }}+\\phi_{\\text {ext }}\\right)=-\\frac{i k \\phi_{\\text {ext }}}{\\epsilon(k, \\omega)}$. Then we obtain the following by using Eq. (6.103):


\\begin{equation*}
\\delta f(k, \\omega)=-\\frac{q}{m} \\frac{k \\varphi_{\\mathrm{ext}}(k, \\omega)}{(\\omega-k v) \\epsilon(k, \\omega)} \\frac{\\partial f_{0}}{\\partial v} \\tag{6.106}
\\end{equation*}


Now we integrate $\\delta f(k, \\omega)$ to obtain $\\delta f(x, t)$.


\\begin{equation*}
\\delta f(x, v, t) \\propto-\\frac{q}{m} \\frac{k_{0} \\varphi_{0}}{\\omega_{p e}} \\int \\mathrm{d} \\omega \\frac{1}{\\left(\\omega-k_{0} v\\right) \\epsilon\\left(k_{0}, \\omega\\right)} \\frac{\\partial f_{0}}{\\partial v} \\exp \\left(i k_{0} x-i \\omega t\\right)( \\tag{6.107}
\\end{equation*}


In addition to the poles of $\\epsilon(k, \\omega)=0$, another pole of $\\omega=k v$ also contributes to $\\delta f(x, v, t)$. Therefore, the perturbation proportional to $\\exp \\left(i k_{0} x-i k_{0} v t\\right)$ appears and does not damp by the Landau damping. The perturbation proportional to
$\\exp \\left(i k_{0} x-i k_{0} v t\\right)$ survives long. In real situations, particle collision would contribute to mix or damp the perturbation. However, in the Sect. 6.8 we considered collisionless plasmas, and the discussions here may valid for $t<1 / v$, where $v$ shows the collision frequency.

The result above shows that plasma perturbations would survive for a long period, though the collective modes damp through the Landau damping for collisionless plasmas. The remarkable property of plasmas presents interesting phenomenon, for example, plasma echo, which will be shown in Sect. 8.3.

\\subsection*{Physical Meaning of the Landau Damping}
In the Sect. 6.9 we consider the physical meaning of the Landau damping. The Landau damping was found based on the distribution function.

In order to understand the physical meaning of the Landau damping, the Maxwell distribution in Fig. 6.3 is first considered. We assume that in the plasma a wave with the phase velocity of $v_{w}$ exists. The wave moving with $v_{w}$ interacts with particles. The wave obtains some energy from fast particles, which move slightly faster than the wave. On the other hand, particles, moving slightly slower than the wave, get some energy from the wave.

In the distribution function shown in Fig. 6.3, $\\partial f / \\partial v<0$, and the number of particles, with the velocity larger than the wave velocity $v_{w}$, is small compared

Fig. 6.3 Physical meaning of the Landau damping. A plasma wave moves with $v_{w}$ in a plasma. Particles, moving slow slightly compared with the wave, get energy from the wave. The wave obtains some energy from particles, which move a little bit faster than the wave. In the Maxwell distribution the particle number, moving slower than the wave, is large compared with the number of the faster particles. In this distribution function, $\\frac{\\partial f}{\\partial v}<0$, and the waves in the plasma damp

\\begin{center}
\\includegraphics[max width=\\textwidth]{2024_02_26_83e36543483eb7d284c1g-143}
\\end{center}

Fig. 6.4 Distribution function, in which waves grow. Near the phase velocity of $v_{w}$, the number of particles, having a larger velocity than $v_{w}$, is large compared with the number of the particles moving slower than $v_{w}$. In this case, the waves get some energy from the particles, and consequently grow. This is one of plasma instability mechanisms

\\begin{center}
\\includegraphics[max width=\\textwidth]{2024_02_26_83e36543483eb7d284c1g-144}
\\end{center}

with that for the particles which have the slower velocity than $v_{w}$. For example, the Maxwell distribution is one of typical examples, in which the Landau damping appears.

Now we consider another case, in which waves grow. A wave with the phase velocity $v_{w}$ in Fig. 6.4 grows, and in this case $\\partial f / \\partial v>0$ near $v_{w}$. Here we use Eq. (6.66), and the integral path in Fig. 6.2b for $\\omega_{i}>0$, because of Eq. (6.95). By using the Plemelj formula in Appendix C.2, we obtain the following:


\\begin{align*}
\\epsilon & =1-\\frac{\\omega_{p}^{2}}{k^{2}} \\int \\frac{\\frac{\\partial f_{0}}{\\partial v_{z}}}{v_{z}-\\omega / k} \\mathrm{~d} v_{z} \\\\
& =1+\\frac{\\omega_{p}^{2}}{k^{2}} \\int_{-\\infty}^{\\infty} \\frac{k \\frac{\\partial f_{0}}{\\partial v_{z}}}{\\omega-k v_{z}} \\mathrm{~d} v_{z}-\\left.\\pi i \\frac{\\omega_{p}^{2}}{k^{2}} \\frac{\\partial f_{0}}{\\partial v_{z}}\\right|_{v_{z}=\\frac{\\omega_{r}}{k}} \\tag{6.108}
\\end{align*}


By Eq. (6.98), $\\omega_{i}$ is obtained:


\\begin{equation*}
\\left.\\omega_{i} \\propto \\frac{\\partial f_{0}}{\\partial v}\\right|_{v_{z}=\\frac{\\omega_{r}}{k}} \\tag{6.109}
\\end{equation*}


Therefore, in the case of the distribution function shown in Fig. 6.3, waves damp. In the case of Fig. 6.4, waves in the region of $\\frac{\\partial f_{0}}{\\partial v}>0$ grow.

Similar to the Landau damping, another damping (or wave growing) would be found in a magnetized plasma: the cyclotron damping. The energy absorption by rotating charged particles in the magnetized plasma would induce the cyclotron damping. The cyclotron rotating motion of charged particles may resonate with waves, and the energy exchange may appear between charged particles and waves. In this book we do not touch the cyclotron damping, and readers can find it in Chap. 6 in Ref. [3], Chap. 11 in Ref. [18] and Chap. 10 in Ref. [19].

\\subsection*{Dispersion Relation for Transverse Electromagnetic Waves}
So far the dielectric response function or the dispersion relation introduced above was applied to static waves in plasma. The Poisson equation was employed.

Here we consider the electromagnetic wave or the transverse wave in plasmas. ${ }^{3}$ We consider one Fourier component for Eqs. (1) and (2) in the Maxwell equation (4.5).


\\begin{align*}
\\mathbf{k} \\times \\mathbf{E} & =\\omega \\mathbf{B}  \\tag{6.110}\\\\
\\mathbf{k} \\times \\mathbf{H} & =-\\omega \\epsilon_{0} \\boldsymbol{\\epsilon} \\cdot \\mathbf{E}-i \\mathbf{J}_{\\mathrm{ext}} \\tag{6.111}
\\end{align*}


Here the dielectric tensor is written by $\\epsilon_{0} \\boldsymbol{\\epsilon}$, and $\\mathbf{B}=\\mu_{0} \\mathbf{H}$.


\\begin{align*}
\\frac{1}{\\mu_{0} \\omega} \\mathbf{k} \\times(\\mathbf{k} \\times \\mathbf{E}) & =\\frac{1}{\\mu_{0} \\omega}\\left((\\mathbf{k} \\cdot \\mathbf{E}) \\mathbf{k}-\\mathbf{k}^{2} \\mathbf{E}\\right) \\\\
& =\\frac{1}{\\mu_{0} \\omega}\\left(\\mathbf{k} \\mathbf{k}-\\mathbf{k}^{2}\\right) \\cdot \\mathbf{E} \\\\
\\therefore \\quad\\left\\{\\boldsymbol{\\epsilon}+\\left(\\frac{\\mathbf{k}^{2} c^{2}}{\\omega^{2}}\\right)\\left(\\frac{\\mathbf{k k}}{\\mathbf{k}^{2}}-\\mathbf{I}\\right)\\right\\} \\cdot \\mathbf{E} & =-\\frac{i \\mathbf{J}_{\\mathrm{ext}}}{\\epsilon_{0} \\omega} \\tag{6.112}
\\end{align*}


Here I is the unit tensor:

\\[
\\mathbf{I} \\equiv\\left(\\begin{array}{lll}
1 & 0 & 0  \\tag{6.113}\\\\
0 & 1 & 0 \\\\
0 & 0 & 1
\\end{array}\\right)
\\]
\\footnotetext{${ }^{3}$ The details for the transverse waves can be found in, for example, Chap. 4 in Refs. [1] and [19].
}

Here $\\mathbf{k k}$ is also a tensor. Even when $\\mathbf{J}_{\\mathrm{ext}}=0$, for $\\mathbf{E} \\neq 0$ the following should be satisfied:


\\begin{equation*}
\\left|\\boldsymbol{\\epsilon}+\\left(\\frac{\\mathbf{k}^{2} c^{2}}{\\omega^{2}}\\right)\\left(\\frac{\\mathbf{k k}}{\\mathbf{k}^{2}}-\\mathbf{I}\\right)\\right|=|\\mathscr{D}|=0 \\tag{6.114}
\\end{equation*}


Here $\\mathscr{D}$ is the dispersion tensor. Equation (6.114) is the dispersion relation, and $\\boldsymbol{\\epsilon}$ shows the dielectric tensor.

From the dielectric tensor $\\epsilon$, the dielectric response function of $\\epsilon(k, \\omega)$ for the longitudinal waves is obtained:


\\begin{equation*}
\\epsilon(k, \\omega)=\\frac{\\mathbf{k} \\cdot \\boldsymbol{\\epsilon} \\cdot \\mathbf{k}}{\\mathbf{k}^{2}} \\tag{6.115}
\\end{equation*}


Different from Eq. (6.111), the induced current of $\\mathbf{J}_{1}$ is expressed explicitly as follows:


\\begin{align*}
& \\frac{\\partial \\mathbf{D}}{\\partial t}=\\mathbf{J}_{1}+\\epsilon_{0} \\frac{\\partial \\mathbf{E}}{\\partial t} \\\\
& \\therefore \\quad i \\omega \\epsilon_{0} \\boldsymbol{\\epsilon} \\cdot \\mathbf{E}=\\mathbf{J}_{1}-i \\omega \\epsilon_{0} \\mathbf{E} \\\\
& \\therefore \\quad \\mathbf{J}_{1}=-i \\omega \\epsilon_{0}(\\boldsymbol{\\epsilon}-\\mathbf{I}) \\cdot \\mathbf{E} \\tag{6.116}
\\end{align*}


For plasmas, which have no external magnetic field, the Vlasov equation is linearized by using $f=f_{0}+\\delta f$ as follows:

\\[
\\begin{array}{r}
\\frac{\\partial \\delta f}{\\partial t}+\\mathbf{v} \\cdot \\frac{\\partial \\delta f}{\\partial \\mathbf{x}}+\\frac{q}{m}\\left(\\mathbf{E}_{1}+\\mathbf{v} \\times \\mathbf{B}_{1}\\right) \\cdot \\frac{\\partial f_{0}}{\\partial \\mathbf{v}}=0 \\\\
\\therefore \\quad(-i \\omega+i \\mathbf{k} \\cdot \\mathbf{v}) \\delta f=-\\frac{q}{m} \\frac{\\partial f_{0}}{\\partial \\mathbf{v}} \\cdot\\left\\{\\frac{\\mathbf{k v}}{\\omega}+\\left(1-\\frac{\\mathbf{k} \\cdot \\mathbf{v}}{\\omega}\\right) \\mathbf{I}\\right\\} \\cdot \\mathbf{E}_{1} \\tag{6.117}
\\end{array}
\\]

Now $\\mathbf{J}_{1}$ is obtained:


\\begin{equation*}
\\mathbf{J}_{1}=n q \\int \\mathbf{v} \\delta f \\mathrm{~d} \\mathbf{v} \\tag{6.118}
\\end{equation*}


Together with Eq. (6.116), the dielectric tensor $\\boldsymbol{\\epsilon}$ is obtained:


\\begin{align*}
\\boldsymbol{\\epsilon}(\\mathbf{k}, \\omega) & \\left.=\\mathbf{I}-\\frac{\\omega_{p}^{2}}{\\omega^{2}} \\int \\mathrm{d} \\mathbf{v} \\frac{\\mathbf{v}}{\\mathbf{k} \\cdot \\mathbf{v}-\\omega} \\frac{\\partial f_{0}}{\\partial \\mathbf{v}} \\cdot[\\mathbf{k v}+(\\omega-\\mathbf{k} \\cdot \\mathbf{v}) \\mathbf{I})\\right] \\\\
& =\\mathbf{I}\\left(1-\\frac{\\omega_{p}^{2}}{\\omega^{2}}\\right)-\\frac{\\omega_{p}^{2}}{\\omega^{2}} \\int \\mathrm{d} \\mathbf{v} \\frac{\\mathbf{v} \\mathbf{v}}{\\mathbf{k} \\cdot \\mathbf{v}-\\omega} \\frac{\\partial f_{0}}{\\partial \\mathbf{v}} \\cdot \\mathbf{k} \\tag{6.119}
\\end{align*}


The dielectric tensor $\\epsilon$ is separated by the dielectric response function $\\epsilon$ for the longitudinal waves and $\\epsilon_{\\perp}$ for the transverse electromagnetic waves:


\\begin{equation*}
\\boldsymbol{\\epsilon}=\\epsilon \\mathbf{I}_{/ /}+\\epsilon_{\\perp} \\mathbf{I}_{\\perp} \\tag{6.120}
\\end{equation*}


Here $\\mathbf{I}_{/ /}$and $\\mathbf{I}_{\\perp}$ are defined as follows:


\\begin{equation*}
\\mathbf{I}_{/ /}=\\frac{\\mathbf{k} \\mathbf{k}}{\\mathbf{k}^{2}}, \\quad \\mathbf{I}_{\\perp}=\\mathbf{I}-\\mathbf{I}_{/ /} \\tag{6.121}
\\end{equation*}


Now the dispersion relation is described as follows by Eq. (6.114):


\\begin{align*}
& \\epsilon=0  \\tag{6.122}\\\\
& \\epsilon_{\\perp}-\\left(\\frac{k c}{\\omega}\\right)^{2}=0 \\tag{6.123}
\\end{align*}


Now we reproduce the dielectric response function in Eq. (6.66) for the static longitudinal waves. We assume again that the wave vector $\\mathbf{k}$ points to $z: \\mathbf{k}=k \\hat{z}$. By using Eqs. (6.119) and (6.115), we obtain Eq. (6.66). Because of $\\mathbf{k}=k \\hat{z}$, the component $\\epsilon_{z z}$ of the dielectric tensor $\\epsilon$ shows $\\epsilon$ for the longitudinal wave and should be Eq. (6.66). From Eqs. (6.119) and (6.115), $\\epsilon_{z z}=1-\\frac{\\omega_{p}^{2}}{\\omega^{2}}+\\frac{\\omega_{p}^{2}}{\\omega^{2}} \\int \\mathrm{d} v_{z} \\frac{v_{z}^{2}}{\\omega-k v_{z}} k \\frac{\\partial f_{0}}{\\partial v_{z}}$. Here $\\frac{v_{z}^{2}}{\\omega-k v_{z}}=-\\left(\\omega+k v_{z}-\\frac{\\omega^{2}}{\\omega-k v_{z}}\\right) / k^{2}$. Then we obtain the dielectric response function for the longitudinal wave: $\\epsilon_{z z}=1-\\frac{\\omega_{p}^{2}}{\\omega^{2}}-\\frac{\\omega_{p}^{2}}{k^{2} \\omega^{2}} \\int \\mathrm{d} v_{z}\\left(\\omega+k v_{z}\\right) k \\frac{\\partial f_{0}}{\\partial v_{z}}+$ $\\frac{\\omega_{p}^{2}}{k^{2}} \\int \\mathrm{d} v_{z} \\frac{1}{\\omega-k v_{z}} k \\frac{\\partial f_{0}}{\\partial v_{z}}=1-\\frac{\\omega_{p}^{2}}{\\omega^{2}}+\\frac{\\omega_{p}^{2}}{\\omega^{2}}+\\frac{\\omega_{p}^{2}}{k^{2}} \\int \\mathrm{d} v_{z} \\frac{k \\frac{\\partial f_{0}}{\\partial v_{z}}}{\\omega-k v_{z}}$. Now we reproduced Eq. (6.66).

\\subsection*{Dispersion Relation for Magnetized Uniform Plasma}
So far in the Chap. 6, we considered the dispersion relation for unmagnetized uniform plasmas. In the Sect. 6.11 the dispersion relation is considered for a magnetized plasma, which was considered by the macroscopic fluid model in Sect. 5.9. Similar to the way performed in Sect. 6.5 for the dispersion relation for electrostatic unmagnetized plasma, we employ the linearized Vlasov equation with the Maxwell equations, instead of the Poisson equation.

The Maxwell equations are transformed by the Fourier transform for $\\mathbf{r}$ and the Laplace transform for $t$. The results are written again below:


\\begin{align*}
\\mathbf{k} \\times \\mathbf{E}_{1} & =\\omega \\mathbf{B}_{1}  \\tag{6.124}\\\\
\\frac{i}{\\mu_{0}} \\mathbf{k} \\times \\mathbf{B}_{1} & =i \\epsilon_{0} \\boldsymbol{\\epsilon} \\omega \\mathbf{E}_{1}+\\mathbf{J}_{1} \\tag{6.125}
\\end{align*}


The current $\\mathbf{J}_{1}$ can be obtained by the following relation:


\\begin{equation*}
\\mathbf{J}_{1}(t, \\mathbf{r})=q \\int \\mathrm{d}^{3} \\mathbf{v} \\delta f(t, \\mathbf{r}, \\mathbf{v}) \\tag{6.126}
\\end{equation*}


When we have multi-species of charged particles in the plasma, the summation is taken over all the species: $q \\int \\mathrm{d}^{3} \\mathbf{v} \\delta f \\rightarrow \\sum_{j} q_{j} \\int \\mathrm{d}^{3} \\mathbf{v} \\delta f_{j}$. The linearized Vlasov equation is as follows:


\\begin{gather*}
\\left\\{\\frac{\\partial}{\\partial t}+(\\mathbf{v} \\cdot \\nabla)+\\left(\\mathbf{v} \\times \\mathbf{B}_{0}\\right) \\cdot \\frac{\\partial}{\\delta \\mathbf{v}}\\right\\} \\delta f \\\\
=-\\frac{q}{m}\\left(\\mathbf{E}_{1}+\\mathbf{v} \\times \\mathbf{B}_{1}\\right) \\cdot \\frac{\\partial f_{0}}{\\partial \\mathbf{v}} \\tag{6.127}
\\end{gather*}


If we do not have $\\mathbf{B}_{0}$ as in Sect. 6.5 for the dispersion relation for electrostatic unmagnetized plasma, the integration of $\\delta f$ is simple. However, now we have $\\mathbf{B}_{0}$. The integration of Eq. (6.127) is not so straightforward. However, as we see in Sect. 6.1.3, along the particle trajectory the distribution function $f_{0}$ is conserved as a constant of motion, and the left side of Eq. (6.127) can be written by a total derivative: $\\frac{\\mathrm{d} \\delta f}{\\mathrm{~d} t}$.

Now we focus on a component of the Fourier-Laplace transformation for $\\mathbf{E}_{1}$, $\\mathbf{B}_{1}$ and $\\delta f$. Then the linearized Vlasov equation of Eq. (6.127) is transformed as follows:


\\begin{align*}
& \\frac{\\mathrm{d}}{\\mathrm{d} t}\\{\\delta f \\exp (-i \\omega t+i \\mathbf{k} \\cdot \\mathbf{r})\\}=-\\frac{q}{m}\\left(\\mathbf{E}_{1}+\\mathbf{v} \\times \\mathbf{B}_{1}\\right) \\cdot \\frac{\\partial f_{0}}{\\partial \\mathbf{v}} \\exp (-i \\omega t+i \\mathbf{k} \\cdot \\mathbf{r}) \\\\
& \\quad=-\\frac{q}{m}\\left(\\mathbf{E}_{1}+\\frac{1}{\\omega} \\mathbf{v} \\times\\left(\\mathbf{k} \\times \\mathbf{E}_{1}\\right)\\right) \\cdot \\frac{\\partial f_{0}}{\\partial \\mathbf{v}} \\exp (-i \\omega t+i \\mathbf{k} \\cdot \\mathbf{r}) \\tag{6.128}
\\end{align*}


After integrating Eq. (6.128) by $t$, the following is obtained:


\\begin{align*}
& \\delta f(\\mathbf{k}, \\mathbf{v}, \\omega)=-\\frac{q}{m} \\int_{-\\infty}^{t} \\mathrm{~d} t^{\\prime} \\frac{\\partial f_{0}}{\\partial \\mathbf{v}^{\\prime}} \\cdot\\left(\\mathbf{E}_{1}+\\frac{1}{\\omega} \\mathbf{v}^{\\prime} \\times\\left(\\mathbf{k} \\times \\mathbf{E}_{1}\\right)\\right) \\\\
& \\exp \\left(i \\omega\\left(t-t^{\\prime}\\right)-i \\mathbf{k} \\cdot\\left(\\mathbf{r}-\\mathbf{r}^{\\prime}\\right)\\right) \\\\
&=- \\frac{q}{m} \\int_{-\\infty}^{t} \\mathrm{~d} t^{\\prime} \\frac{\\partial f_{0}}{\\partial \\mathbf{v}^{\\prime}} \\cdot\\left[\\mathbf{E}_{1}+\\frac{1}{\\omega}\\left\\{\\mathbf{k}\\left(\\mathbf{v}^{\\prime} \\cdot \\mathbf{E}_{1}\\right)-\\left(\\mathbf{k} \\cdot \\mathbf{v}^{\\prime}\\right) \\mathbf{E}_{1}\\right\\}\\right] \\\\
& \\exp \\left(i \\omega\\left(t-t^{\\prime}\\right)-i \\mathbf{k} \\cdot\\left(\\mathbf{r}-\\mathbf{r}^{\\prime}\\right)\\right) \\\\
&=- \\frac{q}{m} \\int_{-\\infty}^{t} \\mathrm{~d} t^{\\prime} \\frac{\\partial f_{0}}{\\partial \\mathbf{v}^{\\prime}} \\cdot\\left\\{\\mathbf{I}\\left(1-\\frac{\\mathbf{k} \\cdot \\mathbf{v}^{\\prime}}{\\omega}\\right)+\\frac{\\mathbf{k v}^{\\prime}}{\\omega}\\right\\} \\cdot \\mathbf{E}_{1} \\\\
& \\exp \\left(i \\omega\\left(t-t^{\\prime}\\right)-i \\mathbf{k} \\cdot\\left(\\mathbf{r}-\\mathbf{r}^{\\prime}\\right)\\right) \\\\
&=-\\frac{q}{m} \\int_{-\\infty}^{t} \\mathrm{~d} t^{\\prime} \\frac{\\partial f_{0}}{\\partial \\mathbf{v}^{\\prime}} \\cdot \\mathfrak{T} \\cdot \\mathbf{E}_{1} \\exp \\left(i \\omega\\left(t-t^{\\prime}\\right)-i \\mathbf{k} \\cdot\\left(\\mathbf{r}-\\mathbf{r}^{\\prime}\\right)\\right) \\tag{6.129}
\\end{align*}


In Eq. (6.129), the tensor of $\\mathfrak{T}$ is as follows:


\\begin{equation*}
\\mathfrak{T}=\\mathbf{I}\\left(1-\\frac{\\mathbf{k} \\cdot \\mathbf{v}^{\\prime}}{\\omega}\\right)+\\frac{\\mathbf{k} \\mathbf{v}^{\\prime}}{\\omega} \\tag{6.130}
\\end{equation*}


The unit tensor is denoted by $\\mathbf{I}$. Here $\\mathbf{v}^{\\prime}$ and $\\mathbf{r}^{\\prime}$ should be the particle velocity and position along the trajectory, respectively. We assume that the external constant uniform magnetic field $\\left(\\mathbf{B}_{0}\\right)$ is parallel to $z$. When $\\mathbf{v}^{\\prime}=\\left(v_{x}^{\\prime}, v_{y}^{\\prime}, v_{z}^{\\prime}\\right)=\\mathbf{v}(t)=$ $\\left(v_{\\perp} \\cos (\\theta), v_{\\perp} \\sin (\\theta), v_{\\|}\\right)$at $t^{\\prime}=t, \\mathbf{v}=\\mathbf{v}^{\\prime} \\exp \\left(-i \\Omega\\left(t-t^{\\prime}\\right)\\right)$ (see also Eq. (3.6)). The velocity $\\mathbf{v}^{\\prime}$ and the position $\\mathbf{r}^{\\prime}$ are expressed as follows:


\\begin{align*}
\\mathbf{v}^{\\prime} & =\\left(v_{x}^{\\prime}, v_{y}^{\\prime}, v_{z}^{\\prime}\\right)=\\left(v_{\\perp} \\cos (\\Omega \\tau+\\theta), v_{\\perp} \\sin (\\Omega \\tau+\\theta), v_{\\|}\\right)  \\tag{6.131}\\\\
\\mathbf{r}^{\\prime} & =\\left(x^{\\prime}, y^{\\prime}, z^{\\prime}\\right) \\\\
& =\\left(x-\\frac{v_{\\perp}}{\\Omega}\\left\\{\\sin ((\\Omega \\tau+\\theta)-\\sin (\\theta)\\}, y+\\frac{v_{\\perp}}{\\Omega}\\{\\cos (\\Omega \\tau+\\theta)-\\cos (\\theta)\\}, z-v_{\\|} \\tau\\right)\\right. \\tag{6.132}
\\end{align*}


Here $\\tau=t-t^{\\prime}, \\Omega=q B_{0} / m$, and $v_{\\|}$and $v_{\\perp}$ are the parallel and transverse components of the the particle velocity $\\mathbf{v}$, respectively. The spiral motion along the magnetic field $B_{0}$ is expressed, and its trajectory is a characteristic line. Along the particle trajectory, $f_{0}$ is a constant of motion.

Then the following is obtained:


\\begin{align*}
\\frac{\\partial f_{0}}{\\partial \\mathbf{v}} & =\\frac{\\partial f_{0}}{\\partial v_{\\perp}} \\mathbf{e}_{\\perp}+\\frac{\\partial f_{0}}{\\partial v_{\\|}} \\mathbf{e}_{z} \\\\
& =\\frac{\\partial f_{0}}{\\partial v_{\\perp}}\\left\\{\\cos (\\Omega \\tau+\\theta) \\mathbf{e}_{x}+\\sin (\\Omega \\tau+\\theta) \\mathbf{e}_{y}\\right\\}+\\frac{\\partial f_{0}}{\\partial v_{\\|}} \\mathbf{e}_{z} \\tag{6.133}
\\end{align*}


Fig. 6.5 Vector arrangement employed in the Sect. 6.11. The uniform constant magnetic field is in $z$ and the wave vector $\\mathbf{k}$ is located in the $x-z$ plane

\\begin{center}
\\includegraphics[max width=\\textwidth]{2024_02_26_83e36543483eb7d284c1g-150}
\\end{center}

Here $\\mathbf{e}_{\\perp}, \\mathbf{e}_{x}, \\mathbf{e}_{y}$ and $\\mathbf{e}_{z}$ show the unit vectors in $v_{\\perp}, x, y$ and $z$, respectively. Without loss of generality, we choose the wave vector $\\mathbf{k}$ in the $x-z$ plane: $\\mathbf{k}=$ $\\left(k_{\\perp}\\left(=k_{x}\\right), 0, k_{\\|}\\right)\\left(=k_{z}\\right)$, as shown in Fig. 6.5. Then the tensor $\\mathfrak{T}$ in Eq. (6.130) is described as follows:


\\begin{align*}
\\mathfrak{T} & =\\mathbf{I}\\left(1-\\frac{\\mathbf{k} \\cdot \\mathbf{v}^{\\prime}}{\\omega}\\right)+\\frac{\\mathbf{k v}^{\\prime}}{\\omega} \\\\
& =\\left[\\begin{array}{ccc}
1-\\frac{k_{\\|} v_{\\|}^{\\prime}}{\\omega}, & \\frac{k_{\\perp} v_{y}^{\\prime}}{\\omega}\\left(=\\frac{k_{\\perp} v_{\\perp}^{\\prime} \\sin (\\Omega \\tau+\\theta)}{\\omega}\\right), & \\frac{k_{\\perp} v_{\\|}^{\\prime}}{\\omega} \\\\
0, & 1-\\frac{\\mathbf{k} \\cdot \\mathbf{v}^{\\prime}}{\\omega}\\left(=1-\\frac{k_{\\perp} v_{\\perp}^{\\prime} \\cos (\\Omega t+\\theta)+k_{\\|} v_{\\|}^{\\prime}}{\\omega}\\right), & 0 \\\\
\\frac{k_{\\|} v_{\\|}^{\\prime} \\cos (\\Omega \\tau+\\theta)}{\\omega}, & \\frac{k_{\\|} v_{\\perp}^{\\prime} \\sin (\\Omega \\tau+\\theta)}{\\omega}, & 1-\\frac{k_{\\perp} v_{\\perp}^{\\prime} \\cos (\\Omega t+\\theta)}{\\omega}
\\end{array}\\right] \\tag{6.134}
\\end{align*}


The integrand of Eq. (6.129) is written as follows:


\\begin{align*}
& \\frac{\\partial f_{0}}{\\partial \\mathbf{v}^{\\prime}} \\cdot\\left\\{\\mathbf{E}_{1}+\\frac{1}{\\omega} \\mathbf{v}^{\\prime} \\times\\left(\\mathbf{k} \\times \\mathbf{E}_{1}\\right)\\right\\} \\\\
= & \\left.\\frac{\\partial f_{0}}{\\partial \\mathbf{v}^{\\prime}} \\cdot\\left[\\mathbf{E}_{1}+\\frac{1}{\\omega}\\left\\{\\mathbf{k}\\left(\\mathbf{v}^{\\prime} \\cdot \\mathbf{E}_{1}\\right)-\\left(\\mathbf{k} \\cdot \\mathbf{v}^{\\prime}\\right) \\mathbf{E}_{1}\\right)\\right\\}\\right] \\\\
= & \\frac{\\partial f_{0}}{\\partial \\mathbf{v}^{\\prime}} \\cdot \\mathfrak{T} \\cdot \\mathbf{E}_{1} \\\\
(= & \\left\\{\\frac{\\partial f_{0}}{\\partial v_{\\perp}}\\left(1-\\frac{k_{\\|} v_{\\|}^{\\prime}}{\\omega}\\right)+\\frac{\\partial f_{0}}{\\partial v_{\\|}} \\frac{k_{\\|} v_{\\perp}^{\\prime}}{\\omega}\\right\\} E_{1 x} \\cos (\\Omega \\tau+\\theta)+ \\\\
& \\left\\{\\frac{\\partial f_{0}}{\\partial v_{\\perp}}\\left(1-\\frac{k_{\\|} v_{\\|}^{\\prime}}{\\omega}\\right)+\\frac{\\partial f_{0}}{\\partial v_{\\|}} \\frac{k_{\\|} v_{\\perp}^{\\prime}}{\\omega}\\right\\} E_{1 y} \\sin (\\Omega \\tau+\\theta)+ \\\\
& \\left.\\left\\{\\frac{\\partial f_{0}}{\\partial v_{\\perp}} \\frac{k_{\\perp} v_{\\perp}^{\\prime}}{\\omega} \\cos (\\Omega \\tau+\\theta)+\\frac{\\partial f_{0}}{\\partial v_{\\|}}\\left(1-\\frac{k_{\\perp} v_{\\perp}^{\\prime}}{\\omega} \\cos (\\Omega \\tau+\\theta)\\right)\\right\\} E_{1 z}\\right) \\tag{6.135}
\\end{align*}


The exponential factor in the integrand in Eq. (6.129) is written as follows:


\\begin{align*}
& \\exp \\left\\{i \\omega \\tau-i \\mathbf{k} \\cdot\\left(\\mathbf{r}-\\mathbf{r}^{\\prime}\\right)\\right\\} \\\\
& =\\exp \\left[i \\omega \\tau-i \\frac{k_{\\perp} v_{\\perp}}{\\Omega}\\{\\sin (\\Omega \\tau+\\theta)-\\sin (\\theta)\\}-i k_{\\|} v_{\\|} \\tau\\right] \\\\
& =\\sum_{n=-\\infty}^{+\\infty} \\sum_{n^{\\prime}=-\\infty}^{+\\infty} J_{n}\\left(\\frac{k_{\\perp} v_{\\perp}}{\\Omega}\\right) J_{n^{\\prime}}\\left(\\frac{k_{\\perp} v_{\\perp}}{\\Omega}\\right) \\\\
& \\quad \\times \\exp \\left\\{-i n(\\Omega \\tau+\\theta)+i n^{\\prime} \\theta+i \\omega \\tau-i k_{\\|} v_{\\|} \\tau\\right\\} \\tag{6.136}
\\end{align*}


Here we used the following relation with the Bessel function of $J_{n}$ of the $n$th order:


\\begin{equation*}
\\exp (-i \\psi \\sin \\theta)=\\sum_{n=-\\infty}^{+\\infty} J_{n}(\\psi) \\exp (-i n \\theta) \\tag{6.137}
\\end{equation*}


Inserting Eqs. (6.135) and (6.136) into Eq. (6.129), now we can calculate $\\delta f(\\mathbf{k}, \\mathbf{v}, \\omega)$, from which $\\mathbf{J}_{1}$ in Eq. (6.118) is also calculated. Then Eq. (6.116) provides the dielectric tensor $\\boldsymbol{\\epsilon}$.

When $\\delta f(\\mathbf{k}, \\mathbf{v}, \\omega)$ is obtained, the following relations on the Bessel function are used:

\\[
\\begin{array}{r}
J_{n-1}(\\psi)+J_{n+1}(\\psi)=\\frac{2 n}{\\psi} J_{n}(\\psi) \\\\
J_{n-1}(\\psi)-J_{n+1}(\\psi)=2 J_{n}^{\\prime}(\\psi) \\tag{6.139}
\\end{array}
\\]

Here $\\psi=\\frac{k_{\\perp} v_{\\perp}}{\\Omega}$ and $J_{n}^{\\prime}(\\psi)=\\frac{\\mathrm{d} J_{n}(\\psi)}{\\mathrm{d} \\psi}$. For the integration over $\\mathbf{v}$ in Eq. (6.118), the following expression is convenient: $\\int \\mathrm{d}^{3} \\mathbf{v}=\\int_{0}^{2 \\pi} \\mathrm{d} \\theta \\int_{0}^{\\infty} \\mathrm{d} v_{\\perp} v_{\\perp} \\int_{-\\infty}^{\\infty} \\mathrm{d} v_{\\|}$. Here the definition of $\\theta$ in Fig. 6.5 should be again reminded. At the integration over $\\theta$ the following selection rules are used:

\\[
\\int_{0}^{2 \\pi} \\mathrm{d} \\theta \\exp (i(l-m) \\theta)= \\begin{cases}0, & \\text { for } l \\neq m  \\tag{6.140}\\\\ 2 \\pi, & \\text { for } l=m\\end{cases}
\\]

Now we obtain the dielectric tensor $\\boldsymbol{\\epsilon}$ for a uniform magnetized plasma:


\\begin{equation*}
\\boldsymbol{\\epsilon}=\\left(1-\\frac{\\omega_{p}^{2}}{\\omega^{2}}\\right) \\mathbf{I}-\\frac{\\omega_{p}^{2}}{\\omega^{2}} \\sum_{n=-\\infty}^{+\\infty} \\int \\mathrm{d} \\mathbf{V}\\left(\\frac{n \\Omega}{v_{\\perp}} \\frac{\\partial f_{0}}{\\partial v_{\\perp}}+k_{\\|} \\frac{\\partial f_{0}}{\\partial v_{\\|}}\\right) \\frac{\\Pi}{n \\Omega+k_{\\|} v_{\\|}-\\omega} \\tag{6.141}
\\end{equation*}


Here $\\int \\mathrm{d} \\mathbf{V}=2 \\pi \\int_{0}^{\\infty} \\mathrm{d} v_{\\perp} v_{\\perp} \\int_{-\\infty}^{\\infty} \\mathrm{d} v_{\\|}$.

\\[
\\Pi=\\left[\\begin{array}{ccc}
\\left(\\frac{n \\Omega}{k_{\\perp}}\\right)^{2} J_{n}^{2}, & i v_{\\perp} \\frac{n \\Omega}{k_{\\perp}} J_{n} J_{n}^{\\prime}, & v_{\\|} \\frac{n \\Omega}{k_{\\perp}} J_{n}^{2}  \\tag{6.142}\\\\
-i v_{\\perp} \\frac{n \\Omega}{k_{\\perp}} J_{n} J_{n}^{\\prime}, & v_{\\perp}^{2} J_{n}^{\\prime 2}, & -i v_{\\perp} v_{\\|} J_{n} J_{n}^{\\prime} \\\\
v_{\\|} \\frac{n \\Omega}{k_{\\perp}} J_{n}^{2}, & i v_{\\perp} v_{\\|} J_{n} J_{n}^{\\prime}, & v_{\\|}^{2} J_{n}^{2}
\\end{array}\\right]
\\]

Here $J_{n}=J_{n}(\\psi)$ and $\\psi=\\frac{k_{\\perp} v_{\\perp}}{\\Omega}$. When Eq. (6.141) is obtained, the following relation is used: $\\frac{\\omega-k_{\\|} v_{\\|}}{\\omega-k_{\\|} v_{\\|}-n \\Omega}=1+\\frac{n \\Omega}{\\omega-k_{\\|} v_{\\|}-n \\Omega}$ and $\\frac{\\omega-n \\Omega}{\\omega-k_{\\|} v_{\\|}-n \\Omega}=1+$ $\\frac{k_{\\|} v_{\\|}}{\\omega-k_{\\|} v_{\\|}-n \\Omega}$. When we obtain Eq. (6.142), the following relations are also needed: $\\sum_{n=-\\infty}^{+\\infty} J_{n}^{2}(z)=1, \\quad \\sum_{n=-\\infty}^{+\\infty} J_{n}(z) J_{n}^{\\prime}(z)=0, \\quad \\sum_{n=-\\infty}^{+\\infty} n J_{n}^{2}(z)=0, \\quad \\sum_{n=-\\infty}^{+\\infty} n J_{n}(z) J_{n}^{\\prime}(z)=0$ and $\\sum_{n=-\\infty}^{+\\infty} n^{2} J_{n}^{2}(z)=z^{2} / 2$. The detailed derivation can be found in Chap. 4 in Ref. [1], Chap. 6 in Ref. [3], Chap. 12 in Ref. [18], Chap. 10 in Refs. [19, 20] and Chap. 8 in Ref. [21]. Then the dispersion relation is obtained by Eq. (6.114):


\\begin{gather*}
\\left|\\boldsymbol{\\epsilon}+\\left(\\frac{\\mathbf{k}^{2} c^{2}}{\\omega^{2}}\\right)\\left(\\frac{\\mathbf{k k}}{\\mathbf{k}^{2}}-\\mathbf{I}\\right)\\right|=\\left|\\boldsymbol{\\epsilon}-\\left(\\frac{\\mathbf{k}^{2} c^{2}}{\\omega^{2}}\\right) \\mathbf{I}_{\\perp}\\right|=|\\mathscr{D}|=0  \\tag{6.114}\\\\
\\mathbf{I}_{\\perp}=\\mathbf{I}-\\mathbf{I}_{\\|}=\\mathbf{I}-\\frac{\\mathbf{k k}}{\\mathbf{k}^{2}} \\\\
=\\left[\\begin{array}{ccc}
1-\\frac{k_{\\perp}^{2}}{k^{2}}\\left(=\\frac{k_{\\|}^{2}}{k_{\\perp}^{2}+k_{\\|}^{2}}\\right), & 0, & \\frac{k_{\\perp} k_{\\|}}{k_{\\perp}^{2}+k_{\\|}^{2}} \\\\
0, & 1, & 0 \\\\
-\\frac{k_{\\perp} k_{\\|}}{k_{\\perp}^{2}+k_{\\|}^{2}}, & 0, & 1-\\frac{k_{\\|}^{2}}{k^{2}}\\left(=\\frac{k_{\\perp}^{2}}{k_{\\perp}^{2}+k_{\\|}^{2}}\\right)
\\end{array}\\right] \\tag{6.143}
\\end{gather*}


When we calculate further on plasma behavior, the distribution function of $f_{0}$ should be given. The dispersion relation is also found in Chap. 4 in Ref. [1], Chap. 6 in Ref. [3], Chap. 12 in Ref. [18] for the Maxwell distributions.

\\subsection*{Waves in Magnetized Uniform Plasma}
Now we consider a wave propagating to the magnetic field of $B_{0} \\mathbf{e}_{z}$. In this case the wave vector $\\mathbf{k} \\| \\mathbf{e}_{z}$ and $k_{\\perp} \\rightarrow 0: \\mathbf{k}=\\left(0,0, k_{\\|}\\right)$and $\\psi=\\frac{k_{\\perp} v_{\\perp}}{\\Omega} \\rightarrow 0$. The Bessel function $J_{n}(\\Psi)$ is approximately written as follows for a small $\\Psi$ :


\\begin{equation*}
J_{n}(\\Psi) \\sim \\frac{\\Psi^{n}}{2^{n} n !}\\left\\{1-\\frac{\\Psi^{2}}{2(2 n+2)}+\\cdots\\right\\} \\tag{6.144}
\\end{equation*}


Therefore, $J_{0}=1, \\lim _{\\Psi \\rightarrow 0} J_{1}(\\Psi) \\sim \\Psi / 2$ and $\\lim _{\\Psi \\rightarrow 0} J_{n}(\\Psi) \\sim \\lim _{\\Psi \\rightarrow 0} \\frac{\\Psi^{n}}{2^{n} n !} \\rightarrow 0$ for $|n|>$ 1. In addition, $J_{-n}(\\Psi)=(-1)^{n} J_{n}(\\Psi)$ and $J_{n}^{\\prime}=\\left(J_{n-1}-J_{n+1}\\right) / 2$. Now the tensor $\\Pi$ in Eq. (6.142) becomes as follows in this case:

\\[
\\boldsymbol{\\Pi}=\\left[\\begin{array}{ccc}
\\frac{v_{\\perp}^{2}}{4}\\left(\\delta_{n, 1}+\\delta_{n,-1}\\right), & i \\frac{v_{\\perp}^{2}}{4}\\left(\\delta_{n, 1}-\\delta_{n,-1}\\right), & 0  \\tag{6.145}\\\\
-i \\frac{v_{\\perp}^{2}}{4}\\left(\\delta_{n, 1}-\\delta_{n,-1}\\right), & \\frac{v_{\\perp}^{2}}{4}\\left(\\delta_{n, 1}+\\delta_{n,-1}\\right), & 0 \\\\
0, & 0, & v_{\\|}^{2} \\delta_{n, 0}
\\end{array}\\right]
\\]

Here $\\delta_{n, m}$ is the Kronecker delta, and $\\delta_{n, m}=1$ for $n=m$ and 0 for $n \\neq m$.

The component of $\\mathscr{D}_{z z}$ should express longitudinal waves, including plasma oscillation shown in Sects. 5.4, 6.6 and 6.7. When we focus on the Alfvén wave, the dispersion relation should be $\\mathscr{D}_{x x} \\mathscr{D}_{y y}-\\mathscr{D}_{x y} \\mathscr{D}_{y x}=0$, because in our case $\\mathbf{k}=\\left(0,0, k_{\\|}\\right)$ and the tensor $\\mathbf{I}_{\\perp}$ is as follows:

\\[
\\mathbf{I}_{\\perp}=\\left[\\begin{array}{l}
1,0,0  \\tag{6.146}\\\\
0,1,0 \\\\
0,0,0
\\end{array}\\right]
\\]

Here for simplicity we consider a low temperature limit of $T \\rightarrow 0$. So the distribution function $f_{0}$ is described by the delta functions: $f_{0} \\propto \\delta\\left(v_{\\perp}\\right) \\delta\\left(v_{\\|}\\right)$. Then we obtain $\\mathscr{D}_{x x}=\\mathscr{D}_{y y}=1-\\frac{\\omega_{p}^{2}}{\\omega^{2}-\\Omega^{2}}$ and $\\mathscr{D}_{x y}=\\mathscr{D}_{y x}=i \\frac{\\omega \\Omega}{\\omega^{2}-\\Omega^{2}} \\frac{\\omega_{p}^{2}}{\\omega^{2}}$. Now we obtain the dispersion relation for waves propagating along $z: \\mathscr{D}_{x x} \\mathscr{D}_{y y}-\\mathscr{D}_{x y} \\mathscr{D}_{y x}=\\mathscr{D}_{x x}^{2}-$ $\\mathscr{D}_{x y}^{2}=\\left(\\mathscr{D}_{x x}+i \\mathscr{D}_{x y}\\right)\\left(\\mathscr{D}_{x x}-i \\mathscr{D}_{x y}\\right)=0$. The dispersion relations are obtained for a cold uniform magnetized plasma as shown below:


\\begin{equation*}
\\left(\\frac{k_{\\|} c}{\\omega}\\right)^{2}=1-\\frac{\\omega_{p}^{2} / \\omega^{2}}{1 \\pm \\Omega / \\omega} \\tag{6.147}
\\end{equation*}


Here $\\omega_{p}$ shows the plasma frequency, the solution with the + sign shows the right hand polarization ( $R$ wave) and the minus sign shows the left hand polarization ( $L$ wave). When $f_{0}$ is the distribution for electrons, $\\Omega=\\Omega_{e}$ is nega-
tive, and the electron rotation direction around the magnetic field $B_{z}$ is the same with the $R$ wave rotation direction. Therefore, electrons can resonate with the $R$ wave. At the resonance frequency $\\omega=\\left|\\Omega_{e}\\right|, k_{\\|} \\rightarrow \\infty$ and the $R$ wave is resonantly absorbed by the electrons. The energy absorption by charged particles in the magnetized plasma would induce the wave damping, called the cyclotron damping (see, for example, Chap. 6 in Ref. [3], Chap. 11 in Ref. [18] and Chap. 10 in Ref. [19]) .

Equation (6.147) also shows that the $R$ wave has a larger phase speed than that of the $L$ wave for high-frequency waves. By this effect, which is called the Faraday rotation, one can measure the plasma electron density.

At the low-frequency region $\\left(\\omega_{p e}^{2}, \\Omega_{e}^{2} \\gg \\omega^{2}\\right)$ for the $R$ wave, the dispersion relation becomes as follows for the long wavelength limit:


\\begin{equation*}
\\omega=\\frac{\\left|\\Omega_{e}\\right| k_{\\|}^{2} c^{2}}{k_{\\|}^{2} c^{2}+\\omega_{p}^{2}} \\sim \\frac{k_{\\|}^{2} c^{2}}{\\omega_{p}^{2}}\\left|\\Omega_{e}\\right| \\tag{6.148}
\\end{equation*}


The wave in Eq. (6.148) is called the helicon wave, which can propagate in plasmas at the low frequency [22, 23].

So far in this Sect. 6.12, we have focused on electron plasmas. When we consider ions in plasma, the dispersion relation in Eq. (6.147) should be considered for ion modes: $\\omega_{p}=\\omega_{p i}$ and $\\Omega=\\Omega_{i}$. For the low-frequency region for ion modes, we find a wave in the conditions of $\\omega \\ll \\Omega_{i}$. Then one can find the following Alfvén mode, which was considered in Eq. (5.138):


\\begin{equation*}
\\frac{\\omega^{2}}{k_{\\|}^{2}} \\sim \\frac{\\omega}{\\Omega_{i}} v_{A}^{2} \\tag{6.149}
\\end{equation*}


Here at the right side of Eq. (6.147), the first term is ignored and $1 \\ll|\\Omega| / \\omega$. In addition $\\Omega_{i} c^{2} / \\omega_{p i}=v_{A}^{2} / \\Omega_{i}$ was used.

Here we performed the analysis as one component plasma approximately. When we need to analyze plasma precisely, both the ions and electrons are included at the same time. So far we introduced the plasma waves by an approximate one-component plasma.

Now we move to waves propagating perpendicular to $B_{z}$ in a uniform magnetized plasma: $\\mathbf{k}=\\left(k_{\\perp}\\left(=k_{x}\\right), 0, k_{\\|}=0\\right)$. In this case, $\\mathscr{D}_{x z}=0, \\mathscr{D}_{z x}=0, \\mathscr{D}_{y z}=0$ and $\\mathscr{D}_{z y}=0$. Then we obtain the dispersion tensor as follows:

\\[
\\mathscr{D}=\\boldsymbol{\\epsilon}-\\left(\\frac{\\mathbf{k}^{2} c^{2}}{\\omega^{2}}\\right) \\mathbf{I}_{\\perp}=\\left[\\begin{array}{ccc}
\\epsilon_{x x}, & -i \\epsilon_{x y}, & 0  \\tag{6.150}\\\\
i \\epsilon_{y x}, & \\epsilon_{y y}-\\frac{k_{\\perp}^{2} c^{2}}{\\omega^{2}}, & 0 \\\\
0, & 0, & \\epsilon_{z z}-\\frac{k_{\\perp}^{2} c^{2}}{\\omega^{2}}
\\end{array}\\right]
\\]

The dispersion relations are $\\epsilon_{x x}\\left(\\epsilon_{y y}-\\frac{k_{\\perp}^{2} c^{2}}{\\omega^{2}}\\right)-\\epsilon_{x y}^{2}=0$ and $\\epsilon_{z z}-\\frac{k_{\\perp}^{2} c^{2}}{\\omega^{2}}=0$. Here $\\epsilon_{z z}-\\frac{k_{\\perp}^{2} c^{2}}{\\omega^{2}}=0$ presents the ordinary wave (the $O$ wave). The relation of $\\epsilon_{x x}\\left(\\epsilon_{y y}-\\frac{k_{\\perp}^{2} c^{2}}{\\omega^{2}}\\right)-\\epsilon_{x y}^{2}=0$ includes the extraordinary wave (the $X$ wave), in which $\\mathbf{E}_{1}$ is in the $x-y$ plain. When $\\mathbf{E}_{1}\\left\\|x, \\mathbf{k}=\\left(k_{\\perp}\\left(=k_{x}\\right), 0,0\\right)\\right\\| \\mathbf{E}_{1}$, which means a longitudinal mode. For this mode, the dispersion relation becomes $\\epsilon_{x x}=0$. The longitudinal mode is called as the Bernstein mode, which can propagate in a high-density plasma (see, for example, Chap. 4 in Ref. [1], Chap. 6 in Ref. [3], Chap. 12 in Ref. [18], Chap. 11 in Refs. [19, 20, 24]). Let us now borrow the result for $\\epsilon_{x x}=0$ from Chap. 4 in Ref. [1], Chap. 12 in Ref. [18] or Chap. 10 in Ref. [19] for the Maxwell distribution of $f_{0}=\\left(\\frac{m}{2 \\pi T}\\right)^{3 / 2} \\exp \\left\\{-\\frac{m\\left(v_{\\perp}^{2}+v_{\\|}^{2}\\right)}{2 T}\\right\\}$. The summary of the derivation for $\\epsilon_{x x}$ is also shown in Appendix G.


\\begin{equation*}
\\epsilon_{x x} \\sim 1-\\frac{k_{D}^{2} \\Omega^{2}}{k_{\\perp}^{2} \\omega} \\sum_{n=-\\infty}^{+\\infty} \\frac{n^{2}}{\\omega-n \\Omega} I_{n}\\left(\\frac{k_{\\perp}^{2} T}{m \\Omega^{2}}\\right) \\exp \\left(-\\frac{k_{\\perp}^{2} T}{m \\Omega^{2}}\\right)\\left(1-W\\left(\\frac{\\omega-n \\Omega}{k_{\\|} \\sqrt{T / m}}\\right)\\right) \\tag{6.151}
\\end{equation*}


Here $I_{n}$ shows the modified Bessel function of the $n$th order (see, for example, Chap. 10 in Ref. [25]). We are now considering the Bernstein wave propagating to the perpendicular direction to $B_{0}$. So $k_{\\|} \\rightarrow 0$, and then $W \\rightarrow 0$. The dispersion relation for the Bernstein mode was studied in detail in Ref. [24]. The relation $\\epsilon_{x x}=0$ provides the following:


\\begin{align*}
0 & \\sim 1-\\frac{k_{D}^{2}}{k_{\\perp}^{2} \\omega} \\sum_{n=-\\infty}^{+\\infty} \\frac{n^{2} \\Omega^{2}}{\\omega-n \\Omega} I_{n}\\left(\\frac{k_{\\perp}^{2} T}{m \\Omega^{2}}\\right) \\exp \\left(-\\frac{k_{\\perp}^{2} T}{m \\Omega^{2}}\\right) \\\\
& \\sim 1-\\frac{k_{D}^{2}}{k_{\\perp}^{2}} \\sum_{n=1}^{+\\infty} \\frac{2 n^{2} \\Omega^{2}}{\\omega^{2}-n^{2} \\Omega^{2}} I_{n}\\left(\\frac{k_{\\perp}^{2} T}{m \\Omega^{2}}\\right) \\exp \\left(-\\frac{k_{\\perp}^{2} T}{m \\Omega^{2}}\\right) \\tag{6.152}
\\end{align*}


Bernstein analyzed the waves in Ref. [24] and found allowable waves at multiple frequencies separating with the cyclotron frequency $\\Omega$. The discussion is valid for electrons and ions in plasmas.

Now again for simplicity we consider a low temperature limit of $T \\rightarrow 0$. In this case $\\epsilon_{x x}=1-\\frac{\\omega_{p}^{2}}{\\omega^{2}-\\Omega^{2}}, \\epsilon_{x y}=\\frac{\\omega_{p}^{2} \\Omega}{\\omega\\left(\\omega^{2}-\\Omega^{2}\\right)}=\\epsilon_{y x}, \\epsilon_{y y}=\\epsilon_{x x}$ and $\\epsilon_{z z}=1-\\frac{\\omega_{p}^{2}}{\\omega^{2}}$.

From $\\epsilon_{z z}-\\frac{k_{\\perp}^{2} c^{2}}{\\omega^{2}}=1-\\frac{\\omega_{p}^{2}}{\\omega^{2}}-\\frac{k_{\\perp}^{2} c^{2}}{\\omega^{2}}=0$, the following relation is easily obtained:


\\begin{equation*}
\\omega^{2}=\\omega_{p}^{2}+k_{\\perp}^{2} c^{2} \\tag{6.153}
\\end{equation*}


The wave is the ordinary wave (the $O$ wave) propagating transversely to $B_{z}$. The $O$ wave appears from the dispersion relation of $\\mathscr{D}_{z z}=0$, and so $\\mathbf{E}_{1}$ should be parallel to $z$ and the magnetic field $B_{z}$. Equation (6.153) shows that the $O$ wave is not affected by $B_{z}$.

The detailed discussions are also found in Chap. 4 in Ref. [1], Chap. 6 in Ref. [3], Chap. 12 in Ref. [18], Chaps. 10 and 11 in Refs. [19] and [20] for the waves in magnetized plasmas.

\\section*{References}
\\begin{enumerate}
  \\item S. Ichimaru, Statistical Plasma Physics, Vol. 1: Basic Principles (CRC Press, Boca Raton, 2004)

  \\item S. Ichimaru, Statistical Plasma Physics, Vol. 2: Condensed Plasmas (CRC Press, Boca Raton, 2004)

  \\item D.R. Nicholson, Introduction to Plasma Theory (Wiley, New York, 1983)

  \\item Yu.L. Klimontovich, The Statistical Theory of Non-equilibrium Processes in a Plasma (Pergamon Press, Oxford, 2013)

  \\item L.D. Landau, E.M. Lifshitz, Statistical Physics Part 1 (Pergamon Press Ltd., Oxford, 1980)

  \\item N.N. Bogoliubov, Problems of Dynamic Theory in Statistical Physics (Technical Information Service, United States Atomic Energy Commission, Oak Ridge, Tennessee, 1960). Available from Office of Technical Services, Department of Commerce, Washington, D.C

  \\item M. Born, H.S. Green, A General Kinetic Theory of Liquids (Cambridge University Press, Cambridge, 1949); A general kinetic theory of liquids. I. The molecular distribution functions. Proc. R. Soc. Lond. A 188, 10-18 (1946); H.S. Green, A general kinetic theory of liquids. II. Equilibrium properties. Proc. R. Soc. Lond. A 189, 103-117 (1947); M. Born, H.S. Green, A general kinetic theory of liquids. III. Dynamical properties. Proc. R. Soc. Lond. A 190, 455-474 (1947)

  \\item J.G. Kirkwood, The statistical mechanical theory of transport processes. I. General theory. J. Chem. Phys. 14, 180-201 (1946); The statistical mechanical theory of transport processes. II. Transport in gases. J. Chem. Phys. 15, 72-76 (1947)

  \\item J. Yvon, La théorie statistique des fluides et l'équation d'état. Actualités scientifiques et industrielles (1935)

  \\item L.D. Landau, E.M. Lifshitz, Physical Kinetics (Pergamon Press Ltd., Oxford, 1981)

  \\item M.N. Rosenbluth, W.M. MacDonald, D.L. Judd, Fokker-Planck equation for an inverse-square force. Phys. Rev. 107, 1-6 (1957). \\href{https://doi.org/10.1103/PhysRev.107.1}{https://doi.org/10.1103/PhysRev.107.1}

  \\item A.D. Fokker, Die mittlere Energie rotierender elektrischer Dipole im Strahlungsfeld. Ann. Phys. 348, 810-820 (1914)

  \\item M. Planck, Über einen Satz der statistischen Dynamik und seine Erweiterung in der Quantentheorie. Sitzungsberichte der Preussischen Akademie der Wissenschaften zu Berlin 24, 324-341 (1917)

  \\item P.L. Bhatnagar, E.P. Gross, M. Krook, A model for collision processes in gases. I. Small amplitude processes in charged and neutral one-component systems. Phys. Rev. 94, 511-525 (1954). \\href{https://doi.org/10.1103/PhysRev.94.511}{https://doi.org/10.1103/PhysRev.94.511}

  \\item L.D. Landau, E.M. Lifshitz, Fluid Dynamics (Pergamon Press Ltd., Oxford, 1987)

  \\item F.F. Chen, Introduction to Plasma Physics and Controlled Fusion, 3rd edn. (Springer, Berlin, 2015)

  \\item B.D. Fried, S.D. Conte, The Plasma Dispersion Function (Academic Press, New York, 1961)

  \\item K. Miyamoto, Plasma Physics and Controlled Nuclear Fusion (Springer, Berlin, 2013)

  \\item T.H. Stix, The Theory of Plasma Waves (MacGraw Hill, New York, 1962)

  \\item R.C. Davidson, Kinetic waves and instabilities in a uniform plasma, in Handbook of Plasma Physics, sect. 3.3, ed. by A.A. Galeev, R.N. Sudan. Vol. 1: Basic Plasma Physics (North Holland Pub., Amsterdam, 1983), pp. 519-586

  \\item N.A. Krall, A.W. Trivelpiece, Principles of Plasma Physics (McGraw-Hill, New York, 1973)

  \\item S. Shinohara, Helicon high-density plasma sources: physics and applications. Adv. Phys. X 3, 1420424 (2018). \\href{https://doi.org/10.1080/23746149.2017.1420424}{https://doi.org/10.1080/23746149.2017.1420424}

  \\item S. Shinohara, High-Density Helicon Plasma Science: From Basics to Applications. Springer Series in Plasma Science and Technology (Springer, Berlin, 2022)

  \\item I.B. Bernstein, Waves in a plasma in a magnetic field. Phys. Rev. 109, 10-21 (1958)

  \\item F.W. Olver, D.W. Lozier, R.F. Boivert, C.W. Clark (eds.), NIST Handbook of Mathematical Functions (National Institute of Standards and Technology and Cambridge University Press, Cambridge, 2010). \\href{https://dlmf.nist.gov}{https://dlmf.nist.gov}

\\end{enumerate}

\\section*{Chapter 7 Plasma Instability }
\\begin{abstract}
In Chapters 2, 5 and 6, equilibrium plasmas are treated. In or near equilibrium plasmas are stable, and perturbations do not grow. In non-equilibrium plasmas excess energy would contribute to instabilities. For example, in Fig. 6.3 the distribution function of $f(v)$, which may be spatially uniform, is in equilibrium and decreases with $|v|$. In the equilibrium state collective modes do not grow and are damped. However, in Fig. 6.4 the distribution function of $f(v)$ has a bump in the velocity space, and collective modes would grow with the factor of $\\exp \\left(+\\omega_{i} t\\right)$, in which $\\omega_{i}>0$. The distribution function in Fig. 6.4 can be found in beam-plasma interaction, in which beam introduces an excess energy into the plasma. For example, two-stream instability is an example for the spatially uniform instability. On the other hand, spatially non-uniform instabilities include sausage and kink instabilities of magnetized column. Interchange instabilities including the Rayleigh-Taylor instability and tearing mode instability are also nonuniform in space. In this chapter, plasma instabilities are discussed. First the two-stream instability is introduced by the fluid model and also by the distribution functions. Then some instabilities are presented with example computer simulation results.
\\end{abstract}

\\subsection*{Two-Stream Instability}
\\subsubsection*{Two-Stream Instability by Fluid Model}
The two-stream instability would appear, when two plasmas have different macroscopic velocities. For example, when an electron beam is injected into a plasma, the two-stream instability would be induced.

Here we assume that ions are rest, and electrons move in the direction $x$ with $V_{0}$. We also assume no magnetic field and zero temperature. The basic equations are below in the fluid model:


\\begin{align*}
& \\frac{\\partial n_{j}}{\\partial t}+\\frac{\\partial n_{j} v_{j}}{\\partial x}=0 \\quad(j=\\text { ion (i) or electron (e) })  \\tag{7.1}\\\\
& m_{j} n_{j}\\left(\\frac{\\partial v_{j}}{\\partial t}+v_{j} \\frac{\\partial v_{j}}{\\partial x}\\right)=q_{j} n_{j} E  \\tag{7.2}\\\\
& \\frac{\\partial E}{\\partial x}=\\frac{e}{\\varepsilon_{0}}\\left(n_{i}-n_{e}\\right) \\tag{7.3}
\\end{align*}


After linearization of the equations, the first-order equations are obtained:


\\begin{align*}
& \\frac{\\partial n_{e 1}}{\\partial t}+n_{0} \\frac{\\partial v_{e 1}}{\\partial x}+V_{0} \\frac{\\partial n_{e 1}}{\\partial x}=0  \\tag{7.4}\\\\
& m_{e} n_{0}\\left(\\frac{\\partial v_{e 1}}{\\partial t}+V_{0} \\frac{\\partial v_{e 1}}{\\partial x}\\right)=-e n_{0} E_{1}  \\tag{7.5}\\\\
& \\frac{\\partial n_{i 1}}{\\partial t}+n_{0} \\frac{\\partial v_{i 1}}{\\partial x}=0  \\tag{7.6}\\\\
& m_{i} n_{0} \\frac{\\partial v_{i 1}}{\\partial t}=e n_{0} E_{1}  \\tag{7.7}\\\\
& \\frac{\\partial E_{1}}{\\partial x}=\\frac{e}{\\varepsilon_{0}}\\left(n_{i 1}-n_{e 1}\\right) \\tag{7.8}
\\end{align*}


Here we assume that the ion charge is 1 and the zeroth-order number density is $n_{0}$. Here we focus on one Fourier component: $\\exp (i k x-i \\omega t)$


\\begin{align*}
& -i \\omega n_{e 1}+n_{0}\\left(i k v_{e 1}\\right)+V_{0}\\left(i k n_{e 1}\\right)=0  \\tag{7.9}\\\\
& m_{e} n_{0}\\left\\{-i \\omega v_{e 1}+V_{0}\\left(i k v_{e 1}\\right)\\right\\}=-e n_{0} E_{1}  \\tag{7.10}\\\\
& -i \\omega n_{i 1}+n_{0}\\left(i k v_{i 1}\\right)=0  \\tag{7.11}\\\\
& m_{i} n_{0}\\left(-i \\omega v_{i 1}\\right)=e n_{0} E_{1}  \\tag{7.12}\\\\
& i k E_{1}=\\frac{e}{\\varepsilon_{0}}\\left(n_{i 1}-n_{e 1}\\right) \\tag{7.13}
\\end{align*}


Then the dispersion relation is obtained:


\\begin{equation*}
\\epsilon(k, \\omega)=1-\\frac{\\omega_{p i}^{2}}{\\omega^{2}}-\\frac{\\omega_{p e}^{2}}{\\left(\\omega-k V_{0}\\right)^{2}}=0 \\tag{7.14}
\\end{equation*}


$\\omega_{p j}(j=i, e)$ shows the plasma frequency for ions or electrons.

Here we obtain a solution $\\omega$ under the following condition:


\\begin{equation*}
\\omega_{p i} \\ll|\\omega| \\ll \\omega_{p e} \\tag{7.15}
\\end{equation*}


From the condition of Eq. (7.15),


\\begin{equation*}
\\frac{\\omega_{p i}^{2}}{\\omega^{2}} \\ll 1 \\tag{7.16}
\\end{equation*}


For $\\epsilon(k, \\omega)=0$, the following relation would be approximately correct:


\\begin{equation*}
\\frac{\\omega_{e}^{2}}{\\left(\\omega-k V_{0}\\right)^{2}} \\sim 1 \\tag{7.17}
\\end{equation*}


In addition, since $\\omega \\ll \\omega_{p e}$, the following would hold:


\\begin{equation*}
\\left|k V_{0}\\right| \\sim \\omega_{p e} \\tag{7.18}
\\end{equation*}


Therefore, the following is obtained:


\\begin{align*}
\\epsilon(k, \\omega) & \\sim 1-\\frac{\\omega_{i}^{2}}{\\omega^{2}}-\\frac{\\omega_{p e}^{2}}{\\left(\\omega-\\omega_{p e}\\right)^{2}} \\\\
& =1-\\frac{\\omega_{i}^{2}}{\\omega^{2}}-\\frac{1}{\\left(1-\\omega / \\omega_{p e}\\right)^{2}} \\\\
& \\simeq 1-\\frac{\\omega_{i}^{2}}{\\omega^{2}}-\\left(1+\\frac{2 \\omega}{\\omega_{p e}}\\right) \\\\
& =-\\frac{\\omega_{p i}^{2}}{\\omega^{2}}-\\frac{2 \\omega}{\\omega_{p e}}=0  \\tag{7.19}\\\\
& \\omega^{3}+\\frac{1}{2} \\omega_{p i}^{2} \\omega_{p e}=0 \\tag{7.20}
\\end{align*}


The solutions for Eq. (7.20) are obtained:


\\begin{equation*}
\\omega=-\\left(\\frac{\\omega_{p i}^{2} \\omega_{p e}}{2}\\right)^{1 / 3}, \\quad\\left(\\frac{1}{2} \\pm i \\frac{\\sqrt{3}}{2}\\right)\\left(\\frac{\\omega_{p i}^{2} \\omega_{p e}}{2}\\right)^{1 / 3} \\tag{7.21}
\\end{equation*}


Physical quantities are proportional to $\\exp (-i \\omega t)$. The imaginary part of $\\omega$ shows the two-stream instability as follows:


\\begin{equation*}
\\gamma=\\frac{\\sqrt{3}}{2}\\left(\\frac{\\omega_{p i}^{2} \\omega_{p e}}{2}\\right)^{1 / 3} \\tag{7.22}
\\end{equation*}


The instability grows with $\\exp (\\gamma t)$, and the growth rate is $\\gamma$ shown above.

Here the physical mechanism of the two-stream instability is presented. The perturbation $\\delta n$ of the electric charge appears as shown in Fig. 7.1a. The electric field perturbation $\\delta E$ in Fig. 7.1b is induced by the electric charge $\\delta n$. In our case, the electron beam moves in $+x$ with the speed of $V_{0}$, and its averaged number density is $n_{e 0}$. Ions stay rest as a background. At this time, the electron number density is perturbed as shown in Fig. 7.1c. The electron beam moving with $V_{0}$ is affected by

Fig. 7.1 Two-stream instability. Physical explanation

\\begin{center}
\\includegraphics[max width=\\textwidth]{2024_02_26_83e36543483eb7d284c1g-161(1)}
\\end{center}

b) $\\delta E$

\\begin{center}
\\includegraphics[max width=\\textwidth]{2024_02_26_83e36543483eb7d284c1g-161(3)}
\\end{center}

c) $\\delta n_{e}$
\\includegraphics[max width=\\textwidth, center]{2024_02_26_83e36543483eb7d284c1g-161(2)}

e) $\\delta n_{e}^{\\prime}$ : the initial $\\delta n_{e}$ is enhanced.

\\begin{center}
\\includegraphics[max width=\\textwidth]{2024_02_26_83e36543483eb7d284c1g-161}
\\end{center}

$\\delta E$, and the speed perturbation $\\delta v_{e}^{\\prime}$ is induced (see Fig. 7.1d). As a result, the electron number density is perturbed, and $\\delta n_{e}^{\\prime}$ appears as shown in Fig. 7.1e. Then $\\delta n$ is enhanced.

\\subsubsection*{Two-Stream Instability by Distribution Function}
In the last Sect. 7.1.1 the two-stream instability is considered by the fluid model. Here we treat the two-stream instability by the distribution function microscopically. Here we assume again that ions are rest and the temperature is zero. The electron beam moves with $V_{0}$ in $x$.

\\[
\\begin{array}{r}
f_{i}(\\mathbf{v})=\\delta(\\mathbf{v}) \\\\
f_{e}(\\mathbf{v})=\\delta\\left(\\mathbf{v}-V_{0} \\hat{x}\\right) \\\\
f(\\mathbf{v})=f_{i}(\\mathbf{v})+f_{e}(\\mathbf{v}) \\tag{7.25}
\\end{array}
\\]

The distribution functions are inserted to Eq. (6.65). Then we obtain the following dispersion relation:


\\begin{equation*}
\\epsilon(k, \\omega)=1-\\frac{\\omega_{i}^{2}}{\\omega^{2}}-\\frac{\\omega_{e}^{2}}{\\left(\\omega-k V_{0}\\right)^{2}}=0 \\tag{7.26}
\\end{equation*}


Equation (7.14) is again obtained.

When we consider the finite temperature of $T_{i}$ and $T_{e}$.

\\[
\\begin{array}{r}
f_{i}\\left(v_{x}\\right)=\\sqrt{\\frac{m_{i}}{2 \\pi T_{i}}} \\exp \\left(-\\frac{m_{i} v_{x}^{2}}{2 T_{i}}\\right) \\\\
f_{e}\\left(v_{x}\\right)=\\sqrt{\\frac{m_{e}}{2 \\pi T_{e}}} \\exp \\left(-\\frac{m_{e}\\left(v_{x}-V_{0}\\right)^{2}}{2 T_{e}}\\right) \\tag{7.28}
\\end{array}
\\]

From Eq. (6.72), the following dispersion relation is obtained:


\\begin{align*}
\\epsilon(k, \\omega) & =1+\\frac{1}{k^{2} \\lambda_{i}^{2}} W\\left(\\frac{\\omega}{k} \\sqrt{\\frac{m_{i}}{T_{i}}}\\right)+\\frac{1}{k^{2} \\lambda_{e}^{2}} W\\left(\\frac{\\omega-k V_{0}}{k} \\sqrt{\\frac{m_{e}}{T_{e}}}\\right) \\\\
& =0 \\tag{7.29}
\\end{align*}


Here we consider the following conditions:


\\begin{align*}
\\left|\\frac{\\omega}{k} \\sqrt{\\frac{m_{i}}{T_{i}}}\\right| & \\gg 1  \\tag{7.30}\\\\
\\left|\\frac{\\omega-k V_{0}}{k} \\sqrt{\\frac{m_{e}}{T_{e}}}\\right| & \\gg 1 \\tag{7.31}
\\end{align*}


Then $\\epsilon(k, \\omega)$ is obtained explicitly:


\\begin{align*}
\\epsilon(k, \\omega) \\simeq & 1-\\frac{\\omega_{i}^{2}}{\\omega^{2}}-\\frac{\\omega_{e}^{2}}{\\left(\\omega-k V_{0}\\right)^{2}} \\\\
& +i \\sqrt{\\frac{\\pi}{2}} \\frac{\\omega}{k^{3} \\lambda_{i}^{2}} \\sqrt{\\frac{m_{i}}{T_{i}}} \\exp \\left[-\\frac{\\omega^{2}}{2 k^{2}} \\sqrt{\\frac{m_{i}}{T_{i}}}\\right] \\\\
& +i \\sqrt{\\frac{\\pi}{2}} \\frac{\\omega}{k^{3} \\lambda_{e}^{2}} \\sqrt{\\frac{m_{e}}{T_{e}}} \\exp \\left[-\\frac{\\left(\\omega-k V_{0}\\right)^{2}}{2 k^{2}} \\sqrt{\\frac{m_{e}}{T_{e}}}\\right]=0 \\tag{7.32}
\\end{align*}


The real part shows Eq. (7.26) and reproduces Eqs. (7.21) and (7.22). The imaginary part of Eq. (7.32) shows the Landau damping, which was discussed in Sects. 6.7 and 6.9 .

\\subsubsection*{Example Simulation for Two-Stream Instability}
In this subsection one example simulation is shown for the two-stream instability by the EPOCH 1D code [1,2]. In this example case in the Sect. 7.1.3, for simplicity, ions are immobile as a background, and electric charge is initially neutralized. Two electron beams flow in counter directions in the $x$ direction with a velocity of $\\pm 0.1 c$. Each electron beam density is $5 \\times 10^{17} \\mathrm{~m}^{-3}$. Figures 7.2 present a) the initial momentum of $p_{x}=\\gamma m v_{x}$ spatial distribution, b) $p_{x}$ versus $x$ at $t=0.40 \\mathrm{~ns}$, c) the electric field $E_{x}$ distribution at $t=0.40 \\mathrm{~ns}$ and d) the electron number density $n_{e 1}$ distribution for one electron beam moving in $+x$ at $t=0.4 \\mathrm{~ns}$. The electron plasma wave appears clearly. In this example case, the electron plasma frequency $\\omega_{p e} \\sim 4.0 \\times 10^{10} / \\mathrm{s}$, and so the growth rate $\\gamma$ is about $\\omega_{p e}$. The instability growth time scale $\\tau \\sim 1 / \\gamma$ is about $\\tau \\sim 0.25 \\mathrm{~ns}$.

Figure 7.3 shows some results at the nonlinear stage of the two-stream instability: a) $E_{x}$ at $t=0.48 \\mathrm{~ns}$, b) the electron density distribution $n_{e 1}$ for the electron beam moving $+x$ at $t=0.48 \\mathrm{~ns}, \\mathrm{c}$ ) the particle distribution in the $p_{x}-x$ space at $t=0.48$ $\\mathrm{ns}$ and $\\mathrm{d}$ ) the particle distribution in the $p_{x}-x$ space at $t=1.84 \\mathrm{~ns}$. Figure $7.3 \\mathrm{~d}$ may show that some electrons are trapped by the plasma wave with its finite amplitude (see, for example, Chap. 6 in [3] and Chap. 7 in Ref. [4] ).
\\includegraphics[max width=\\textwidth, center]{2024_02_26_83e36543483eb7d284c1g-163}

Fig. 7.2 Example simulation results for the two-stream instability
a) $t=0.48 \\mathrm{~ns}$
\\includegraphics[max width=\\textwidth, center]{2024_02_26_83e36543483eb7d284c1g-164(1)}
b) $t=0.48 \\mathrm{~ns}$
\\includegraphics[max width=\\textwidth, center]{2024_02_26_83e36543483eb7d284c1g-164}

Fig. 7.3 Example simulation results for the two-stream instability at a nonlinear stage

\\subsection*{Ion Acoustic Instability}
In Sect. 5.5 in Chap. 5, the ion acoustic wave was considered. For the ion acoustic wave, Eq. (5.83) was obtained. Here we consider the ion acoustic wave instability. The physical assumptions employed are the same with those in Sect. 5.5. Here we consider the ion acoustic instability under the following conditions.


\\begin{align*}
\\left|\\frac{\\omega}{k} \\sqrt{\\frac{m_{i}}{T_{i}}}\\right| & \\gg 1  \\tag{7.33}\\\\
\\left|\\frac{\\omega-k V_{0}}{k} \\sqrt{\\frac{m_{e}}{T_{e}}}\\right| & \\ll 1 \\tag{7.34}
\\end{align*}


Then the following dispersion relation is obtained:


\\begin{align*}
\\epsilon(k, \\omega) \\simeq & 1-\\frac{\\omega_{i}^{2}}{\\omega^{2}}+\\frac{1}{k^{2} \\lambda_{e}^{2}} \\\\
& +i \\sqrt{\\frac{\\pi}{2}} \\frac{\\omega}{k^{3} \\lambda_{i}^{2}} \\sqrt{\\frac{m_{i}}{T_{i}}} \\exp \\left(-\\frac{\\omega^{2}}{2 k^{2}} \\sqrt{\\frac{m_{i}}{T_{i}}}\\right) \\\\
& +i \\sqrt{\\frac{\\pi}{2}} \\frac{\\omega-k V_{0}}{k^{3} \\lambda_{e}^{2}} \\sqrt{\\frac{m_{e}}{T_{e}}}=0 \\tag{7.35}
\\end{align*}


From the real part, the real part of $\\omega_{r}$ is obtained:


\\begin{equation*}
\\omega_{r}^{2}=\\frac{\\omega_{i}^{2}}{1+\\frac{1}{k^{2} \\lambda_{e}^{2}}}=\\frac{k^{2} \\lambda_{e}^{2} \\omega_{i}^{2}}{1+k^{2} \\lambda_{e}^{2}}=\\frac{k^{2}}{1+k^{2} \\lambda_{e}^{2}}\\left(\\frac{T_{e}}{m_{i}}\\right) \\tag{7.36}
\\end{equation*}


When we consider the following conditions of $k^{2} \\lambda_{e}^{2} \\ll 1$ and $T_{e} \\gg T_{i}$, Eq. (7.35) becomes approximately Eq. (5.83), and the wave concerned is the ion acoustic wave. The 4 th term at the right hand side of Eq. (7.35) shows the Landau damping. However, the 5th term shows the growth rate $\\gamma$ for the ion acoustic instability.


\\begin{equation*}
\\gamma=-\\frac{\\epsilon_{i}}{\\left.\\frac{\\partial \\epsilon_{r}}{\\partial \\omega}\\right|_{\\omega=\\omega_{r}}}=-\\sqrt{\\frac{\\pi}{8}} \\frac{\\omega_{r}^{3}}{\\omega_{i}^{2}} \\frac{\\omega-k V_{0}}{k^{3} \\lambda_{e}^{2}} \\sqrt{\\frac{m_{i}}{T_{e}}}+\\text { Landau damping } \\tag{7.37}
\\end{equation*}


The result shows that the ion acoustic instability grows, when the following condition is satisfied:


\\begin{equation*}
V_{0}>\\frac{\\omega}{k} \\tag{7.38}
\\end{equation*}


\\subsection*{Instability of Magnetized Plasma Column}
In the last Sects. 7.1 and 7.2, the microscopic instabilities were introduced. Here some of macroscopic plasma instabilities are introduced, including the sausage instability.

Now we consider a magnetized plasma column in Fig. 7.4a, through which a current $I$ flows along the cylindrical axis. The column is infinitely long in the axis direction. The column is in an equilibrium with the magnetic field $B_{\\theta}$ in the $\\theta$ direction: The column plasma has a pressure and can expand radially by the pressure $P$.

Fig. 7.4 Sausage instability
\\includegraphics[max width=\\textwidth, center]{2024_02_26_83e36543483eb7d284c1g-166}

The magnetic field of $B_{\\theta}$ provides the Lorentz force of $\\mathbf{J} \\times \\mathbf{B}$, which balances with the pressure force.

Let us assume that an axisymmetric perturbation in the column radius $r$ appears as shown in Fig. 7.4b. It is also assumed that the perturbation amplitude is sufficiently small: $\\delta r \\ll r$. In Fig. 7.4a, the magnetic field $B_{\\theta}$ is obtained by the second equation of Eq. (4.5) ( $\\left.\\nabla \\times \\mathbf{B}=\\mu_{0} \\mathbf{J}\\right)$ for the steady state $\\frac{\\partial}{\\partial t}=0$.

\\[
\\begin{array}{r}
\\mu_{0} I=\\int B_{\\theta} 2 \\pi d r=2 \\pi r B_{\\theta} \\\\
\\therefore \\quad B_{\\theta}=\\mu_{0} I / 2 \\pi r \\tag{7.39}
\\end{array}
\\]

As shown in Fig. 7.4b, when the radius of a part of the column radius is shrinked, $B_{\\theta}$ becomes strong and the force to shrink the column becomes large. Equation (7.39) also shows this fact. This instability is called the sausage instability.

In order to stabilize the sausage instability, we can apply the axial magnetic field $B_{z}$ inside the column plasma. When the column radius is reduced as shown in Fig. 7.4b, the magnetic pressure by $B_{z}$ of $B_{z}^{2} / 2 \\mu_{0}$ contributes to prevent the column from shrinking.

When the plasma pressure $P$ is sufficiently low and negligible, it may be assumed $P \\sim 0$. At the equilibrium state the following relation holds:


\\begin{equation*}
\\frac{B_{z}^{2}}{2 \\mu_{0}}=\\frac{B_{\\theta}^{2}}{2 \\mu_{0}} \\tag{7.40}
\\end{equation*}


Now a perturbation $\\delta r<0$ is imposed. When the following condition is fulfilled, the plasma column is stable:


\\begin{equation*}
\\frac{B_{z} \\delta B_{z}}{\\mu_{0}}>\\frac{B_{\\theta} \\delta B_{\\theta}}{\\mu_{0}} \\tag{7.41}
\\end{equation*}


Here $\\delta B_{\\theta}$ is obtained by Eq. (7.39) as follows:


\\begin{equation*}
\\delta B_{\\theta}=-\\frac{\\mu_{0} I}{2 \\pi r^{2}} \\delta r=-B_{\\theta} \\frac{\\delta r}{r} \\tag{7.42}
\\end{equation*}


We can also assume that the total $B_{z}$ inside the column, that is, $\\pi r^{2} B_{z}$ is conserved:

\\[
\\begin{array}{r}
0=\\delta\\left(\\pi r^{2} B_{z}\\right)=2 \\pi r \\delta r B_{z}+\\pi r^{2} \\delta B_{z} \\\\
\\therefore \\quad \\delta B_{z}=-B_{z}\\left(\\frac{2 \\delta r}{r}\\right) \\tag{7.43}
\\end{array}
\\]

By Eqs. (7.42) and (7.43), the stability condition for the sausage instability (7.41) is written as follows:

$$
\\begin{aligned}
& -\\frac{2 B_{z}^{2}}{\\mu_{0}} \\frac{\\delta r}{r}>-\\frac{B_{\\theta}^{2}}{\\mu_{0}} \\frac{\\delta r}{r} \\\\
& \\therefore \\quad \\frac{\\delta r}{\\mu_{0} r}\\left(2 B_{z}^{2}-B_{\\theta}^{2}\\right)<0
\\end{aligned}
$$

Therefore, for $\\delta r<0$, the stability condition for the sausage instability is as follows:


\\begin{equation*}
2 B_{z}^{2}>B_{\\theta}^{2} \\tag{7.45}
\\end{equation*}


When a perturbation shown in Fig. 7.5 appears in the magnetized plasma column, it becomes also unstable. This is called the kink instability. When some parts of the column shift transversely, the magnetic field inside the bent cylinder becomes strong compared with that outside. It leads to the kink instability.

The detail discussions on the kink instability and on the magnetized plasma column can be found in Chap. 8 in Ref. [5] and Chap. 2 in Ref. [6]. When the additional magnetic field $B_{z}$ is applied again to the plasma column, it may contribute to stretch the column along its axis and it may be stabilized. However the kink modes with long wavelengths cannot be stabilized by $B_{z}[5,6]$. When the column plasma is bounded by a hollow metal liner, the magnetic field at the bent part outward becomes strong and prevent the kink instability [5, 6].

When $B_{\\theta}$ and $B_{z}$ are applied to the plasma column, the total magnetic field becomes spiral around the cylindrical axis. If perturbation pitches correspond to the magnetic field pitch, which is $2 \\pi r B_{z} /\\left.B_{\\theta}\\right|_{r=r}=4 \\pi^{2} r^{2} B_{z} /\\left(\\mu_{0} I\\right)$, the perturbations would grow. It would be called the screw instability or the helical instability. When the column length $L$ is short compared with the perturbation pitch, the kinkmode perturbation would be stabilized: $L<\\frac{2 \\pi r B_{z}}{\\left.B_{\\theta}\\right|_{r=r}}$. This is the Kruskal-Shafranov limit $[5,6]$ :


\\begin{equation*}
I<\\frac{4 \\pi^{2} r^{2} B_{z}}{\\mu_{0} L} \\tag{7.46}
\\end{equation*}


Fig. 7.5 Kink instability

\\begin{center}
\\includegraphics[max width=\\textwidth]{2024_02_26_83e36543483eb7d284c1g-168}
\\end{center}

\\subsubsection*{Example Simulation for the Sausage and Kink Instabilities}
In the Sect. 7.3.1, the sausage and kink instabilities are simulated [7] by the EPOCH code $[1,2]$. A magnetized z-current-driven plasma column is employed to simulate the sausage instability. The column plasma consists of protons and electrons. The protons are stationary and uniform in space. Its initial density is $f \\times n_{0}$, and the protons provide the partial charge neutralization of the $z$-current electrons with the neutralization ratio of $f$. In this Sect. 7.3.1 $f$ is set to be 0.99 . The electron beam speed in $z$ is $0.1 \\mathrm{c}$, and the electron number density is $n_{0}$. Here $\\mathrm{c}$ is the speed of light and $n_{0}=10^{16} / \\mathrm{m}^{3}$. The electron velocity and density are also uniform. The initial ion and electron temperatures are $1 \\mathrm{eV}$. The plasma column is $20 \\mathrm{~cm}$ long in $z$, and its radius is $1.5 \\mathrm{~cm}$. In the column equilibrium state the radial force is balanced between the outward electrostatic force and the inward magnetic pinching force by the azimuthal magnetic field generated by the electron net current [8].

Figure 7.6 demonstrates the sausage instability growth and shows a) the electron distribution at $t=30 \\mathrm{~ns}$ and b) the proton distribution at $t=40 \\mathrm{~ns}$. In Fig. 7.6 the initial perturbation has 4 wavelengths in the $20 \\mathrm{~cm}$ simulation box in the longitudinal direction of $z$.

Figure 7.7 shows the kink instability growth, and initially the plasma electron column is displaced in $y$ by $5 \\%$ of the column radius. In $20 \\mathrm{~cm}, 2$ modes are accommodated in $z$. The electron distributions are presented in Fig. 7.7a at $t=50 \\mathrm{~ns}$, and the protons are shown in Fig. $7.7 \\mathrm{~b}$ at $t=50 \\mathrm{~ns}$. Figure 7.7 presents the kink instability growth clearly.
\\includegraphics[max width=\\textwidth, center]{2024_02_26_83e36543483eb7d284c1g-169(2)}

Fig. 7.6 Example simulation results for the sausage instability by EPOCH [1,2]. The electron density at (a) $t=30 \\mathrm{~ns}$ and the proton density at (b) $t=40 \\mathrm{~ns}$. Source Ref. [7]. (Reprinted figures with permission from Kawata et al. [7] Copyright 2020 by the American Physical Society

\\section*{a) $t=50 \\mathrm{~ns}$ electron $\\operatorname{density}\\left(/ \\mathrm{m}^{3}\\right)$}
\\includegraphics[max width=\\textwidth, center]{2024_02_26_83e36543483eb7d284c1g-169}
b) $t=50 \\mathrm{~ns}$ proton $\\operatorname{density}\\left(/ \\mathrm{m}^{3}\\right)$

\\begin{center}
\\includegraphics[max width=\\textwidth]{2024_02_26_83e36543483eb7d284c1g-169(1)}
\\end{center}

Fig. 7.7 Example simulation results for the kink instability by EPOCH [1,2]. The electron density at (a) $t=50 \\mathrm{~ns}$ and the proton density at (b) $t=50 \\mathrm{~ns}$. Source Ref. [7]. Reprinted figures with permission from Kawata et al. [7]. Copyright 2020 by the American Physical Society

\\begin{center}
\\includegraphics[max width=\\textwidth]{2024_02_26_83e36543483eb7d284c1g-170}
\\end{center}

Fig. 7.8 Interchange instability. In this case, a plasma is supported by a magnetic field $B_{z}$ against gravity $\\mathbf{g}$. When a perturbation appears at the plasma surface as shown in Fig. 7.8, the drift of $\\mathbf{g} \\times \\mathbf{B}$ induces the charge separation, which creates $E_{y}$. Then the $\\mathbf{E} \\times \\mathbf{B}$ drift enhances the original perturbation

\\subsection*{Interchange Instability-The Rayleigh-Taylor instability and an Example Simulation}
In the Sect.7.4, interchange instability is introduced (see, for example, Chap. 10 in Ref. [9]). The interchange instability includes the Rayleigh-Taylor instability, in which a light plasma (or fluid) supports a heavy plasma (or fluid) against gravity $\\mathbf{g}$. In the interchange instability shown in Fig. 7.8, a plasma is supported by a magnetic field against gravity. When a perturbation appears at the plasma surface as shown in Fig. 7.8, the $\\mathbf{g} \\times \\mathbf{B}$ drift induces the charge separation, which creates $E_{y}$. Then the $\\mathbf{E} \\times \\mathbf{B}$ drift enhances the original perturbation. By the exchange between a part of the magnetic field and a part of the plasma, the system total energy becomes smaller. The gravity $\\mathbf{g}$ here may be a force of a centrifugal force by a bending magnetic field.

Even without the magnetic field, the interchange instability appears in a system, in which a light plasma (or fluid) supports a heavy plasma (or fluid) against gravity. The system is unstable, and it is also called the Rayleigh-Taylor instability. In this case the growth rate is written by $\\gamma=\\sqrt{g k}$.

At $x=0$ the two-plasmas interface is located. Against the gravity $g$, a heavy plasma is located at $x>0$ with the density $\\rho_{02}$ and a light plasma is at $x<0$ with the density $\\rho_{01}$. Figure 7.9 shows a schematic figure for the Rayleigh-Taylor instability. The heavy plasma goes down and the light plasma goes up.

As basic equations, the continuity equation and the equation of motion are used, together with the incompressibility condition of $\\operatorname{div} \\mathbf{v}=\\nabla \\cdot \\mathbf{v}=0$. The incompressibility condition comes from the condition of $\\rho=$ constant. As the first-order quantities, $v_{x}, v_{y}, \\rho_{1}$ and $p_{1}$ appear:


\\begin{align*}
\\rho \\frac{\\partial v_{x}}{\\partial t} & =-\\frac{\\partial p_{1}}{\\partial x}-g \\rho_{1}  \\tag{7.47}\\\\
\\rho \\frac{\\partial v_{y}}{\\partial t} & =-\\frac{\\partial p_{1}}{\\partial y} \\tag{7.48}
\\end{align*}


Fig. 7.9 Rayleigh-Taylor instability. Against the gravity, a heavy plasma is supported by a light plasma. The system is unstable

\\begin{center}
\\includegraphics[max width=\\textwidth]{2024_02_26_83e36543483eb7d284c1g-171}
\\end{center}


\\begin{align*}
& \\frac{\\partial \\rho_{1}}{\\partial t}+v_{x} \\frac{\\partial \\rho}{\\partial x}=0  \\tag{7.49}\\\\
& \\frac{\\partial v_{x}}{\\partial x}+\\frac{\\partial v_{y}}{\\partial y}=0 \\tag{7.50}
\\end{align*}


Equation (7.50) shows the incompressibility $(\\nabla \\cdot \\mathbf{v}=0)$. The first-order quantity is proportional to $\\exp \\left(i k_{y} y+\\gamma t\\right)$. Then we obtain the following algebraic equations:


\\begin{align*}
& \\gamma \\rho v_{x}=-\\frac{\\mathrm{d} p_{1}}{\\mathrm{~d} x}-g \\rho_{1}  \\tag{7.51}\\\\
& \\gamma \\rho v_{y}=-i k_{y} p_{1}  \\tag{7.52}\\\\
& \\gamma \\rho_{1}+v_{x} \\frac{\\mathrm{d} \\rho}{\\mathrm{d} x}=0  \\tag{7.53}\\\\
& \\frac{\\mathrm{d} v_{x}}{\\mathrm{~d} x}+i k_{y} v_{y}=0 \\tag{7.54}
\\end{align*}


From these equations, the following is obtained:


\\begin{equation*}
\\frac{\\mathrm{d}}{\\mathrm{d} x}\\left(\\rho \\frac{\\mathrm{d} v_{x}}{\\mathrm{~d} x}\\right)-\\rho k_{y}^{2} v_{x}=0 \\tag{7.55}
\\end{equation*}


Except the boundary at $x=0$, in each plasma the density is constant. At each area, the following holds:


\\begin{equation*}
\\frac{\\mathrm{d}^{2}}{\\mathrm{~d} x^{2}} v_{x}-k_{y}^{2} v_{x}=0 \\tag{7.56}
\\end{equation*}


Therefore, we obtain the solution below:

\\[
v_{x}= \\begin{cases}A \\exp \\left(+k_{y} x\\right) & \\text { for } x<0  \\tag{7.57}\\\\ A \\exp \\left(-k_{y} x\\right) & \\text { for } x>0\\end{cases}
\\]

At the interface $x=0, v_{x}$ is continuous. At the same time, $\\mathrm{d} v_{x} / \\mathrm{d} x$ is also continuous. Substituting Eq. (7.53) into Eq. (7.51), the following is obtained:


\\begin{equation*}
\\frac{\\mathrm{d} p_{1}}{\\mathrm{~d} x}=-\\gamma \\rho v_{x}+\\frac{g}{\\gamma} \\frac{\\mathrm{d} \\rho}{\\mathrm{d} x} v_{x} \\tag{7.58}
\\end{equation*}


By Eqs. (7.52) and (7.54), the following is obtained:


\\begin{equation*}
k_{y}^{2} p_{1}=-\\gamma \\rho \\frac{\\mathrm{d} v_{x}}{\\mathrm{~d} x} \\tag{7.59}
\\end{equation*}


After integrating Eq. (7.58) in $x$ over the infinitesimal interval across the interface at $x=0$, the following is obtained:


\\begin{equation*}
\\Delta p_{1}=\\frac{g}{\\gamma} \\Delta \\rho v_{x} \\tag{7.60}
\\end{equation*}


Here $\\Delta \\rho=\\rho_{02}-\\rho_{01}$. By Eqs. (7.59), (7.60) gives the following:


\\begin{equation*}
\\Delta\\left(\\frac{\\gamma \\rho}{k_{y}^{2}} \\frac{\\mathrm{d} v_{x}}{\\mathrm{~d} x}\\right)=\\frac{g}{\\gamma} v_{x} \\Delta \\rho \\tag{7.61}
\\end{equation*}


By using Eqs. (7.57), we obtain the following:


\\begin{equation*}
\\frac{\\gamma}{k_{y}} v_{x}\\left(\\rho_{1}+\\rho_{2}\\right)=\\frac{g}{\\gamma} v_{x}\\left(\\rho_{2}-\\rho_{1}\\right) \\tag{7.62}
\\end{equation*}


The growth rate for the Rayleigh-Taylor instability is obtained:


\\begin{equation*}
\\gamma=\\sqrt{g k_{x} \\frac{\\rho_{2}-\\rho_{1}}{\\rho_{1}+\\rho_{2}}} \\tag{7.63}
\\end{equation*}


When water is supported by air against the gravity, we can estimate the time scale of the replacement between the water and the air by $1 / \\gamma$. If we take $k \\sim 1 / \\mathrm{cm}$, $g=9.8 \\mathrm{~m} / \\mathrm{s}^{2}$ and then $1 / \\gamma \\sim 0.032 \\mathrm{~s}$. Therefore, the water is interchanged with the air quickly.

Figure 7.10 shows an example three-dimensional (3D) fluid simulation result for the Rayleigh-Taylor instability by a 3D Euler fluid code shown in Sect. 5.2 and in Appendix F. The heavy plasma goes down and penetrates into the light plasma. Instead, the light plasma goes up and replaces. In Fig. 7.10, $\\rho_{02}=2.0$ and $\\rho_{01}=1.0$. The gravity is applied in the $-y$ direction. Initially the system is in equilibrium, and a small perturbation is applied in $v_{y}$ near the boundary of the two plasmas. The initial perturbation has two modes in $x$ and $z$. For this simulation the 3D Euler fluid code introduced in Sect. 5.2 was used. For simplicity the reflection boundary conditions are employed in $x$ and $z$, and in $z$ the free boundary condition is employed. The

Fig. 7.10 Example simulation result for the Rayleigh-Taylor instability. The density $\\rho$ contour surfaces are presented. Against the gravity, a heavy plasma is supported by a light plasma

\\begin{center}
\\includegraphics[max width=\\textwidth]{2024_02_26_83e36543483eb7d284c1g-173}
\\end{center}

computation box has 200 meshes in $x, 400$ in $y$ and 200 meshes in $z$. In addition to the Rayleigh-Taylor instability, it seems that the Kelvin-Helmholtz instability, which is introduced in the next Sect.7.5, appears at the surface of a spike in Fig. 7.10.

\\subsection*{The Kelvin-Helmholtz Instability and an Example Simulation}
Here we derive the growth rate of the Kelvin-Helmholtz instability (see, for example, Chap. 11 in Ref. [9] and Ref. [10]). Two stratified incompressible fluids or neutral plasmas move with different speeds of $U=U(z)$ in the $x$ direction: In our case $U=U_{1}$ for $z<0$ and $U=U_{2}$ for $z>0$. Both $U_{1}$ and $U_{2}$ are constant in each region. The linearized basic equations are below for $\\rho_{1}, \\mathbf{v}_{\\mathbf{1}}=\\left(v_{x}, v_{y}, v_{z}\\right), p_{1}$ :


\\begin{align*}
\\frac{\\partial \\rho_{1}}{\\partial t}+U \\frac{\\partial \\rho_{1}}{\\partial x}+v_{z} \\frac{\\partial \\rho}{\\partial z} & =0  \\tag{7.64}\\\\
\\rho \\frac{\\partial v_{x}}{\\partial t}+\\rho U \\frac{\\partial v_{x}}{\\partial x}+\\rho v_{z} \\frac{\\partial U}{\\partial z} & =-\\frac{\\partial p_{1}}{\\partial x} \\tag{7.65}
\\end{align*}



\\begin{align*}
\\rho \\frac{\\partial v_{y}}{\\partial t}+\\rho U \\frac{\\partial v_{y}}{\\partial x} & =-\\frac{\\partial p_{1}}{\\partial y}  \\tag{7.66}\\\\
\\rho \\frac{\\partial v_{z}}{\\partial t}+\\rho U \\frac{\\partial v_{z}}{\\partial x} & =-\\frac{\\partial p_{1}}{\\partial z}  \\tag{7.67}\\\\
\\frac{\\partial z}{\\partial t}+U \\frac{\\partial z}{\\partial x} & =v_{z}, \\text { at the interface (z) between two fluids }  \\tag{7.68}\\\\
\\frac{\\partial v_{x}}{\\partial x}+\\frac{\\partial v_{y}}{\\partial y}+\\frac{\\partial v_{z}}{\\partial z} & =0 \\tag{7.69}
\\end{align*}


The perturbations are decomposed by the Fourier transformation in $x$ and $y$, and one Fourier component is again focused. The perturbation components are proportional to $\\exp \\left(-i \\omega t+i k_{x} x+i k_{y} y\\right)$. The linearized equations above become as follows:


\\begin{align*}
\\left(-i \\omega+i k_{x} U\\right) \\rho_{1}+v_{z} \\frac{\\partial \\rho}{\\partial z} & =0  \\tag{7.70}\\\\
\\rho\\left(-i \\omega+i k_{x} U\\right) v_{x}+\\rho v_{z} \\frac{\\partial U}{\\partial z} & =-i k_{x} p_{1}  \\tag{7.71}\\\\
\\rho\\left(-i \\omega+i k_{x} U\\right) v_{y} & =-i k_{y} p_{1}  \\tag{7.72}\\\\
\\rho\\left(-i \\omega+i k_{x} U\\right) v_{z} & =-\\frac{\\partial p_{1}}{\\partial z}  \\tag{7.73}\\\\
i k_{x} v_{x}+i k_{y} v_{y}+\\frac{\\partial v_{z}}{\\partial z} & =0 \\tag{7.74}
\\end{align*}


In addition to these equations, the following boundary condition is introduced:


\\begin{equation*}
\\left(-i \\omega+i k_{x} U\\right) z=v_{z} \\text {, at the interface (z) between two fluids } \\tag{7.75}
\\end{equation*}


In the Kelvin-Helmholtz instability, the flowing speed of the two plasmas (fluids) is different: $U=U_{1}$ for $z<0$ and $U=U_{2}$ for $z>0$. At the boundary between the two plasmas, $z$ is continuous.

From these equations we can obtain the following equation:


\\begin{equation*}
\\frac{\\partial}{\\partial z}\\left\\{\\rho\\left(-\\omega+k_{x} U\\right) \\frac{\\partial v_{z}}{\\partial z}-\\rho k_{x} \\frac{\\partial U}{\\partial z} v_{z}\\right\\}-\\left(k_{x}^{2}+k_{y}^{2}\\right) \\rho\\left(-\\omega+k_{x} U\\right) v_{z}=0 \\tag{7.76}
\\end{equation*}


Now we integrate Eq. (7.76) over the interface of the two plasmas over the infinitesimal interval across the interface. Then we can obtain the jump condition of $\\Delta\\left\\{\\rho\\left(-\\omega+k_{x} U\\right) \\frac{\\partial v_{z}}{\\partial z}\\right\\}=0$. In both regions of the plasmas, physical quantities of $\\rho$ and $U$ are uniform. Then we may obtain $\\frac{\\partial^{2} v_{z}}{\\partial z^{2}}-\\left(k_{x}^{2}+k_{y}^{2}\\right) v_{z}=0$. Then $v_{z}$ is proportional to a combination of $\\exp ( \\pm k z)$ and should decrease with the increase

Fig. 7.11 Kelvin-Helmholtz instability in 3D. A jet of a plasma (fluid) with the normalized density of 1.2 is injected from the surface at $x=0$ with a speed of $50 \\mathrm{~m} / \\mathrm{s}$ into another plasma (fluid) with the density of 1.0 . The jet radius is $0.2 \\mathrm{~m}$ in this example. The jet is transported over $3.5 \\mathrm{~m}$ in $x$. However, the Kelvin-Helmoltz instability may prevent to transport the jet over a long distance
\\includegraphics[max width=\\textwidth, center]{2024_02_26_83e36543483eb7d284c1g-175}

in $z$. In addition, the boundary condition of Eq. (7.75) should be satisfied. In each region, we can obtain the following solution:

\\[
v_{z} \\propto \\begin{cases}\\left(-\\omega+k_{x} U_{1}\\right) e^{+k z} & (z<0)  \\tag{7.77}\\\\ \\left(-\\omega+k_{x} U_{2}\\right) e^{-k z} & (z>0)\\end{cases}
\\]

Here $k^{2}=k_{x}^{2}+k_{y}^{2}$. In this solution $z$ is continuous at the boundary of $z=0$. The jump condition above leads to the following equation:


\\begin{equation*}
\\rho_{2}\\left(-\\omega+k_{x} U_{2}\\right)^{2}+\\rho_{1}\\left(-\\omega+k_{x} U_{1}\\right)^{2}=0 \\tag{7.78}
\\end{equation*}


Now we obtain the growth rate $\\gamma$ :


\\begin{equation*}
\\gamma=\\frac{k_{x} \\sqrt{\\rho_{1} \\rho_{2}\\left(U_{1}-U_{2}\\right)^{2}}}{\\left(\\rho_{1}+\\rho_{2}\\right)} \\tag{7.79}
\\end{equation*}


The growth rate for the Kelvin-Helmholtz instability shows that when we have a small velocity difference of $\\Delta U$, the Kelvin-Helmholtz appears. The growth rate in Eq. (7.79) does no include $k_{y}$. It means the growth rate is sensitive to perturbations parallel to $\\Delta U$.

In Fig. 5.7 in Sect. 5.2.5, we computed the Kelvin-Helmholtz instability in the twodimensional (2D) space. The stability condition shows that any relative streaming induces the Kelvin-Helmholtz instability (see Chap. 11 in Ref. [9]). Here a jet flow in a plasma is again simulated in 3D by a 3D Euler fluid code in Sect. 5.2 and in Appendix F. A plasma (fluid) jet is injected in another plasma (fluid) with $50 \\mathrm{~m} / \\mathrm{s}$ from the surface at $x=0$. The jet normalized density is 1.2, and the background plasma density is set to be 1.0. The background plasma has no macroscopic velocity
initially. The jet is transported in a long distance. The simulation box is $3.5 \\mathrm{~m}$ long in $x, 1.8 \\mathrm{~m}$ in $y$ and $z$. The jet radius is $20 \\mathrm{~cm}$. Figure 7.11 shows a result for the Kelvin-Helmholtz instability in 3D. Figure 7.11 may present that the long-distance transport of the plasma jet would be prevented by the Kelvin-Helmholtz instability.

\\subsection*{Parametric Instability}
When three waves in plasmas have the following resonance conditions, the parametric instabilities appear (see, for example, Chap. 7 in Ref. [3], Chap. 3 in Ref. [6], and Refs. [11-13]):


\\begin{equation*}
\\omega_{0}=\\omega_{1}+\\omega_{2}, \\mathbf{k}_{0}=\\mathbf{k}_{1}+\\mathbf{k}_{2} \\tag{7.80}
\\end{equation*}


The first condition shows the energy conservation, and the second one shows the momentum conservation. Figure 7.12 shows a schematic figure for the stimulated Brillouin scattering (SBS), in which an electromagnetic wave $\\left(\\omega_{0}, \\mathbf{k}_{0}\\right)$ couples with another electromagnetic wave $\\left(\\omega_{1}, \\mathbf{k}_{1}\\right)$ and the ion plasma wave $\\left(\\omega_{2}, \\mathbf{k}_{2}\\right)$. When the resonance conditions in Eqs. (7.80) are fulfilled, the parametric instabilities would appear.

For example, an electromagnetic wave $\\left(\\omega_{0}, \\mathbf{k}_{0}\\right)$ would excite an electron plasma static wave $\\left(\\omega_{1}, \\mathbf{k}_{1}\\right)$ and an ion static wave $\\left(\\omega_{2}, \\mathbf{k}_{2}\\right)$. This is called the parametric decay instability. When an electromagnetic wave $\\left(\\omega_{0}, \\mathbf{k}_{0}\\right)$ couples with another electromagnetic wave $\\left(\\omega_{1}, \\mathbf{k}_{1}\\right)$ and the electron plasma wave $\\left(\\omega_{2}, \\mathbf{k}_{2}\\right)$, it is called the stimulated Raman scattering (SRS). If an electromagnetic wave energy is converted to static waves in a plasma, the energy would be absorbed effectively by the plasma. Electromagnetic waves propagate in a vacuum and escape from plasmas. However, static waves need plasma media and cannot escape from plasmas.

Fig. 7.12 Parametric instability. Figure 7.12 shows a schematic figure for the stimulated Brillouin scattering (SBS). When the resonance conditions in Eqs. (7.80) are fulfilled, the parametric instabilities would appear

\\begin{center}
\\includegraphics[max width=\\textwidth]{2024_02_26_83e36543483eb7d284c1g-176}
\\end{center}

\\subsection*{The Weibel Instability}
In the Sect.7.7, an interesting instability called the Weibel instability is introduced [14]. When plasmas have anisotropic temperatures, a magnetic field perturbation grows absolutely. Figure 7.13a shows a distribution function, which has an anisotropic electron temperature distribution: $T_{x} \\gg T_{y}$ and $T_{y}$. When a magnetic field perturbation $B_{z}$ appears, electrons moving in $-x$ are focused at zero points of $B_{z}$, and electrons oppositely moving in $+x$ are focused at the other $B_{z}$ zero points as shown in Fig. 7.13b. The filamented electron current enhances the initial perturbation of $B_{z}$. This is the Weibel instability [14].

Now the electron distribution function would be described as follows:


\\begin{equation*}
f\\left(v_{x}, v_{y}, v_{z}\\right)=n_{e}\\left(\\frac{m}{2 \\pi T_{x}}\\right)^{1 / 2}\\left(\\frac{m}{2 \\pi T_{\\perp}}\\right) \\exp \\left(-\\frac{v_{x}^{2}}{2 T_{x} / m}-\\frac{v_{y}^{2}+v_{z}^{2}}{2 T_{\\perp} / m}\\right) \\tag{7.81}
\\end{equation*}


Here $m$ shows the electron mass, and $T_{x} \\gg T_{\\perp}\\left(=T_{y}=T_{z}\\right)$. In this example case for the Weibel instability, the wave vector $\\mathbf{k}$ is parallel to $y$, the magnetic field perturbation is parallel to $z$ and the perturbation of the electric field is in the $x$ direction, as shown in Fig. 7.13b. Therefore, $\\mathbf{k}=(0, k, 0), \\mathbf{B}_{1}=\\left(0,0, B_{1 z}\\right)$ and $\\mathbf{E}_{1}=$ $\\left(E_{1}, 0,0\\right)$. For this case, from Eqs. (6.114) or (6.123), one can easily obtain the dispersion relation for the Weibel instability:


\\begin{align*}
\\epsilon_{x x} & =\\frac{k^{2} c^{2}}{\\omega^{2}} \\\\
& =1-\\frac{\\omega_{p}^{2}}{\\omega^{2}}+\\frac{\\omega_{p}^{2}}{\\omega^{2}} \\frac{T_{x}}{T_{\\perp}} W\\left(\\frac{\\omega}{k} \\sqrt{\\frac{m}{T_{\\perp}}}\\right) \\tag{7.82}
\\end{align*}


Here we ignore the ion contribution, and $\\omega_{p}$ shows the electron plasma frequency. For $\\frac{\\omega}{k} \\sqrt{\\frac{m}{T_{\\perp}}} \\gg 1$, approximately $W(Z) \\sim-1 / Z^{2}$ from Eq. (6.78). Then we obtain $\\omega$ for the Weibel instability:


\\begin{equation*}
\\omega^{2} \\sim-\\frac{\\omega_{p}^{2} k^{2} T_{x}}{m\\left(\\omega_{p}^{2}+k^{2} c^{2}\\right)} \\tag{7.83}
\\end{equation*}


The perturbations are proportional to $\\exp (-i \\omega t)$ in our case, and $\\omega=\\omega_{r}+i \\omega_{i}=$ $\\omega_{r}+i \\gamma$. Therefore, the growth rate $\\gamma$ is as follows:


\\begin{equation*}
\\gamma \\sim \\omega_{p} k \\sqrt{\\frac{T_{x}}{m\\left(\\omega_{p}^{2}+k^{2} c^{2}\\right)}} \\tag{7.84}
\\end{equation*}


\\section*{a) $f(v)$ with directiondependent temperature}
\\begin{center}
\\includegraphics[max width=\\textwidth]{2024_02_26_83e36543483eb7d284c1g-178}
\\end{center}

b) The Weibel instability

Current Plasma

\\begin{center}
\\includegraphics[max width=\\textwidth]{2024_02_26_83e36543483eb7d284c1g-178(1)}
\\end{center}

Fig. 7.13 Weibel instability, in which the distribution function has an anisotropic temperature. In this example case, the temperature in the $x$ direction is large compared with that in other directions. The perturbation of $B_{z}$ is enhanced by the electron filament generation by $B_{z}$

Here we found the real part of $\\omega_{r}=0$. It means that the Weibel instability is one of absolute instabilities, which have no real part $\\omega_{r}=0$ and do not propagate in plasmas.

\\subsection*{Filamentation Instability and an Example Simulation}
When an electron beam is injected into a plasma, background electrons respond and compensate the electron beam current. The background immobile ions neutralize the electron charge. In this situation, the energy of the electron beam may induce a magnetic field and makes filaments in the electron beam. Here the filamentation instability is introduced. The filamentation instability is very similar to the Weibel instability shown in Sect.7.7. Instead of the anisotropy of the temperature, the electron beam drives the filamentation instability.

Figure 7.14 shows the mechanism of the filamentation instability schematically. When a magnetic field perturbation $B_{1 z}$ appears initially, electrons moving in $x$ are focused at the $B=0$ points. The electron current focused at the null magnetic field points (or sheets) enhance the original magnetic field perturbation. This is the filamentation instability [15-20].

In the filamentation instability, the electron distribution function would be as follows:

Fig. 7.14 Filamentation instability. When a perturbation of magnetic field appears in plasmas, beam electrons and its return current are focused at the null points of the magnetic field. The electron current focused enhances the perturbed magnetic field

\\begin{center}
\\includegraphics[max width=\\textwidth]{2024_02_26_83e36543483eb7d284c1g-179}
\\end{center}


\\begin{align*}
f\\left(v_{x}, v_{y}, v_{z}\\right) & =n_{b}\\left(\\frac{m}{2 \\pi T_{b}}\\right)^{3 / 2} \\exp \\left(-\\frac{\\left(v_{x}-V_{b}\\right)^{2}+v_{y}^{2}+v_{z}^{2}}{2 T_{b} / m}\\right) \\\\
& +n_{p}\\left(\\frac{m}{2 \\pi T_{p}}\\right)^{3 / 2} \\exp \\left(-\\frac{\\left(v_{x}+V_{p}\\right)^{2}+v_{y}^{2}+v_{z}^{2}}{2 T_{p} / m}\\right) \\tag{7.85}
\\end{align*}


Here $n_{b}$ shows the number density of the beam electrons, $T_{b}$ the beam electron temperature, and $V_{b}$ shows the beam electron drift speed in the $x$ direction. The background electron number density is $n_{p}$ and its temperature is $T_{p}$. The beam electron current is compensated by the return current of the background electrons:


\\begin{equation*}
n_{b} V_{b}=n_{p} V_{p} \\tag{7.86}
\\end{equation*}


The electric charge is neutralized by the background immobile ions. The ion motion is again ignored here. In this case, the follows are also assumed: $\\mathbf{k}=(0, k, 0)$, $\\mathbf{B}_{1}=\\left(0,0, B_{1 z}\\right)$ and $\\mathbf{E}_{1}=\\left(E_{1}, 0,0\\right)$.

In the filamentation instability the dispersion relation is as follows:


\\begin{align*}
\\epsilon_{x x}=\\frac{k^{2} c^{2}}{\\omega^{2}} & \\\\
=1 & -\\frac{\\omega_{b}^{2}}{\\omega^{2}}+\\frac{\\omega_{b}^{2}}{\\omega^{2}}\\left(1+\\frac{V_{b}^{2}}{T_{b} / m}\\right) W\\left(\\frac{\\omega}{k} \\sqrt{\\frac{m}{T_{b}}}\\right) \\\\
& -\\frac{\\omega_{p}^{2}}{\\omega^{2}}+\\frac{\\omega_{p}^{2}}{\\omega^{2}}\\left(1+\\frac{V_{p}^{2}}{T_{p} / m}\\right) W\\left(\\frac{\\omega}{k} \\sqrt{\\frac{m}{T_{p}}}\\right) \\tag{7.87}
\\end{align*}


Here in the long wavelength limit or the the low temperature limit $\\left(\\left|\\frac{\\omega}{k} \\sqrt{\\frac{m}{T_{b}}}\\right| \\gg 1\\right.$ and $\\left|\\frac{\\omega}{k} \\sqrt{\\frac{m}{T_{p}}}\\right| \\gg 1$ ), one can obtain the following growth rate for the filamentation instability:


\\begin{equation*}
\\gamma \\sim k \\frac{\\sqrt{\\left(T_{b} / m+V_{b}^{2}\\right) \\omega_{b}^{2}+\\left(T_{p} / m+V_{p}^{2}\\right) \\omega_{b}^{2}}}{\\sqrt{\\omega_{b}^{2}+\\omega_{p}^{2}+k^{2} c^{2}}} \\tag{7.88}
\\end{equation*}


when $V_{b} \\gg V_{p}$, the growth rate may become $\\gamma \\sim k V_{b} \\omega_{b} / \\sqrt{\\omega_{p}^{2}+k^{2} c^{2}}$.

Here example 2D PIC (EPOCH [1, 2]) simulation results are shown in Fig.7.15. The electron beam speed in $x$ is $V_{b}=0.99 c$ and $V_{p}=-0.11 c$ in the $-x$ direction. Here $c$ is the speed of light. In order to compensate the beam current initially, $n_{p}=9 n_{b}$ and $n_{b}=1.0 \\times 10^{19} / \\mathrm{cm}^{3}$. The background ions compensate the electron charge. The beam temperature $T_{b}=100 \\mathrm{eV}$ and $T_{p}=T_{i}=T_{b}$, where $T_{i}$ shows the ion temperature. Figure 7.15a shows the electron beam number density $n_{b}$ and in b) $B_{z}$ is shown. The results show the clear filamentation instability.

In some cases, the beam electron speed is fast and the background electron number density is very high. In these cases, collision may contribute to the background electron behavior $[17,18]$. When the collisions for the background electrons cannot be ignored, the Vlasov equation in Sect. 6.1.3 for the background electron should be modified by including the collision term [21]. The simple Krook collision term was introduced in Eq. (6.43):


\\begin{equation*}
\\frac{\\partial f}{\\partial t}+(\\mathbf{v} \\cdot \\nabla) f+\\frac{q}{m}(\\mathbf{E}+\\mathbf{v} \\times \\mathbf{B}) \\cdot \\frac{\\partial f}{\\partial \\mathbf{v}}=-v\\left(f-f_{0}\\right) \\tag{6.43}
\\end{equation*}


When the collision effect is included in the background electrons in the filamentation instability, an interesting phenomenon appears. When the electron beam speed $V_{b}$ is high and $n_{b} \\ll n_{p}$, the current neutralization of Eq. (7.86) requires $V_{p}=n_{b} V_{b} / V_{p}$, and $V_{p} \\ll v_{b}$. Therefore, the collision effect may be notable for the return current electrons. In this case, the dispersion relation in Eq. (7.87) is modified as follows:


\\begin{align*}
& \\epsilon_{x x}=\\frac{k^{2} c^{2}}{\\omega^{2}}=1-\\frac{\\omega_{b}^{2}}{\\omega^{2}}+\\frac{\\omega_{b}^{2}}{\\omega^{2}}\\left(1+\\frac{V_{b}^{2}}{T_{b} / m}\\right) W\\left(\\frac{\\omega}{k} \\sqrt{\\frac{m}{T_{b}}}\\right) \\\\
& -\\frac{\\omega_{p}^{2}}{\\omega(\\omega+i \\nu)}+\\frac{\\omega_{p}^{2}}{\\omega(\\omega+i \\nu)}\\left\\{\\left(1+\\frac{V_{p}^{2}}{T_{p} / m}\\right)+\\frac{i v}{\\omega}\\left(\\frac{V_{p}^{2}}{T_{p} / m}\\right)\\right\\} W\\left(\\frac{\\omega}{k} \\sqrt{\\frac{m}{T_{p}}}\\right) \\tag{7.89}
\\end{align*}


References $[17,18]$ present that the growth rate for the filamentation instability is enhanced by the collision $v$. The background electrons flow with $V_{p}$ to compensate the electron beam current in plasmas, However, the collisions would reduce the return current, and the net current would be enhanced. Therefore, the filamentation growth

Fig. 7.15 Example 2D PIC simulation results for the filamentation instability. a The beam electron number density $n_{b}$ and $\\mathbf{b}$ the magnetic field $B_{z}$. The electron beam moving in the $x$ direction is filamented

\\begin{center}
\\includegraphics[max width=\\textwidth]{2024_02_26_83e36543483eb7d284c1g-181}
\\end{center}

rate becomes large compared with that in the collisionless cases. Here the explicit form for the growth rate $\\gamma$ is not shown. One can find the result in Fig. 1 in Ref. [18], which shows a clear enhancement of $\\gamma$ by the collision.

\\subsection*{Tearing Mode Instability and an Example Simulation}
A current $\\left(J_{x}\\right)$ sheet creates an antiparallel magnetic field in plasmas. Figure 7.16 shows the tearing mode instability and the magnetic reconnection [22-39]. One can find the current sheet formation and the magnetic field reconnection in plasmas in various areas: for example, solar flares [22], the earth magnetosphere [23, 38], controlled magnetic confinement fusion [25], particle accelerations [23] and laser interaction with matter in laboratory astrophysics [26-28].

Fig. 7.16 Tearing mode instability. The electron beam moving in the $+x$ direction is filamented, and the antiparallel magnetic field is reconnected. Source Ref. [29]

\\begin{center}
\\includegraphics[max width=\\textwidth]{2024_02_26_83e36543483eb7d284c1g-182}
\\end{center}

In the Sect.7.9, a collisionless tearing mode instability in plasmas[29] is introduced by a 3D RPOCH PIC simulation [1, 2]. The small amplitude modulations imposed on the current sheet induce the reconnection of the magnetic field and the electric current filamentation. The equilibrium state for the current sheet is based on the Harris solution presented in Ref. [40], which is used as the initial condition.

The electron density $n_{e}$ is $1.0 \\times 10^{14} \\mathrm{~m}^{-3}$ at the center of the current sheet, and the electrons move with $V=0.1 c$ in the $+x$ direction. The ions move with $-V$ in $x$. The plasma is a hydrogen plasma, and the ions compensate the electron charge: $n_{i}=n_{e}$. The electron and ion temperatures are $10 \\mathrm{eV}=T_{e}=T_{i}=T$. The initial density distribution is as follows [40]:


\\begin{equation*}
n(z)=\\frac{n_{0}}{\\cosh ^{2}(z / L)} \\tag{7.90}
\\end{equation*}


The initial magnetic field $B_{y}$ is as follows:


\\begin{equation*}
B_{y}(z)=B_{0} \\tanh (z / L) \\tag{7.91}
\\end{equation*}


a) $\\boldsymbol{n}_{\\mathrm{e}}$ at $t=0.7 \\mu \\mathrm{s}$

\\begin{center}
\\includegraphics[max width=\\textwidth]{2024_02_26_83e36543483eb7d284c1g-183}
\\end{center}

b) $n_{\\mathrm{i}}$ at $t=0.7 \\mu \\mathrm{s}$

\\begin{center}
\\includegraphics[max width=\\textwidth]{2024_02_26_83e36543483eb7d284c1g-183(1)}
\\end{center}

Fig. 7.17 Electron density $n_{e}$ and ion density $n_{i}$ for the tearing mode instability 3D simulation. Source Ref. [29]

The current sheet thickness $L$ is as follows:


\\begin{equation*}
L=\\frac{c}{V} \\lambda_{D}=\\frac{\\lambda_{D}}{\\beta} \\tag{7.92}
\\end{equation*}


Here $\\beta=V / c$. In our case, $L=0.1 \\mathrm{~m}$. For simplicity, we move to another frame, in which ions are at rest in the simulation. In the frame, the electrons move with the velocity $-2 V$ in the $+x$ direction and the electron density is related with the electron density by $n_{e}=n_{i}+2 \\beta^{2} n / \\gamma$ [41]. The electric field from the electric charge separation is $E_{z}=\\gamma \\beta B_{0} \\tanh (z / L)$ [41]. The growth rate $\\gamma$ of the tearing mode may be estimated by $v_{A} / L$, where $v_{A}$ is the Alfven speed and is about $2.18 \\times 10^{4} \\mathrm{~m} / \\mathrm{s}$. Therefore, $\\gamma \\sim 2.18 \\times 10^{6} / \\mathrm{s}$. The growth time scale $\\tau$ is $\\tau=1 / \\gamma=\\sim 0.458 \\mu \\mathrm{s}$.

In the simulation, the initial small perturbation is imposed on the electron density $n_{e}$ along the $y$ direction. The perturbation amplitude is $5 \\%$, and the wavelength $L_{y}$ is $10 \\mathrm{~cm}$.

Figures 7.17 show a) the electron number density $n_{e}$ and b) the ion number density $n_{i}$ at $t=0.7 \\mu \\mathrm{s}$. Figures 7.18 present the magnetic field strength $|\\mathbf{B}|$ at $x=0$ at a) $t=0$ and b) $t=0.7 \\mu \\mathrm{s}$. The tearing mode instability appears and the plasma filamentation is clearly shown in Fig. 7.17. At the same time, the magnetic field lines are reconnected and form the $X$ points (magnetic field null points).

The detail discussions can be found in Ref. [29] for the simulation of the collisionless tearing mode instability.

\\begin{center}
\\includegraphics[max width=\\textwidth]{2024_02_26_83e36543483eb7d284c1g-184}
\\end{center}

Fig. 7.18 Magnetic field strength at $x=0$ for the tearing mode instability 3D simulation. Source Ref. [29]

\\subsection*{Drift Instability}
When plasmas have spatial gradients in physical quantities, for example, density, temperature, pressure, magnetic field strength, etc., drift waves would appear and the plasmas would become unstable. For example, plasmas confined by magnetic field may universally have the gradients in physical quantities. The drift waves may propagate with the drift velocity introduced in Sect. 3.3.

Here we consider an electrostatic perturbation with $\\mathbf{E}=\\left(E_{x}, 0, E_{z}\\right)$ and $\\exp \\left(i k_{x} x+i k_{z} z-i \\omega t\\right)$ in a static magnetic field $B_{z}$. The electron inertia is ignored, and the ions are represented by the $E_{x} \\times B_{z}$ drift. The density gradient is in $y$ as shown in Fig. 7.19. The linearized equations are shown below. The ion continuity equation is approximated as follows:


\\begin{align*}
& -i \\omega n_{i 1}+i k_{x} n_{0} v_{i 1 x}+\\frac{\\partial n_{0} v_{i 1 y}}{\\partial y}+i k_{z} n_{0} v_{i 1 z} \\\\
\\sim & -i \\omega n_{i 1}+\\frac{\\partial n_{0}}{\\partial y} v_{i 1 y}=0 \\tag{7.93}
\\end{align*}


Here we ignored the second term of $i k_{x} n_{0} v_{i 1 x}$, because we assume that $v_{i 1 x}$ is small. The fourth term of $i k_{z} n_{0} v_{i 1 z}$ is also ignored, because $k_{z}<k_{x}$.

The ion perturbation of $v_{i 1}$ is estimated by the $E_{x 1} \\times B_{z}$ drift.


\\begin{equation*}
v_{i 1 y} \\sim-\\frac{E_{x 1}}{B_{z}} \\tag{7.94}
\\end{equation*}


The electron density may be expressed by the equilibrium distribution:


\\begin{align*}
n_{i 1} & \\sim n_{0}(\\exp (e \\phi / T)-1) \\sim \\frac{n_{0} e \\phi}{T_{0}}  \\tag{7.95}\\\\
E_{x 1} & \\sim-i k_{x} \\phi \\tag{7.96}
\\end{align*}


\\begin{center}
\\includegraphics[max width=\\textwidth]{2024_02_26_83e36543483eb7d284c1g-185}
\\end{center}

Fig. 7.19 Drift instability. The diamagnetic drift is created by $\\nabla p$. When an electrostatic perturbation of $\\mathbf{E}_{1}$ appears in the $x-z$ plane and $k_{x} \\gg k_{z}$, the $E_{1} \\times B_{z}$ drift modifies the electron density profile along $z$. At the places of $n_{1}>0$ along $z$ the electron number density becomes large and the electrons are decelerated by $\\mathbf{E}_{1}$. The electron energy goes to the wave. On the other hand, at the places of $n_{1}<0$ along $z$ the electron density decreases, and the electrons here absorb the wave energy. Totally the wave obtains the net energy from the electrons

Then we obtain the following:


\\begin{equation*}
\\frac{\\omega}{k_{x}}=\\frac{T_{0}}{e B_{z}} \\frac{1}{n_{0}} \\frac{\\partial n_{0}}{\\partial y} \\tag{7.97}
\\end{equation*}


The phase velocity of the drift wave is the diamagnetic drift velocity in Eq. (3.16). Figure 7.19 shows a schematic figure for the drift instability. The diamagnetic drift, which is created by the density gradient of $\\nabla n$ in Fig. 7.19, is the energy source, which drives the drift instability. When $\\mathbf{E}_{1}$ appears in the $x-z$ plane and $k_{x}>k_{z}$, the $E_{x 1} \\times B_{z}$ drift appears as shown in Fig. 7.19. Here we have a small but finite $k_{z}$. At the places of $n_{1}>0$ along $z$ the electron number density becomes large and the electrons are decelerated by $\\mathbf{E}_{1}$. The electron energy goes to the wave. This is just like an inverse process of the Landau damping. On the other hand, at the places of $n_{1}<0$ along $z$ the electron density decreases, and the electrons there absorb the wave energy. Totally the wave obtains the net energy from the electrons, and then the drift wave is enhanced. Along $z$ the electrons move freely. On the other hand, when the electrons meet collisions or resistance, the collision would contribute to sustain the potential $\\phi$, so that the resistive drift instability would appear (see, for example, Chap. 7 in [5], Chap. 6 in Ref. [42] and Refs. [43-47]).

Various drift waves are well studied and are important for stable plasma confinement in magnetic confinement fusion (MCF) devices and for studies on structure formation and zonal flows in MCF plasmas and space (see, for example, Chap. 14
in Ref. [4], Chap. 7 in Ref. [5], Chap. 6 in Ref. [42], Ref. [48], Chap. 7 in Ref. [49], Chap. 20 in Ref. [50], Chap. 9 in Ref. [51] and Ref. [52]). It is well known that the Jupiter shows the beautiful zonal flow, which would be related to the turbulence driven by the drift wave.

\\section*{References}
\\begin{enumerate}
  \\item T.D. Arber, K. Bennett et al., Contemporary particle-in-cell approach to laser-plasma modelling. Arber. Plasma Phys. Control. Fusion 57, 113001 (2015)

  \\item Bennett, K.: Users manual for the EPOCH PIC codes, EPOCH Version 4.3, (2014)

  \\item D.R. Nicholson, Introduction to Plasma Theory (Wiley, New York, 1983)

  \\item T.H. Stix, The Theory of Plasma Waves (MacGraw Hill, New York, 1962)

  \\item K. Miyamoto, Plasma Physics and Controlled Nuclear Fusion (Springer, Springer, Berlin, Heidelberg, 2013)

  \\item K. Niu, Nuclear Fusion (Cambridge University Press, Cambridge, 2009)

  \\item S. Kawata, T. Karino, Y.J. Gu, Phase control of a $z$-current-driven plasma column. Phys. Rev. E 101, 041201 (2020). \\href{https://doi.org/10.1103/PhysRevE.101.041201}{https://doi.org/10.1103/PhysRevE.101.041201}

  \\item R.C. Davidson, Theory of nonneutral plasmas, Frontiers in Phys., W. A. Benjamin (1974)

  \\item S. Chandrasekhar, Hydrodynamics and Hydromagnetic Stability (Dover pub, New York, 1970)

  \\item The summary on the Kelvin-Helmholtz can be found below (2021). \\href{https://www.sciencedirect}{https://www.sciencedirect}. com/topics/earth-and-planetary-sciences/kelvin-helmholtz-instability. Cited 15 August 2021

  \\item K. Nishikawa, Parametric excitation of coupled waves I. General formulation. J. Phys. Soc. Jpn. 24, 916-922 (1968)

  \\item K. Nishikawa, Parametric excitation of coupled waves. II. Parametric plasmon-photon interaction. J. Phys. Soc. Jpn., 24, 1152-1158 (1968)

  \\item K. Mima, K. Nishikawa, Parametric instabilities and wave dissipation, in plasmas, in Handbook of Plasma Physics, Basic Plasma Physics II, vol. 2, Sec. 6.5. ed. by A.A. Galeev, R.N. Sudan (North Holland Pub, Amsterdam, New York, Oxford, 1983), pp.451-517

  \\item E.S. Weibel, Spontaneously growing transverse waves in a plasma due to an anisotropic velocity distribution. Phys. Rev. Lett. 2, 83-84 (1959)

  \\item A. Bret, M.-C. Firpo, C. Deutsch, Characterization of the initial filamentation of a relativistic electron beam passing through a plasma. Phys. Rev. Lett. 94, 115002 (2005)

  \\item A. Bret, M.-C. Firpo, C. Deutsch, Collective electromagnetic modes for beam-plasma interaction in the whole k space. Phys. Rev. E 70, 046401 (2004)

  \\item T. Okada, K. Niu, Filamentation and two-stream instabilities of light ion beams in fusion target chambers. J. Phys. Soc. Jpn. 50, 3845-3846 (1981)

  \\item T. Okada, K. Niu, Effect of collisions on the relativistic electromagnetic instability. J. Plasma Phys. 24, 483-488 (1980)

  \\item R.F. Hubbard, D.A. Tidman, Filamentation instability of Ion beams focused in pellet-fusion reactors. Phys. Rev. Lett. 41, 866-870 (1978)

  \\item M. Tatarakis, F.N. Beg, E.L. Clark, A.E. Dangor, R.D. Edwards, R.G. Evans, T.J. Goldsack, K.W.D. Ledingham, P.A. Norreys, M.A. Sinclair, M.-S. Wei, M. Zepf, K. Krushelnick, Propagation instabilities of high-intensity laser-produced electron beams. Phys. Rev. Lett. 90, 175001 (2003)

  \\item P.L. Bhatnagar, E.P. Gross, M. Krook, A model for collision processes in gases. I. Small amplitude processes in charged and neutral one-component systems. Phys. Rev. 94, 511-525 (1954). \\href{https://doi.org/10.1103/PhysRev.94.511}{https://doi.org/10.1103/PhysRev.94.511}

  \\item E. Priest, T. Forbes, Magnetic Reconnection (Cambridge University Press, Cambridge, MHD Theory and Applications, 2000)

  \\item L.M. Zelenyi, H.V. Malova, A.V. Artemyev, V.Y. Popov, A.A. Petrukovich, Thin current sheets in collisionless plasma: equilibrium structure, plasma instabilities, and particle acceleration. Plasma Phys. Rep. 37, 118-160 (2011)

  \\item J. Birn, A.V. Artemyev, D.N. Baker, M. Echim, M. Hoshino, L.M. Zelenyi, Particle acceleration in the magnetotail and aurora. Space Sci. Rev. 173, 49-102 (2012)

  \\item Chapter 6, J. Wesson, Tokamaks, International Series of Monographs on Physics, 4th edn. (Oxford Science Publications, Oxford, 2011)

  \\item B.A. Remington, R.P. Drake, D.D. Ryutov, Experimental astrophysics with high power lasers and Z pinches. Rev. Mod. Phys. 78, 755-808 (2006)

  \\item S.V. Bulanov, TZh. Esirkepov, D. Habs, F. Pegoraro, T. Tajima, Relativistic laser-matter interaction and relativistic laboratory astrophysics. Eur. Phys. J. D, 55, 483-507 (2009)

  \\item S.V. Bulanov, Magnetic reconnection: from MHD to QED. Plasma Phys. Controlled Fusion 59, 014029 (2017)

  \\item Y.J. Gu, S. Kawata, S.V. Bulanov, Dynamic mitigation of the tearing mode instability in a collisionless current sheet. Sci Rep 11, 11651 (2021). \\href{https://doi.org/10.1038/s41598-02191111-8}{https://doi.org/10.1038/s41598-02191111-8}

  \\item Z. Chang, J.D. Callen, E.D. Fredrickson, R.V. Budny, C.C. Hegna, K.M. McGuire, M.C. Zarnstorff, TFTR group: observation of nonlinear neoclassical pressure-gradient-driven tearing modes in TFTR. Phys. Rev. Lett. 74, 4663-4666 (1995)

  \\item Y.J. Gu, F. Pegoraro, P.V. Sasorov, D. Golovin, A. Yogo, G. Korn, S.V. Bulanov, Electromagnetic burst generation during annihilation of magnetic field in relativistic laser-plasma interaction. Sci. Rep. 9, 19462 (2019)

  \\item M. Yamada, Y. Ren, H. Ji, J. Breslau, S. Gerhardt, R. Kulsrud, A. Kuritsyn, Experimental study of two-fluid effects on magnetic reconnection in a laboratory plasma with variable collisionality. Phys. Plasmas 13, 052119 (2006)

  \\item K. Fujimoto, R.D. Sydora, Plasmoid-induced turbulence in collisionless magnetic reconnection. Phys. Rev. Lett. 109, 265004 (2012)

  \\item S. Zenitani, M. Hesse, A. Klimas, C. Black, M. Kuznetsova, The inner structure of collisionless magnetic reconnection: the electron-frame dissipation measure and Hall fields. Phys. Plasmas 18, 122108 (2011)

  \\item S. Zenitani, M. Hesse, A. Klimas, C. Black, M. Kuznetsova, New measure of the dissipation region in collisionless magnetic reconnection. Phys. Rev. Lett. 106, 195003 (2011)

  \\item J.W. Dungey, Interplanetary magnetic field and the auroral zones. Phys. Rev. Lett. 6, 47-48 (1961)

  \\item M. Ottaviani, F. Porcelli, Nonlinear collisionless magnetic reconnection. Phys. Rev. Lett. 71, 3802-3805 (1993)

  \\item J. Birn, J.F. Drake, M.A. Shay, B.N. Rogers, R.E. Denton, M. Hesse, M. Kuznetsova, Z.W. Ma, A. Bhattacharjee, A. Otto, P.L. Pritchett, Geospace environmental modeling (GEM) magnetic reconnection challenge. J. Geophys. Res. 106, 3715-3719 (2001)

  \\item M. Yamada, R. Kulsrud, H. Ji, Magnetic reconnection. Rev. Mod. Phys. 82, 603-664 (2010)

  \\item E.G. Harris, On a plasma sheath separating regions of oppositely directed magnetic field. Nuovo Cimento 23, 115-121 (1962)

  \\item Chapter1, L.D. Landau, E.M. Lifshitz, The Classical Theory of Fields (ButterworthHeinemann, Amsterdam, 1980)

  \\item F.F. Chen, Introduction to Plasma Physics and Controlled Fusion, 3rd ed. (Springer, 2015)

  \\item L.I. Rudakov, R.Z. Sagdeev, Oscillations of an Inhomogeneous Plasma in a Magnetic Field. Soviet Phys. JETP 10, 952-954 (1960)

  \\item A.A. Galeev, V.N. Oraevskii, R.Z. Sagdeev, "Universal" instability of an inhomogeneous plasma in a magnetic field. Soviet Phys. JETP 17, 615-620 (1963)

  \\item A.B. Mikhailovskii, L.I. Rudakov, The stability of a spatially inhomogeneous plasma in a magnetic field. Soviet Phys. JETP 17, 621-625 (1963)

  \\item S.S. Moiseev, R.Z. Sagdeev, On the bohm diffusion coefficient. Soviet Phys. JETP 17, 515517 (1963)

  \\item W. Horton, Drift waves and transport. Rev. Mod. Phys. 71, 735-778 (1999)

  \\item K. Itoh, S.I. Itoh, P.H. Diamond, T.S. Hahm, A. Fujisawa, G.R. Tynan, M. Yagi, Y. Nagashima, Physics of zonal flows. Phys. Plasmas 13, 055502 (2006)

  \\item P.H. Diamond, S.-I. Itoh, K. Itoh, Modern Plasma Physics, vol. 1: Physical Kinetics of Turbulent Plasmas (Cambridge University Press, 2010)

  \\item R.J. Goldston, P.H. Rutherford, Introduction to Plasma Physics (Inst. of Physics Pub, Bristol and Philadelphia, 1995)

  \\item P. Davidson, Turbulence: An Introduction for Scientists and Engineers, 2nd edn. (Oxford University Press, Oxford, 2015)

  \\item F. Jenko, B.D. Scott, Numerical computation of collisionless drift wave turbulence. Phys. Plasmas 6, 2418-2424 (1999)

\\end{enumerate}

\\section*{Chapter 8 Short Introduction to Nonlinear Plasma Physics }
\\begin{abstract}
So far we have already seen some examples for nonlinear phenomena in plasmas: one is the ponderomotive force in Sect. 3.6 and Appendix D. Other examples are shown numerically in Chap. 7 for plasma instabilities, including the two-stream instability, the Kelvin-Helmholtz instability and the parametric instability. In Chap. 8, a short introduction is presented for plasma nonlinear phenomena, including soliton, plasma echo and plasma turbulence [1-7].
\\end{abstract}

\\subsection*{Solitary Wave: The Korteweg-de Vries (KdV) Equation}
A solitary wave was reported in Ref. [8], in which one can find beautiful figures on the waves at water surface. The solitary wave propagates with a constant velocity without changing its shape. Experimentally in plasmas solitary waves were observed $[9,10]$. Later, in 1985, one equation, called the Korteweg-de Vries (KdV) equation, was derived to describe the solitary waves in water in a channel [11]:


\\begin{equation*}
\\frac{\\partial u}{\\partial t}+u \\frac{\\partial u}{\\partial x}+\\delta^{2} \\frac{\\partial^{3} u}{\\partial x^{3}}=0 \\tag{8.1}
\\end{equation*}


The Korteweg-de Vries (KdV) equation was also obtained in plasma physics [12, 13]. As we have seen in Tips 5.1 in Sect. 5.1, the second term of the Korteweg-de Vries (KdV) equation is a nonlinear convection term, which induces wave steepening. The convection term appears commonly in the fluid equations in the Euler form. The third term introduces the wave dispersion. When a wave moves with an averaged specific speed of $C$ in Eq. (8.1) and $\\frac{\\partial u}{\\partial t}+C \\frac{\\partial u}{\\partial x} \\sim 0, u \\sim C+u^{\\prime}$ and Eq. (8.1) may be approximated as follows (see, for example, Chap. 2 in Ref. [5]): $\\frac{\\partial u^{\\prime}}{\\partial t}+$ $\\delta^{2} \\frac{\\partial^{3} u^{\\prime}}{\\partial x^{3}}=0$. Then the third term in Eq. (8.1) gives the dispersion proportional to $k^{3}$. For example, the dispersion relation for the ion acoustic wave in Eq. (5.82) also provides the dispersion relation proportional to $k^{3}$ for $T_{i}=0$. In Ref. [13], they found the Korteweg-de Vries ( $\\mathrm{KdV}$ ) equation for the ion acoustic wave.

\\includegraphics[max width=\\textwidth, center]{2024_02_26_83e36543483eb7d284c1g-190(2)}
b) $t=1 / \\pi$

\\includegraphics[max width=\\textwidth, center]{2024_02_26_83e36543483eb7d284c1g-190}
c) $t=3.6 / \\pi$

\\begin{center}
\\includegraphics[max width=\\textwidth]{2024_02_26_83e36543483eb7d284c1g-190(1)}
\\end{center}

Fig. 8.1 Numerical solutions for solitons by the Korteweg-de Vries (KdV) equation at $\\mathbf{a} t=0, \\mathbf{b}$ $t=1 / \\pi$ and $\\mathbf{c} t=3.6 / \\pi[14]$

In Ref. [14], Zabusky and Kruskal analyzed the Korteweg-de Vries (KdV) equation numerically and found the solitary waves, which are called "solitons". Figure 8.1 shows the numerical solutions for the solitons by the Korteweg-de Vries (KdV) equation. Following Zabusky and Kruskal in Ref. [14], the initial condition is $u(t=0)=\\cos (\\pi x)$ as shown in Fig. 8.1a, and $\\delta=0.022$. Figures $8.1 \\mathrm{~b}$ and $\\mathrm{c}$ show the profiles of $u$ (b) at $t=1.0 / \\pi$ and (c) at $t=3.6 / \\pi$, respectively. Zabusky and Kruskal found that the solitons keep their shape and are isolated with each other [14].

In Ref. [14], Eq. (8.1) is discretized as follows on uniformly discretized spatial meshes:


\\begin{align*}
\\frac{u_{i}^{n+1}-u_{i}^{n-1}}{2 \\Delta t} & +\\left.\\frac{1}{3}\\left(u_{i+1}+u_{i}+u_{i-1}\\right) \\frac{u_{i+1}-u_{i-1}}{2 \\Delta x}\\right|^{n} \\\\
& +\\left.\\delta^{2} \\frac{u_{i+2}-2 u_{i+1}+2 u_{i-1}-u_{i-2}}{2(\\Delta x)^{3}}\\right|^{n}=0 \\tag{8.2}
\\end{align*}


The discretization method is the central difference one in time and in space (see also Eqs. (5.14) and (5.17) in Sect. 5.2.1). In Fig. 8.1, the spatial mesh size $\\Delta x$ is $1.0 / N$, and $N=128$ in this example analysis. The time step of $\\Delta t$ was selected to satisfy the numerical stability presented in Sects. 5.2.3 and 5.2.4. One can also find other discretization methods, for example, in Refs. [15, 16].

\\subsubsection*{The Korteweg-de Vries (KdV) Equation for Ion Acoustic Wave}
In Sect. 5.5, the ion acoustic wave was analyzed linearly. As shown above, the linear dispersion relation in Eq. (5.82) in Sect. 5.5 also shows the dispersive property. Here the Korteweg-de Vries (KdV) equation is derived for the ion acoustic wave with a finite amplitude, following Washimi and Taniuchi in Ref. [13]. In Sect. 5.5, the electron inertia was neglected, and the finite ion temperature was included in Eqs. (5.75):

\\[
\\left\\{\\begin{array}{l}
\\frac{\\partial n_{e}}{\\partial t}+\\frac{\\partial\\left(n_{e} v_{e x}\\right)}{\\partial x}=0  \\tag{5.75}\\\\
0=-\\gamma T_{e} \\frac{\\partial n_{e}}{\\partial x}-e n_{e} E \\\\
\\frac{\\partial n_{i}}{\\partial t}+\\frac{\\left(n_{i} v_{i x}\\right)}{\\partial x}=0 \\\\
m_{i} n_{i}\\left(\\frac{\\partial v_{i x}}{\\partial t}+v_{i x} \\frac{\\partial v_{i x}}{\\partial x}\\right)=-\\gamma T_{i} \\frac{\\partial n_{i}}{\\partial x}+e n_{i} E \\\\
\\frac{\\partial E}{\\partial x}=\\frac{1}{\\epsilon_{0}} e\\left(n_{i}-n_{e}\\right)
\\end{array}\\right.
\\]

Now $T_{i} \\rightarrow 0$, and the first equation in Eqs. (5.75) is not needed, since the electron mass $m_{e}$ is ignored. After appropriate normalizations, the following equations are used in Ref. [13]:


\\begin{align*}
\\frac{\\partial n_{i}}{\\partial t}+\\frac{\\partial n_{i} v_{i}}{\\partial x} & =0  \\tag{8.3}\\\\
\\frac{\\partial v_{i}}{\\partial t}+v_{i} \\frac{\\partial v_{i}}{\\partial x} & =E  \\tag{8.4}\\\\
\\frac{\\partial n_{e}}{\\partial x} & =-n_{e} E  \\tag{8.5}\\\\
\\frac{\\partial E}{\\partial x} & =n_{i}-n_{e} \\tag{8.6}
\\end{align*}


Now the following scale transformation is introduced in Ref. [13]:


\\begin{align*}
& \\xi=\\epsilon^{1 / 2}(x-t)  \\tag{8.7}\\\\
& \\eta=\\epsilon^{3 / 2} x \\tag{8.8}
\\end{align*}


Here $\\epsilon$ is a small parameter. Based on the transformation, $\\frac{\\partial}{\\partial t}=-\\epsilon^{1 / 2} \\frac{\\partial}{\\partial \\xi}$ and $\\frac{\\partial}{\\partial x}=$ $\\epsilon^{1 / 2} \\frac{\\partial}{\\partial \\xi}+\\epsilon^{3 / 2} \\frac{\\partial}{\\partial \\eta}$

Then we obtained the following equations:


\\begin{align*}
-\\frac{\\partial n_{i}}{\\partial \\xi}+\\frac{\\partial n_{i} v_{i}}{\\partial \\xi}+\\epsilon \\frac{\\partial n_{i} v_{i}}{\\partial \\eta} & =0  \\tag{8.9}\\\\
-\\frac{\\partial v_{i}}{\\partial \\xi}+v_{i}\\left(\\frac{\\partial v_{i}}{\\partial \\xi}+\\epsilon \\frac{\\partial v_{i}}{\\partial \\eta}\\right) & =\\tilde{E}  \\tag{8.10}\\\\
\\frac{\\partial n_{e}}{\\partial \\xi}+\\epsilon \\frac{\\partial n_{e}}{\\partial \\eta} & =-n_{e} \\tilde{E}  \\tag{8.11}\\\\
\\epsilon \\frac{\\partial \\tilde{E}}{\\partial \\xi}+\\epsilon^{2} \\frac{\\partial \\tilde{E}}{\\partial \\eta} & =n_{i}-n_{e} \\tag{8.12}
\\end{align*}


Here the following relation is introduced: $E=\\epsilon^{1 / 2} \\tilde{E}$. In Ref. [13], it is assumed that the following series expansions in $\\epsilon$ hold:


\\begin{align*}
n_{i(, e)} & =1+\\epsilon n_{i(, e)}^{(1)}+\\epsilon^{2} n_{i(, e)}^{(2)}+\\cdots  \\tag{8.13}\\\\
v_{i} & =\\epsilon v_{i}^{(1)}+\\epsilon^{2} v_{i}^{(2)}+\\cdots  \\tag{8.14}\\\\
\\tilde{E} & =\\epsilon \\tilde{E}^{(1)}+\\epsilon^{2} \\tilde{E}^{(2)}+\\cdots \\tag{8.15}
\\end{align*}


As the first-order equations, we obtain the following equations: $\\frac{\\partial n_{i}^{(1)}}{\\partial \\xi}=\\frac{\\partial v_{i}^{(1)}}{\\partial \\xi}$, $\\frac{\\partial v_{i}^{(1)}}{\\partial \\xi}=-\\tilde{E}, \\frac{\\partial n_{e}^{(1)}}{\\partial \\xi}=-\\tilde{E}$ and $n_{i}^{(1)}=n_{e}^{(1)}$. It is also assumed that disturbances are localized, and then $n_{i}^{(1)}=v_{i}^{(1)}$. From the second-order terms, the followings are obtained:


\\begin{align*}
-\\frac{\\partial n_{i}^{(2)}}{\\partial \\xi}+\\frac{\\partial n_{i}^{(1)} v_{i}^{(1)}}{\\partial \\xi}+\\frac{\\partial v_{i}^{(2)}}{\\partial \\xi}+\\frac{\\partial v_{i}^{(1)}}{\\partial \\eta} & =0  \\tag{8.16}\\\\
-\\frac{\\partial v_{i}^{(2)}}{\\partial \\xi}+v_{i}^{(1)} \\frac{\\partial v_{i}^{(1)}}{\\partial \\xi} & =\\tilde{E}^{(2)}  \\tag{8.17}\\\\
\\frac{\\partial n_{e}^{(2)}}{\\partial \\xi}+\\frac{\\partial n_{e}^{(1)}}{\\partial \\eta} & =-n_{e}^{(1)} \\tilde{E}^{(1)}-\\tilde{E}^{(2)}  \\tag{8.18}\\\\
\\frac{\\partial \\tilde{E}^{(1)}}{\\partial \\xi} & =n_{i}^{(2)}-n_{e}^{(2)} \\tag{8.19}
\\end{align*}


After eliminating the second-order terms, Washimi and Taniuchi obtained the Korteweg-de Vries (KdV) equation for ion acoustic wave [13]:


\\begin{equation*}
\\frac{\\partial v_{i}^{(1)}}{\\partial \\eta}+v_{i}^{(1)} \\frac{\\partial v_{i}^{(1)}}{\\partial \\xi}+\\frac{1}{2} \\frac{\\partial^{3} v_{i}^{(1)}}{\\partial \\xi^{3}}=0 \\tag{8.20}
\\end{equation*}


\\subsubsection*{Property of the Korteweg-de Vries (KdV) Equation}
In Fig. 8.1, the numerical results are shown for the Korteweg-de Vries (KdV) Eq. (8.1). After the time Figs. 8.1c, 8.2 show the results for the propagation of the solitons at (a) $t=4.0 / \\pi$, (b) $t=4.75 / \\pi$ and $t=6.7 / \\pi$. Figure 8.2 shows that each soliton keeps its shape and propagates with a constant velocity. Each soliton is isolated and stable. One soliton is not scattered by another and keeps its identity.

The Korteweg-de Vries (KdV) in Eq. (8.1) has interesting solutions, including one soliton, and the following solution outlines the characteristic of the single soliton (see, for example, Chap. 5 in Ref. [1], Chap. 2 in Ref. [5] and Ref. [14] and Chap. 7 in Ref. [17]):


\\begin{equation*}
u(t, x)=u_{\\infty}+3\\left(C-u_{\\infty}\\right) \\operatorname{sech}^{2}\\left(\\frac{x-C t}{2 \\delta / \\sqrt{C-u_{\\infty}}}\\right) \\tag{8.21}
\\end{equation*}


Here $C$ shows the soliton speed, and $u_{\\infty}$ is $u$ at $x= \\pm \\infty$. Equation (8.21) for the single soliton would show that the larger the soliton amplitude $3 C$, the faster the soliton and the narrower the width. In Fig. 8.2, the soliton " 1 " has the larger amplitude, and the soliton "1" overtakes other smaller solitons as shown in Fig. 8.2b and $\\mathrm{c}$.

In Ref. [18], Gardner, et al. presented a remarkable finding to solve the Kortewegde Vries (KdV) equation by the inverse scattering transform method, which is explained in detail in Chap. 5 in Ref. [1], Chap. 2 in Ref. [5] and Chap. 7 in Ref. [17] and Refs. [18-20]. The Korteweg-de Vries (KdV) equation in Eq. (8.1) also has conservation laws (see, for example, Chap. 2 in Ref. [5] and Refs. [20, 21]).

In Ref. [18], Gardner et al. started from the following Korteweg-de Vries (KdV) equation:


\\begin{equation*}
\\frac{\\partial u}{\\partial t}-6 u \\frac{\\partial u}{\\partial x}+\\frac{\\partial^{3} u}{\\partial x^{3}}=0 \\tag{8.22}
\\end{equation*}


Equation (8.1) becomes Eq. (8.22), when $u \\rightarrow-6 u, x \\rightarrow x \\delta$ and $t \\rightarrow t \\delta$ are introduced in Eq. (8.1). Then the following time-independent Schrödinger equation is introduced:


\\begin{equation*}
-\\frac{\\partial^{2} \\Psi}{\\partial x^{2}}+u \\Psi=\\lambda \\Psi \\tag{8.23}
\\end{equation*}


a) $t=4.0 / \\pi$

\\includegraphics[max width=\\textwidth, center]{2024_02_26_83e36543483eb7d284c1g-194}
b) $t=4.75 / \\pi$

\\includegraphics[max width=\\textwidth, center]{2024_02_26_83e36543483eb7d284c1g-194(2)}
c) $t=6.7 / \\pi$

\\begin{center}
\\includegraphics[max width=\\textwidth]{2024_02_26_83e36543483eb7d284c1g-194(1)}
\\end{center}

Fig. 8.2 Propagation of solitons by the Korteweg-de Vries (KdV) equation after the results in Fig. 8.1 at $\\mathbf{a} t=4.0 / \\pi, \\mathbf{b} t=4.75 / \\pi$ and $\\mathbf{c} t=6.7 / \\pi[14]$

Here $u$ is the potential which satisfies the Korteweg-de Vries (KdV) equation in Eq. (8.22), and $\\lambda$ would be eigenvalue. The Korteweg-de Vries (KdV) equation is invariant in the Galilean transformation: $u \\rightarrow u-\\lambda, x \\rightarrow x-6 \\lambda t$ and $t \\rightarrow t$. Now $u$ is solved by Eq. (8.23): $u=\\lambda+\\frac{\\partial^{2} \\Psi}{\\partial x^{2}} / \\Psi$. Then $u$ is inserted into the Korteweg-de Vries (KdV) equation in Eq. (8.22), and the following is obtained:


\\begin{equation*}
\\frac{\\partial \\lambda}{\\partial t} \\Psi^{2}+\\frac{\\partial}{\\partial x}\\left(\\Psi \\frac{\\partial Q}{\\partial x}-\\frac{\\partial \\Psi}{\\partial x} Q\\right)=0 \\tag{8.24}
\\end{equation*}


Here $Q$ is as follows:


\\begin{equation*}
Q=\\frac{\\partial \\Psi}{\\partial t}+\\frac{\\partial^{3} \\Psi}{\\partial x^{3}}-3(u+\\lambda) \\frac{\\partial \\Psi}{\\partial x} \\tag{8.25}
\\end{equation*}


Here Eq. (8.24) is integrated by $x$ over $(-\\infty, \\infty)$. If $\\Psi$ goes to zero at $x \\rightarrow \\pm \\infty$, the right side of Eq. (8.24) vanishes. Now we may obtain $\\frac{\\partial \\lambda}{\\partial t}=0$. Therefore, the eigenvalue $\\lambda$ of Eq. (8.23) is constant. Then the following is obtained:


\\begin{equation*}
\\Psi \\frac{\\partial Q}{\\partial x}-\\frac{\\partial \\Psi}{\\partial x} Q=\\mathrm{constant} \\equiv C_{1} \\tag{8.26}
\\end{equation*}


After integrating Eq. (8.26) again, the following is obtained:


\\begin{equation*}
\\frac{Q}{\\Psi}=C_{1} \\int \\frac{d x}{\\Psi^{2}}+C_{2} \\tag{8.27}
\\end{equation*}


Here $C_{1}$ and $C_{2}$ are the integration constants. Then Eq. (8.27) becomes as follows:


\\begin{equation*}
\\frac{\\partial \\Psi}{\\partial t}+\\frac{\\partial^{3} \\Psi}{\\partial x^{3}}-3(u+\\lambda) \\frac{\\partial \\Psi}{\\partial x}=C_{1} \\Psi \\int \\frac{d x}{\\Psi^{2}}+C_{2} \\Psi \\tag{8.28}
\\end{equation*}


\\subsubsection*{Inverse Scattering Transform for the Korteweg-de Vries (KdV) Equation}
For Eq. (8.23), bound states would be found as $\\lambda_{n}=-k_{n}^{2}$. For the continuum, the energy would be $\\lambda=k^{2}$. Now In Ref. [18] Gardner et al. considered a wave scattering problem by the potential of $u$ in Eq. (8.22). An incoming wave of $\\exp (-\\mathrm{ikx})$ from $x=+\\infty$ is considered, and now the amplitude is set to 1 . A part of the wave is transmitted with the amplitude of $|a|$, and the rest is reflected with the amplitude of $|b|$. Therefore, $|a|^{2}+|b|^{2}=1$ :


\\begin{align*}
& \\Psi=\\exp (-\\mathrm{ikx})+b \\exp (+\\mathrm{kx}), \\text { at } x \\rightarrow \\infty  \\tag{8.29}\\\\
& \\Psi=a \\exp (-\\mathrm{ikx}), \\text { at } x \\rightarrow-\\infty \\tag{8.30}
\\end{align*}


For the discrete eigenvalues of $\\lambda_{n}=-k_{n}^{2}$, when $x \\rightarrow \\pm \\infty, \\Psi \\rightarrow 0$ and $1 / \\Psi^{2} \\rightarrow$ $\\infty$. Therefore, $C_{1}=0$ should be satisfied in Eq. (8.28). After multiplying Eq. (8.28) by $\\Psi$ and also using Eq. (8.23), we obtain $\\frac{\\partial}{\\partial t} \\int d x \\Psi^{2} / 2=C_{2} \\int d x \\Psi^{2}$. Here the integral domain in $x$ is $(-\\infty,+\\infty)$. Because $\\Psi$ is normalized as $\\int d x \\Psi^{2}=1, C_{2}=0$ for the discrete eigenvalues of $\\lambda_{n}=-k_{n}^{2}$. At $x \\rightarrow \\infty, \\Psi \\sim c_{n} \\exp \\left(-k_{n} x\\right)$, which is inserted into Eq. (8.28) with $C_{1}=C_{2}=0$ to obtain the following:


\\begin{equation*}
\\frac{\\partial c_{n}}{\\partial t}-4 k_{n}^{3} c_{n}=0 \\tag{8.31}
\\end{equation*}


Then we obtain $c_{n}(t)=c_{n}(0) \\exp \\left(4 k_{n}^{3} t\\right)$. For the continuum $\\lambda=k^{2}$, Eqs. (8.29) and (8.30) are used in Eq. (8.28). Then the followings are obtained:


\\begin{align*}
a(t) & =a(0)  \\tag{8.32}\\\\
b(t) & =b(0) \\exp \\left(8 i k^{3} t\\right) \\tag{8.33}
\\end{align*}


Now we got the information for the scattering by the potential $u$. In this case, Gel'fand and Levitan in Ref. [22,23] show that one can reconstruct the original potential $u$ from the data for the scattering [22-25]. Based on the Gel'fand-Levitan equation one can obtain $u$ by the following:


\\begin{equation*}
u(x, t)=-2 \\frac{\\partial K(x, x)}{\\partial x} \\tag{8.34}
\\end{equation*}


The Gel'fand-Levitan equation is as follows:

\\[
\\begin{array}{r}
\\qquad K(x, y)+B(x+y)+\\int_{x}^{\\infty} K(x, z) B(y+z) \\mathrm{d} z=0 \\\\
\\text { Here } B(x)=\\frac{1}{2 \\pi} \\int_{-\\infty}^{\\infty} d k b(k) \\exp (\\mathrm{ikx})+\\sum_{n} c_{n}^{2} \\exp \\left(-k_{n} x\\right) . \\tag{8.36}
\\end{array}
\\]

In order to summarize the calculation process in the inverse scattering transform method, here the initial part of an example calculation is shown for two solitons with the eigenvalue of $\\lambda_{1}=-k_{1}^{2}$ and $\\lambda_{2}=-k_{2}^{2}$. Now we have no reflection: $b=0$. Then $B(x)=c_{1}(t)^{2} \\exp \\left(-k_{1} x\\right)+c_{2}(t)^{2} \\exp \\left(-k_{2} x\\right)$. From Eq. (8.35), The Gel'fandLevitan equation becomes as follows:


\\begin{align*}
K(x, y) & +\\sum_{n=1,2} c_{n}^{2}(0) \\exp \\left(-k_{n}(x+y)+8 k_{n}^{3} t\\right) \\\\
& +\\int_{x}^{\\infty} K(x, z) \\sum_{n=1,2} c_{n}^{2}(0) \\exp \\left(-k_{n}(y+z)+8 k_{n}^{3} t\\right) \\mathrm{d} z=0 \\tag{8.37}
\\end{align*}


From this Gel'fand-Levitan equation, $K(x, y)$ is solved; for example, $K(x, y)$ may be written as follows: $K(x, y)=\\sum_{n=1,2} h_{n} \\exp \\left(-k_{n} y\\right)$. Then two equations are obtained for $h_{1}$ and $h_{2}$. By Eq. (8.34), $u$ is obtained for two solitons and is a solution of Eq. (8.22), the Korteweg-de Vries (KdV) equation.

In the Korteweg-de Vries (KdV) equation in Eq. (8.1) or in Eq. (8.22), the nonlinear convection term of $u \\frac{\\partial u}{\\partial x}$ and the dispersion term of $\\frac{\\partial^{3} u}{\\partial u^{3}}$ are extracted from the basic equations for plasmas.

\\subsection*{The Burgers Equation and Shock Wave}
In the basic equation for plasmas, other physical properties are included. One of them is a diffusion, which is well known, for example, as heat conduction for temperature
or viscosity of fluid or so. The Burgers equation includes the nonlinear term of $u \\frac{\\partial u}{\\partial x}$ and the diffusion term of $\\frac{\\partial^{2} u}{\\partial u^{2}}[26]$ :


\\begin{equation*}
\\frac{\\partial u}{\\partial t}+u \\frac{\\partial u}{\\partial x}-v \\frac{\\partial^{2} u}{\\partial x^{2}}=0 \\tag{8.38}
\\end{equation*}


The Burgers equation can be solved analytically by the Cole-Hopf transformation [27, 28]:


\\begin{equation*}
u=-2 v \\frac{\\Psi_{x}}{\\Psi}=-2 v \\frac{\\partial}{\\partial x} \\ln (\\psi) \\tag{8.39}
\\end{equation*}


The Burgers equation is transformed to a diffusion equation as follows:


\\begin{equation*}
\\frac{\\partial \\Psi}{\\partial t}-v \\frac{\\partial^{2} \\Psi}{\\partial x^{2}}=0 \\tag{8.40}
\\end{equation*}


The diffusion equation was already introduced and numerically solved with its numerical stability in Sect. 5.2.6.

The diffusion effect was also important to describe shock waves in the fluid model, shown in Sect. 5.2.5. In Sect. 5.2.5, the artificial viscosity was introduced to describe shock waves numerically (see, for example, Chap. 12 in Ref. [29]). At the shock waves, fluid ordered motion is transformed to disordered motion. The shock wave thickness does not affect the jump condition, that is, the Rankine-Hugoniot condition at the shock wave (see, for example, Chaps. 9 and 10 in Ref. [30]). Therefore, the shock wave thickness is artificially expanded to the spatial mesh size in fluid numerical simulations, as discussed in Sect. 5.2.5. In Sect. 5.2.5 the second derivative, that is, the dissipation term, was explicitly introduced to describe shock waves in the fluid simulations (see Eqs. (5.32)- (5.34)).

In Fig. 8.3, a numerical solution for the Burgers equation in Eq. (8.38) is presented. Initially a sinusoidal wave is localized (see the dotted line). During the propagation, a wave-like shock wave is formed as shown in Fig. 8.3 by the solid line. In Fig. 8.3, $v=0.1$. In the computation the following discretization equation is used:


\\begin{align*}
\\frac{u_{i}^{n+1}-u_{i}^{n}}{\\Delta t} & +\\left\\{\\frac{u_{i}^{n}+\\left|u_{i}^{n}\\right|}{2}\\left(\\frac{u_{i}^{n}-u_{i-1}^{n}}{\\Delta x}\\right)+\\frac{u_{i}^{n}-\\left|u_{i}^{n}\\right|}{2}\\left(\\frac{u_{i+1}^{n}-u_{i}^{n}}{\\Delta x}\\right)\\right\\} \\\\
& -v \\frac{u_{i+1}^{n}-2 u_{i}^{n}+u_{i-1}^{n}}{\\Delta x^{2}}=0 \\tag{8.41}
\\end{align*}


Here $n$ shows the time index, and $n$ means $t=n \\Delta t$. The index $i$ shows the spatial position in $x$.

\\begin{center}
\\includegraphics[max width=\\textwidth]{2024_02_26_83e36543483eb7d284c1g-198}
\\end{center}

Fig. 8.3 Solution of the Burgers equation. Initially a sinusoidal wave is localized as shown by the dotted line. Then a wave-like shock wave appears as shown in figure by the solid line. In this example, $v=0.1$

\\subsection*{Plasma Echo}
In Sect. 6.8, perturbation propagation in plasmas is studied briefly. When an impulse of $\\varphi_{\\mathrm{ext1}}(x, t)=\\varphi_{01} \\delta\\left(\\omega_{\\mathrm{pe}} t\\right) \\exp \\left(i k_{1} x\\right)$ is applied to plasma at $t=0$, collective modes and macroscopic quantities damp through the Landau damping, though the microscopic information of the particle distribution function is kept in the following form:


\\begin{equation*}
\\left.\\delta f\\right|_{1}(x, v, t) \\propto \\exp \\left(i k_{1} x-i k_{1} v t\\right) \\tag{8.42}
\\end{equation*}


Based on this interesting plasma property, plasma echo was found theoretically and experimentally [31-40]. Two pulses are applied to plasmas separately in time or in space. After macroscopic quantities disappear through the Landau damping, macroscopic quantities reappear. Here briefly its physics is introduced below. The detailed discussions and derivation can be found in Chap. 5 in Ref. [1], Chap. 5 in Ref. [5], Chap. 2 in Ref. [41] and Chap. 8 in Ref. [42].

At $t=0$ the first impulse is applied to a plasma as shown above. At $t$, which is larger than the Landau damping time $1 /\\left|\\omega_{i}\\right|$, the potential induced by the first impulse damps. Then at $t=\\tau>>1 /\\left|\\omega_{i}\\right|$ the second impulse $\\varphi_{\\mathrm{ext} 2}(x, t)=$ $\\varphi_{02} \\delta\\left(\\omega_{p e}(t-\\tau)\\right) \\exp \\left(i k_{2} x\\right)$ is applied to the plasma. Then the perturbation of the distribution function by the second pule survives again as follows: $\\left.\\delta f\\right|_{2}(x, v, t) \\propto$ $\\left.\\exp \\left(i k_{2} x-i k_{2} v(t-\\tau)\\right)\\right)$. The Vlasov equation and the Poisson equation are again shown below for our 1D case:

\\[
\\begin{array}{r}
\\frac{\\partial f(x, v, t)}{\\partial t}+v \\frac{\\partial f}{\\partial x}-\\frac{q}{m} \\frac{\\partial \\varphi}{\\partial x} \\frac{\\partial f}{\\partial v}=0 \\\\
\\frac{\\partial^{2} \\varphi}{\\partial x^{2}}=-\\frac{q}{\\epsilon_{0}}\\left(\\int \\mathrm{d} v f-1\\right)=-\\frac{q}{\\epsilon_{0}} \\int \\delta f \\tag{8.44}
\\end{array}
\\]

The external potential $\\varphi(x, t)=\\varphi_{01} \\delta\\left(\\omega_{p e} t\\right) \\exp \\left(i k_{1} x\\right)+\\varphi_{02} \\delta\\left(\\omega_{p e}(t-\\tau)\\right)$ $\\exp \\left(i k_{2} x\\right)$. When $f$ is expanded as $f=f_{0}+\\delta f$, the second-order term comes from the last term of the left side of the Vlasov equation.

From the second impulse, the perturbation of the distribution function may be as follows: $\\left.\\left.\\delta f\\right|_{2}(x, v, t) \\propto \\exp \\left(i k_{2} x-i k_{2} v(t-\\tau)\\right)\\right)$. The $\\left.\\delta f\\right|_{1}$ in Eq. (8.42) and $\\left.\\delta f\\right|_{2}$ couple with each other through the second-order nonlinear term:


\\begin{equation*}
\\exp \\left[i\\left(k_{1} \\pm k_{2}\\right) x-\\left\\{i k_{1} v t \\pm i k_{2} v(t-\\tau)\\right\\}\\right] \\tag{8.45}
\\end{equation*}


Therefore, when $k_{1} v t \\pm k_{2} v(t-\\tau)=0$ is satisfied, the the wave with form of $\\exp \\left\\{i\\left(k_{1} \\pm k_{2}\\right) x\\right\\}$ would reappear. This is the plasma echo. Now $t>\\tau$, and so $k_{1} v t-$ $k_{2} v(t-\\tau)=0$. Therefore, at $t=\\frac{k_{2} \\tau}{k_{2}-k_{1}}$ for $k_{2}>k_{1}$, the plasma echo appears.

\\subsection*{A Glimpse at Turbulence}
In the Sect. 8.4, turbulence in fluids and plasmas is briefly introduced. The scientific area of plasma turbulence has been studied intensively, and good instructive texts have been published [1-7]. From perturbations and fluctuations, complex flows and waves would appear with eddies or vortices in plasmas and fluids. Through instabilities, plasmas and fluid may move to turbulent state. Usually turbulence is abundant in plasmas in space and laboratory plasmas including magnetic confinement fusion (MCF), except some plasmas including short time or pulse plasmas. For example, in inertial confinement fusion (ICF) the time scale of pulse plasmas would be about several nanoseconds, and before going into turbulence or strong nonlinear state ICF fuel implosion, ignition and burning would end. Excluding these exceptions, we can find plasma turbulence abundantly.

Turbulence is basically nonlinear. It would be difficult to solve the nonlinear equations directly, for example, for wave-wave and wave-particle interactions. The standard way would be perturbation expansion in fluid models or in kinetic models to obtain solutions. In this book we do not go into details of the turbulence theoretical frameworks. However, one can find the excellent References [1-7, 43-51].

In our daily life, we can also find turbulence of cream in a coffee cup, river stream, ocean currents, air flow, storm, typhoon, etc. In fluid dynamics, study on turbulence has been explored intensively [3, 52-58]. In fluid dynamics frequently the equation of motion, that is, the Navier-Stokes equation is employed to express the fluid motion together with the continuity equation in Eq. (5.1) and the energy equation in Eq. (5.7)
or Eq. (5.8). Equation (5.4) was introduced for the inviscid (zero viscosity) fluid. The Navier-Stokes equation including the viscous term is shown below for the constant density $\\rho=$ constant:


\\begin{equation*}
\\frac{\\partial \\mathbf{v}}{\\partial t}+(\\mathbf{v} \\cdot \\nabla) \\mathbf{v}=\\frac{D \\mathbf{v}}{D t}=-\\nabla\\left(\\frac{P}{\\rho}\\right)+v \\Delta \\mathbf{v} \\tag{8.46}
\\end{equation*}


Here $v$ is called as the coefficient of kinetic viscosity, and $v=\\mu / \\rho$, where $\\mu$ is the coefficient of viscosity. For the constant density $\\rho=$ constant, the fluid is incompressible and $\\nabla \\cdot \\mathbf{v}=0$ from the equation of continuity Eq. (5.1).

When $\\mathbf{v}$ is normalized by $U$, the coordinate $\\mathbf{x}$ is normalized by $L, P$ is normalized by $\\rho U^{2}$, and the time $t$ is normalized by $L / U$, the Navier-Stokes equation has a similarity:


\\begin{equation*}
\\frac{\\partial \\mathbf{v}}{\\partial t}+(\\mathbf{v} \\cdot \\nabla) \\mathbf{v}=\\frac{D \\mathbf{v}}{D t}=-\\nabla P+\\left(\\frac{v}{U L}\\right) \\Delta \\mathbf{v}=-\\nabla P+\\frac{1}{R_{e}} \\Delta \\mathbf{v} \\tag{8.47}
\\end{equation*}


Here the physical quantities are now dimensionless, and $R_{e}=U L / v$, which is called the Reynolds number. When $R_{e}$ increases beyond the critical Reynolds number, the flow becomes turbulence. At the critical Reynolds number of around $R_{e} \\sim 2000 \\sim 10,000$, fluid turbulence would develop, depending on fluid status and boundary condition (see, for example, Chap. 1 in Ref. [3] and Ref. [59]).

In fluid dynamics, vorticity $\\omega$ is one of key quantities: $\\omega=\\nabla \\times \\mathbf{v}$. Now we take a rotation operation ( $\\nabla \\times$ ) to Eq. (8.46). Then we obtain the following equation for the vorticity:


\\begin{equation*}
\\frac{\\partial \\omega}{\\partial t}=\\nabla \\times(\\mathbf{v} \\times \\omega)+v \\Delta \\omega \\tag{8.48}
\\end{equation*}


Here we used the following vector formula: $\\nabla \\times \\nabla(P / \\rho)=0, \\nabla \\cdot \\nabla \\times \\mathbf{v}=0$, and $\\nabla \\times\\{(\\mathbf{v} \\cdot \\nabla) \\cdot \\mathbf{v}\\}=-\\nabla \\times(\\mathbf{v} \\times \\omega)$. Since $\\nabla \\times(\\mathbf{v} \\times \\omega)=(\\omega \\cdot \\nabla) \\mathbf{v}-(\\mathbf{v} \\cdot \\nabla) \\omega$, the following form is also obtained:


\\begin{equation*}
\\frac{\\partial \\omega}{\\partial t}+(\\mathbf{v} \\cdot \\nabla) \\omega=\\frac{D \\omega}{D t}=(\\omega \\cdot \\nabla) \\mathbf{v}+v \\Delta \\omega \\tag{8.49}
\\end{equation*}


The first term of the right side shows the vorticity intensity change by the vortex stretching or compression and the change in the vortex direction or tilting. Here $\\frac{\\omega^{2}}{2}$ can be defined and is called the enstrophy.

Now in order to obtain the fluid energy, we operate the velocity $\\mathbf{v}$ to Eq. (8.46). Then we obtain the following:


\\begin{equation*}
\\frac{\\partial}{\\partial t}\\left(\\frac{\\mathbf{v}^{2}}{2}\\right)=-\\nabla \\cdot\\left[\\mathbf{v}\\left(\\frac{\\mathbf{v}^{2}}{2}+\\frac{P}{\\rho}\\right)+v\\{(\\nabla \\times \\mathbf{v}) \\times \\mathbf{v}\\}\\right]-v(\\nabla \\times \\mathbf{v})^{2} \\tag{8.50}
\\end{equation*}


Here we define a spatial average integral over a volume of $V:<f>=\\frac{1}{V} \\int \\mathrm{d} V f$, where $f$ is a physical quantity. We assume that the volume $V$ is an arbitrary fixed volume, and at the boundary of $V$ the fluid field vanishes. Then from Eq. (8.50), the following is obtained:


\\begin{equation*}
\\frac{\\mathrm{d}}{\\mathrm{d} t} \\frac{1}{V} \\int \\mathrm{d} V\\left(\\frac{\\mathbf{v}^{2}}{2}\\right)=-v \\frac{1}{V} \\int \\mathrm{d} V(\\nabla \\times \\mathbf{v})^{2}=-v \\frac{1}{V} \\int \\mathrm{d} V \\omega^{2} \\tag{8.51}
\\end{equation*}


Now we define the averaged kinetic energy per unit mass $<E>=\\left\\langle\\frac{\\mathbf{v}^{2}}{2}\\right\\rangle$ and the averaged enstrophy $<Z>=\\left\\langle\\frac{\\omega^{2}}{2}\\right\\rangle$. We obtain the following:


\\begin{equation*}
\\frac{\\mathrm{d}}{\\mathrm{d} t}<E>=-2 v\\left\\langle\\frac{\\omega^{2}}{2}\\right\\rangle=-2 v<Z> \\tag{8.52}
\\end{equation*}


The quantity of $H=\\mathbf{v} \\cdot \\omega$ is the helicity, which describes a measure for helically twisting structure of streamlines. The averaged helicity $\\langle H\\rangle$ is obtained as follows:


\\begin{equation*}
\\frac{\\partial}{\\partial t}<H>=-2 v \\frac{1}{V} \\int \\mathrm{d} V \\omega \\cdot(\\nabla \\times \\omega)=-2 v<\\omega \\cdot(\\nabla \\times \\omega)> \\tag{8.53}
\\end{equation*}


Now we move to the Fourier space $\\mathbf{k}$ from the real space $\\mathbf{x}: \\mathbf{v}(\\mathbf{x}, t)=\\int \\mathrm{d} \\mathbf{k} \\mathbf{v}(\\mathbf{k}, t)$ $\\exp (i \\mathbf{k} \\cdot \\mathbf{x})$. Equation (8.46) is first transformed and is multiplied by $\\mathbf{v}(-\\mathbf{k}, t)$. After averaging statistically (ensemble averaging), the energy $\\operatorname{spectrum}(E(k, t))$ equation may be obtained as follows (see, for example, Chap. 9 in Ref. [3]):


\\begin{equation*}
\\frac{\\partial}{\\partial t} E(k, t)=T(k, t)-2 \\nu k^{2} E(k, t)+F(k, t) \\tag{8.54}
\\end{equation*}


Here $k=|\\mathbf{k}|, T(k, t)$ shows the energy transfer flux, and $F(k, t)$ is the energy injection rate. In Eq. (8.54), $2 v k^{2} E(k, t)$ is the energy dissipation spectrum function.

In Chap. 4 in Ref. [60], Richardson introduced the energy cascade for high Reynolds number in which energy is injected into "big whirls". The energy is transferred to "little whirls", whose energy is dissipated by viscosity. Kolmogorov proposed the energy spectrum for spatially homogeneous and isotropic turbulence by a dimension analysis [61-63] for the energy transport range, that is called the inertial range:


\\begin{equation*}
E(k)=C_{k} \\epsilon^{2 / 3} k^{-5 / 3} \\tag{8.55}
\\end{equation*}


Here $C_{k} \\sim 1.5$ is the Kolmogorov constant [64], and $\\epsilon$ is the energy flux. The energy may be injected at around $k_{0}$, and the energy is transferred to the large $k$ region. The energy transferred may be dissipated at around the Kolmogorov wavenumber $k_{K}$. Here the Kolmogorov scale of $\\eta=1 / k_{K}=\\left(v^{3} / \\epsilon\\right)^{1 / 4}$ may be assumed to be less than $1 / k_{0}: \\eta=1 / k_{K} \\ll 1 / k_{0}$. It was assumed that the energy spectrum does not depend on the viscosity $v$ in the inertial range and is expressed by the energy flux $\\epsilon$ and the wavenumber $k$.

The energy spectrum $E(k)$ was obtained from the kinetic energy per unit mass by the Fourier transform. The dimension of energy spectrum $E(k)$ is $[E(k)]=L^{3} / t^{2}$, and the dimension of $\\epsilon$ is $[\\epsilon]=L^{2} / t^{3}$ and $[k]=1 / L$. Then we obtain the Kolmogorov energy spectrum in Eq. (8.55). The Kolmogorov works have attracted scientists, and the relating works have been explored in Chaps. 6 and 7 in Ref. [53] and Refs. [65, 66].

In plasmas and fluids, 2D turbulences are found in various situations at, for example, Jupiter, tidal flow, typhoon, zonal flow in magnetized plasmas, soap film surface, etc. (see, for example, Ref. [2], Chap. 10 in Ref. [3] and Refs. [67-72] ). In 2D turbulence, it was suggested that the energy is transported to the inverse direction (inverse cascade) from small scales to large scales [73-75]. In 2D fluid model, in addition to the energy balance equation of Eq. (8.52), the enstrophy balance equation is obtained as follows:


\\begin{equation*}
\\frac{\\mathrm{d}}{\\mathrm{d} t}<Z>=\\frac{\\mathrm{d}}{\\mathrm{d} t}\\left\\langle\\frac{\\omega^{2}}{2}\\right\\rangle=-v<(\\nabla \\omega)^{2}> \\tag{8.56}
\\end{equation*}


Here we assume that the 2D fluid motion is restricted in $(x, y)$, and that the direction of the vorticity is in $z$. Then Eq. (8.49) in 3D becomes $D \\omega / D t=v \\Delta \\omega$ in $2 \\mathrm{D}$, and we obtain the above equation. The energy equation of Eq. (8.52) and the enstrophy equation of Eq. (8.56) suggested that in 2D the inverse energy cascade would appear with the enstrophy cascade ( see, for example, Ref. [2], Chap. 10 in Ref. [3] and Refs. [67-72] ).

In plasmas a model equation for the 2D turbulence was proposed in Refs. [77, 78]. The model equation is called the Hasegawa-Mima equation and was derived for the drift wave turbulence in magnetized nonuniform plasmas. The following continuity equation for the ion perturbation $n$ is used:


\\begin{equation*}
\\frac{\\partial n}{\\partial t}+\\nabla_{\\perp} \\cdot\\left\\{n_{0}\\left(\\mathbf{v}_{E \\times B}+\\mathbf{v}_{p}\\right)\\right\\}=0 \\tag{8.57}
\\end{equation*}


Here, the homogeneous magnetic field $\\mathbf{B}_{0}$ is parallel to $z$, the electrostatic perturbation is considered $\\left(\\mathbf{E}=-\\nabla_{\\perp} \\varphi\\right), \\nabla_{\\perp}=\\left(\\frac{\\partial}{\\partial x}, \\frac{\\partial}{\\partial y}, 0\\right), \\mathbf{v}_{E \\times B}=\\left(-\\nabla_{\\perp} \\varphi \\times \\mathbf{B}_{0}\\right) / B_{0}^{2}$ shows the $E \\times B$ drift, $\\mathbf{v}_{p}=\\frac{1}{\\omega_{c i} B_{0}}\\left\\{\\frac{\\partial}{\\partial t}+\\left(\\mathbf{v}_{E \\times B} \\cdot \\nabla_{\\perp}\\right)\\right\\}\\left(-\\nabla_{\\perp} \\varphi\\right)$ is the polarization drift, $\\omega_{c i}$ is the ion cyclotron frequency, and the quasi-neutrality condition of $n / n_{0}=e \\varphi / T_{e}$ is imposed. There is a density gradient of $\\nabla n_{0}$, and the ion temperature
is low. Equation (8.57) is deduced to one equation for $\\varphi$, that is, the Hasegawa-Mima Equation. The detailed derivation of the Hasegawa-Mima equation can be found in Appendix 1 in Ref. [2] and Ref. [76].

In Sect.8.4, some topics on turbulence are briefly introduced. As shown in Sect. 8.4, the turbulence has been studied intensively for stable plasma confinement in magnetic confinement fusion (MCF) and for structure formation and zonal flows in MCF plasmas and space (see, for example, Chap. 9 in Ref. [1] and Refs. [2, 4-7, $79,80]$ ).

\\section*{References}
\\begin{enumerate}
  \\item S. Ichimaru, in Statistical Plasma Physics, Vol. 1: Basic Principles (CRC Press, Boca Raton, 2004)

  \\item P.H. Diamond, S.-I. Itoh, K. Itoh, in Modern Plasma Physics, Vol. 1: Physical Kinetics of Turbulent Plasmas (Cambridge University Press, 2010)

  \\item P. Davidson, Turbulence: An Introduction for Scientists and Engineers, 2nd edn. (Oxford University Press, Oxford, 2015)

  \\item B.B. Kadomtsev, Plasma Turbulence (Academic Press, London and New York, 1965)

  \\item R.C. Davidson, Methods in Nonlinear Plasma Theory (Academic Press, New York and London, 1972)

  \\item R.Z. Sagdeev, A.A. Galeev, in Nonlinear Plasma Theory, Reviews of Plasma Physics, ed. by M.A. Leontovich, vol. VII (Consultants Bureau, New York, 1965), pp. 1-180; Lectures on the non-linear plasma theory of plasma, International center for theoretical physics, IAEA, IC/66/64 (1966)

  \\item J.A. Krommes, A tutorial introduction to the statistical theory of turbulent plasmas, a halfcentury after Kadomtsev's Plasma Turbulence and the resonance-broadening theory of Dupree and Weinstock. J. Plasma Phys. 81, 205810601 (2015)

  \\item J.S. Russell, Report on waves, in Report of the Fourteenth Meeting of the British Association for the Advancement of Science, York, vol. 1845 (1844), pp. 311-390

  \\item H. Ikezi, R.J. Taylor, D.R. Baker, Formation and interaction of ion-acoustic solutions. Phys. Rev. Lett. 25, 11-14 (1970)

  \\item H. Ikezi, Experiments on ion-acoustic solitary waves. Phys. Fluids 16, 1668-1765 (1973)

  \\item D.J. Korteweg, G. de Vries, On the change of form of long waves advancing in a rectangular canal, and on a new type of long stationary waves. Phil. Mag. 39, 422-443 (1895)

  \\item C.S. Gardner, G.K. Morikawa, Similarity in the asymptotic behavior of collision-free hydrodynamic waves and water waves, in Report NYU-9082. Courant Instruction of Mathematical Science (New York University, New York, 1960)

  \\item H. Washimi, T. Taniuchi, Propagation of ion-acoustic solitary waves of small amplitude. Phys. Rev. Lett. 17, 996-998 (1966)

  \\item N.J. Zabusky, M.D. Kruskal, Interaction of "solitonsâtž in a collisionless plasma and the recurrence of initial states. Phys. Rev. Lett. 15, 240-243 (1965)

  \\item M. Shahrill, M.S.F. Chong, H.N.H.M. Nor, Applying explicit schemes to the Korteweg-de vries equation. Mode. Appl. Sci. 9, 200-224 (2015)

  \\item P.J. Roache, Fundamentals of Computational Fluid Dynamics (Hermosa Pub, New Mexico, 2003)

  \\item D.R. Nicholson, Introduction to Plasma Theory (John Wiley \\& Sons, New York, 1983)

  \\item C.S. Gardner, J.M. Greene, M.D. Kruskal, R.M. Miura, Method for solving the Korteweg-de Vries equation. Phys. Rev. Lett. 19, 1095-1097 (1967)

  \\item R.M. Miura, Korteweg-de Vries equation and generalizations. I. A remarkable explicit nonlinear transformation. J. Math. Phys. 9 1202-1204 (1968)

  \\item R.M. Miura, The Korteweg-de Vries equation: a survey of results. SIAM Rev. 18, 412-459 (1976)

  \\item R.M. Miura, C.S. Gardner, M.D. Kruskal, Korteweg-devries equation and generalizations. II. Existence of conservation laws and constants of motion. J. Math. Phys. 9 1204-1209 (1968)

  \\item I.M. Gel'fand, B.M. Levitan, On the determination of a differential equation from its spectral function., Izv. Akad. Nauk SSSR Ser. Mat. 15 309-360 (1951). (Translation: American mathematical society. Trans. Ser. 2, 1 (1955) p. 253)

  \\item V.A. Marchenko, On reconstruction of the potential energy from phases of the scattered waves. Dokl. Akad. Nauk SSSR 104, 695-698 (1955)

  \\item I. Kay, H.E. Moses, The determination of the scattering potential from the spectral measure function. Nuovo Cimento 3, 276-304 (1956)

  \\item Z.S. Agranovich, V.A. Marchenko, The Inverse Problem of Scattering Theory (Gordon \\& Breach Science Publishers Ltd., New York, 1964)

  \\item J.M. Burgers, Application of a model system to illustrate some points of the statistical theory of free turbulence. Proc. Roy. Neth. Acad. Sci. Amsterdam 43 2-12 (1940)

  \\item E. Hopf, The partial differential equation ut $+\\mathrm{uux}=\\mu \\mathrm{uxx}$. Commun. Pure Appl. Math. 3, 201-230 (1950)

  \\item J.D. Cole, On a quasi-linear parabolic equation occurring in aerodynamics. Quart. Appl. Math. 9, 225-236 (1951)

  \\item R.D. Richtmeyer, K.W. Morton, Difference Methods for Initial-Value Problems (Interscience Pub, New York, 1967)

  \\item L.D. Landau, E.M. Lifshitz, Fluid Dynamics (Pergamon Press Ltd., Oxford, 1987)

  \\item R.W. Gould, T.M. O'Neil, J.H. Malmberg, Plasma wave echo. Phys. Rev. Lett. 19, 219-222 (1967)

  \\item T.M. O'Neil, R.W. Gould, Temporal and spatial plasma wave echoes. Phys. Fluids 11, 134-142 (1968)

  \\item R.M. Hill, D.E. Kaplan, Cyclotron resonance echo. Phys. Rev. Lett. 14, 1062-1063 (1965)

  \\item I.D. Abella, N.A. Kurnit, S.R. Hartmann, Photon echoes. Phys. Rev. 141, 391-406 (1966)

  \\item T. Kamimura, A. Hasegawa, Plasma wave echo excited by two cyclotron waves. Phys. Fluids 12, 1480-1482 (1969)

  \\item E.L. Hahn, Spin echoes. Phys. Rev. 80, 580-594 (1950)

  \\item W.H. Kegel, R.W. Gould, On the theory of pulse stimulated radiation from plasma. Phys. Lett. 19, 531-532 (1965)

  \\item H. Ikezi, N. Takahashi, Observation of spatial ion-wave echoes. Phys. Rev. Lett. 20, 140-142 (1968)

  \\item D.R. Baker, N.R. Ahern, A.Y. Wong, Phys. Rev. Lett. 20, 318-321 (1968)

  \\item H. Ikezi, N. Takahashi, K. Nishikawa, Spatial ion-wave echoes. Phys. Fluids 12, 853-865 (1969)

  \\item L.D. Landau, E.M. Lifshitz, Physical Kinetics (Pergamon Press Ltd., Oxford, 1981)

  \\item F.F. Chen, Introduction to Plasma Physics and Controlled Fusion, 3rd edn. (Springer, 2015)

  \\item N. Rostoker, M.N. Rosenbluth, Test particles in a completely ionized plasma. Phys. Fluids 3, $1-14(1960)$

  \\item M. Lax, Fluctuations from the nonequilibrium steady state. Rev. Modern Phys. 32, 25-64 (1960)

  \\item N. Rostoker, Fluctuations of a plasma (I). Nuclear Fusion 1, 101-120 (1961)

  \\item N. Rostoker, Superposition of dressed test particles. Phys. Fluids 7, 479-490 (1964)

  \\item T.H. Dupree, A perturbation theory for strong plasma turbulence. Phys. Fluids 9, 1773-1782 (1966)

  \\item R.C. Davidson, Statistical frameworks for weak turbulence. Phys. Fluids 10, 1707-1713 (1967)

  \\item R.C. Davidson, Weak turbulence in a homogeneous plasma, Ph. D. Thesis, Princeton University, AEC (Atomic Energy Commission) Research and Development Report, MAT-496 (1966)

  \\item D. Montgomery, A BBGKY framework for turbulence. Phys. Fluids 19, 802-810 (1976)

  \\item A.A. Galeev, R.Z. Sagdeev, Nonlinear plasma theory. Rev. Plasma Phys. ed. by M.A. Leontovich, 7 1-180 (1979)

  \\item J.R. Holton, An Introduction to Dynamic Meteorology, 4th edn. (Elsevier Academic Press, Amsterdam, 2004)

  \\item U. Frisch, Turbulence-The legacy of A.N. Kolmogorov, Cambridge University Press, New York (1995)

  \\item A.S. Monin, A.M. Yaglom, Statistical Fluid Mechanics: Mechanics of Turbulence, vol. 1 (MIT Press, Cambridge, Massachusetts, and London, England, 1973)

  \\item A.S. Monin, A.M. Yaglom, Statistical Fluid Mechanics: Mechanics of Turbulence, vol. 2 (MIT Press, Cambridge, Massachusetts, and London, England, 1975)

  \\item M. Lesieur, Turbulence in Fluids, Fourth Revised and Enlarged Edition (Springer, Dordrecht, The Netherlands, 2008)

  \\item S.B. Pope, Turbulent Flows (Cambridge University Press, 2000)

  \\item Y. Zhou, T.T. Clark, D.S. Clark, S.G. Glendinning, M.A. Skinner, C.M. Huntington, O.A. Hurricane, A.M. Dimits, B.A. Remington, Turbulent mixing and transition criteria of flows induced by hydrodynamic instabilities. Phys. Plasmas 26, 080901 (2019)

  \\item L. Schiller, Über den Strömungswiderstand von Rohren verschiedenen Querschnitts und Rauhigkeitsgrades. Zeitschrift Angew. Math. Mech. 3, 2-13 (1923)

  \\item L.F. Richardson, Weather Prediction by Numerical Process (Cambridge at the University Press, 1922), p. 66

  \\item A. Kolmogorov, The local structure of turbulence in incompressible viscous fluid for very large reynolds' numbers. Dokl. Akad. Nauk SSSR 30, 301-305 (1941)

  \\item A.N. Kolmogorov, On degeneration (decay) of isotropic turbulence in an incompressible viscous fluid. Dokl. Akad. Nauk SSSR 31, 538-541 (1941)

  \\item A.M. Obukhov, On energy distribution in the spectrum of turbulent flow. Izv. RAN (Akad. Nauk., SSSR). Ser. Geogr. Geofiz., 5 453-466 (1941)

  \\item K.R. Sreenivasan, On the universality of the Kolmogorov constant. Phys. Fluids 7, 2778-2784 (1995)

  \\item Z.-S. She, E. Leveque, Universal scaling laws in fully developed turbulence. Phys. Rev. Lett. 72, 336-339 (1994)

  \\item Z.-S. She, Hierarchical structures and scalings in turbulence, in Lecture Notes in Physics, Turbulence Modeling and Vortex Dynamics, ed by O. Boratav, A. Eden, A. Erzan, vol. 491 (Springer, 1997), pp. 28-52

  \\item J. Paret, P. Tabeling, Experimental observation of the two-dimensional inverse energy cascade. Phys. Rev. Lett. 79, 4162-4165 (1997)

  \\item R.T. Cerbus, W.I. Goldburg, Intermittency in 2D soap film turbulence. Phys. Fluids 25, 105111 (2013)

  \\item J.G. Charney, The dynamics of long waves in a baroclinic westerly current. J. Meteor. 4, 135-163 (1947)

  \\item J.G. Charney, On the scale of atmospheric motions. Geofys. Publ. 17, 1-17 (1948)

  \\item J.G. Charney, A. Eliassen, A numerical method for predicting the perturbations of the middle latitude westerlies. Tellus 1, 38-54 (1949)

  \\item J.G. Charney, J.G. DeVore, Multiple flow equilibria in the atmosphere and blocking. J. Atmos. Sci. 36, 1205-1216 (1979)

  \\item R.H. Kraichnan, Inertial ranges in two-dimensional turbulence. Phys. Fluids 10, 1417-1423 (1967)

  \\item G.K. Batchelor, Computation of the energy spectrum in homogeneous two-dimensional turbulence. Phys. Fluids 12 (1969) pp. II-233-II-239

  \\item C.E. Leith, Diffusion approximation for two-dimensional turbulence. Phys. Fluids 11, 671-673 (1968)

  \\item F. Jenko, Nonlinear dynamics, Tutorial lecture at Les Houches plasmas school 2017 of from laboratories to astrophysics: The expanding universe of plasma physics (2017), \\href{https://ipag}{https://ipag}. \\href{http://osug.fr/}{osug.fr/} lesurg/plasmas2017/talks/Jenko-Houches-2017.pdf. Cited 28 March 2022

  \\item A. Hasegawa, K. Mima, Stationary spectrum of strong turbulence in magnetized nonuniform plasma. Phys. Rev. Lett. 39, 205-208 (1977)

  \\item A. Hasegawa, K. Mima, Pseudo-three-dimensional turbulence in magnetized nonuniform plasma. Phys. Fluids 21, 87-92 (1978)

  \\item A. Hasegawa, M. Wakatani, Self-organization of electrostatic turbulence in a cylindrical plasma. Phys. Rev. Lett. 59, 1581-1584 (1987)

  \\item K. Itoh, S.-I. Itoh, P.H. Diamond, T.S. Hahm, A.G.R. Fujisawa, Tynan, M. Yagi, Y. Nagashima, Physics of zonal flows. Phys. Plasmas 13, 055502 (2006)

\\end{enumerate}

\\section*{Chapter 9 Applications of Plasmas }
\\begin{abstract}
So far the properties of plasmas have been focused, as well as the theoretical and numerical methods for plasmas. In Chap. 9, applications of plasmas are introduced, including plasma process, single-probe method to measure the plasma temperature, plasma jet, nuclear fusion, laser particle acceleration, cluster ion interaction with plasma and control of plasma instabilities and non-uniformity.
\\end{abstract}

\\subsection*{Plasma Process}
Since microfabrication of electronic components is required, plasma process has been used for thin-film formation and etching [1]. The plasma etching enables the finestructure fabrication in the scale of $1 \\mu \\mathrm{m}$ or less. By using the dry plasma process, drawbacks of wet processes are improved. In the fabrication process, a mask is needed to protect parts that are not etched. The solution swells the mask in the wet process, and the pattern accuracy would not be improved. In the dry plasma process, the mask swelling is avoided and the fine pattern can be accurately produced.

In plasmas chemically activated particles are created and react with materials. The activated particles interact with the unmasked parts of the material and produce fine structures (Fig. 9.1).

In the plasma etching, the activated particles move in all the directions. Therefore, the interacting surface may not be perfectly flattened.

\\subsection*{Electron Temperature Measurement by Single-Probe Method}
In Sect. 9.2, a measurement method, that is, a single-probe method for electron temperature is introduced (see, for example, Chap. 8 in Ref. [2]). In this method a thin probe is inserted into a plasma, and it is assumed that plasmas are in equilibrium and described by the Maxwell distributions (see Eq. (2.1) and Fig. 2.1). In addition to the probe method, there are various measurement methods to diagnose plasmas.

Fig. 9.1 Plasma etching

\\begin{center}
\\includegraphics[max width=\\textwidth]{2024_02_26_83e36543483eb7d284c1g-208}
\\end{center}

For example, when electromagnetic wave passes through a plasma, its wavelength would change depending on the electron density. The plasma electron density would be also measured. Plasmas also absorb some wavelength ranges of light. It would be possible to examine components in plasmas from the spectrum of the transmitted light.

Now we consider the single-probe method. A single probe, which has a surface area of $S$, is inserted into a plasma. The probe voltage is set to a negative voltage $V_{p}(<0)$, and ions in the plasma come to the probe. Slow electrons are reflected from the probe surface, though the faster electrons would come and reach the probe surface. So all the ions contribute to the current measured by the probe:


\\begin{equation*}
I_{i}=S \\int_{0}^{\\infty} q_{i} f_{M} v_{x} \\mathrm{~d} v_{x}=\\frac{S}{4} n_{0} q_{i}\\left\\langle v_{i}\\right\\rangle \\tag{9.1}
\\end{equation*}


Here $f_{M}$ shows the Maxwell distribution, $\\left\\langle v_{i}\\right\\rangle$ the ion averaged velocity and $q_{i}$ the ion charge. The fast electrons also contribute to the probe current:


\\begin{equation*}
I_{e}=S \\int_{v_{e}}^{\\infty}(-e) f_{M} v_{x} \\mathrm{~d} v_{x}=-\\frac{S}{4} n_{0} e\\left\\langle v_{e}\\right\\rangle \\exp \\left(\\frac{e V_{p}}{T_{e}}\\right) \\tag{9.2}
\\end{equation*}


Here $v_{e}$ should be as follows:


\\begin{equation*}
\\frac{1}{2} m_{e} v_{e}^{2}=e\\left|V_{p}\\right| \\tag{9.3}
\\end{equation*}


Here $\\left\\langle v_{e}\\right\\rangle$ shows the electron averaged velocity, with which electrons overcome the potential $V_{p}$. When $V_{p}<0$, the total current measured by the probe is as follows:


\\begin{equation*}
I=I_{i}+I_{e}=\\frac{S}{4} n_{0}\\left[q_{i}\\left\\langle v_{i}\\right\\rangle-e\\left\\langle v_{e}\\right\\rangle \\exp \\left(\\frac{e V_{p}}{T_{e}}\\right)\\right] \\text { for } V_{p}<0 \\tag{9.4}
\\end{equation*}


On the other hand, when $V_{p}>0$, all the electrons moving toward the probe are collected by the probe, and the slow ions are reflected:


\\begin{gather*}
I_{i}=\\frac{S}{4} n_{0} q_{i}\\left\\langle v_{i}\\right\\rangle \\exp \\left(\\frac{-q_{i} V_{p}}{T_{i}}\\right)  \\tag{9.5}\\\\
I_{e}=-\\frac{S}{4} n_{0} e\\left\\langle v_{e}\\right\\rangle \\tag{9.6}
\\end{gather*}


The total current measured by the probe is as follows:


\\begin{equation*}
I=I_{i}+I_{e}=\\frac{S}{4} n_{0} q_{i}\\left[\\left\\langle v_{i}\\right\\rangle \\exp \\left(\\frac{-q_{i} V_{p}}{T_{i}}\\right)-e\\left\\langle v_{e}\\right\\rangle\\right] \\text { for } V_{p}>0 \\tag{9.7}
\\end{equation*}


If we can ignore the ion current, the total current can be estimated as follows:

\\[
I= \\begin{cases}\\frac{-S}{4} n_{0} e\\left\\langle v_{e}\\right\\rangle \\exp \\left(\\frac{e V_{p}}{T_{e}}\\right) & \\left(V_{p}<0\\right)  \\tag{9.8}\\\\ \\frac{-S}{4} n_{0} e\\left\\langle v_{e}\\right\\rangle & \\left(V_{p}>0\\right)\\end{cases}
\\]

Here we apply the logarithmic operation of $\\ln$ to the first equation of Eq. (9.8):


\\begin{equation*}
\\ln (-I)=\\ln \\left(\\frac{S}{4} n_{0} e\\left\\langle v_{e}\\right\\rangle\\right)+\\frac{e}{T_{e}} V_{p} \\tag{9.9}
\\end{equation*}


From the gradient of the graph of $\\ln (-I)$ versus $V_{p}$, we can obtain the electron temperature $T_{e}$.

In the probe measurement method, the plasmas are assumed to be in equilibrium, and it is also assumed that the probe does not disturb the plasmas.

\\subsection*{Plasma Jet}
Another interesting application of plasmas is a plasma jet [3]. As shown in Fig. 9.2, a plasma, generated by a discharge, can be accelerated, and the reactive force would propel rockets or objects. Through the plasma generated by a discharge, an electric current flows as shown in Fig. 9.2 and generates the magnetic field B. The Lorentz force of $\\mathbf{J} \\times \\mathbf{B}$ pushes the plasma out of the device. Then rockets or objects are propelled. In addition, plasma jets would be used to material surface treatment or detection of small amounts of chemicals.

Fig. 9.2 Plasma jet. In the device shown, a current flows through the plasma generated by a discharge, and the Lorentz force of $\\mathbf{J} \\times \\mathbf{B}$ accelerates the plasma

\\begin{center}
\\includegraphics[max width=\\textwidth]{2024_02_26_83e36543483eb7d284c1g-210}
\\end{center}

\\subsection*{Nuclear Fusion}
In Sect. 9.4, nuclear fusion is introduced. Fusion fuel reaches a high temperature of several hundred million Kelvin or higher. At the high temperature the fusion fuel is in a plasma state.

Unlike nuclear fusion, nuclear fission uses Uranium $235\\left({ }^{235} \\mathrm{U}\\right)$, which splits to extract nuclear fission energy. The following is an example fission reaction:


\\begin{equation*}
{ }_{92}^{235} \\mathrm{U}+{ }_{0}^{1} n \\rightarrow A_{1}+A_{2}+2.43{ }_{0}^{1} n \\tag{9.10}
\\end{equation*}


Here $A_{1}$ and $A_{2}$ are the fission products, which may induce a problem of the nuclear waste. The nuclear binding energy becomes large around $\\mathrm{Fe}$. The total binding energy for a nucleus is the energy required to separate all the nucleons (protons and neutrons) with each other. Therefore, $\\mathrm{Fe}$ is the rather stable nucleus. For heavy nuclei like ${ }^{235} \\mathrm{U}$ the binding energy becomes small, and for light nuclei like $H$ it is also small.

In nuclear fusion, light nuclei fuse into one heavier nucleon, and the fusion reaction releases energy. The following DT reaction is a typical example of the fusion reaction:


\\begin{equation*}
{ }^{2} D+{ }^{3} T \\rightarrow{ }^{4} \\mathrm{He}+{ }^{1} n+17.6 \\mathrm{MeV} \\tag{9.11}
\\end{equation*}


The deuterium $D$ and the tritium $T$ are isotopes of hydrogen. The DT reactions produce $\\alpha$ particles $\\left({ }^{4} \\mathrm{He}\\right)$ and neutrons $n$. One DT fusion reaction produces the energy of $17.6 \\mathrm{MeV}=2.82 \\times 10^{-12} \\mathrm{~J}$. The following reaction is the DD fusion reaction:

\\[
{ }^{2} D+{ }^{2} D \\rightarrow\\left\\{\\begin{array}{l}
{ }^{3} \\mathrm{He}+{ }^{1} n+3.27 \\mathrm{MeV}  \\tag{9.12}\\\\
{ }^{3} \\mathrm{He}+{ }^{1} \\mathrm{H}+4.03 \\mathrm{MeV}
\\end{array}\\right.
\\]

The binding energy comes from the mass defect. By the special relativity, mass is equivalent to energy. For example, in the DT reaction in Eq. (9.11) the total mass of $D$ and $T$ is larger than the total mass of $\\mathrm{He}$ and $n$ :

$$
D(2.01471)+T(3.01700)-\\operatorname{He}(4.00390)-n(1.00893)=+0.01888(9.13)
$$

Here we employ the atomic mass unit, in which ${ }^{16} \\mathrm{O}$ is 16 .

The nuclear fusion reactor has not been commercialized yet. The nuclear fusion energy has been studied intensively for our society future energy source. In nature, we find the nuclear fusion reactions in stars like the sun. The solar energy is the nuclear fusion energy. In one day the sun releases the energy of about $3.3 \\times 10^{31}$ $\\mathrm{J} /$ day, and about $1.5 \\times 10^{22} \\mathrm{~J} /$ day comes to our earth in one day, which is much larger than the energy of $\\sim 6 \\times 10^{20} \\mathrm{~J} / \\mathrm{y}$ we consumed in 2018 for one year. The enormous amount of energy comes from the nuclear fusion energy.

In the reaction of Eq. (9.11), the deuterium of $D$ exists in the sea, and the tritium of $T$ is produced by $\\mathrm{Li}$, which absorbs a neutron in a fusion reactor. Therefore, the fusion energy may free us from the energy resource problem.

Here we compare the energy of mass with the kinetic energy. Assume a car running with $40 \\mathrm{~km} / \\mathrm{h}$, and its mass is $M$. The kinetic energy of the car is $M \\times(11 \\mathrm{~m} / \\mathrm{s})^{2} / 2$. On the other hand, the energy of mass is $M c^{2}$, where $c$ is the speed of light.

$($ Energy of mass $) /($ Kinetic energy $)=M c^{2} /\\left\\{\\frac{M}{2}(11 \\mathrm{~m} / \\mathrm{s})^{2}\\right\\} \\simeq 1.5 \\times 10^{15}$

The energy of mass is enormous.

\\subsubsection*{Fusion Reaction}
In nuclear fusion researches, the DT reaction has been focused, because the cross section of the DT reaction is larger than those in other fusion reactions. The DT reaction itself can be realized by several methods: for example, a $D$ beam is produced by a particle accelerator and irradiates the stationary $T$ particles. In this case the $D$ particle energy of $\\sim 66 \\mathrm{keV}$ gives the maximum cross section. One DT reaction produces the energy of $17.6 \\mathrm{MeV}=2.82 \\times 10^{-12} \\mathrm{~J}$. In order to construct a commercial power reactor with one million $\\mathrm{kW}(=1 \\mathrm{GJ} / \\mathrm{s})$ of electricity, the fusion output energy of $\\sim 3 \\mathrm{GJ} / \\mathrm{s}$ would be required. The amount of the mass energy is enormous. But one DT reaction is not sufficient. In a second $\\sim 10^{21}$ DT reactions are needed for a power reactor. One of the difficulties in nuclear fusion is the huge number of the reactions per second. The huge number of reactions should be sustained stably for a long period.

The nuclei of $D$ and $T$ have the plus charge and repel each other by the Coulomb force. It is necessary to overcome the Coulomb force and fuse the nuclei together. As described above, particle accelerators can be used to fuse the nuclei. This method is mostly used to create a neutron source. For nuclear fusion power reactors, so far thermonuclear fusion has been focused. The fusion fuel of $D$ and $T$ is heated up to a high temperature of $\\sim 10 \\mathrm{keV}$ or so. The DT reaction cross section peaks at about $66 \\mathrm{keV}$. Even at the lower temperature of $\\sim 10 \\mathrm{keV}$, the fuel nuclei contain fast nuclei at the tail of the Maxwell distrobution. We also need to reduce the input energy to heat the fuel plasma. The controlled thermonuclear fusion has been studied intensively.

The fusion reaction number is considered here. The cross section of the DT fusion reaction is denoted by $\\sigma$, and the relative speed between $D$ and $T$ is $v$. The number density of $D$ is $n_{D}$, and the number density of $T$ is shown by $n_{T}$. The fusion reaction number per second per unit volume is as follows:


\\begin{equation*}
n_{D} n_{T} \\sigma v \\tag{9.15}
\\end{equation*}


The nuclei DT are in the Maxwell distribution. The cross section $\\sigma$ depends also on the relative speed of $v$ and so on the temperature. Here $\\sigma v$ should be averaged over the Maxwell distribution as follows:


\\begin{equation*}
n_{D} n_{T}\\langle\\sigma v\\rangle \\tag{9.16}
\\end{equation*}


$\\langle\\sigma v\\rangle$ is called the reaction rate.

\\subsubsection*{The Lawson Criterion-To Sustain Fusion Reaction}
On the sun the high-temperature plasma is confined by the gravity and the fusion reactions continue. On the earth, the fusion fuel plasma should be also confined to sustain the fusion reactions.

Here we consider the condition to sustain the fusion reaction, that is called the Lawson criterion. Here the DT plasma is considered with the temperature of $\\sim 10$ $\\mathrm{keV}$. From the high-temperature plasma, electrons emit the radiation, which can escape from the fusion fuel plasma. It is the bremsstrahlung loss: $E_{\\mathrm{BL}}$ (see, for example, Chap. 15 in Ref. [4], Chap. 5 in Ref. [5] and Chap. 1 in Ref. [6]). If the plasma expands, the temperature drops. Therefore, the plasma should be confined. Alternatively, the fusion reactions should end, before the fuel plasma expands and disassembles. On the other hand, the DT reaction product of the $\\alpha$ particle $\\left({ }^{4} \\mathrm{He}\\right)$ in Eq. (9.11) heats the fuel plasma. The other fusion product of neutron $n$ in Eq. (9.11) escapes almost freely from the plasma. The $\\alpha$ particles are the charged particles. They interact with other charged particles by the Coulomb collisions and deposit their energy into the fusion fuel plasma. Assume the plasma density is $n$, and $n_{D}=$ $n_{T}=n / 2$. The plasma heating energy is $\\left(3 n_{T} / 2\\right) \\times 2$, where the factor 2 comes from electrons and ions of the fusion fuel. The temperature equation would be as follows:

\\begin{center}
\\includegraphics[max width=\\textwidth]{2024_02_26_83e36543483eb7d284c1g-213}
\\end{center}

Fig. 9.3 Sustaining DT fusion reactions


\\begin{equation*}
\\frac{\\mathrm{d}\\left(3 n_{T}\\right)}{\\mathrm{d} t} \\sim-E_{\\mathrm{BL}}-\\frac{3 n_{T}}{\\tau}+\\frac{n^{2}\\langle\\sigma v\\rangle}{4} E_{\\alpha} \\tag{9.17}
\\end{equation*}


The second term at the right hand side comes from the plasma expansion, and $\\tau$ shows the plasma expansion time. The third term shows the $\\alpha$ heating, and $E_{\\alpha}=3.5$ $\\mathrm{MeV}$ (see Eq. (9.11)). The energy of the $\\alpha$ particle in Eq. (9.11) is obtained by 3.5$17.6 \\mathrm{MeV} \\times 1 /(4+1)$. The rest of the fusion reaction energy $(14.1-17.6 \\mathrm{MeV} \\times 4 / 5)$ goes to neutron by the momentum conservation (Fig. 9.3).

In order to sustain the DT fusion reaction, the left hand side of Eq. (9.17) should be larger than 0. From Eq. (9.17), the Lawson criterion to sustain the DT fusion reaction is obtained, though we do not show the explicit form here for $E_{\\mathrm{BL}}$ (see, for example, Chap. 15 in Ref. [4], Chap. 5 in Ref. [5] and Chap. 1 in Ref. [6]):


\\begin{equation*}
n \\tau \\geq 10^{20} \\mathrm{~s} / \\mathrm{m}^{3} \\tag{9.18}
\\end{equation*}


In addition to Eq. (9.18), the fusion fuel temperature should be larger than $\\sim 10$ $\\mathrm{keV}$. This is the Lawson criterion, which is the condition to sustain the DT fusion reactions (see, for example, Chap. 1 in Ref. [6]).

For the DD reaction in Eq. (9.12), the condition becomes difficult seriously: $n \\tau>10^{22} \\mathrm{~s} / \\mathrm{m}^{3}$. In addition, the temperature required becomes also higher.

\\subsubsection*{MCF: Magnetic Confinement Fusion}
Now the plasma confinement is considered. In Chap. 3, we found that charged particles rotate around the magnetic field lines. The magnetic field can confine charged particles and the fusion fuel plasmas. The magnetic confinement fusion (MCF) has been studied intensively for many years [7-16]. In MCF the fusion fuel gas plasma is confined magnetically in a large volume.

Fig. 9.4 Simple magnetic field configuration

\\begin{center}
\\includegraphics[max width=\\textwidth]{2024_02_26_83e36543483eb7d284c1g-214}
\\end{center}

Figure 9.4 shows the magnetic field $B_{z}$. If the strength of $B_{z}$ is sufficiently strong, plasma particles are hard to escape in the lateral direction. However, from the column ends charged particles can escape freely. Both the ends of the column are connected to form a doughnut, that is called a torus as shown in Fig.9.5. Then the plasma particles do not escape from the column ends.

If plasma charged particles are confined only by the magnetic field of $B_{z}$ in the torus magnetic field configuration in Fig. 9.5, the plasma particles perform the $\\nabla B$ drift, as shown in Sect. 3.3. The magnetic field $B_{z}$ is strong at the inside of the torus compared with that at the torus outside. The $\\nabla B$ drift induces the charge separation and the electric field shown in Fig. 9.5 (see also Eq. (3.12)). In addition to the $\\nabla B$ drift, another drift by the centrifugal force appears in the bending magnetic field $B_{z}$ along the torus axis direction, that is called the toroidal direction. The drift induced by the centrifugal force is called as the curvature drift, which also induces the charge separation. By the resulting electric and magnetic fields, the $E \\times B$ drift appears and pushes the plasma outward of the torus.

As shown in Fig. 9.5, when $B_{\\theta}$ is applied, the magnetic field configuration becomes a spiral one. The plasma particles, moving along with the magnetic field line, move their positions from the outside to the inside of the torus. The spiral magnetic field would prevent the plasma outward motion.

Tokamak is a typical magnetic confinement device of plasma, as shown in Fig. 9.5. A large tokamak of ITER ("The Way" in Latin) is currently under construction with international cooperations $[12,13]$. One of the goals of ITER is to produce $500 \\mathrm{MW}$ of the nuclear fusion power with the $50 \\mathrm{MW}$ input power for a long pulse of 400

\\begin{center}
\\includegraphics[max width=\\textwidth]{2024_02_26_83e36543483eb7d284c1g-215}
\\end{center}

Fig. 9.5 Torus magnetic field configuration. The $\\nabla B$ and curvature drifts induce the charge separation. The resulting $E \\times B$ drift pushes the torus plasma outward. The magnetic field $B_{\\theta}$ modifies the magnetic configuration to a spiral one, along which plasma particles move and would change their positions from the outside to the inside of the torus

$600 \\mathrm{~s}$. The plasma volume is $840 \\mathrm{~m}^{3}$ and the plasma major radius, that is, the radius of the torus is $6.2 \\mathrm{~m}$. The fusion fuel plasma is heated up to $150 \\times 10^{6} \\mathrm{~K}$. The first plasma in ITER is scheduled for December 2025.

In tokamak, the current of $I_{z}$ produces the magnetic field $B_{\\theta}$. Plasma is heated by the Joule heating by the plasma current, neutral beam injections and electromagnetic wave heatings.

In the magnetic confinement fusion (MCF), including ITER, scientific issues may include a plasma disruption and material developments. Through the torus plasma a strong current flows. When the plasma becomes unstable, the plasma disruption may occur. In the DT nuclear fusion reactions high-energy neutrons may activate structural materials surrounding the plasma and the fusion devices. The materials should strong against the neutron irradiation. Intensive studies have been performed to solve these issues in MCF for our future energy source, including mitigation methods of the disruption and successful new material developments [12-16].

\\subsubsection*{ICF: Inertial Confinement Fusion}
In the last Sect. 9.4.3, MCF was introduced. In MCF plasma is confined by the magnetic field. On the sun, plasma is confined by gravity. In order to sustain the nuclear fusion reactions and to release the sufficient fusion energy, the fusion fuel temperature should reach $\\sim 10 \\mathrm{keV}$, and the Lawson criterion in Eq. (9.18) should be satisfied for the DT fusion: $n \\tau>10^{20} \\mathrm{~s} / \\mathrm{m}^{3}$. In MCF the plasma confinement time $\\tau$ is the order of $1 \\mathrm{~s}$ or longer. When the fuel density $n$ is, for example, $n \\sim 10^{29} / \\mathrm{m}^{3}$, $\\tau \\sim 10^{-9}$ s satisfies the Lawson criterion. This method in controlled fusion is called

Fig. 9.6 Plasma expansion

\\begin{center}
\\includegraphics[max width=\\textwidth]{2024_02_26_83e36543483eb7d284c1g-216}
\\end{center}

inertial confinement fusion (ICF) [6, 17]. In ICF, fusion fuel is compressed to a high density of $\\sim 1000$ times the solid density or more. In Sect. 9.4.4, ICF is focused. In ICF, the fusion fuel is not confined, and before the fuel expansion or disassembling, the fusion reactions terminate.

Here let us consider the time of $10^{-9} \\mathrm{~s}$. A high-temperature plasma is in a box. The box outer wall is taken out at $t=0$. The plasma expands, and the information of the plasma expansion propagates into the plasma with the sound speed of $C_{s}$ as shown in Fig.9.6. When the expansion wave reaches the plasma center, the plasma is disassembled. Before the disassemble time, it is necessary to release the fusion energy. The sound speed $C_{s} \\sim \\sqrt{T / m}$ is estimated to $C_{s} \\sim 10^{6} \\mathrm{~m} / \\mathrm{s}$ for the DT fusion plasma at the temperature of $\\sim 10 \\mathrm{keV}$.

For example, the expansion time is estimated by $\\sim$ (Radius) $/ C_{s}$ for a DT sphere. When the sphere radius is $\\sim 1 \\mathrm{~mm}$, the expansion disassembling time is about $0.001 \\mathrm{~m} /\\left(10^{6} \\mathrm{~m} / \\mathrm{s}\\right)=10^{-9} \\mathrm{~s}=1 \\mathrm{~ns}$. In ICF the inertia of the compressed DT fuel is used to release the fusion energy.

In ICF the confinement time $\\tau$ is so short that the DT fuel density should become high in order to satisfy the Lawson criterion in Eq. (9.18): $n>\\sim 10^{29} / \\mathrm{m}^{3}$.

Here we estimate the input energy to heat the DT spherical fuel with its radius of $r$ up to $T$ :


\\begin{equation*}
E_{\\text {in }}=2 \\times\\left(\\frac{3 n_{T}}{2}\\right) \\times\\left(\\frac{4 \\pi r^{3}}{3}\\right)=4 \\pi n_{T} r^{3} \\tag{9.19}
\\end{equation*}


The factor 2 in Eq. (9.19) comes from ions and electrons of the DT fuel plasma. In the Lawson criterion Eq. (9.18), the confinement time $\\tau$ is written by $\\tau \\sim r / C_{s}$ :


\\begin{equation*}
n r>10^{20} C_{s}\\left(\\mathrm{~s} / \\mathrm{m}^{3}\\right) \\tag{9.20}
\\end{equation*}


Fig. 9.7 Input energy $E_{\\text {in }}$ heats a material, and the hated material plasma pushes the DT fuel $m$ and the outer layer $M$. In a rocket propulsion, $m \\gg M$

\\begin{center}
\\includegraphics[max width=\\textwidth]{2024_02_26_83e36543483eb7d284c1g-217}
\\end{center}

Equation (9.19) becomes as follows:


\\begin{equation*}
E_{\\text {in }}>4 \\pi T\\left(10^{2} C_{s}\\right)^{3}\\left(\\frac{1}{n^{2}}\\right) \\tag{9.21}
\\end{equation*}


Therefore, the input energy $E_{\\text {in }}$ to heat the DT fuel is proportional to $1 / n^{2}$. The input energy $E_{\\text {in }}$ becomes smaller, when the fuel density becomes higher.

In addition, the fusion reaction rate is proportional to $n^{2}\\langle\\sigma v\\rangle$ and becomes large with the increase in $n^{2}$.

In ICF, the increase in the density is quite important. It has been considered to compress the DT fuel to $1000 \\sim$ a few 1000 times the solid density. When the DT fuel density is around the solid or DT liquid density, the input energy $E_{\\text {in }}$ becomes huge. However, if the DT density is compressed to about a thousand times the solid density, the input energy $E_{\\text {in }}$ becomes a reasonable value of about $1 \\sim$ several MJ.

Now we consider the method how to compress the DT fuel density to a high density of $1000 \\sim$ a few 1000 times the solid density. For instance, one solid ball with its radius of $1 \\mathrm{~cm}$ is compressed to $1 \\mathrm{~mm}$. Then we can attain the high density of the 1000 times the solid density. However, it seems that it is not so easy to compress the target material to a 1000 times the solid density.

In order to compress the DT fuel to a high density, a rocket effect shown in Fig. 9.7 is employed. The input energy $E_{\\text {in }}$ heats a material between $m$ and $M$. If we ignore the thermal energy and kinetic energy of the heated material, the following is obtained by the momentum conservation approximately:


\\begin{equation*}
M V=m v \\tag{9.22}
\\end{equation*}


The energy conservation provides the following:

\\includegraphics[max width=\\textwidth, center]{2024_02_26_83e36543483eb7d284c1g-218(3)}
heated by driver $E_{\\text {in }}$

b) Outer layer is scattered \\& DT is imploded

\\begin{center}
\\includegraphics[max width=\\textwidth]{2024_02_26_83e36543483eb7d284c1g-218}
\\end{center}

\\begin{center}
\\includegraphics[max width=\\textwidth]{2024_02_26_83e36543483eb7d284c1g-218(1)}
\\end{center}

\\section*{c) DT is stagnated, compressed, ignited and burned.}
\\begin{center}
\\includegraphics[max width=\\textwidth]{2024_02_26_83e36543483eb7d284c1g-218(2)}
\\end{center}

Fig. 9.8 Schematic diagram of the DT fuel implosion, compression, ignition and burning. a Initially the input driver energy $E_{\\text {in }}$ is deposited at the outer region of the fuel target. $\\mathbf{b}$ The outer layer expands outward, and the DT fuel spherical shell is imploded. c Then the compressed DT fuel is ignited and burned. The basic idea in Fig. 9.7 is employed to compress the DT fuel


\\begin{equation*}
E_{\\text {in }}=\\frac{M V^{2}}{2}+\\frac{m v^{2}}{2} \\tag{9.23}
\\end{equation*}


Here we focus on the energy carrying by $m$, which is considered to be the DT fuel in ICF:


\\begin{equation*}
\\eta=\\frac{m v^{2} / 2}{E_{\\text {in }}}=\\frac{M}{M+m} \\tag{9.24}
\\end{equation*}


Therefore, the energy efficiency $\\eta$ becomes large, when $M$, which is explosively scattered out, becomes large.

The idea of the rocket effect in Fig. 9.7 is used in a spherical DT fuel implosion shown schematically in Fig. 9.8: (a) Initially the input driver energy $E_{\\text {in }}$ is deposited at the outer region of the fuel target, (b) the outer layer expands outward, and the spherical shell of the DT fuel is imploded. (c) Then the compressed DT fuel is ignited and burned.

In Fig. 9.8 the DT fuel spherical target has a DT hollow shell, which helps to convert the implosion kinetic energy to the DT fuel thermal energy efficiently $[18$, 19]. When the DT fuel sphere has no hollow inside, the DT fuel central area may be heated significantly before the DT compression at the spherical center. In the hollow shell target structure in Fig.9.8a, b, first the driver input energy is converted to the DT shell kinetic energy, and at the stagnation of the DT fuel the DT kinetic energy is converted to the DT thermal energy to be ignited.

In general the DT shell is in a very low temperature initially. The D and T are isotopes of hydrogen, and so at the room temperature they are in a gas state. Therefore, the DT fuel should be cooled down to be at least a liquid to form a DT layer in Fig. 9.8a, b. Usually the implosion speed $v_{\\text {imp }}$ is larger than the sound speed $C_{s}$ of
the DT fuel during the implosion, and it would be difficult to avoid shock wave generations during the implosion. Multiple shock waves may appear and heat up the DT fuel. The shock wave heating of the DT fuel before the compression may prevent the DT compression to a high density. When the target has a hollow inside, the serious shock heating may be avoided.

Unlike nuclear fission in Eq. (9.10), the fusion reactions in Eqs. (9.11) or (9.12) need two nuclei to fuse. The second law of thermodynamics tells that the entropy increases in an isolated physical system. Therefore, the input energy $E_{\\text {in }}$ should be added first to the fusion fuel system from the outside.

As an input energy driver, it has been considered to use intense lasers [20, 21]. Lasers have a strong directionality, can be focused on a small area and are relatively simple to transport energy in the form of light in a long distance. In addition to lasers, it has been also considered to employ heavy ion beams (HIB) [22-24] and pulse power [25].

When laser beams are focused on a target as shown in Fig.9.8a, the lasers are absorbed at the material surface. The target surface illuminated by the lasers is ionized, and a plasma is generated at the surface. The laser is one of the electromagnetic waves and so cannot penetrate into overdense plasmas with high-density electrons as learned in Sect. 5.6. In the laser plasma interaction, $M \\ll m$ in Fig. 9.7, and the efficiency $\\eta$ becomes small in Eq. (9.24). However, the DT fuel layer is imploded toward the target center (see Fig. 9.8b). If the DT fuel reaches a sufficiently high temperature $(>5-10) \\mathrm{keV}$ and also meets the Lawson criterion, the DT fuel is ignited and burned.

In laser fusion, it has been proposed to improve the energy efficiency $\\eta$ in Eq. (9.24) to follow the way suggested in Fig. 9.7: if laser beams enter, for example, from holes of a cylindrical hohlraum (a cavity in a cylindrical shape) and illuminates the inner surface of the cylinder, the laser energy can be absorbed efficiently [20, 26-32]. The spherical DT fuel target is housed in the hohlraum. In this scheme, that is called the indirect drive scheme, the laser energy is first converted to radiation, and the radiation drives the DT fuel implosion and ignition. In the summer of 2021, NIF experiments presented a very good result, in which the $\\sim 1.92 \\mathrm{MJ}$ NIF laser energy input led the fusion output more than $1.37 \\mathrm{MJ}[31,33,34]$. In addition, in December 2022, NIF showed the breakeven ${ }^{1}$ with the fusion output energy of $3.15 \\mathrm{MJ}$ by the input laser energy of $2.05 \\mathrm{MJ}[35,36]$. Toward future energy source for our society, inertial fusion would move to repetitive and stable energy output [37].

As an input driver energy, ion beam is another candidate, instead of laser. Especially heavy ion beams have been considered for the ICF driver [17, 22-24, 38-40]. As learned in Sect. 2.4, impinging ions collide with target charged particles with the Coulomb collision cross section of Eq. (2.25). When the ions become slow during the interaction, the cross section for the Coulomb collision becomes larger. Therefore, ions deposit their energy largely inside the target material and form an energy deposition peak, called the Bragg peak [40-44]. In ion beam inertial fusion, the energy
\\footnotetext{${ }^{1}$ The "breakeven" means that more energy is produced from fusion reactions than the input energy used to drive fusion fuel implosion, ignition and burning.
}
deposition range is rather long compared with that in laser solid interaction. For example, the heavy ion stopping range, that is, the energy deposition layer thickness, is about several hundred micron $\\mathrm{m}$, depending on the ion particle energy. The concept in Fig. 9.7 would be realized, and the relation of $M \\gg m$ in Eq. (9.24) is naturally realized. In addition to these preferable features of heavy ion inertial fusion, the driver efficiency from electricity to ion beam energy is rather high: about $30-40 \\%$. Because of the high driver efficiency, the requirement for the fusion target gain, that is defined by (the fusion energy output)/(the driver input energy), would be relaxed [24]. The heavy ion inertial fusion (HIF) will be introduced briefly in Sect. 9.4.4.

During the implosion and compression of the DT fuel in ICF target, the DT fuel density should reach a high density of 1000 a few thousand times the solid density as we see above in this section. During the DT fuel layer acceleration, the Rayleigh-Taylor instability may appear. As we see the Rayleigh-Taylor instability in Sect. 7.4, it appears, when a light material supports a heavy material against acceleration. During the ICF fuel acceleration and compression, it may occur [45]. In addition, to the Rayleigh-Taylor instability, the non-uniform implosion acceleration may also appear, because the total number of driver beams is not infinite, and it is quite difficult to obtain a perfect spherical implosion. The requirement for the implosion non-uniformity is very stringent. The non-uniformity of the implosion acceleration should be less than a few \\% [46, 47]. However, the experimental results on NIF have demonstrated that the DT fuel is compressed to a few thousand times the solid density $[29,30]$, and they also have approached just before the DT fuel ignition [27, 28, 34].

In addition, another alternative ignition scheme of fast ignition was also proposed [48] in order to separate the DT fuel ignition from the DT fuel implosion or compression. First the DT fuel is compressed by lasers or driver, and at around the maximum compression additional ignition energy is added to ignite the DT fusion fuel [48-52].

\\section*{HIF: Heavy Ion Beam Inertial Confinement Fusion}
In the preceding Sect. 9.4.4 the ICF concept is introduced. In Sect. 9.4.4 heavy ion beam (HIB) ICF is briefly introduced [24, 38-40, 53, 54].

Heavy ion beam (HIB) is generated by particle accelerators, in which particle energy, pulse length and pulse shape designed are controlled well. HIB accelerators have a remarkable advantage of a high energy efficiency of $\\sim 30-40 \\%$ from the electricity to the HIB's energy. The HIB ICF (HIF) energy system presents that a relatively low fusion energy gain, which is defined by the fusion energy output/the input driver HIBs energy, is sufficient to construct the HIF energy power plant to supply $\\sim 1 \\mathrm{GW}$ electricity to our society, because of the high energy efficiency. A standard HIBs driver input energy may be about $4-5 \\mathrm{MJ} /$ shot. The DT fusion fuel should be also compressed to about a thousand times of the solid density to reduce the input driver energy, and the total DT fuel mass is about several mg. The HIB accelerator would provide a flexible controllability of the HIB parameters to release the fusion energy stably. In HIB accelerators the following key points in HIF are well controlled:
for example, focusing position, beam focal radius, particle energy distribution in a beam, ion energy, pulse length, pulse shape, HIB axis oscillation/wobbling motion, operating repetition rate, etc. [22-24, 38-40, 53-60]. Another important preferable feature is to deposit HIB energy inside a material. The interaction of HIB ions with materials is almost classical, that is, the Coulomb interaction [4, 41-43]. This means that HIB ion interaction is well understood and defined inside materials. The energy deposition spatial profile in materials is sufficiently predictable, and the fusion fuel design would be relatively simple [24, 38, 40, 61].

When HIBs interact with a thin material layer, depending on the ion energy the heavy ions may penetrate the thin heavy layer and deposit the particle energy near the stop position inside the material. The scheme illustrated in Fig. 9.7 is realized naturally by the HIB's nature. The high HIB driver efficiency of about $30-40 \\%$ leads to a relatively low fusion gain to construct a fusion power plant. The fusion energy gain $\\mathrm{G}$ of $\\sim 50-70$ is appropriate to construct a fusion energy power plant, which delivers $1 \\mathrm{GW}[24]$.

The HIB ion deposition range is defined by the HIB ions stopping length, which would be $\\sim 500 \\mu \\mathrm{m}$ or so depending on the material and the particle energy. Therefore, a relatively large-density scale length appears in the fuel target material. The temperature at the energy deposition layer in a HIF target does not reach a very high temperature: normally about $300 \\mathrm{eV}$ or so is realized in the energy absorption region. One of the critical issues in inertial fusion would be a spherically uniform target compression, which would be degraded by a non-uniform implosion. The implosion non-uniformity would be introduced by the HIBs energy deposition non-uniformity or the Rayleigh-Taylor (R-T) instability. The large-density gradient-scale length may help to reduce the $\\mathrm{R}-\\mathrm{T}$ growth rate $[24,40,44,62,63]$. The wide density valley appears in the energy deposition region, and in the density valley a part of the HIBs deposited energy is converted to the radiation, and the radiation may be confined in the density valley. The converted and confined radiation energy is not negligible, and it would be the order of $\\sim 100 \\mathrm{~kJ}$ in a HIF reactor-size DT target. The confined radiation in the density valley would also contribute to reduce the non-uniformity of the HIBs energy deposition [64].

The HIB ion interaction is relatively simple and is almost the classical Coulomb collision, except the plasma range-shortening effect (see, for example, Ref. [42] and Chap. 3 in Ref. [65]). The DT fuel compression dynamics itself in HIF is the same as that in the laser fusion, though the driver target interaction is quite different from that in the laser fusion. A typical HIB ion species would be $\\mathrm{Pb}$ or $\\mathrm{Cs}$ or so. For $\\mathrm{Pb}$ ions, the appropriate $\\mathrm{Pb}$ ion energy would be about $8-10 \\mathrm{GeV}$ or so. The HIBs illumination scheme should be studied intensively to realize a uniform energy deposition in a HIF target. The uniformity requirement must be fulfilled to release the fusion energy. The HIB pulse shape should be also designed to obtain a high implosion efficiency $\\eta$.

The HIF reactor would have a dilute reactor gas inside the reactor chamber after each fuel target shot. The reactor operation frequency would be $10-15 \\mathrm{~Hz}$ in HIF. In HIF the debris gas would supply cold electrons to compensate the HIB self-charge in the vicinity of the fuel target and/or during the HIBs transport in the reactor. The reactor radius may be $3-5 \\mathrm{~m}$ or so. The HIB transport in the reactor is another key
issue. The HIB ion mass is large so that the ion trajectory is almost straight between the HIB accelerator exit and the reactor center. However, each HIB carries a large current of $1 \\mathrm{kA}$ or so. The remaining large current and its self-charge would provide a slight HIB radial expansion [66]. However, HIB would be safely transported almost ballistically.

The reactor design is also another issue in ICF [67-70]. The first wall could be a wet wall with a molten salt or so or a dry wall. The reactor design must accommodate a large number of HIBs beam port, for example, 32 beam ports. At the first wall and the outer reactor vessel the beam port holes should have mechanical shutters or so to prevent the fusion debris exhaust gas toward the accelerator upstream. In addition, the target debris remains inside of the first wall or mixes with the liquid molten salt, which may be circulated. The target debris treatment should be also studied as a part of the reactor design. The tritium (T) also remains inside of each target after burning. Usually about $30 \\%$ or so of the DT fuel is reacted and depleted in the target during the burning process. So a large part of $\\mathrm{T}$ in each target is mixed in the reactor gas and would be melted in the liquid first wall. The rest $\\mathrm{T}$ and the radioactivated target materials must be distributed inside of the reactor vessel right after the target burning, and they would not accrete in a large lump in the reactor. The distributed radioactive materials must be collected and separated in the fusion reactor system safely.

In ICF reactor, DT fuel target is repetitively injected and precisely aligned. When a spherical DT fuel is injected from outside a reactor, the DT fuel pellet may be accelerated to a high speed of $\\sim 100 \\mathrm{~m} / \\mathrm{s}$ to avoid the DT fuel melting by the reactor exhaust reactor gas [71-78]. The DT fuel is cooled down to be a liquid, and the DT fuel layer is formed in the hollow fuel capsule. In HIF fuel target, the most outer layer of $\\mathrm{Pb}$ becomes a superconductor [71-73] in a cryostat. The cryo-target is transferred to an injection gun, by which the cryo-target is injected into reactor. The traveling time of the fuel target injected is about 0.05-0.06 s [71, 72], which is estimated by the traveling distance $(\\sim 5-6 \\mathrm{~m}$ ) divided by the target injection speed of $\\sim 100 \\mathrm{~m} / \\mathrm{s}$. During the time interval of about $0.05-0.06 \\mathrm{~s}$, it was found that the superconducting state of the cryo-target is not melted [72]. The results in Refs. [71-73] demonstrate that the target trajectory and speed are controlled by magnetic force. The HIF directdrive target is designed to have a heavy outer layer which becomes a superconducting state. This point would be also a merit of the HIF target. The outer superconducting shell would help reduce the target alignment error in reactor chamber in HIF [7173]. In addition, Ref. [74] proposed that charged fuel target alignment would be also controlled actively by electric field.

The following points should be also studied for a realistic HIB reactor system. The ICF reactor operation is intermittent. The DT neutrons and the target debris transfer the fusion pulse energy to the heat recovery system. The pulsation effect of the heat transfer on fusion reactor should be studied to avoid a significant influence on the steady electric power generation.

\\subsection*{Laser Particle Acceleration}
Laser technology has been explored extraordinarily, and very intense lasers are now available $[79,80]$. Currently the laser intensity has reached $\\sim 10^{23} \\mathrm{~W} / \\mathrm{cm}^{2}$ [80].

Based on the development of the laser technology and science, studies on laser particle acceleration have been proposed [81-84] and intensively investigated in the world, as well as studies on radiation generation and various application of lasermatter interaction [85]. In laser particle acceleration, the very strong laser field is used to accelerate electrons and ions instead of conventional accelerators. The maximal acceleration field would be $\\sim 100 \\mathrm{MeV} / \\mathrm{m}$ for conventional accelerators. However, the acceleration field would be $>100 \\mathrm{GeV} / \\mathrm{m}$ for laser plasma acceleration. Lasers may make particle accelerators very short.

Laser plane wave is schematically shown in Fig.9.9, in which laser propagates in $+z$. The electric field $E$ is directed in $y$ and is transverse to the magnetic field $B$ in $x$. When one electron enters in the strong laser field in Fig. 9.9, the electron may be accelerated strongly. Now the laser propagates in the $+z$ direction. The electron may reach a high speed close to the speed of light $c$. Since electron has mass, electron cannot reach $c$ exactly. Therefore, the laser passes through the electron. In that case, Fig. 9.10 shows the electron energy change schematically along the electron trajectory in $z$. First the electron receives the laser energy, but after that the electron loses its energy and returns to the initial energy. Figure 9.10 presents the electron energy change, when the electron passes through one wavelength of the laser field.

The result shown in Fig. 9.10 may present that particles cannot obtain a net energy from lasers [86-88]. This result comes from the fact that lasers have a symmetry in space and time [84, 86-88] as shown in Fig. 9.9. Figure 3.13 in Sect. 3.6 also presents this fact. The electron motion in a uniform plane laser wave in Fig. 3.13 is a simple zigzag one, which seems to be consistent with the result in Fig. 9.10. However, the laser field is not always uniform, and the interaction between laser and particles may be limited in time and space. So far various ideas have been proposed, and experimental, theoretical and numerical studies have been performed for the particle accelerations [81-84, 89-109].

Fig. 9.9 Laser field, propagating in $+z$

\\begin{center}
\\includegraphics[max width=\\textwidth]{2024_02_26_83e36543483eb7d284c1g-223}
\\end{center}

Fig. 9.10 Electron energy change along $z$ in a uniform plane electromagnetic wave. The electron first obtains the laser energy, and later decelerated. The electron final energy returns back to the initial energy, after the electron passes one laser wavelength

\\begin{center}
\\includegraphics[max width=\\textwidth]{2024_02_26_83e36543483eb7d284c1g-224}
\\end{center}

For example, when a static electric or magnetic field is applied uniformly in space in addition to a laser field, the spatial and temporal symmetry would be broken. Then electrons can obtain net energy from the laser field [107, 108]. If an electron can escape from the laser deceleration field at the electron energy peak in Fig.9.10, the electron obtains a net energy. This can be realized by several methods. One of them is a plasma mirror or separator, which may be placed at an appropriate place to reflect only the laser. The electrons pass through the thin plasma [102, 110]. After the electrons obtain the laser energy, the electrons can escape from the laser deceleration field by the thin plasma mirror or separator as shown in Fig. 4.7 in Sect. 4.4.4. Another example can be found in Fig. 3.14 in Sect. 3.6, which shows that an electron is expelled by a Gaussian laser pulse. The ponderomotive force shown in Sect. 3.6 or in Appendix D pushes charged particles, and the particles can be accelerated.

When an intense laser pulse enters into a dilute plasma gas, the plasma electrons would be expelled by the laser strong field and the rest ions create a wake field behind the laser pulse. It is called as the laser wake field acceleration [81, 83, 84, 89-94] .

Figure $9.11 \\mathrm{~b}$ shows the laser wake field generation by the intense short-pulse laser in Fig. 9.11 a. The laser pulse length is $20 \\mathrm{fs}$, the laser intensity is $2.14 \\times 10^{20} \\mathrm{~W} / \\mathrm{cm}^{2}$, and the pulse shape is the Gaussian one in time and space. The spot radius is $2 \\mu \\mathrm{m}$. The plasma is a hydrogen plasma, and the density is $0.005 n_{c}=8.75 \\times 10^{18} / \\mathrm{cm}^{3}$ for the laser wavelength of $\\lambda=0.8 \\mu \\mathrm{m}$. Here $n_{c}$ shows the critical density, at which the laser frequency $\\omega$ is equal to the electron plasma frequency $\\omega_{p e}$ shown in Sect. 5.6. A bubble of the electron density is created, and at the tail (the left end of the bubble in this case) of the bubble electrons are accelerated as shown by the green color in Fig. 9.11b.

When laser intensity is very high, for example, $10^{21} \\mathrm{~W} / \\mathrm{cm}^{2}$ or higher and the laser illuminates a target, the target electrons are quickly accelerated by the laser electric field and soon become relativistic. However, ions are heavy and so at the beginning of the laser-target interaction the ions would stay the original position. Between the target ions and the electrons accelerated, a strong electric field is created at the target surface. The direction of the electric field is normal to the target surface, and so the

\\begin{center}
\\includegraphics[max width=\\textwidth]{2024_02_26_83e36543483eb7d284c1g-225}
\\end{center}

Fig. 9.11 a Intense laser enters a gas plasma, and the plasma electrons are expelled quickly. b A bubble of the electron density is created, and at the tail (the left end of the bubble in this case) of the bubble electrons are accelerated as shown by the green color

ions of the target surface are also gradually accelerated to the normal direction to the target surface. This ion acceleration mechanism is called as the target normal sheath acceleration (TNSA) and has been studied in detail [95-100, 103, 106].

When a laser with its intensity of $10^{21} \\mathrm{~W} / \\mathrm{cm}^{2}$ illuminates a hydrogen thin-disk target with its density of $28 n_{c}$, the EPOCH 3D simulation shows the electron acceleration and proton acceleration as shown in Fig. 9.12. The dense plasma thin disk is irradiated by the intense laser, whose pulse duration is $40 \\mathrm{fs}$. The laser spot size is $2 \\mu \\mathrm{m}$ with the Gaussian distribution. In Fig. 9.12a, the target electron temperature is shown. The electrons are accelerated quickly from the thin-disk target and move around the target. The electron temperature is very high at the target front and back surfaces. In this case, the electrons move to the right side of the target induces the strong charge separation and consequently the strong electric field, which accelerates the target ions, as shown in Fig. 9.12b, c.

In the laser ion acceleration various efforts have been performed to improve the ion quality, energy, pulse shape, etc. For the energy conversion efficiency from lasers to ions was improved significantly by subwavelength structured targets $[105,106,111]$. The short-pulse laser energy is absorbed efficiently by the subwavelength structure of the foil target surface. The absorbed energy is converted well to the hot electrons, and the stronger electric field is created. The ions are efficiently accelerated. In order to improve the ion beam quality, the ion beam divergence would be reduced by the transverse TNSA electric field, which appears at the transverse walls of holes at the solid target rear $[106,109]$. The ion beam quality in longitudinal would be realized by the beam bunching [106]. The tail part of the ion beam generated can be accelerated, for example, by the additional stage. Then the longitudinal beam elongation would be reduced. After the beam bunching, additional post-acceleration stages may be used to obtain a target ion energy [106].

\\begin{center}
\\includegraphics[max width=\\textwidth]{2024_02_26_83e36543483eb7d284c1g-226}
\\end{center}

Fig. 9.12 Intense laser accelerates electrons and ions. An intense laser illuminates a high-density plasma thin disk. The plasma is a hydrogen plasma with its initial density of liquid hydrogen. a The target electrons are expelled quickly from the thin disk target and move around the target. The electron temperature is very high at the target front and back surfaces. b In this case, the electrons move to the right side of the target that induces the strong charge separation and consequently the strong electric field, which accelerates the $\\mathbf{c}$ target ions

\\subsection*{Cluster Ion Interaction with Plasma}
In Fig. 9.11, an intense laser generates a wake field, which accelerates background plasma electrons. Instead of lasers, high-energy ions would also generate the wake field [112-121]. The wake field is also an example phenomenon of the plasma collective behavior. Relating to the ion-generated wake field, the cluster ion interaction with matters has been also studied intensively [112-121].

Figure 9.13 shows the wake field behind a swift $\\mathrm{Si}$ ion in a solid-density $\\mathrm{Al}$ plasma. For the fast-moving ion with $v_{0} \\gg \\sqrt{T_{e} / m_{e}}$, the wake field, localized in the transverse and backward directions, extends beyond $\\lambda_{D e}$ [121]. In Fig. 9.13, $v_{0}=0.221 c$ and $T_{e}=10 \\mathrm{eV}$. Here, $T_{e}$ is the electron temperature, $m_{e}$ the electron mass, and $c$ is the speed of light. The Debye shielding length is $\\lambda_{D e} \\sim 1 \\AA$.

Here we consider a fast-moving $\\mathrm{Si}$ cluster interaction with the solid $\\mathrm{Al}$ plasma. In one cluster moving with a high speed of $v_{0} / \\sqrt{T_{e} / m_{e}} \\gg 1$, a Si ion located behind a forward-moving Si ion may catch up and overtake the forward-moving $\\mathrm{Si}$ in the cluster during the Si cluster interaction with the high-density Al plasma. The Si cluster interionic distance is $l_{c} \\sim 5 \\AA>\\lambda_{D e}(\\sim 1 \\AA)$. In our case, the Si cluster composed of six Si ions interacts with the solid Al plasma. The six-Si cluster structure is a regular octahedron with the side length of $5 \\AA$.

Therefore, in our case shown in Sect.9.6, the wake field does not cover the whole cluster ions. When the plasma density is very low and/or the $v_{0}$ is very low, the wake field may cover all the ions of one cluster. Then the so-called vicinage effect may appear [112-121], in which the enhancement in the stopping power for all the ions in one cluster would be expected. However, in our case, the plasma density is so high, and the $\\mathrm{Si}$ ion speed is large.

\\begin{center}
\\includegraphics[max width=\\textwidth]{2024_02_26_83e36543483eb7d284c1g-227}
\\end{center}

Fig. 9.13 Schematic figure of the wake potential generated by a swift Si ion. In high-density solid Al plasma, the wake field is localized in the transverse direction and elongated longitudinally behind the $\\mathrm{Si}$ ion, so that only the $\\mathrm{Si}$ ion just behind the $\\mathrm{Si}$ ion in the linearly aligned direction is influenced, $\\lambda_{D e}<l_{c}$. Here, $l_{c}$ is the Si cluster interionic distance. Source Ref. [112] (Reprinted figure with permission from Kawata, S., Deutsch, C. and Gu, Y. J.: Peculiar behavior of Si cluster ions in a high-energy-density solid Al plasma, Phys. Rev. E, 99 (2019) 011201(R). \\href{https://link.aps.org/doi/}{https://link.aps.org/doi/} 10.1103/PhysRevE.99.011201 Copyright 2019 by the American Physical Society)

Figure 9.14 shows the time evolution of six $\\mathrm{Si}$ ions in one $\\mathrm{Si}$ cluster in the solid Al plasma at (a) $t=0$, (b) $0.5 \\mathrm{fs}$, (c) $1.0 \\mathrm{fs}$ and (d) $1.5 \\mathrm{fs}$. In each figure, the number besides each $\\mathrm{Si}$ ion shows the identity number of each $\\mathrm{Si}$ ion: For example, (1) in the term $\\mathrm{Si}(1)$ shows the identity number for the $\\mathrm{Si}$. Each $\\mathrm{Si}$ ion creates a wake field, and the wake electric field is localized in an elongated transversely limited area as shown in Fig.9.13. The six-Si cluster structure is a regular octahedron with a side length of $5 \\AA$ : therefore, the length between $\\operatorname{Si}(1)$ and $\\operatorname{Si}(2)$ is $5 \\times \\sqrt{2.0} \\AA$. As shown in Fig. 9.14, the forward-moving $\\mathrm{Si}(1)$ is caught and overtaken by $\\mathrm{Si}(2)$, which is located just behind $\\mathrm{Si}(1)$ in parallel.

Among the six Si ions consisting of the Si cluster, only Si(2) feels the wake field by $\\mathrm{Si}(1)$. Figure $9.15 \\mathrm{a}$ presents the time sequence of the $x$ position, and Fig. $9.15 \\mathrm{~b}$ shows the history of the $\\mathrm{Si}$ ion kinetic energy $(\\gamma-1)$ for $\\mathrm{Si}(1)$ and $\\operatorname{Si}(2)$. The $\\mathrm{Si}(1)$ ion dissipates the energy normally as clearly shown in Fig. 9.15b, though $\\mathrm{Si}(2)$ dissipates its energy slightly by the wake field until $\\mathrm{Si}(2)$ catches up $\\mathrm{Si}(1)$.

After $\\mathrm{Si}(2)$ passes through $\\mathrm{Si}(1)$, both $\\mathrm{Si}(1)$ and $\\mathrm{Si}(2)$ move slightly in transverse. Then the longitudinal alignment is broken during the interaction, and $\\mathrm{Si}(1)$ does not catch up with $\\mathrm{Si}(2)$ again.

The interesting ion motion is presented in a cluster interacting with high-density solid $\\mathrm{Al}$ plasma. In the case shown here, the target electron number density is high, and the ion or cluster speed is also large, $v_{0} / \\sqrt{T_{e} / m_{e}} \\gg 1$. Therefore, the wake field created by each ion in one cluster influences the ions just behind the ion concerned. In this extreme situation, the front ions lose their energy as an isolated ion loses its energy, and the ion just behind the forward-moving ion in the cluster may catch up with the forward ion and exchanges positions with that of the forward ion.
(a) $\\mathrm{t}=0$

\\begin{center}
\\includegraphics[max width=\\textwidth]{2024_02_26_83e36543483eb7d284c1g-228(3)}
\\end{center}

(c) $\\mathrm{t}=1.0 \\mathrm{fs}$

\\begin{center}
\\includegraphics[max width=\\textwidth]{2024_02_26_83e36543483eb7d284c1g-228(2)}
\\end{center}

(b) $t=0.5 \\mathrm{fs}$

\\begin{center}
\\includegraphics[max width=\\textwidth]{2024_02_26_83e36543483eb7d284c1g-228}
\\end{center}

(d) $\\mathrm{t}=1.5 \\mathrm{fs}$

\\begin{center}
\\includegraphics[max width=\\textwidth]{2024_02_26_83e36543483eb7d284c1g-228(1)}
\\end{center}

Fig. 9.14 Si cluster, composed of six Si ions, interacts with the solid Al plasma. The Si cluster velocity is $v_{x}=0.221 c$. The six $\\mathrm{Si}$ ions consist of a regular octahedron, and the distance between the adjacent two $\\mathrm{Si}$ ions is $5 \\AA$. The six $\\mathrm{Si}$ ion positions are shown at a $t=0, \\mathrm{~b} 0.5 \\mathrm{fs}, \\mathrm{c} 1.0 \\mathrm{fs}$ and d 1.5 fs. The forward-moving $\\mathrm{Si}(1)$ is caught and overtaken by $\\mathrm{Si}(2)$, which is located just behind $\\mathrm{Si}(1)$ in parallel. Source Ref. [112] (Reprinted figures with permission from Kawata, S., Deutsch, C. and Gu, Y. J.: Peculiar behavior of Si cluster ions in a high-energy-density solid Al plasma, Phys. Rev. E, 99 (2019) 011201(R). \\href{https://link.aps.org/doi/10.1103/PhysRevE}{https://link.aps.org/doi/10.1103/PhysRevE}. 99.011201 Copyright 2019 by the American Physical Society)

\\subsection*{Control of Plasma: Dynamic Mitigation of Plasma Instabilities and Non-uniformities}
In general, plasma instabilities emerge from perturbations. Normally perturbation phase in plasmas is unknown, and so instability growth rate is discussed. However, if the perturbation phase is known even in plasmas, the instability growth can be controlled by a superimposition of perturbations imposed actively. In Sect.9.7, we

\\begin{center}
\\includegraphics[max width=\\textwidth]{2024_02_26_83e36543483eb7d284c1g-229}
\\end{center}

(b)

\\begin{center}
\\includegraphics[max width=\\textwidth]{2024_02_26_83e36543483eb7d284c1g-229(1)}
\\end{center}

Fig. 9.15 a History of the position in $x$, and $\\mathbf{b}$ the kinetic energy history for $\\operatorname{Si}(1)$ and $\\operatorname{Si}(2)$ in one $\\mathrm{Si}$ cluster, which is composed of six $\\mathrm{Si}$ ions moving in the solid $\\mathrm{Al}$ in the $x$ direction. The forward-moving $\\mathrm{Si}(1)$ dissipates its energy normally in the $\\mathrm{Al}$ target, and the following $\\mathrm{Si}(2)$ feels the acceleration wake field, generated by the forward-moving Si(1). Source Ref. [112] (Reprinted figures with permission from Kawata, S., Deutsch, C. and Gu, Y. J.: Peculiar behavior of Si cluster ions in a high-energy-density solid Al plasma, Phys. Rev. E, 99 (2019) 011201(R). \\href{https://link.aps}{https://link.aps}. org/doi/10.1103/PhysRevE.99.011201 Copyright 2019 by the American Physical Society)

Fig. 9.16 Inverted pendulum, stabilized by adding a strong oscillating acceleration of $A \\sin (\\omega t)$ to create a new stabilization window [127]

\\begin{center}
\\includegraphics[max width=\\textwidth]{2024_02_26_83e36543483eb7d284c1g-230}
\\end{center}

introduce plasma control theory and its applications to several plasma instabilities [24, 61, 122-126], as well as another type of plasma control theory which was proposed by Kapitza [127].

\\subsubsection*{Control of Plasma: Theory}
\\section*{Dynamic Stabilization of Instability by Adding Strong Oscillation}
One of the most typical and well-known mechanisms is the feedback control in which the perturbation phase is detected and the perturbation growth is controlled or mitigated [128]. The feedback control has been applied successfully to many mechanical systems, including tall buildings and towers, to be stabilized against earthquakes. In plasmas and fluids, it is difficult to detect the perturbation phase and amplitude.

In Refs. [129-131], one dynamic stabilization mechanism was proposed to stabilize the Rayleigh-Taylor ( $\\mathrm{R}-\\mathrm{T}$ ) instability based on the strong oscillation of acceleration, which was realized, for example, by the picket fence pulse train or the laser intensity modulation in laser inertial fusion [132]. In this mechanism, the total acceleration oscillates strongly, and so the additional oscillating force is added to create a new stable window in the system. Originally this dynamic stabilization mechanism was proposed by Kapitza [127], and it was applied to the stabilization of an inverted pendulum. The inverted pendulum is an unstable system, a strongly and rapidly oscillating acceleration is applied on the system in Ref. [127], and then the inverted pendulum system has a new stable window. In this method, the equation for the unstable system is modified and has another force term coming from the oscillating acceleration. In this mechanism, the growth rate is modified by the strongly oscillating acceleration. Figure 9.16 shows a schematic figure for the inverted pendulum, stabilized by adding a strong oscillating acceleration.

When the inverted pendulum in Fig. 9.16 is subjected by a strongly oscillating acceleration $A \\sin \\omega t$, we obtain the following Mathieu-type equation (see, for example, Chap. 28 in Ref. [133] and Chap. 13 in Ref. [134]) for $\\theta(t)$ :


\\begin{equation*}
\\frac{\\mathrm{d}^{2} \\theta(t)}{\\mathrm{d} t^{2}}=\\frac{g}{l} \\theta(t)-A \\omega^{2} \\theta(t) \\sin \\omega t \\tag{9.25}
\\end{equation*}


Here $l$ shows the pendulum length. When $A=0$, the inverted pendulum becomes unstable. When the second term at the right hand side is added to the system, a stable window appears in the inverted pendulum (see Ref. [127], Chap. 28 in Ref. [133] and Chap. 13 in Ref. [134]). The stabilization condition is $A-0.5<2 g /\\left(l \\omega^{2}\\right)<A^{2}$. The stability condition shows that the additional acceleration oscillation at the second term of the right hand side of Eq. (9.25) should be very fast, and the amplitude of $A$ must satisfy the stability condition. The dynamic stabilization mechanism shown here works on the inverted pendulum. However, it may be difficult to apply this mechanism to our tall buildings, bridges or large structures in our society.

\\section*{Dynamic Mitigation of Plasma Instabilities and Non-uniformity by Phase Control}
In plasmas the perturbation phase and amplitude cannot be measured dynamically. However, by using a dynamically controlled driver, for example, a wobbling beam or an oscillating beam or a rotating beam or so $[59,60]$, the initial perturbation is actively imposed, and the initial perturbation phase and amplitude are defined actively. In this case, the amplitude and phase of the perturbation can be defined by the input driver beam wobbling at least in the linear phase. In plasmas it would be difficult to realize a perfect feedback control, but a feedforward control, in which perturbation phase and amplitude are predicted, can be adapted to the instability mitigation or the nonuniformity mitigation in plasmas. For example, heavy ion beam accelerators would provide controlled wobbling or oscillating beams with high frequency [24, 39, 58$60,122-124]$. An intense electron beam axis can be also wobbled in its controlled manner and thus provides the defined phase and amplitude of perturbations.

If the energy driver beam wobbles or oscillates uniformly in time, the imposed perturbation for a physical quantity of $F$ at $t=\\tau$ may be written as follows:


\\begin{equation*}
F=\\delta F \\exp (i \\omega t) \\exp (\\gamma(t-\\tau)+i \\mathbf{k} \\cdot \\mathbf{x}) \\tag{9.26}
\\end{equation*}


Here $\\delta F$ shows the amplitude, $\\Omega$ is the oscillation or wobbling frequency, and $\\Omega \\tau$ is the phase shift of a superposed perturbation. When system is unstable, $\\gamma$ shows the growth rate and is positive. Even for stable system $(\\gamma<0)$, Eq. (9.26) serves a smoothing of non-uniformity in system concerned. At each time $t=\\tau$, the wobbler provides a new perturbation with the controlled phase shifted and amplitude defined by the driving wobbler itself. After the superposition of the perturbations, the overall perturbation would be described as follows:


\\begin{align*}
\\int_{0}^{t} \\mathrm{~d} \\tau & \\delta F \\exp (i \\Omega t) \\exp (\\gamma(t-\\tau)+i \\mathbf{k} \\cdot \\mathbf{x}) \\\\
& \\propto \\frac{\\gamma+i \\Omega}{\\gamma^{2}+\\Omega^{2}} \\delta F \\exp (\\gamma t+i \\mathbf{k} \\cdot \\mathbf{x}) \\tag{9.27}
\\end{align*}


At each time of $t=\\tau$, the driving wobbler provides a new perturbation with the shifted phase. Then each perturbation grows with the factor of $\\exp (\\gamma t)$. At $t>\\tau$, the superimposed overall perturbation growth is modified as shown in Eq. (9.27). When $\\Omega \\gg \\gamma$, the perturbation amplitude is reduced by the factor of $\\sim|\\gamma| /|\\Omega|$, compared with the pure instability growth $(\\Omega=0)$ based on the driver-induced non-uniformity.

When $\\gamma>0$, the system is unstable. When $\\gamma<0$, the system is stable, and the non-uniformities would be also smoothed. The phase control mitigation theory works well to mitigate the instability growth and to smooth the non-uniformity, as long as the relation of $|\\Omega| \\geq|\\gamma|$. Physical systems may have many modes, and modes, fulfilling the condition of $|\\Omega| \\geq|\\gamma|$, are mitigated by the dynamic phase control presented here. In addition, if there are other sources of perturbations in physical systems and if the perturbation phase and amplitude are not controlled, the dynamic phase control mechanism does not work.

Figure 9.17 shows the superimposed perturbations decomposed, and at each time the phase-defined perturbation is imposed actively by the driver wobbler. The perturbations are superimposed at the time $t$. The wobbling trajectory is under control, for example, by a beam accelerator or so, the superimposed perturbation phase and amplitude are controlled, and the overall perturbation growth is also controlled. In Fig.9.17a at $t=t$, a perturbation is introduced, and after $\\Delta t$ it grows, when the system is unstable. After $\\Delta t=2 \\pi / \\Omega$, another perturbation with the inverse phase is imposed. The new perturbation with the inverse phase would compensate the perturbation shown in Fig. 9.17a. In Fig. 9.17c, the overall amplitude of all the perturbations would be reduced.

The control system shown in Sect.9.7.1 may work on the mitigation of plasma instabilities and for the smoothing of plasma non-uniformities. For the unstable plasmas, after linear stage plasmas would move to nonlinear phases and to turbulent states as shown in Chap. 8. When plasmas are in the turbulent states, the phase of the turbulences may not be controlled and the theory presented here may not work for the mitigation.

\\subsubsection*{Dynamic Control of Plasma Instabilities}
In Sect. 9.7.2, some example simulation results are briefly shown to demonstrate the dynamic phase control of plasma instabilities and non-uniformity.

Fig. 9.17 Concept of phase control of plasma instabilities and non-uniformities. a At $t=t$, a perturbation is introduced, and after $\\Delta t$ it grows, when the system is unstable. $\\mathbf{b}$ After $\\Delta t=2 \\pi / \\Omega$, another perturbation with the inverse phase is imposed. The new perturbation with the inverse phase would compensate the perturbation shown in a. c The overall amplitude of all the perturbations would be reduced

\\begin{center}
\\includegraphics[max width=\\textwidth]{2024_02_26_83e36543483eb7d284c1g-233(2)}
\\end{center}

b) $t=t+2 \\pi / \\Omega$

\\begin{center}
\\includegraphics[max width=\\textwidth]{2024_02_26_83e36543483eb7d284c1g-233}
\\end{center}

c) $t=t+2 \\pi / \\Omega$

\\begin{center}
\\includegraphics[max width=\\textwidth]{2024_02_26_83e36543483eb7d284c1g-233(1)}
\\end{center}

\\section*{Dynamic Control of the Rayleigh-Taylor Instability}
Figure 9.18 show example simulation results for the Rayleigh-Taylor instability growth, which has two modes. Figure 9.18 presents the density and the vorticity $(\\nabla \\times \\mathbf{v})$ by the $2 \\mathrm{D}$ fluid simulations introduced in Chap. 5 . In this example, two stratified fluids are superimposed under an acceleration of $g=g_{0}+\\delta g$. The density ratio between the two fluids is $10 / 3$. In this specific case the wobbling frequency $\\Omega$ is the growth rate of the Rayleigh-Taylor instability of $\\gamma$, the amplitude of $\\delta g$ is $0.1 g_{0}$, and the results shown in Fig. 9.18 are those at $t=5 / \\gamma$. In Fig. 9.18a, $\\delta g$ is constant and drives the Rayleigh-Taylor instability as usual, and in Fig. $9.18 \\mathrm{~b}$ the phase of $\\delta g$ is shifted or oscillates with the frequency of $\\Omega$ as stated above for the dynamic instability mitigation. The growth mitigation ratio of the Rayleigh-Taylor instability is $76.0 \\%$ in this case. The growth mitigation ratio is defined by $\\left(H_{0}-H_{\\text {mitigate }}\\right) / H_{0} \\times 100$ $\\%$. Here $H_{0}$ shows the deviation amplitude of the two-fluid interface in the case in Fig. 9.18a without the oscillation $(\\Omega=0)$, and $H_{\\text {mitigate }}$ presents the deviation for the other case with the oscillation $(\\Omega=\\gamma)$. The example simulation results support well the effect of the dynamic mitigation mechanism. In Ref. [157], it is presented that the dynamic phase control mechanism is rather robust against changes in the wobbling or oscillating frequency, amplitude and phases.

\\section*{a) $t=5 / \\gamma$ w/o wobbling}
\\begin{center}
\\includegraphics[max width=\\textwidth]{2024_02_26_83e36543483eb7d284c1g-234}
\\end{center}

\\section*{b) $\\boldsymbol{t}=\\mathbf{5} / \\gamma$, with wobbling: $\\Omega=\\gamma$}
\\begin{center}
\\includegraphics[max width=\\textwidth]{2024_02_26_83e36543483eb7d284c1g-234(1)}
\\end{center}

Fig. 9.18 Dynamic mitigation of the Rayleigh-Taylor instability growth in 2D fluid simulations: a the density and the vorticity $\\nabla \\times \\mathbf{v}$ at $\\gamma t=5$ without the wobbling motion $(\\Omega=0)$, and $\\mathbf{b}$ the density and the vorticity at $\\gamma t=5$ with the wobbling motion $(\\Omega=\\gamma)$. The results presented here show that the mechanism of the dynamic instability mitigation works well for the Rayleigh-Taylor instability growth

\\section*{Dynamic Control of Instabilities of Magnetized Plasma Column}
Here first we simulate the sausage instability by the 3D EPOCH PIC code [135, 136]. The column plasma consists of protons and electrons in Sect.9.7.2, and a part of the results was shown in Fig. 7.6 in Sect. 7.3.1. The simulation conditions are described below again. The protons are initially stationary and uniform in space inside the column. Its initial density is $f \\times n_{0}$, and the protons provide the partial charge neutralization of the $z$-current electrons with the neutralization ratio of $f$, which is 0.99 in the example cases. The radial force on the electron beam, consisting of
the radial electric and magnetic forces, is balanced for the arbitrary density profile, provided $1-f=\\beta_{0}=v_{0} / c$ (see Chap. 2 in Ref. [137]). Here the electron beam speed $v_{0}$ in $z$ is set to be $0.1 c$, and so in this case $f$ is 0.99 . Here $c$ is the speed of light, and the electron number density is $n_{0}=10^{16} / \\mathrm{m}^{3}$. The electron velocity and density are also uniform. The initial ion and electron temperatures are $1 \\mathrm{eV}$. The plasma column is $20 \\mathrm{~cm}$ long in $z$, and its radius is $1.5 \\mathrm{~cm}$. In the $z$ direction, the cyclic boundary condition is employed, and in the transverse directions the open boundary conditions are employed. In the column equilibrium state the radial force is balanced between the outward electrostatic force and the inward magnetic pinch force by the azimuthal magnetic field generated by the electron net current (see Chap. 2 in Ref. [137]).

Figure 9.19 presents the sausage instability evolution without the mitigation mechanism. Figure 9.19 shows a clear instability growth and presents the electron distributions at (a) $t=20 \\mathrm{~ns}$ and (b) $t=30 \\mathrm{~ns}$, and the proton distributions at (c) $t=20 \\mathrm{~ns}$ and (d) $t=40 \\mathrm{~ns}$. As presented in Fig. 9.19, the initial perturbation has four modes in the $20-\\mathrm{cm}$ simulation box in the longitudinal direction of $z$. The growth rate $\\gamma$ of the sausage instability is about $\\gamma \\sim 6.16 \\times 10^{6}$ /s in this example case. Figures $9.19 b$, d correspond to Figs. 7.6a, b, respectively.

In Fig.9.20, an electron beam rotation (wobbling) is added around the column axis by one rotation in $20 \\mathrm{~cm}$ with $20 \\%$ amplitude of the initial plasma column radius to smooth the initial perturbation for the sausage instability. The rotation frequency $\\Omega$ is $\\Omega \\sim 3.00 \\times 10^{8} / \\mathrm{s}$. Therefore, in this example $\\Omega \\gg \\gamma$. Figure $9.20 \\mathrm{a}$ shows the rotation profile, and the stationary ion column has no perturbation initially. Compared with Figs. $9.19 \\mathrm{~d}$ and 9.20c demonstrates that the sausage instability is well mitigated by the rotation behavior imposed.

The field energy, which is proportional to $\\varepsilon_{0} \\mathbf{E}^{2}+\\mu_{0} \\mathbf{B}^{2}$, was measured for the sausage instability presented in Figs.9.19 and 9.20. As explained above, the rotating electron driver beam smooths and mitigates the sausage instability growth clearly. The growth mitigation is significant, and the growth mitigation ratio is $58.3 \\%$ at $t=40 \\mathrm{~ns}$. In the growth mitigation mechanism for plasma instabilities, as mentioned earlier, the growth rate does not change, but the perturbation amplitude is reduced significantly. Figures 9.19 and 9.20 demonstrate that the amplitude reduction of the sausage instability is remarkable in the dynamic phase control [125].

In addition to the sausage instability, the kink instability is also simulated by the EPOCH 3D simulations. Figure 9.21 shows the kink instability growth without the mitigation mechanism, and initially the plasma electron column is displaced in $y$ by $5 \\%$ of the column radius. In $20 \\mathrm{~cm}, 2$ waves are accommodated in $z$. The growth rate $\\gamma$ of the kink instability is about $\\gamma \\sim 1.17 \\times 10^{7} / \\mathrm{s}$.

The electron distributions are presented in Fig. 9.21 (a) at $t=0 \\mathrm{~ns}$ and (b) $t=50$ ns, and the protons are shown in Fig. 9.21 (c) at $t=0 \\mathrm{~ns}$ and (d) $t=50 \\mathrm{~ns}$. Figures $9.21 \\mathrm{~b}, \\mathrm{~d}$ present the kink instability growth clearly. Figures $9.21 \\mathrm{~b}$, d correspond to Fig. 7.7a, b in Sect. 7.3.1, respectively.

On the other hand, Fig. 9.22 presents the kink instability growth under the phase control mitigation of 4 rotations in $20 \\mathrm{~cm}$ in $z$ with the amplitude of $20 \\%$ of the column radius. The rotation frequency $\\Omega$ is $\\Omega \\sim 1.20 \\times 10^{9} / \\mathrm{s}$. Therefore, in this

\\section*{w/o mitigation}
\\begin{center}
\\includegraphics[max width=\\textwidth]{2024_02_26_83e36543483eb7d284c1g-236}
\\end{center}

\\section*{(b) $t=30$ ns electron density $\\left(/ \\mathrm{m}^{3}\\right)$}
\\begin{center}
\\includegraphics[max width=\\textwidth]{2024_02_26_83e36543483eb7d284c1g-236(1)}
\\end{center}

Fig. 9.19 Sausage instability without the mitigation mechanism. The 3D Particle-in-Cell simulations are performed by EPOCH3D [135, 136]. The electron densities at $\\mathbf{a} t=20 \\mathrm{~ns}$ and $\\mathbf{b} t=30 \\mathrm{~ns}$ and the proton densities at $\\mathbf{c} t=20 \\mathrm{~ns}$ and $\\mathbf{d} t=40 \\mathrm{~ns}$. Source Ref. [125]. (Reprinted figures with permission from Kawata, S., Karino, T. and Gu, Y. J.: Phase control of a z-current-driven plasma column, Phys. Rev. E, 101 (2020) 041201. \\href{https://link.aps.org/doi/10.1103/PhysRevE.101.041201}{https://link.aps.org/doi/10.1103/PhysRevE.101.041201} Copyright 2020 by the American Physical Society)
with mitigation

\\begin{center}
\\includegraphics[max width=\\textwidth]{2024_02_26_83e36543483eb7d284c1g-237(1)}
\\end{center}

(b) $t=36$ ns electron $\\operatorname{density}\\left(/ \\mathrm{m}^{3}\\right)$

\\begin{center}
\\includegraphics[max width=\\textwidth]{2024_02_26_83e36543483eb7d284c1g-237(2)}
\\end{center}

\\section*{(c) $t=40$ ns proton $\\operatorname{density}\\left(/ \\mathrm{m}^{3}\\right)$}
\\begin{center}
\\includegraphics[max width=\\textwidth]{2024_02_26_83e36543483eb7d284c1g-237}
\\end{center}

Fig. 9.20 Sausage instability with the mitigation mechanism. The electron densities at $\\mathbf{a} t=0 \\mathrm{~ns}$ and $\\mathbf{b} t=36 \\mathrm{~ns}$ and the proton density at $\\mathbf{c} t=40 \\mathrm{~ns}$. The rotating motion around the column axis mitigates the sausage instability growth successfully. Source Ref. [125]. (Reprinted figures with permission from Kawata, S., Karino, T. and Gu, Y. J.: Phase control of a $z$-current-driven plasma column, Phys. Rev. E, 101 (2020) 041201. \\href{https://link.aps.org/doi/10.1103/PhysRevE.101.041201}{https://link.aps.org/doi/10.1103/PhysRevE.101.041201} Copyright 2020 by the American Physical Society)

example $\\Omega \\gg \\gamma$. The electrons are presented in Fig. 9.22 (a) at $t=8 \\mathrm{~ns}$ and (b) at $t$ $=50 \\mathrm{~ns}$, and the protons are shown in Fig. 9.22 (c) at $t=14 \\mathrm{~ns}$ and (d) $t=50 \\mathrm{~ns}$. The electron and proton spatial distributions reflect the electron beam rotation around the column axis in Fig. 9.22. The kink instability shown in Fig. 9.21 is well mitigated by the rotation (wobbling or oscillating) behavior as presented in Fig. 9.22.

For the kink instability simulations, the field energy is again measured, and the mitigation ratio was $30.0 \\%$ at $t=50 \\mathrm{~ns}$.

The dynamic mitigation mechanism works well to mitigate the sausage and kink instabilities of the z-current-driven plasma column. The z-current-driven plasma col-
\\includegraphics[max width=\\textwidth, center]{2024_02_26_83e36543483eb7d284c1g-238(1)}

(d) $t=50$ ns proton $\\operatorname{density}\\left(/ \\mathrm{m}^{3}\\right)$

\\begin{center}
\\includegraphics[max width=\\textwidth]{2024_02_26_83e36543483eb7d284c1g-238}
\\end{center}

Fig. 9.21 Kink instability without the mitigation mechanism. The 3D Particle-in-Cell simulations are performed by EPOCH3D [135, 136]. The electron densities at $\\mathbf{a} t=0 \\mathrm{~ns}$ and $\\mathbf{b} t=50 \\mathrm{~ns}$ and the proton densities at $\\mathbf{c} t=0 \\mathrm{~ns}$ and $\\mathbf{d} t=50 \\mathrm{~ns}$. Source Ref. [125]. (Reprinted figures with permission from Kawata, S., Karino, T. and Gu, Y. J.: Phase control of a $z$-current-driven plasma column, Phys. Rev. E, 101 (2020) 041201. \\href{https://link.aps.org/doi/10.1103/PhysRevE}{https://link.aps.org/doi/10.1103/PhysRevE}. 101.041201 Copyright 2020 by the American Physical Society)
\\includegraphics[max width=\\textwidth, center]{2024_02_26_83e36543483eb7d284c1g-239(2)}

(c) $t=14 \\mathrm{~ns}$ proton $\\operatorname{density}\\left(/ \\mathrm{m}^{3}\\right)$

\\begin{center}
\\includegraphics[max width=\\textwidth]{2024_02_26_83e36543483eb7d284c1g-239}
\\end{center}

(d) $t=50$ ns proton $\\operatorname{density}\\left(/ \\mathrm{m}^{3}\\right)$

\\begin{center}
\\includegraphics[max width=\\textwidth]{2024_02_26_83e36543483eb7d284c1g-239(1)}
\\end{center}

Fig. 9.22 Kink instability with the mitigation mechanism. The electron densities at $\\mathbf{a} t=8 \\mathrm{~ns}$ and $\\mathbf{b} t=50 \\mathrm{~ns}$ and the proton densities at $\\mathbf{c} t=14 \\mathrm{~ns}$ and at $\\mathbf{d} t=50 \\mathrm{~ns}$. The oscillating or wobbling motion mitigates the kink instability growth successfully by the dynamic phase control mechanism. Source Ref. [125]. (Reprinted figures with permission from Kawata, S., Karino, T. and Gu, Y. J.: Phase control of a $z$-current-driven plasma column, Phys. Rev. E, 101 (2020) 041201. \\href{https://link}{https://link}. \\href{http://aps.org/doi/10.1103/PhysRevE}{aps.org/doi/10.1103/PhysRevE}. 101.041201 Copyright 2020 by the American Physical Society)
umn may have multiple modes. Even in this case, the dynamic mitigation mechanism, shown in Sect. 9.7.1, works well to reduce the instability growth as far as $\\Omega \\geq \\gamma$ in the $z$-current-driven plasma column. The axis rotation or wobbling of particle beams is widely realized by accelerators [58-60], and the electron beam wobbling motion would be available in accelerators for near-future experiments to mitigate the instability growth of the plasma column.

\\section*{Dynamic Control of Filamentation Instability}
Here the dynamic control of the filamentation instability growth is presented. In Sect. 7.8, 2D EPOCH simulation results were presented. The input electron beam is injected into a plasma, and the electron beam has a current modulation in the transverse direction. The electron beam current modulation defines actively the filamentation phase. After a short time of $\\Delta t$, the filamentation instability grows. Then the electron beam oscillates in the transverse direction, and the electron beam modulation also moves in the transverse direction. The new perturbation with the shifted phase is applied, and the perturbations grow. The overall instability growth should be defined by the sum of all the perturbations at $t$, and the filamentation instability is dynamically stabilized as shown in Fig.9.17.

We use the following parameter values: $\\alpha=n_{b e} / n_{p}=1 / 9, \\beta=v_{b e} / c=0.9$, $v_{p e} / c=-0.1, n_{b e}$ is the electron beam number density, and $n_{p e}$ the number density of the background plasma electrons. The temperatures of the beam electrons, the background electrons and the background ions are $100 \\mathrm{eV}$. In the simulations shown here, $n_{p}=1.0 \\times 10^{-3} \\times 4 \\pi^{2} \\epsilon_{0} m_{e} c^{2} /(\\lambda e)^{2}$, the time is normalized by $1 / \\omega_{p e}$ and the scale length is normalized by $\\lambda$. In Sect.9.7.2, 3D EPOCH simulations are performed. Figure 9.23 shows the filamentation instability growth at (a) $t=0$ and (b) $t=32 \\omega_{p e}$ without the electron beam wobbling motion. Then we add the wobbling motion on the electron beam with the frequency of $\\Omega=2 \\omega_{p e}$. The growth rate of the filamentation instability may be expressed by $\\gamma_{F} \\sim \\beta \\sqrt{\\alpha / \\gamma_{b}} \\omega_{p e}$, where $\\gamma_{b}=1 / \\sqrt{1-\\beta^{2}}$. Therefore, the dynamic mitigation mechanism would work well to mitigate the filamentation instability growth.

Figure 9.24 presents a clear mitigation of the filamentation instability growth. In Figs. 9.23 and 9.24, $j_{x}, B_{z}$ and $n_{e}$ are shown. The magnetic field energy growth was also measured for both the cases in 3D. The mitigation ratio was about $58.6 \\%$ at $t=35 / \\omega_{p e}[124]$.

\\section*{Dynamic Control of Tearing Mode Instability}
In Sect. 7.9, the collisionless tearing mode instability was introduced and simulated by the 3D EPOCH simulations. A current $\\left(J_{x}\\right)$ sheet creates an antiparallel magnetic field in plasmas. Figure 7.16 in Sect. 7.9 showed the tearing mode instability and the magnetic reconnection (see, for example, Refs. [126, 138-140], Chap. 7 in Ref. [141]
\\includegraphics[max width=\\textwidth, center]{2024_02_26_83e36543483eb7d284c1g-241}

Fig. 9.23 Filamentation instability without the mitigation mechanism. The 3D PIC simulation results for the filamentation instability growth at $\\mathbf{a} t=0$ and $\\mathbf{b} t=32 / \\omega_{p e}$ without the electron beam wobbling motion. Source [124]

and Refs. [142-154] ). In Sect. 9.7.2, the wobbling motion is added to the electron sheet current along the sheet as shown in Fig. 9.25 [126]. The parameter values and setup for the 3D EPOCH simulation are the same with those in Sect. 7.9, except the wobbling motion with the wobbling frequency $\\Omega=300 \\mathrm{MHz}$. The growth rate for the tearing mode instability was about $\\gamma \\sim 2.18 \\times 10^{6} / \\mathrm{s}$. Therefore, $\\Omega \\gg \\gamma$. The electron sheet current oscillates along the sheet. It is expected that the perturbation phase is defined by the wobbling electron beam, and the wobbling motion would mitigate the magnetic reconnection.

Figure 9.26a shows the electron number density $n_{e}$ at $t=0.7 \\mu$ s, and Fig. 9.26b shows the proton density $n_{i}$ at $t=0.7 \\mu$ s. Figure 9.26a also presents the oscillation
\\includegraphics[max width=\\textwidth, center]{2024_02_26_83e36543483eb7d284c1g-242}

Fig. 9.24 Filamentation instability with the mitigation mechanism. The 3D PIC simulation results for the filamentation instability growth at $\\mathbf{a} t=0$ and $\\mathbf{b} t=32 / \\omega_{p e}$ with the electron beam wobbling motion. The wobbling frequency is $\\Omega=2 \\omega_{\\text {pe }}$. Source [124]

or wobbling motion of the electron sheet current. The filamentation does not grow much compared with the results in Fig. 7.17 [126].

Figure 9.27a shows the magnetic field strength at $x=0$, and the magnetic field reconnection is not remarkable. Comparing with the results in Fig. 7.18b, the dynamic mitigation or onset delay of the magnetic reconnection is clearly shown. In order to compare the filamentation and magnetic reconnection in the electron sheet current schematically shown in Figs. 7.16 and 9.25, the histories of the normalized field energy of $B_{z}^{2}$ are presented in Fig. 9.27b. The electron current is filamented along with the magnetic reconnection. Associated with the electron current filamentation and the magnetic reconnection, $B_{z}$ is induced. Figure $9.27 \\mathrm{~b}$ presents the clear difference in the field energy between the two cases and shows the clear onset delay of the magnetic

\\begin{center}
\\includegraphics[max width=\\textwidth]{2024_02_26_83e36543483eb7d284c1g-243}
\\end{center}

Fig. 9.25 Wobbling sheet electron current. Schematic diagram for dynamic phase control in electron current sheet sustained plasma system, in which electron filamentation and magnetic reconnection grow. The electron sheet current oscillates along the sheet, and the perturbation phases introduced by the wobbling electron beam smooth and mitigate the perturbation amplitude. Source Ref. [126]

reconnection and filamentation. The theoretical considerations and 3D numerical computations show the clear effectiveness and viability of the dynamic phase control method to mitigate the plasma instability and the magnetic reconnection.

The sheet current plasma system can be found in magnetic fusion devices, space, terrestrial magnetic system, etc. The dynamic mitigation mechanism may contribute to mitigate the magnetic fusion plasma disruptive behavior or to understand the stable structure of the sheet current sustained plasma system.

\\section*{Dynamic Control of the Kelvin-Helmholtz Instability}
In Sect. 7.5 and in Fig. 7.11, the Kelvin-Helmholtz instability was introduced and studied (see Sect. 11 in Ref. [155]). Figure 9.28 shows again the Kelvin-Helmholtz instability in 3D: (a) the density $\\rho$ at the normalized time $t=30,000$, (b) $\\rho$ at $t=180,000$ and (c) the speed $|\\mathbf{v}|(\\mathrm{m} / \\mathrm{s})$ at $t=180,000$. A fluid with the initial density of $\\rho=1.2$ is injected into another fluid of $\\rho=1.0$ from the left boundary with the speed of $50 \\mathrm{~m} / \\mathrm{s}$. The fluid injection port radius is $20 \\mathrm{~cm}$ at the left boundary of $x=0$. The fluid simulation results may show the Kelvin-Helmholtz instability clearly. The injected fluid is not transported smoothly in the distance of $3.5 \\mathrm{~m}$.

Fig. 9.26 Tearing mode instability mitigated by wobbling sheet electron current. Figure (a) shows the electron number density $n_{e}$ at $\\mathbf{a} t=0.7 \\mu \\mathrm{s}$, and Fig. (b) shows the proton density $n_{i}$ at $\\mathbf{b} t=0.7 \\mu \\mathrm{s}$. Figure (a) presents the oscillation or wobbling motion of the electron sheet current. The filamentation does not grow much compared with the results in Fig. 7.17. Source Ref. [126]

\\begin{center}
\\includegraphics[max width=\\textwidth]{2024_02_26_83e36543483eb7d284c1g-244}
\\end{center}

Figure 9.29 presents the Kelvin-Helmholtz instability in 3D with the wobbling motion at the normalized time $t=180,000$. Figure 9.29 (a) shows the density $\\rho$ and (b) shows the speed $|\\mathbf{v}|(\\mathrm{m} / \\mathrm{s})$ with the wobbling frequency of $\\Omega=10,000 / \\mathrm{s}$. Figure 9.29 (c) shows the density $\\rho$ and (d) shows the speed $|\\mathbf{v}|(\\mathrm{m} / \\mathrm{s})$ with the wobbling frequency of $\\Omega=1000 / \\mathrm{s}$. The fluid, injected from the injection port at the left boundary, rotates with the frequency of $\\Omega$ with the rotation radius of $5(\\mathrm{~cm})$ at $x=0$. By Eq. (7.79), the growth rate $\\gamma$ estimated may be $\\gamma \\sim 782 / \\mathrm{s}$, when the perturbation wave number $k_{x}$ is estimated by the using the radius $0.2 \\mathrm{~m}$ of the injected fluid and so by $k_{x} \\sim 2 \\pi / 0.2 / \\mathrm{m}$. The fluid simulation results may show that the Kelving-Helmholtz instability is mitigated by the wobbling motion. The injected fluid is transported better in the distance of $3.5 \\mathrm{~m}$, compared with the results shown in Figs. 9.28 and 7.11, in which no wobblling motion is added.

Fig. 9.27 Tearing mode instability mitigated by the wobbling sheet electron current. Figure (a) shows the distribution of magnetic field strength at a $t=0.7 \\mu \\mathrm{s}$, and Fig. (b) shows the histories of normalized field energy of $B_{z}^{2}$ for the sheet electron current plasma systems with (solid line) and without (dotted line) the wobbling or oscillating motion of the sheet electron current along $y$. Figures present that the filamentation does not grow much compared with the results in Figs. 7.17 and 7.18b. Source [126]

$$
\\text { a) }|\\vec{B}| \\text { at } t=0.7 \\mu s \\text { at } x=0 \\text {; with wobbling }
$$

\\begin{center}
\\includegraphics[max width=\\textwidth]{2024_02_26_83e36543483eb7d284c1g-245}
\\end{center}

b) Normalized $B_{z}^{2}$

\\begin{center}
\\includegraphics[max width=\\textwidth]{2024_02_26_83e36543483eb7d284c1g-245(1)}
\\end{center}

\\section*{Dynamic Control of Two-Stream Instability}
In Sect.9.7.2, two identical counter-streaming cold electron components employed to simulate the two-stream instability. The immobile background cold ions neutralize the electron charge. The number density $n_{e}$ of each electron beam is $10^{4} / \\mathrm{m}^{3}$, and each streaming speed $v_{d e}$ in the $x$ direction is $\\pm 5.77 \\times 10^{-3} c$. Here $c$ shows the speed of light. The electron plasma frequency $\\omega_{p e}$ is about $5.64 \\times 10^{3} / \\mathrm{s}$. The maximal growth rate $\\gamma_{\\max }$ of the two-stream instability is $\\sim 1.41 \\times 10^{3} / \\mathrm{s}$ at the wavelength of $\\lambda_{\\max } \\sim 22.3 \\mathrm{~km}$. We employ again EPOCH3D [135, 136] to simulate the electron two-stream instability. The periodic boundary conditions are employed in all the directions in the EPOCH simulations.

Figure 9.30 presents the two-stream instability growth. The two electron beams of $n_{e+}$ and $n_{e-}$ counter with each other in the $x$ direction in this example case. The subscripts of $\\pm$ at $n_{e}$ represent the identical counter-streaming electron beams moving in $+x$ and $-x$. Figure 9.30a shows the initial electron number density $n_{e+}$ of the electron beam moving in $+x$, Fig. $9.30 \\mathrm{~b}$ presents the electron number density
\\includegraphics[max width=\\textwidth, center]{2024_02_26_83e36543483eb7d284c1g-246(2)}
b) $\\rho$ at $t=180000$

\\begin{center}
\\includegraphics[max width=\\textwidth]{2024_02_26_83e36543483eb7d284c1g-246(1)}
\\end{center}

c) $|v|$ at $t=180000$

\\begin{center}
\\includegraphics[max width=\\textwidth]{2024_02_26_83e36543483eb7d284c1g-246}
\\end{center}

Fig. 9.28 The Kelvin-Helmholtz instability in 3D: a the density $\\rho$ at the normalized time $t=$ $30,000, \\mathbf{b} \\rho$ at $t=180,000$ and $\\mathbf{c}$ the speed $|\\mathbf{v}|(\\mathrm{m} / \\mathrm{s})$ at $t=180,000$. A fluid with the initial density of $\\rho=1.2$ is injected into another fluid of $\\rho=1.0$ from the left boundary with the speed of $50 \\mathrm{~m} / \\mathrm{s}$. The fluid injection port radius is $20 \\mathrm{~cm}$ at the left boundary of $x=0$. The fluid simulation results may show the Kelvin-Helmholtz instability clearly. The injected fluid is not transported smoothly in the distance of $3.5 \\mathrm{~m}$

$n_{e+}$ at $t=1.4 \\mathrm{~ms}$, and Fig. 9.30c shows the phase space map in the momentum $P_{x}$ versus $x$ space at $15 \\mathrm{~km} \\leq z \\leq 16 \\mathrm{~km}$. One of the two electron beams moving in the $+x$ direction has the initial perturbation $\\delta n_{e+}$ in $n_{e+}$ with the amplitude of $5 \\%$ as shown in Fig. 9.30a. In the $x$ direction two modes are imposed, and one mode is introduced in $y$ and $z$ as the initial perturbation.

Figure 9.31 shows the two-stream instability by two counter-streaming electron beams with the wobbling motion in $n_{e+}$. In this case, initially one of the two electron beams moving in $+x$ has the density perturbation in $n_{e+}$ with the wobbling motion. The wobbling motion in this case is introduced in the perturbation $\\delta n_{e+}$ of the amplitude of 5\\% additionally with the sinusoidal oscillation in $y$ and $z$. The spatial amplitude of the wobbling oscillation is $10 \\mathrm{~km}$, that is, the half wavelength of the $\\delta n_{e+}$ perturbation in both $y$ and $z$, and the oscillation frequency is defined as that 3 wavelengths are accommodated in the one wavelength $(10 \\mathrm{~km})$ of the $\\delta n_{e+}$ perturbation. Figure 9.31a shows that the electron number density of $n_{e+}$ is shown at $t=1.8 \\mathrm{~ms}$, and Fig. $9.31 \\mathrm{~b}$ presents the phase space map in $P_{x}$ versus $x$ at $t=1.8 \\mathrm{~ms}$ at $15 \\mathrm{~km} \\leq z \\leq 16 \\mathrm{~km}$. The two-stream instability growth is not significant compared with the results in Fig. 9.30b, c.

Figure 9.32 shows the time sequence of the electric field energy $\\left(\\propto E_{x}^{2}\\right)$ in the $x$ direction for the two-stream instabilities with and without the wobbling motion of $n_{e+}$. At $t=1.8 \\mathrm{~ms}$ the electric field energy is reduced by $95.4 \\%$ in this specific case.
\\includegraphics[max width=\\textwidth, center]{2024_02_26_83e36543483eb7d284c1g-247}

Fig. 9.29 The Kelvin-Helmoholtz instability in 3D with the wobbling motion at the normalized time $t=180,000$. Figure 9.29 a shows the density $\\rho$ and $\\mathbf{b}$ shows the speed $|\\mathbf{v}|(\\mathrm{m} / \\mathrm{s})$ with the wobbling frequency of $\\Omega=10,000 /$ s. Figure 9.29 c shows the density $\\rho$ and $\\mathbf{d}$ shows the speed $|\\mathbf{v}|(\\mathrm{m} / \\mathrm{s})$ with the wobbling frequency of $\\Omega=1000 / \\mathrm{s}$. The fluid, injected from the injection port at the left boundary, rotates with the frequency of $\\Omega$ with the rotation radius of $5(\\mathrm{~cm})$ at $x=0$. By Eq. (7.79), the growth rate $\\gamma$ estimated is $\\gamma \\sim 782 /$ s. The fluid simulation results may show that the Kelvin-Helmholtz instability is mitigated by the wobbling motion. The injected fluid is transported better in the distance of $3.5 \\mathrm{~m}$, compared with the results shown in Figs. 9.28 and 7.11, in which no wobblling motion is added

The two-stream instability onset is significantly delayed by the wobbling motion of the driver electron beam. Figures 9.31 and 9.32 demonstrate that the wobbling behavior of the perturbed electron beam mitigates the two-stream instability growth successfully [156].

\\section*{Dynamic Mitigation of Fuel Target Implosion in Heavy Ion Inertial Fusion}
As shown in Sects. 9.4.4 and 9.4.4, DT fusion fuel spherical target should be compressed to $\\sim 1000$ times the solid density to reduce the input driver energy and to release sufficient fusion energy output in ICF. The implosion non-uniformity of spherical fuel target comes from driver illumination non-uniformity, imperfect sphericity of fuel target, target injection alignment error into a fusion reactor, etc. [46, 47, 71-73, 75-78]. In order to compress the DT fuel to the high density, the driver energy illumination should be uniform on the spherical DT fuel target. The driver may illuminate the target directly or may be converted to radiation to improve the illumination uniformity. The driver energy is converted first to the DT fuel imploding kinetic energy, and then the kinetic energy is converted to heat the DT fuel to
\\includegraphics[max width=\\textwidth, center]{2024_02_26_83e36543483eb7d284c1g-248}

Fig. 9.30 Two-stream instability by two counter-streaming electron beams. The two identical electron beams counter each other. Initially a one of the two electron beams moving in $+x$ has the density perturbation in longitudinal and transverse. Two modes in $x$ and one mode in $y$ and $z$ are accommodated in this example case. b At $t=1.4 \\mathrm{~ms}$, the two-stream instability grows. The phase space map is also shown in $\\mathbf{c}$ in $P_{x}$ versus $x$. Source Ref. [156]

compress the DT fuel at the fuel target center. The spherical fuel target illumination non-uniformity must be less than a few $\\%$ to compress the DT fuel [46, 47].

On the other hand, in heavy ion inertial fusion (HIF), heavy ion beams (HIBs) are generated by particle accelerators, which has a capability to control the HIBs parameter values precisely [24, 54, 58-61]. For example, particle accelerators would give a wobbling oscillation to HIBs, which illuminate on the fusion DT fuel target. One can use the high controllability of the particle accelerators to mitigate the driver HIBs illumination non-uniformity, and then to mitigate the DT fuel implosion nonuniformity [24, 44, 61, 122, 157].

One may use the wobbling heavy ion beam (HIB) as shown in Fig. 9.33. Initially the HIB center stays at the optimal position to reduce the initial imprint of the illumination non-uniformity. After around one rotation, each HIB axis tends to a circle rotation. The HIB wobbling behavior contributes to smooth the illumination nonuniformity and to mitigate the implosion non-uniformity. The dynamic mitigation mechanism would be realized by the HIBs wobbling motion.

As shown in Sect. 9.7.1, the phase of the HIBs illumination non-uniformity on the DT fuel target would be controlled by the wobbling motion in Fig. 9.33. For the HIB DT fuel target implosion, 2D fluid simulations are performed [158], and the detailed 32 HIBs illumination [159-162] is also computed together with the fluid implosion simulations.

Fig. 9.31 Two-stream instability by two counter-streaming electron beams with the wobbling motion in $n_{e+}$. In this case, initially one of the two electron beams moving in $+x$ has the density perturbation in $n_{e+}$ with the wobbling motion. a The electron number density of $n_{e+}$ is shown at $t=1.8 \\mathrm{~ms}$. The phase space map is also shown in $\\mathbf{b}$ in $P_{x}$ versus $x$ at $t=1.8 \\mathrm{~ms}$ at $15 \\mathrm{~km} \\leq z \\leq$ $16 \\mathrm{~km}$. The two-stream instability growth is not significant compared with the results in Figs. 9.30b, c. Source [156]
\\includegraphics[max width=\\textwidth, center]{2024_02_26_83e36543483eb7d284c1g-249}

Figure 9.34a shows the ion temperature distribution at $t=29 \\mathrm{~ns}$ of the DT fuel target in 2D, when the HIBs' wobbling motion is employed. Compared with the result in Fig. 9.34b without the HIBs wobbling, it is apparent that the spatial nonuniformity is smoothed by the wobbling HIBs. For both the cases in Fig. 9.34a, b the difference is just the HIBs wobbling motion in the computation conditions. In this case the HIBs wobbling frequency is $424 \\mathrm{MHz}$ in Fig. 9.34a.

Relating to Eq. (9.27) and Sect.9.4.4 on the dynamic mitigation and smoothing mechanism, we also checked the phase of the implosion acceleration non-uniformity. The implosion acceleration distributions are presented in Fig. 9.35. During the HIBs half rotation between 6 turns and 6.5 turns of the wobbling or rotating HIBs, the phase of the implosion acceleration non-uniformity of the DT fuel is clearly reversed. The result means that the phase of the implosion acceleration non-uniformity is controlled externally by the driver HIBs wobbling behavior. From Eq. (9.27) and the relating theory of the dynamic control mechanism in Sect.9.4.4, the HIBs illumination nonuniformity is smoothed, and the R-T instability would be also mitigated. The fusion

Fig. 9.32 Time histories of the longitudinal electric field energy associated with the two-stream instability by two counter-streaming electron beams with and without the wobbling motion of $n_{e+}$. At $t=1.8 \\mathrm{~ms}$ the electric field energy is reduced by $95.4 \\%$ in this specific case. The two-stream instability onset is significantly delayed by the wobbling motion of the driver electron beam. Source Ref. [156]

\\begin{center}
\\includegraphics[max width=\\textwidth]{2024_02_26_83e36543483eb7d284c1g-250}
\\end{center}

Fig. 9.33 Schematic diagram for a spirally rotating $\\mathrm{HIB}$, illuminating on a spherical fuel target. Initially the HIB center stays at the optimal position to reduce the initial imprint of the illumination non-uniformity. After around one rotation, each HIB axis tends to a circle rotation. The HIB wobbling behavior contributes to smooth the illumination non-uniformity and mitigates the implosion non-uniformity. The dynamic mitigation mechanism would be realized by the HIBs wobbling motion. Source Ref. [24]

\\begin{center}
\\includegraphics[max width=\\textwidth]{2024_02_26_83e36543483eb7d284c1g-250(1)}
\\end{center}

\\section*{Ion temperature at $t=29.0 \\mathrm{~ns}$}
\\begin{center}
\\includegraphics[max width=\\textwidth]{2024_02_26_83e36543483eb7d284c1g-251(1)}
\\end{center}

Fig. 9.34 Ion temperature distributions near the target center at $29 \\mathrm{~ns}$ for the cases a with the HIBs wobbling motion and $\\mathbf{b}$ without the wobbling. Sources Ref. [24, 61]

\\begin{center}
\\includegraphics[max width=\\textwidth]{2024_02_26_83e36543483eb7d284c1g-251}
\\end{center}

Fig. 9.35 Implosion acceleration distributions of DT fuel along the azimuthal direction with the HIBs wobbling behavior. The implosion acceleration non-uniformity is clearly controlled by the wobbling motion. During the half rotation of the wobbling HIBs, the phase of the implosion acceleration non-uniformity is reversed. The results demonstrate that the dynamic mitigation mechanism works well to smooth the HIB illumination non-uniformity and to mitigate the implosion nonuniformity. Sources Refs. [24]

output gain $G$ (= Fusion output energy/input driver energy) becomes 81 for this case at the wobbling frequency of $424 \\mathrm{MHz}$, though without the wobbling behavior $G$ was 62.4.

\\section*{References}
\\begin{enumerate}
  \\item Plasma etching. \\href{https://www.sciencedirect.com/topics/materials-science/plasma-etching}{https://www.sciencedirect.com/topics/materials-science/plasma-etching}. Cited 17 Nov 2021

  \\item F.F. Chen, Introduction to Plasma Physics and Controlled Fusion, 3rd ed. (Springer, 2015)

  \\item S. Shinohara, Helicon high-density plasma sources: physics and applications. Adv. Phys. X 3, 1420424 (2018). \\href{https://doi.org/10.1080/23746149.2017.1420424}{https://doi.org/10.1080/23746149.2017.1420424}

  \\item J.D. Jackson, Classical Electrodynamics (Wiley, Hoboken, 1999)

  \\item Ya.B. Zel'dovich, Yu.P. Raizer, Physics of Shock Waves and High-temperature Hydrodynamic Phenomena (Dover, New York, 2002)

  \\item K. Niu, Nuclear Fusion (Cambridge University Press, Cambridge, 2009)

  \\item K. Miyamoto, Plasma Physics and Controlled Nuclear Fusion (Springer, Berlin, Heidelberg, 2013)

  \\item J. Ongena, R. Koch, R. Wolf, H. Zohm, Magnetic-confinement fusion. Nat. Phys. 12, 398-410 (2016). \\href{https://doi.org/10.1038/nphys3745}{https://doi.org/10.1038/nphys3745}

  \\item M. Kikuchi, M. Azumi, Steady-state tokamak research: core physics. Rev. Mod. Phys. 84, 1807-1854 (2012). \\href{https://doi.org/10.1103/RevModPhys.84.1807}{https://doi.org/10.1103/RevModPhys.84.1807}

  \\item M. Kikuchi, M. Azumi, Frontiers in Fusion Research II: Introduction to Modern Tokamak Physics (Springer, Cham, 2015). \\href{https://doi.org/10.1007/978-3-319-18905-5}{https://doi.org/10.1007/978-3-319-18905-5}

  \\item M. Kikuchi, Frontiers in Fusion Research: Physics and Fusion (Springer, London, 2011). \\href{https://doi.org/10.1007/978-1-84996-411-1}{https://doi.org/10.1007/978-1-84996-411-1}

  \\item ITER, \\href{https://www.iter.org/}{https://www.iter.org/}. Cited 26 Nov 2021

  \\item M. Claessens, ITER: The Giant Fusion Reactor: Bringing a Sun to Earth (Springer Nature, Switzerland, 2020). \\href{https://doi.org/10.1007/978-3-030-27581-5}{https://doi.org/10.1007/978-3-030-27581-5}

  \\item T. Tanabe, Plasma-Material Interactions in a Controlled Fusion Reactor (Springer, Singapore, 2021)

  \\item S. Krasheninnikov, A. Smolyakov, A. Kukushkin, On the Edge of Magnetic Fusion Devices (Springer, Cham, 2020)

  \\item A.V. Melnikov, Electric Potential in Toroidal Plasmas (Springer, Cham, 2019)

  \\item S. Atzeni, J. Meyer-Ter-Vehn, The Physics of Inertial Fusion (Oxford Science Pub., 2004)

  \\item J. Nuckolls, L. Wood, A. Thiessen, G. Zimmerman, Laser compression of matter to superhigh densities: thermonuclear (CTR) applications. Nature 239, 139-142 (1972). \\href{https://doi}{https://doi}. org/10.1038/239139a0

  \\item W. Daiber, A. Hertzberg, C.E. Wittliff, Laser-generated implosions. Phys. Fluids 9, 617-619 (1966). \\href{https://doi.org/10.1063/1.1761718}{https://doi.org/10.1063/1.1761718}

  \\item NIF (National Ignition Facility) project, \\href{https://lasers.llnl.gov/about/what-is-nif}{https://lasers.llnl.gov/about/what-is-nif}. Cited 30 Nov 2021

  \\item LMJ (Laser Mégajoule) Project, \\href{http://www-lmj.cea.fr/}{http://www-lmj.cea.fr/}. Cited 30 Nov 2021

  \\item FAIR Project, \\href{https://fair-center.eu/}{https://fair-center.eu/}. Cited 30 Nov 2021

  \\item HIAF Project, \\href{http://hiaf.impcas.ac.cn/hiaf_en/public/c/news.html}{http://hiaf.impcas.ac.cn/hiaf\\_en/public/c/news.html}. Cited 30 Nov 2021

  \\item S. Kawata, Direct-drive heavy ion beam inertial confinement fusion: a review, toward our future energy source. Adv. Phys. X 6, 1873860 (2021). \\href{https://doi.org/10.1080/23746149}{https://doi.org/10.1080/23746149}. 2021.1873860

  \\item Z machine at Sandia National Laboratory, \\href{https://www.sandia.gov/z-machine/}{https://www.sandia.gov/z-machine/}. Cited 30 Nov 2021

  \\item J.D. Lindl, P. Amendt, R.L. Berger, S.G. Glendinning, S.H. Glenzer, S.W. Haan, R.L. Kauffman, O.L. Landen, L.J. Suter, The physics basis for ignition using indirect-drive targets on the national ignition facility. Phys. Plasmas 11, 339-483 (2004). \\href{https://doi.org/10.1063/1}{https://doi.org/10.1063/1}. 1578638

  \\item A.B. Zylstra, A.L. Kritcher, O.A. Hurricane, D.A. Callahan, K. Baker, T. Braun, D.T. Casey, D. Clark, K. Clark, T. Döppner, L. Divol, D.E. Hinkel, M. Hohenberger, C. Kong, O.L. Landen, A. Nikroo, A. Pak, P. Patel, J.E. Ralph, N. Rice, R. Tommasini, M. Schoff, M. Stadermann, D. Strozzi, C. Weber, C. Young, C. Wild, R.P.J. Town, M.J. Edwards, Record energetics for an inertial fusion implosion at NIF. Phys. Rev. Lett. 126, 025001 (2021). \\href{https://doi.org/10}{https://doi.org/10}. 1103/PhysRevLett.126.025001

  \\item A.L. Kritcher, A.B. Zylstra, D.A. Callahan, O.A. Hurricane, C. Weber, J. Ralph, D.T. Casey, A. Pak, K. Baker, B. Bachmann, S. Bhandarkar, J. Biener, R. Bionta, T. Braun, M. Bruhn, C. Choate, D. Clark, M. Di NicolaJ, L. Divol, T. Doeppner, V. Geppert-Kleinrath, S. Haan, J. Heebner, V. Hernandez, D. Hinkel, M. Hohenberger, H. Huang, C. Kong, S. Le Pape, D. Mariscal, E. Marley, L. Masse, K.D. Meaney, M. Millot, A. Moore, K. Newman, A. Nikroo, P. Patel, L. Pelz, N. Rice, H. Robey, J.S. Ross, M. Rubery, J. Salmonson, D. Schlossberg, S. Sepke, K. Sequoia, M. Stadermann, D. Strozzi, R. Tommasini, P. Volegov, C. Wild, S. Yang, C. Young, M.J. Edwards, O. Landen, R. Town, M. Herrmann, Achieving record hot spot energies with the largest HDC implosions on NIF in HYBRID-E. Phys. Plasmas 28, 072706 (2021). \\href{https://doi.org/10.1063/5.0047841}{https://doi.org/10.1063/5.0047841}

  \\item O.A. Hurricane, D.A. Callahan, D.T. Casey, P.M. Celliers, C. Cerjan, E.L. Dewald, T.R. Dittrich, T. Döppner, D.E. Hinkel, L.F. Berzak Hopkins, J.L. Kline, S. Le Pape, T. Ma, A.G. MacPhee, J.L. Milovich, A. Pak, H.-S. Park, P.K. Patel, B.A. Remington, J.D. Salmonson, P.T. Springer, R. Tommasini, Fuel gain exceeding unity in an inertially confined fusion implosion. Nature 506, 343-348 (2014)

  \\item H.-S. Park, O.A. Hurricane, D.A. Callahan, D.T. Casey, E.L. Dewald, T.R. Dittrich, T. Döppner, D.E. Hinkel, L.F. Berzak Hopkins, S. Le Pape, T. Ma, P.K. Patel, B.A. Remington, H.F. Robey, J.D. Salmonson, High-Adiabat high-foot inertial confinement fusion implosion experiments on the national ignition facility. Phys. Rev. Lett. 112, 055001 (2014)

  \\item Abu-Shawareb, et al. (Indirect Drive ICF Collaboration), Lawson criterion for ignition exceeded in an inertial fusion experiment. Phys. Rev. Lett. 129, 075001 (2022). \\href{https://doi}{https://doi}. org/10.1103/PhysRevLett.129.075001

  \\item A.L. Kritcher, A.B. Zylstra, D.A. Callahan, O.A. Hurricane, C.R. Weber, D.S. Clark, C.V. Young, J.E. Ralph, D.T. Casey, A. Pak, O.L. Landen, B. Bachmann, K.L. Baker, L. Berzak Hopkins, S.D. Bhandarkar, J. Biener, R.M. Bionta, N.W. Birge, T. Braun, T.M. Briggs, P.M. Celliers, H. Chen, C. Choate, L. Divol, T. Döppner, D. Fittinghoff, M.J. Edwards, M. Gatu Johnson, N. Gharibyan, S. Haan, K.D. Hahn, E. Hartouni, D.E. Hinkel, D.D. Ho, M. Hohenberger, J.P. Holder, H. Huang, N. Izumi, J. Jeet, O. Jones, S.M. Kerr, S.F. Khan, H. Geppert Kleinrath, V. Geppert Kleinrath, C. Kong, K.M. Lamb, S. Le Pape, N.C. Lemos, J.D. Lindl, B.J. MacGowan, A.J. Mackinnon, A.G. MacPhee, E.V. Marley, K. Meaney, M. Millot, A.S. Moore, K. Newman, J.-M.G. Di Nicola, A. Nikroo, R. Nora, P.K. Patel, N.G. Rice, M.S. Rubery, J. Sater, D.J. Schlossberg, S.M. Sepke, K. Sequoia, S.J. Shin, M. Stadermann, S. Stoupin, D.J. Strozzi, C.A. Thomas, R. Tommasini, C. Trosseille, E.R. Tubman, P.L. Volegov, C. Wild, D.T. Woods, S.T. Yang, Design of an inertial fusion experiment exceeding the Lawson criterion for ignition. Phys. Rev. E 106025201 (2022). \\href{https://doi.org/10.1103/PhysRevE}{https://doi.org/10.1103/PhysRevE}. 106.025201

  \\item M. Zepf, Fusion turns up the heat. Physics 15, 67 (2022). \\href{https://doi.org/10.1103/Physics}{https://doi.org/10.1103/Physics}. 15. 67

  \\item NIF, achieving a yield of more than 1.3 megajoules ,\\href{https://www.llnl.gov/news/nationalignition-facility-experiment-puts-researchers-threshold-fusion-ignition}{https://www.llnl.gov/news/nationalignition-facility-experiment-puts-researchers-threshold-fusion-ignition}. Cited 1 Dec 2021

  \\item NIF, National Ignition Facility achieves fusion ignition, \\href{https://www.llnl.gov/news/nationalignition-facility-achieves-fusion-ignition}{https://www.llnl.gov/news/nationalignition-facility-achieves-fusion-ignition}. Cited 14 Dec 2022

  \\item National Nuclear Security Administration, NNSA Administrator Hruby's Remarks at the NIF Press Conference (2022). \\href{https://www.energy.gov/nnsa/articles/nnsa-administrator-hrubysremarks-nif-press-conference}{https://www.energy.gov/nnsa/articles/nnsa-administrator-hrubysremarks-nif-press-conference}. Cited 14 Dec 2022

  \\item J. Tollefson, Laser-fusion facility heads back to the drawing board. Nature $\\mathbf{6 0 8}, 20-21$ (2022). \\href{https://doi.org/10.1038/d41586-022-02022-1}{https://doi.org/10.1038/d41586-022-02022-1}

  \\item W.M. Sharp, J.J. Barnard, R.H. Cohen, M. Dorf, A. Friedman, D.P. Grote, S.M. Lund, L.J. Perkins, M.R. Terry, F.M. Bieniosek, A. Faltens, E. Henestroza, J.-Y. Jung, A.E. Koniges, J.W. Kwan, E.P. Lee, S.M. Lidia, B.G. Logan, P.N. Ni, L.R. Reginato, P.K. Roy, P.A. Seidl, J.H. Takakuwa, J.-L. Vay, W.L. Waldron, R.C. Davidson, E.P. Gilson, I. Kaganovich, H. Qin, E. Startsev, I. Haber, R.A. Kishek, Inertial fusion driven by intense heavy-ion beams, in Proceedings of 2011 Particle Accelerator Conference, New York, NY, USA, WEOAS1 (2011), pp. 1386-1393

  \\item I. Hofmann, Review of accelerator driven heavy ion nuclear fusion. Matter Radiat. Extremes 3, 1-11 (2018). \\href{https://doi.org/10.1016/j.mre.2017.12.001}{https://doi.org/10.1016/j.mre.2017.12.001}

  \\item S. Kawata, T. Karino, A.I. Ogoyski, Review of heavy-ion inertial fusion physics. Matter Radiat. Extremes 1, 89-113 (2016). \\href{https://doi.org/10.1016/j.mre.2016.03.003}{https://doi.org/10.1016/j.mre.2016.03.003}

  \\item J.F. Ziegler, J.P. Biersack, U. Littmark, In the Stopping and Range of Ions in Matter, vol. 1 (Pergamon, New York, 1985)

  \\item T.A. Mehlhorn, A finite material temperature model for ion energy deposition in ion-driven inertial confinement fusion targets. J. Appl. Phys. 52, 6522-6532 (1981)

  \\item Bragg peak, \\href{https://www.sciencedirect.com/topics/medicine-and-dentistry/bragg-peak}{https://www.sciencedirect.com/topics/medicine-and-dentistry/bragg-peak}. Cited 1 Dec 2021

  \\item S. Kawata, T. Sato, T. Teramoto, E. Bandoh, Y. Masubichi, I. Takahashi, Radiation effect on pellet implosion and Rayleigh-Taylor instability in light-ion beam inertial confinement fusion. Laser Part. Beams 11, 757-768 (1993)

  \\item Y. Zhou, T.T. Clark, D.S. Clark, S.G. Glendinning, M.A. Skinner, C.M. Huntington, O.A. Hurricane, A.M. Dimits, B.A. Remington, Turbulent mixing and transition criteria of flows induced by hydrodynamic instabilities. Phys. Plasmas 26, 080901 (2019)

  \\item M.H. Emery, J.H. Orens, J.H. Gardner, J.P. Boris, Influence of nonuniform laser intensities on ablatively accelerated targets. Phys. Rev. Lett. 48, 253-256 (1982)

  \\item S. Kawata, K. Niu, Effect of nonuniform implosion of target on fusion parameters. J. Phys. Soc. Jpn. 53, 3416-3426 (1984)

  \\item M. Tabak, J. Hammer, M. Glinsky, W. Kruer, S. Wilks, J. Woodworth, E. Campbell, M. Perry, R. Mason, Ignition and high gain with ultrapowerful lasers. Phys. Plasmas 1, 1626-1634 (1994)

  \\item H.-H. Song, W.-M. Wang, Y.-T. Li, Dense polarized positrons from laser-irradiated foil targets in the QED regime. Phys. Rev. Lett. 129, 035001 (2022)

  \\item Z.-M. Sheng, Y. Sentoku, K. Mima, J. Zhang, W. Yu, J. Meyer-ter-Vehn, Angular distributions of fast electrons, ions, and Bremsstrahlung $\\mathrm{x} / \\gamma$-rays in intense laser interaction with solid targets. Phys. Rev. Lett. 85, 5340-5343 (2000)

  \\item A.L. Lei, K.A. Tanaka, R. Kodama, G.R. Kumar, K. Nagai, T. Norimatsu, T. Yabuuchi, K. Mima, Optimum hot electron production with low-density foams for laser fusion by fast ignition. Phys. Rev. Lett. 96, 255006 (2006)

  \\item J. Zhang, W.M. Wang, X.H. Yang, D. Wu, Y.Y. Ma, J.L. Jiao, Z. Zhang, F.Y. Wu, X.H. Yuan, Y.T. Li, J.Q. Zhu, Double-cone ignition scheme for inertial confinement fusion. Phil. Trans. R. Soc. A. 378, 20200015 (2020)

  \\item K. Horioka, Progress in particle-beam-driven inertial fusion research: activities in Japan. Matter Radiat. Extremes 3, 12-27 (2018)

  \\item B. Sharkov, D. Varentsov, Experiments on extreme states of matter towards HIF at FAIR. Nucl. Instr. Meth. A733, 238-241 (2014)

  \\item R.C. Davidson, H. Qin, Physics of Intense Charged Particle Beams in High Energy Accelerators (Imperial College Press, London and World Scientific, Singapore, 2001)

  \\item P.A. Seidl, J.J. Barnard, A. Faltens, A. Friedman, W.L. Waldron, Multiple beam induction accelerators for heavy ion fusion. Nucl. Instr. Meth. A A733, 193-199 (2014)

  \\item K. Takayama, R. Briggs, Induction Accelerators (Springer, Berlin Heidelberg, 2011)

  \\item B.Yu. Sharkov, D.H.H. Hoffmann, A.A. Golubev, Y. Zhao, High energy density physics with intense ion beams. Matter Radiat. Extremes 1, 28-47 (2016)

  \\item H. Qin, R.C. Davidson, B.G. Logan, Centroid and dynamics of high-intensity charged-particle beams in an external focusing lattice and oscillating wobbler. Phys. Rev. Lett. 104, 254801 (2010)

  \\item R.C. Arnold, E. Colton, S. Fenster, M. Foss, G. Magelssen, A. Moretti, Utilization of high energy, small emittance accelerators for ICF target experiments. Nucl. Inst. Meth. 199, 557561 (1982)

  \\item R. Sato, S. Kawata, T. Karino, K. Uchibori, A.I. Ogoyski, Non-uniformity smoothing of direct-driven fuel target implosion by phase control in heavy ion inertial fusion. Sci. Rep. 9, 6659 (2019)

  \\item S.E. Bodner, Rayleigh-Taylor instability and laser-pellet fusion. Phys. Rev. Lett. 33, 761-764 (1974)

  \\item H. Takabe, K. Mima, L. Montierth, R.L. Morse, Self-consistent growth rate of the RayleighTaylor instability in an ablatively accelerating plasma. Phys. Fluids 28, 3676-3682 (1985)

  \\item J. Sasaki, T. Nakamura, Y. Uchida, T. Someya, K. Shimizu et al., Beam non-uniformity smoothing using density valley formed by heavy ion beam deposition in inertial confinement fusion fuel pellet. Jpn. J. Appl. Phys. 40, 968-971 (2001)

  \\item S. Ichimaru, Statistical Plasma Physics, vol. 1. (Basic Principles, CRC Press, Boca Raton, 2004)

  \\item S. Kawata, R. Sonobe, T. Someya, T. Kikuchi, Final beam transport in HIF. Nucl. Inst. Meth. Phys. Res. A 544, 98-103 (2005)

  \\item D. Böhne, I. Hofmann, G. Kessler, G.L. Kulcinski, J. Meyer-ter-Vehn, U. von Möllendorff, G.A. Moses, R.W. Müller, I.N. Sviatoslavsky, D.K. Szem, W. Vogelsang, HIBALL-a conceptual design study of a heavy-ion driven inertial confinement fusion power plant. Nucl. Eng. Des. 73, 195-200 (1982)

  \\item T. Yamaki, et al., HIBLIC-1, Conceptual Design of a Heavy Ion Fusion Reactor. Institute for Plasma Physics, Nagoya University, Report IPPJ-663 (1985)

  \\item R.W. Moir, R.L. Bieri, X.M. Chen, T.J. Dolan, M.A. Hoffman, P.A. House, R.L. Leber, J.D. Lee, Y.D. Lee, J.C. Liu, G.R. Long hurst, W.R. Meier, P.F. Peterson, R.W. Petzoldt, V.E. Schrock, M.T. Tobin, W.H. Williams, HYLIFE-II: a molten salt inertial fusion energy power plant design-final report. Fusion Technol. 25, 5-25 (1994)

  \\item I. Hofmann, G. Plass, HIDIF Study, Report of the European Study Group on Heavy Ion Driven Inertial Fusion, GSI Report (1998) GSI-98-06

  \\item R. Tsuji, Trajectory adjusting system using a magnetic lens for a $\\mathrm{Pb}$-coated super-conducting IFE target. Fusion Eng. Des. 81, 2877-2885 (2006)

  \\item T. Kubo, T. Karino, H. Kato, S. Kawata, Fuel pellet alignment in heavy-ion inertial fusion reactor. IEEE Trans. Plasma Sci. 47, 2-8 (2019)

  \\item H. Nakamura, T. Kubo, T. Karino, H. Kato, S. Kawata, Fuel pellet injection into heavy-ion inertial fusion reactor. High Energ. Density Phys. 35, 1574-1818 (2020)

  \\item R. Tsuji, In-situ mass-to-charge ratio measurement and trajectory control of a vertically injected laser fusion energy charged target via electric field. Plasma Fusion Res. 17, 1404088 (2022)

  \\item D.T. Goodin, C.R. Gibson, R.W. Petzoldt, N.P. Siegel, L. Thompson, A. Nobile, G.E. Besenbruch, K.R. Schultz, Developing the basis for target injection and tracking in inertial fusion energy power plants. Fusion Eng. Des. 60, 27-36 (2002)

  \\item R.W. Petzoldt, M. Cherry, N.B. Alexander, D.T. Goodin, G.E. Besenbruch, K.R. Schultz, General atomics: design of an inertial fusion energy target tracking and position prediction system. Fusion Tech. 39, 678-683 (2001)

  \\item R.W. Petzoldt, D.T. Goodin, A. Nikroo, E. Stephens, N. Siegel, N.B. Alexander, A.R. Raffray, T.K. Mau, M. Tillack, F. Najmabadi, S.I. Krasheninnikov, R. Gallix, Direct drive target survival during injection in an inertial fusion energy power plant. Nucl. Fusion 42, 1351-1356 (2002)

  \\item R.W. Petzoldt, IFE target injection and tracking experiment. Fusion Tech. 34, 831-839 (1998)

  \\item D. Strickland, G. Mourou, Compression of amplified chirped optical pulses. Opt. Commun. 56, 219-221 (1985)

  \\item J.W. Yoon, Y.G. Kim, I.W. Choi, J.H. Sung, H.W. Lee, S.K. Lee, C.H. Nam, Realization of laser intensity over $10^{23} \\mathrm{~W} / \\mathrm{cm}^{2}$. Optica 8, 630-635 (2021)

  \\item T. Tajima, J. Dawson, Laser electron accelerator. Phys. Rev. Lett. 43, 267-270 (1979)

  \\item K. Shimoda, Proposal for an electron accelerator using an optical maser. Appl. Opt. 1, 33-35 (1962)

  \\item G.A. Mourou, T. Tajima, S.V. Bulanov, Optics in the relativistic regime. Rev. Modern Phys. 78, 309-371 (2006)

  \\item E. Esarey, C.B. Schroeder, W.P. Leemans, Physics of laser-driven plasma-based electron accelerators. Rev. Modern Phys. 81, 1229-1285 (2009)

  \\item H. Takabe, The Physics of Laser Plasmas and Applications-Volume 1. Springer Series in Plasma Science and Technology (2020). \\href{https://doi.org/10.1007/978-3-030-49613-5}{https://doi.org/10.1007/978-3-030-49613-5}

  \\item P.M. Woodward, A method of calculating the field over a plane aperture required to produce a given polar diagram. J. Inst. Electr. Eng. 93, 1554-1558 (1947)

  \\item J. D. Lawson, IEEE Trans. Nucl. Sci, NS-26, 4217-4219 (1979)

  \\item R.B. Palmer, A LASER-DRIVEN GRATING LINAC. Part. Accel. 11, 81-90 (1980)

  \\item C. Joshi, W.B. Mori, T. Katsouleas, J.M. Dawson, J.M. Kindel, D.W. Forslund, Ultrahigh gradient particle acceleration by intense laser-driven plasma density waves. Nature 311, 525$529(1984)$

  \\item W. Leemans, B. Nagler, A. Gonsalves, Cs. Tóth, K. Nakamura, C.G.R. Geddes, E. Esarey, C.B. Schroeder, S.M. Hooker, GeV electron beams from a centimetre-scale accelerator. Nat. Phys. 2, 696-699 (2006). \\href{https://doi.org/10.1038/nphys418}{https://doi.org/10.1038/nphys418}

  \\item I. Blumenfeld, C. Clayton, F.J. Decker, M.J. Hogan, C. Huang, R. Ischebeck, R. Iverson, C. Joshi, T. Katsouleas, N. Kirby, W. Lu, K.A. Marsh, W.B. Mori, P. Muggli, E. Oz, R.H. Siemann, D. Walz, M. Zhou, Energy doubling of $42 \\mathrm{GeV}$ electrons in a metre-scale plasma wakefield accelerator. Nature 445, 741-744 (2007). \\href{https://doi.org/10.1038/nature05538}{https://doi.org/10.1038/nature05538}

  \\item X. Wang, R. Zgadzaj, N. Fazel, Z. Li, S.A. Yi, X. Zhang, W. Henderson, Y.-Y. Chang, R. Korzekwa, H.-E. Tsai, C.-H. Pai, H. Quevedo, G. Dyer, E. Gaul, M. Martinez, A.C. Bernstein, T. Borger, M. Spinks, M. Donovan, V. Khudik, G. Shvets, T. Ditmire, M.C. Downer, Quasimonoenergetic laser-plasma acceleration of electrons to $2 \\mathrm{GeV}$. Nat. Commun. 4, 1988 (2013). \\href{https://doi.org/10.1038/ncomms2988}{https://doi.org/10.1038/ncomms2988}

  \\item W.P. Leemans, A.J. Gonsalves, H.-S. Mao, K. Nakamura, C. Benedetti, C.B. Schroeder, Cs. Tóth, J. Daniels, D.E. Mittelberger, S.S. Bulanov, J.-L. Vay, C.G.R. Geddes, E. Esarey, Multi-GeV electron beams from capillary-discharge-guided subpetawatt laser pulses in the self-trapping regime. Phys. Rev. Lett. 113, 245002 (2014)

  \\item A.R. Maier, N.M. Delbos, T. Eichner, L. Hübner, S. Jalas, L. Jeppe, S.W. Jolly, M. Kirchen, V. Leroux, P. Messner, M. Schnepp, M. Trunk, P.A. Walker, C. Werle, P. Winkler, Decoding sources of energy variability in a laser-plasma accelerator. Phys. Rev. X 10, 031039 (2020)

  \\item S.C. Wilks, A.B. Langdon, T.E. Cowan, M. Roth, M. Singh, S. Hatchett, M.H. Key, D. Pennington, A. MacKinnon, R.A. Snavely, Energetic proton generation in ultra-intense lasersolid interactions. Phys. Plasmas 8, 542-549 (2001)

  \\item E.L. Clark, K. Krushelnick, J.R. Davies, M. Zepf, M. Tatarakis, F.N. Beg, A. Machacek, P.A. Norreys, M.I.K. Santala, I. Watts, A.E. Dangor, Measurements of energetic proton transport through magnetized plasma from intense laser interactions with solids. Phys. Rev. Lett. 84, 670-673 (2000)

  \\item K. Krushelnick, E.L. Clark, M. Zepf, J.R. Davies, F.N. Beg, A. Machacek, M.I.K. Santala, M. Tatarakis, I. Watts, P.A. Norreys, A.E. Dangor, Energetic proton production from relativistic laser interaction with high density plasmas. Phys. Plasmas 7, 2055-2061 (2000)

  \\item A. Maksimchuk, S. Gu, K. Flippo, D. Umstadter, V.Yu. Bychenkov, Forward ion acceleration in thin films driven by a high-intensity laser. Phys. Rev. Lett. 84, 4108-4111 (2000)

  \\item R.A. Snavely, M.H. Key, S.P. Hatchett, T.E. Cowan, M. Roth, T.W. Phillips, M.A. Stoyer, E.A. Henry, T.C. Sangster, M.S. Singh, S.C. Wilks, A. MacKinnon, A. Offenberger, D.M. Pennington, K. Yasuike, A.B. Langdon, B.F. Lasinski, J. Johnson, M.D. Perry, E.M. Campbell, Intense high-energy proton beams from petawatt-laser irradiation of solids. Phys. Rev. Lett. 85, 2945-2948 (2000)

  \\item M. Roth, M. Schollmeier, Ion acceleration: TNSA. in Laser-Plasma Interactions and Applications, Scottish Graduate Series ed. by P. McKenna, D. Neely, R. Bingham, D. Jaroszynski. (Springer International Publishing, Heidelberg, 2013), pp. 303-350

  \\item T. Esirkepov, M. Borghesi, S.V. Bulanov, G. Mourou, T. Tajima, Highly efficient relativisticion generation in the laser-piston regime. Phys. Rev. Lett. 92, 175003 (2004)

  \\item K. Sakai, S. Miyazawa, S. Kawata, S. Hasumi, T. Kikuchi, High-energy-density attosecond electron beam production by intense short-pulse laser with a plasma separator. Laser Part. Beams 24, 321-327 (2006)

  \\item T. Nakamura, S. Kawata, Origin of protons accelerated by an intense laser and the dependence of their energy on the plasma density. Phys. Rev. E 67, 026403 (2003)

  \\item Q. Kong, S. Miyazaki, S. Kawata, K. Miyauchi, K. Sakai, Y.K. Ho, K. Nakajima, N. Miyanaga, J. Limpouch, A.A. Andreev, Electron bunch trapping and compression by an intense focused pulse laser. Phys. Rev. E 69, 056502 (2004)

  \\item Y. Nodera, S. Kawata, N. Onuma, J. Limpouch, O. Klimo, T. Kikuchi, Improvement of energy conversion efficiency from laser to proton beam in a laser-foil interaction. Phys. Rev. E 78, 046401 (2008)

  \\item S. Kawata, T. Nagashima, M. Takano, T. Izumiyama, D. Kamiyama, D. Barada, Q. Kong, Y.J. Gu, P.X. Wang, Y.Y. Ma, W.M. Wang, W. Zhang, J. Xie, H. Zhang, D. Dai, Controllability of intense-laser ion acceleration. High Power Laser Sci. Eng. 2(e4), 1-11 (2014)

  \\item S. Kawata, T. Maruyama, H. Watanabe, I. Takahashi, Inverse-Bremsstrahlung electron acceleration. Phys. Rev. Lett. 22, 2072-2075 (1991)

  \\item S. Kawata, A. Manabe, S. Takeuchi, High-energy electron production by an electromagnetic wave with a static magnetic field. Jpn. J. Appl. Phys. 28, L704-L706 (1989)

  \\item M. Nakamura, S. Kawata, R. Sonobe, Q. Kong, S. Miyazaki, N. Onuma, T. Kikuchi, Robustness of a tailored hole target in laser-produced collimated proton beam generation. J. Appl. Phys. 101, 113305 (2007)

  \\item C. Thaury, F. Quéré, J.-P. Geindre, A. Levy, T. Ceccotti, P. Monot, M. Bougeard, F. Réau, P. dÓliveira, P. Audebert,, R. Marjoribanks, Ph. Martin, Plasma mirrors for ultrahigh-intensity optics. Nature Phys. 3, 424-429 (2007). \\href{https://doi.org/10.1038/nphys595}{https://doi.org/10.1038/nphys595}

  \\item D. Margarone, O. Klimo, I.J. Kim, J. Proküpek, J. Limpouch, T.M. Jeong, T. Mocek, J. Pšikal, H.T. Kim, J. Proška, K.H. Nam, L. Štolcová, I.W. Choi, S.K. Lee, J.H. Sung, T.J. Yu, G. Korn, Laser-driven proton acceleration enhancement by nanostructured foils. Phys. Rev. Lett. 109, 234801 (2012)

  \\item S. Kawata, C. Deutsch, Y.J. Gu, Peculiar behavior of Si cluster ions in a high-energy-density solid Al plasma. Phys. Rev. E 99, 011201 (2019)

  \\item G. Zwicknagel, C. Deutsch, Correlated ion stopping in plasmas. Phys. Rev. E 56, 970-987 (1997)

  \\item A. Bret, C. Deutsch, Correlated stopping power of a chain of N charges. J. Plasma Phys. 74, 595-599 (2008)

  \\item E. Nardi, Z. Zinamon, Interaction of fast $\\mathrm{C}_{60}$ clusters with a Lindhard gas. Phys. Rev. A 51, R3407-R3409 (1995)

  \\item E. Nardi, Z. Zinamon, T.A. Tombrello, N.M. Tanushev, Simulation of the interaction of highenergy $\\mathrm{C}_{60}$ cluster ions with amorphous targets. Phys. Rev. A 66, 013201 (2002)

  \\item D. Ben-Hamu, A. Baer, H. Feldman, J. Levin, O. Heber, Z. Amitay, Z. Vager, D. Zajfman, Energy loss of fast clusters through matter. Phys. Rev. A 56, 4786-4794 (1997)

  \\item N.R. Arista, Stopping of molecules and clusters. Nucl. Instrum. Methods Phys. Res. B 108, 164-165 (2000)

  \\item W. Brandt, A. Ratkowski, R.H. Ritchie, Energy loss of swift proton clusters in solids. Phys. Rev. Lett. 33, 1325-1328 (1974)

  \\item D.S. Gemmell, J. Remillieux, J.-C. Poizat, M.J. Gaillard, R.E. Holland, Z. Vager, Evidence for an alignment effect in the motion of swift ion clusters through solids. Phys. Rev. Lett. 23, $1420-1424$ (1975)

  \\item P. Chenevier, J. Dolique, H. Perès, Potential created by a test particle in one-, two- and threedimensions in a flowing ion-electron plasma. J. Plasma Phys. 10, 185-195 (1973). https:// \\href{http://doi.org/10.1017/S0022377800007753}{doi.org/10.1017/S0022377800007753}

  \\item S. Kawata, Dynamic mitigation of instabilities. Phys. Plasmas 19, 024503 (2012)

  \\item S. Kawata, Y.J. Gu, X.F. Li, T. Karino, H. Katoh, J. Limpouch, O. Klimo, D. Margarone, Q. Yu, Q. Kong, S. Weber, S. Bulanov, A. Andreev, Dynamic stabilization of filamentation instability. Phys. Plasmas 25, 011601 (2018)

  \\item S. Kawata, T. Karino, Y. Gu, Dynamic stabilization of plasma instability. High Power Laser Sci. Eng. 7, e3 (2019)

  \\item S. Kawata, T. Karino, Y.J. Gu, Phase control of a $z$-current-driven plasma column. Phys. Rev. E 101, 041201 (2020). \\href{https://doi.org/10.1103/PhysRevE}{https://doi.org/10.1103/PhysRevE}. 101.041201

  \\item Y.J. Gu, S. Kawata, S.V. Bulanov, Dynamic mitigation of the tearing mode instability in a collisionless current sheet. Sci. Rep. 11, 11651 (2021). \\href{https://doi.org/10.1038/s41598-02191111-8}{https://doi.org/10.1038/s41598-02191111-8}

  \\item P.L. Kapitza, Dynamic stability of the pendulum with vibrating suspension point. Soviet Phys. JETP 21, 588-597 (1951)

  \\item G.F. Franklin, J. Powell, A. Emami-Naeini, Feedback Control of Dynamic Systems, Global ed. (Pearson Education Ltd., 2014)

  \\item G.H. Wolf, Dynamic stabilization of the interchange instability of a liquid-gas interface. Phys. Rev. Lett. 24, 444-446 (1970)

  \\item F. Troyon, R. Gruber, Theory of the dynamic stabilization of the Rayleigh-Taylor instability. Phys. Fluids 14, 2069-2073 (1971)

  \\item A.R. Piriz, G.R. Prieto, I.M. Diaz, J.J.L. Cela, Dynamic stabilization of Rayleigh-Taylor instability in Newtonian fluids. Phys. Rev. E 82, 026317 (2010)

  \\item R. Betti, R.L. McCrory, C.P. Verdon, Stability analysis of unsteady ablation fronts. Phys. Rev. Lett. 71, 3131-3134 (1993)

  \\item F.W.J. Olver, D.W. Lozier, R.F. Boisvert, C.W. Clark (eds.), NIST Handbook of Mathematical Functions (Cambridge University Press, Cambridge, 2010). \\href{https://dlmf.nist.gov/}{https://dlmf.nist.gov/}

  \\item G.B. Arfken, H.J. Weber, Mathematical Methods for Physicists, 6th edn. (Elsevier Academic Press, Amsterdam, 2005)

  \\item T.D. Arber, K. Bennett, et al., Contemporary particle-in-cell approach to laser-plasma modelling. Plasma Phys. Control. Fusion 57, 113001 (2015)

  \\item K. Bennett, Users Manual for the EPOCH PIC codes, EPOCH Version 4.3 (2014)

  \\item R.C. Davidson, W.A. Benjamin, Theory of nonneutral plasmas. Frontiers Phys. (1974)

  \\item E. Priest, T. Forbes, Magnetic Reconnection. MHD Theory and Applications (Cambridge University Press, Cambridge, 2000)

  \\item L.M. Zelenyi, H.V. Malova, A.V. Artemyev, V.Y. Popov, A.A. Petrukovich, Thin current sheets in collisionless plasma: equilibrium structure, plasma instabilities, and particle acceleration. Plasma Phys. Rep. 37, 118-160 (2011)

  \\item J. Birn, A.V. Artemyev, D.N. Baker, M. Echim, M. Hoshino, L.M. Zelenyi, Particle acceleration in the magnetotail and aurora. Space Sci. Rev. 173, 49-102 (2012)

  \\item J. Wesson, Tokamaks. International Series of Monographs on Physics, 4th edn. (Oxford Science Publications, Oxford, 2011)

  \\item B.A. Remington, R.P. Drake, D.D. Ryutov, Experimental astrophysics with high power lasers and Z pinches. Rev. Mod. Phys. 78, 755-808 (2006)

  \\item S.V. Bulanov, T.Zh. Esirkepov, D. Habs, F. Pegoraro, T. Tajima, Relativistic laser-matter interaction and relativistic laboratory astrophysics. Eur. Phys. J. D 55, 483-507 (2009)

  \\item S.V. Bulanov, Magnetic reconnection: from MHD to QED. Plasma Phys. Controlled Fusion 59, 014029 (2017)

  \\item Z. Chang, J.D. Callen, E.D. Fredrickson, R.V. Budny, C.C. Hegna, K.M. McGuire, M.C. Zarnstorff, and TFTR group: observation of nonlinear neoclassical pressure-gradient-driven tearing modes in TFTR. Phys. Rev. Lett. 74, 4663-4666 (1995)

  \\item Y.J. Gu, F. Pegoraro, P.V. Sasorov, D. Golovin, A. Yogo, G. Korn, S.V. Bulanov, Electromagnetic burst generation during annihilation of magnetic field in relativistic laser-plasma interaction. Sci. Rep. 9, 19462 (2019)

  \\item M. Yamada, Y. Ren, H. Ji, J. Breslau, S. Gerhardt, R. Kulsrud, A. Kuritsyn, Experimental study of two-fluid effects on magnetic reconnection in a laboratory plasma with variable collisionality. Phys. Plasmas 13, 052119 (2006)

  \\item K. Fujimoto, R.D. Sydora, Plasmoid-induced turbulence in collisionless magnetic reconnection. Phys. Rev. Lett. 109, 265004 (2012)

  \\item S. Zenitani, M. Hesse, A. Klimas, C. Black, M. Kuznetsova, The inner structure of collisionless magnetic reconnection: the electron-frame dissipation measure and Hall fields. Phys. Plasmas 18, $122108(2011)$

  \\item S. Zenitani, M. Hesse, A. Klimas, C. Black, M. Kuznetsova, New measure of the dissipation region in collisionless magnetic reconnection. Phys. Rev. Lett. 106, 195003 (2011)

  \\item J.W. Dungey, Interplanetary magnetic field and the auroral zones. Phys. Rev. Lett. 6, 47-48 (1961)

  \\item M. Ottaviani, F. Porcelli, Nonlinear collisionless magnetic reconnection. Phys. Rev. Lett. 71, 3802-3805 (1993)

  \\item J. Birn, J.F. Drake, M.A. Shay, B.N. Rogers, R.E. Denton, M. Hesse, M. Kuznetsova, Z.W. Ma, A. Bhattacharjee, A. Otto, P.L. Pritchett, Geospace environmental modeling (GEM) magnetic reconnection challenge. J. Geophys. Res. 106, 3715-3719 (2001)

  \\item M. Yamada, R. Kulsrud, H. Ji, Magnetic reconnection. Rev. Mod. Phys. 82, 603-664 (2010)

  \\item S. Chandrasekhar, Hydrodynamics and Hydromagnetic Stability (Dover pub, New York, 1970)

  \\item S. Kawata, Y.J. Gu, Dynamic mitigation of two-stream instability in plasma (2020). arXiv \\href{https://doi.org/10.48550/ARXIV.2009.03040}{https://doi.org/10.48550/ARXIV.2009.03040}

  \\item S. Kawata, T. Karino, Robust dynamic mitigation of instabilities. Phys. Plasmas 22, 042106 (2015)

  \\item R. Sato, S. Kawata, T. Karino, K. Uchibori, T. Iinuma, H. Katoh, A.I. Ogoyski, Code O-SUKI: simulation of direct-drive fuel target implosion in heavy ion inertial fusion. Comput. Phys. Commun. 40, 83-100 (2019). \\href{https://doi.org/10.1016/j.cpc.2019.03.003}{https://doi.org/10.1016/j.cpc.2019.03.003}

  \\item S. Skupsky, K. Lee, J. Appl. Phys. 54, 3662-3671 (1983)

  \\item A.I. Ogoyski, T. Someya, S. Kawata, Code OK1-simulation of multi-beam irradiation on a spherical target in heavy ion fusion. Comput. Phys. Commun. 157, 160-172 (2004)

  \\item A.I. Ogoyski, S. Kawata, T. Someya, Code OK2-a simulation code of ion-beam illumination on an arbitrary shape and structure target. Comput. Phys. Commun. 161, 143-150 (2004)

  \\item A.I. Ogoyski, S. Kawata, P.H. Popov, Code OK3-an upgraded version of OK2 with beam wobbling function. Comput. Phys. Commun. 181, 1332-1333 (2010)

\\end{enumerate}

\\section*{Appendix A Additional Readings}
Instructive books and texts are listed below in plasma physics and computational methods in plasmas for additional readings. The following references are sequenced alphabetically by the first author in each topic.

Basic plasma physics:

\\begin{enumerate}
  \\item Chen, F. F.: Introduction to Plasma Physics and Controlled Fusion, 3rd Ed., Springer (2015).

  \\item Ichimaru, S.: Statistical Plasma Physics, Vol. 1: Basic Principles, CRC Press, Boca Raton (2004).

  \\item Ichimaru, S.: Statistical Plasma Physics, Vol. 2: Condensed Plasmas, CRC Press, Boca Raton (2004).

  \\item Krall, N. A. and Trivelpiece, A. W.: Principles of Plasma Physics, McGraw-Hill, New York (1973).

  \\item Nicholson, D. R.: Introduction to Plasma Theory, John Wiley \\& Sons, New York (1983).

\\end{enumerate}

Waves and instabilities in plasmas

\\begin{enumerate}
  \\item Davidson, R. C.: Kinetic Waves and Instabilities in a Uniform Plasma, pp. 519586, Sec. 3.3 in Handbook of Plasma Physics, Vol. 1 Basic Plasma Physics, Galeev, A. A. and Sudan, R. N. (Eds.), North Holland Pub., Amsterdam; New York; Oxford (1983).

  \\item Stix, T. H.: The theory of plasma waves, MacGraw Hill, New York (1962).

\\end{enumerate}

\\section*{Turbulence in fluids}
\\begin{enumerate}
  \\item Davidson, P.: Turbulence: An Introduction for Scientists and Engineers, 2nd ed., Oxford University Press, Oxford (2015).
\\end{enumerate}

\\section*{Nonlinear physics and turbulence in plasmas}
\\begin{enumerate}
  \\item Davidson, R. C.: Methods in Nonlinear Plasma Theory, Academic Press, New York and London (1972).

  \\item Diamond, P. H., Itoh, S.-I. and Itoh, K.: Modern Plasma Physics, Vol. 1: Physical Kinetics of Turbulent Plasmas, Cambridge University Press (2010).

  \\item Kadomtsev, B. B.: Plasma Turbulence, Academic Press, London and New York (1965).

  \\item Sagdeev, R. Z. and Galeev, A. A.: Nonlinear plasma theory, Reviews of plasma phys., ed. by Leontovich M. A., VII pp. 1-180, Consultants Bureau, New York (1965) ; Lectures on the non-linear plasma theory of plasma, Int. center for theoretical phys., IAEA, IC/66/64 (1966).

\\end{enumerate}

\\section*{Nuclear fusion}
\\begin{enumerate}
  \\item Atzeni, S. and Meyer-Ter-Vehn, J.: The Physics of Inertial Fusion, Oxford Science Pub. (2004).

  \\item Kikuchi, M. and Azumi, M.: Frontiers in Fusion Research II: Introduction to Modern Tokamak Physics, Springer, Cham (2015).

  \\item Miyamoto, K.: Plasma Physics and Controlled Nuclear Fusion, Springer, Springer, Berlin, Heidelberg (2013).

  \\item Niu, K.: Nuclear Fusion, Cambridge University Press, Cambridge (2009).

\\end{enumerate}

\\section*{Fluid dynamics}
\\begin{enumerate}
  \\item Chandrasekhar, S.: Hydrodynamics and hydromagnetic stability, Dover pub., New York (1970).

  \\item Zel'dovich, Ya. B. and Raizer, Yu. P.: Physics of shock waves and hightemperature hydrodynamic phenomena, Dover, New York (2002).

\\end{enumerate}

\\section*{Computational methods in plasmas}
\\begin{enumerate}
  \\item Birdsall, C. K. and Langdon, A. B.: Plasma Physics via Computer Simulation, Taylor \\& Francis, New York (2005).

  \\item Hockney, R. W. and Eastwood, J. W.: Computer Simulation using Particles, Taylor \\& Francis, New York (1988).

  \\item Richtmeyer, R. D. and Morton, K. W.: Difference Methods for Initial-Value Problems, Interscience Pub., New York (1967).

  \\item Roache, P. J.: Fundamentals of Computational Fluid Dynamics, Hermosa Pub., New Mexico (2003).

\\end{enumerate}

Uncertainty in scientific computation

\\begin{enumerate}
  \\item Dienstfrey, A. and Boisvert, R., (Ed.): Uncertainty Quantification in Scientific Computing, IFIP Advances in Information and Communication Technology, 377, Springer, New York (2012).

  \\item Einarsson, B. (Ed.): Accuracy and Reliability in Scientific Computing, siam (Society for Industrial and Applied Mathematics) (2005).

\\end{enumerate}

\\section*{Appendix B 
 Physical Constants and Mathematical Formulae}
The followings are short summaries of the physical constants and the mathematical formulae used frequently in plasma physics.

\\section*{B. 1 Physical Constants and Relations}
\\begin{center}
\\begin{tabular}{llll}
Elementary charge & $\\mathrm{e}$ & $1.6022 \\times 10^{-19}$ & $\\mathrm{C}($ Coulomb $)$ \\\\
Electron mass & $m_{e}$ & $9.1094 \\times 10^{-31}$ & $\\mathrm{~kg}$ \\\\
Proton mass & $m_{p}$ & $1.6726 \\times 10^{-27}$ & $\\mathrm{~kg}$ \\\\
Permittivity of vacuum & $\\epsilon_{0}$ & $8.8542 \\times 10^{-12}$ & $\\mathrm{~F} / \\mathrm{m}$ \\\\
Permeability of vacuum & $\\mu_{0}$ & $1.2566 \\times 10^{-6}$ & $\\mathrm{H} / \\mathrm{m}$ \\\\
Speed of light in vacuum & $\\mathrm{c}$ & $2.9979 \\times 10^{8}$ & $\\mathrm{~m} / \\mathrm{s}$ \\\\
Electron rest mass energy & $m_{e} c^{2}$ & 510.98 & $\\mathrm{keV}$ \\\\
Proton rest mass energy & $m_{p} c^{2}$ & 0.93823 & $\\mathrm{GeV}$ \\\\
Boltzmann constant & $k\\left(\\right.$ or $\\left.k_{B}\\right)$ & $1.3807 \\times 10^{-23}$ & $\\mathrm{~J} / \\mathrm{K}$ \\\\
\\hline
\\end{tabular}
\\end{center}

\\begin{center}
\\begin{tabular}{lll}
$1 \\mathrm{eV}$ & $1.6022 \\times 10^{-19}$ & $\\mathrm{~J}$ \\\\
$1 \\mathrm{eV}$ & $1.1604 \\times 10^{4}$ & $\\mathrm{~K}$ \\\\
$1 \\mathrm{C}$ & $3 \\times 10^{9}$ & statcoulomb \\\\
$1 \\mathrm{~A}$ (Ampere) & $3 \\times 10^{9}$ & statampere \\\\
Electric field E: $1 \\mathrm{~V} / \\mathrm{m}$ & $\\frac{1}{3} \\times 10^{-4}$ & statvolt/cm \\\\
 & 1 &  \\\\
Voltage: $1 \\mathrm{~V}$ & $\\frac{1}{3} \\times 10^{-2}$ & statvolt \\\\
Energy: $1 \\mathrm{~J}$ & $10^{7}$ & erg \\\\
Force: $1 \\mathrm{~N}$ & $10^{5}$ & dyne \\\\
Magnetic field (induction) B: 1 T & $10^{4}$ & gauss \\\\
\\end{tabular}
\\end{center}

\\section*{B. 2 Vector Formulae}
In the Sect. B.2, vectors are shown by the bold letters of $\\mathbf{a}, \\mathbf{b}, \\mathbf{c}, \\mathbf{d}, \\ldots$, and scalars by $\\varphi, \\xi, \\ldots$ Vector products or outer products are represented by " $\\times$ ", and scalar products or inner products by “.”.

$$
\\begin{aligned}
& \\mathbf{a} \\cdot \\mathbf{b}=\\mathbf{b} \\cdot \\mathbf{a} \\\\
& \\mathbf{a} \\cdot \\mathbf{b} \\times \\mathbf{c}=\\mathbf{b} \\cdot \\mathbf{c} \\times \\mathbf{a}=\\mathbf{c} \\cdot \\mathbf{a} \\times \\mathbf{b} \\\\
& \\mathbf{a} \\times(\\mathbf{b} \\times \\mathbf{c})=(\\mathbf{a} \\cdot \\mathbf{c}) \\mathbf{b}-(\\mathbf{a} \\cdot \\mathbf{b}) \\mathbf{c}=(\\mathbf{c} \\times \\mathbf{b}) \\times \\mathbf{a} \\\\
& \\mathbf{a} \\times(\\mathbf{b} \\times \\mathbf{c})+\\mathbf{b} \\times(\\mathbf{c} \\times \\mathbf{a})+\\mathbf{c} \\times(\\mathbf{a} \\times \\mathbf{b})=0 \\\\
& (\\mathbf{a} \\times \\mathbf{b}) \\cdot(\\mathbf{c} \\times \\mathbf{d})=(\\mathbf{a} \\cdot \\mathbf{c})(\\mathbf{b} \\cdot \\mathbf{d})-(\\mathbf{a} \\cdot \\mathbf{d})(\\mathbf{b} \\cdot \\mathbf{c}) \\\\
& \\nabla(\\varphi \\xi)=\\varphi \\nabla \\xi+\\xi \\nabla \\varphi \\\\
& \\nabla \\cdot(\\varphi \\mathbf{a})=\\varphi \\nabla \\cdot \\mathbf{a}+(\\mathbf{a} \\cdot \\nabla) \\varphi \\\\
& \\nabla \\times(\\varphi \\mathbf{a})=\\varphi \\nabla \\times \\mathbf{a}+\\nabla \\varphi \\times \\mathbf{a} \\\\
& \\nabla \\cdot(\\mathbf{a} \\times \\mathbf{b})=\\mathbf{b} \\cdot(\\nabla \\times \\mathbf{a})-\\mathbf{a} \\cdot(\\nabla \\times \\mathbf{b}) \\\\
& \\nabla \\times(\\mathbf{a} \\times \\mathbf{b})=\\mathbf{a}(\\nabla \\cdot \\mathbf{b})-\\mathbf{b}(\\nabla \\cdot \\mathbf{a})+(\\mathbf{b} \\cdot \\nabla) \\mathbf{a}-(\\mathbf{a} \\cdot \\nabla) \\mathbf{b} \\\\
& \\nabla^{2} \\mathbf{a}=\\Delta \\mathbf{a}=\\nabla(\\nabla \\cdot \\mathbf{a})-\\nabla \\times(\\nabla \\times \\mathbf{a}) \\\\
& \\nabla \\cdot(\\nabla \\times \\mathbf{a})=0 \\\\
& \\nabla \\times(\\nabla \\varphi)=0 \\quad \\iiint_{V} \\varphi \\mathrm{d} V=\\iint_{S} \\varphi \\mathrm{d} S
\\end{aligned}
$$

(Here $V$ shows volume surrounded by the surface $\\mathbf{S}$. The direction of $d \\mathbf{S}$ is normal to the surface of $\\mathbf{S}$ and outward. )

$$
\\begin{aligned}
& \\iiint_{V} \\nabla \\cdot \\mathbf{a d} V=\\iint_{S} \\mathbf{a} \\cdot \\mathrm{d} \\mathbf{S} \\\\
& \\iiint_{V} \\nabla \\times \\mathbf{a d} V=\\iint_{S} \\mathrm{~d} \\mathbf{S} \\times \\mathbf{a} \\\\
& \\iint_{S} \\mathrm{~d} \\mathbf{S} \\times \\nabla \\varphi=\\int_{l} \\varphi \\mathrm{d} \\mathbf{l}
\\end{aligned}
$$

(Here $l$ is a closed curve surrounding the open surface $\\mathbf{S}$, and $\\mathrm{dl}$ is a line element of the closed curve. )

$$
\\iint(\\nabla \\times \\mathbf{a}) \\cdot \\mathrm{d} \\mathbf{S}=\\int_{l} \\mathbf{a} \\cdot \\mathrm{d} \\mathbf{l}
$$

\\section*{B. 3 Differential Operators}
Cartesian coordinate: $(x, y, z)$

$$
\\begin{aligned}
& (\\nabla \\varphi)_{x}=\\frac{\\partial \\varphi}{\\partial x}, \\quad(\\nabla \\varphi)_{y}=\\frac{\\partial \\varphi}{\\partial y}, \\quad(\\nabla \\varphi)_{z}=\\frac{\\partial \\varphi}{\\partial z} \\\\
& \\nabla \\cdot \\mathbf{a}=\\frac{\\partial a_{x}}{\\partial x}+\\frac{\\partial a_{y}}{\\partial y}+\\frac{\\partial a_{z}}{\\partial z} \\\\
& (\\nabla \\times \\mathbf{a})_{x}=\\frac{\\partial a_{z}}{\\partial y}-\\frac{\\partial a_{y}}{\\partial z} \\\\
& (\\nabla \\times \\mathbf{a})_{y}=\\frac{\\partial a_{x}}{\\partial z}-\\frac{\\partial a_{z}}{\\partial x} \\\\
& (\\nabla \\times \\mathbf{a})_{z}=\\frac{\\partial a_{y}}{\\partial x}-\\frac{\\partial a_{x}}{\\partial y} \\\\
& \\nabla^{2} \\varphi=\\Delta \\varphi=\\frac{\\partial^{2} \\varphi}{\\partial x^{2}}+\\frac{\\partial^{2} \\varphi}{\\partial y^{2}}+\\frac{\\partial^{2} \\varphi}{\\partial z^{2}}
\\end{aligned}
$$

Cylindrical coordinate: $(r, \\theta, z)$

$$
\\begin{aligned}
& (\\nabla \\varphi)_{r}=\\frac{\\partial \\varphi}{\\partial r}, \\quad(\\nabla \\varphi)_{\\theta}=\\frac{1}{r} \\frac{\\partial \\varphi}{\\partial \\theta}, \\quad(\\nabla \\varphi)_{z}=\\frac{\\partial \\varphi}{\\partial z} \\\\
& \\nabla \\cdot \\mathbf{a}=\\frac{1}{r} \\frac{\\partial\\left(r a_{r}\\right)}{\\partial r}+\\frac{1}{r} \\frac{\\partial a_{\\theta}}{\\partial \\theta}+\\frac{\\partial a_{z}}{\\partial z} \\\\
& (\\nabla \\times \\mathbf{a})_{r}=\\frac{1}{r} \\frac{\\partial a_{z}}{\\partial \\theta}-\\frac{\\partial a_{\\theta}}{\\partial z} \\\\
& (\\nabla \\times \\mathbf{a})_{\\theta}=\\frac{\\partial a_{r}}{\\partial z}-\\frac{\\partial a_{z}}{\\partial r} \\\\
& (\\nabla \\times \\mathbf{a})_{z}=\\frac{1}{r} \\frac{\\partial\\left(r a_{\\theta}\\right)}{\\partial r}-\\frac{1}{r} \\frac{\\partial a_{r}}{\\partial \\theta} \\\\
& \\nabla^{2} \\varphi=\\Delta \\varphi=\\frac{1}{r} \\frac{\\partial}{\\partial r}\\left(r \\frac{\\partial \\varphi}{\\partial r}\\right)+\\frac{1}{r^{2}} \\frac{\\partial^{2} \\varphi}{\\partial \\theta^{2}}+\\frac{\\partial^{2} \\varphi}{\\partial z^{2}} \\\\
& \\left(\\nabla^{2} \\mathbf{a}\\right)_{r}=\\nabla^{2} a_{r}-\\frac{2}{r^{2}} \\frac{\\partial a_{\\theta}}{\\partial \\theta}-\\frac{a_{r}}{r^{2}} \\\\
& \\left(\\nabla^{2} \\mathbf{a}\\right)_{\\theta}=\\nabla^{2} a_{\\theta}+\\frac{2}{r^{2}} \\frac{a_{\\theta}}{\\partial \\theta}-\\frac{a^{2}}{r^{2}} \\\\
& \\left(\\nabla^{2} \\mathbf{a}\\right)_{z}=\\nabla^{2} a_{z}
\\end{aligned}
$$

Spherical coordinate: $(r, \\theta, \\varphi)$

$$
\\begin{aligned}
& (\\nabla \\xi)_{r}=\\frac{\\partial \\xi}{\\partial r}, \\quad(\\nabla \\xi)_{\\theta}=\\frac{1}{r} \\frac{\\partial \\xi}{\\partial \\theta}, \\quad(\\nabla \\xi)_{\\varphi}=\\frac{1}{r \\sin \\theta} \\frac{\\partial \\xi}{\\partial \\varphi} \\\\
& \\nabla \\cdot \\mathbf{a}=\\frac{1}{r^{2}} \\frac{\\partial\\left(r^{2} a_{r}\\right)}{\\partial r}+\\frac{1}{r \\sin \\theta} \\frac{\\partial\\left(a_{\\theta} \\sin \\theta\\right)}{\\partial \\theta}+\\frac{1}{r \\sin \\theta} \\frac{\\partial a_{\\varphi}}{\\partial \\varphi} \\\\
& (\\nabla \\times \\mathbf{a})_{r}=\\frac{1}{r \\sin \\theta} \\frac{\\partial\\left(a_{\\varphi} \\sin \\theta\\right)}{\\partial \\theta}-\\frac{1}{r \\sin \\theta} \\frac{\\partial a_{\\theta}}{\\partial \\varphi} \\\\
& (\\nabla \\times \\mathbf{a})_{\\theta}=\\frac{1}{r \\sin \\theta} \\frac{\\partial a_{r}}{\\partial \\varphi}-\\frac{1}{r} \\frac{\\partial\\left(r a_{\\varphi}\\right)}{\\partial r} \\\\
& (\\nabla \\times \\mathbf{a})_{\\varphi}=\\frac{1}{r} \\frac{\\partial\\left(r a_{\\theta}\\right)}{\\partial r}-\\frac{1}{r} \\frac{\\partial a_{r}}{\\partial \\theta} \\\\
& \\nabla^{2} \\xi=\\triangle \\xi=\\frac{1}{r^{2}} \\frac{\\partial}{\\partial r}\\left(r^{2} \\frac{\\partial \\xi}{\\partial r}\\right)+\\frac{1}{r^{2} \\sin \\theta} \\frac{\\partial}{\\partial \\theta}\\left(\\sin \\theta \\frac{\\partial \\xi}{\\partial \\theta}\\right)+\\frac{1}{r^{2} \\sin ^{2} \\theta} \\frac{\\partial^{2} \\xi}{\\partial \\varphi^{2}} \\\\
& \\left(\\nabla^{2} \\mathbf{a}\\right)_{r}=\\nabla^{2} a_{r}-\\frac{2 a_{r}}{r^{2}}-\\frac{2}{r^{2}} \\frac{\\partial a_{\\theta}}{\\partial \\theta}-\\frac{2 \\cot \\theta a_{\\theta}}{r^{2}}-\\frac{2}{r^{2} \\sin ^{2} \\theta} \\frac{\\partial a_{\\varphi}}{\\partial \\varphi} \\\\
& \\left(\\nabla^{2} \\mathbf{a}\\right)_{\\theta}=\\nabla^{2} a_{\\theta}+\\frac{2}{r^{2}} \\frac{\\partial a_{r}}{\\partial \\theta}-\\frac{2 \\cos \\theta}{r^{2} \\sin ^{2} \\theta}-\\frac{\\partial a_{\\varphi}}{r^{2} \\sin ^{2} \\theta} \\frac{2 \\cos \\theta}{\\partial \\varphi} \\\\
& \\left(\\nabla^{2} \\mathbf{a}\\right)_{\\varphi}=\\nabla^{2} a_{\\varphi}+\\frac{\\partial a_{\\theta}}{r^{2} \\sin \\theta} \\frac{\\partial a_{r}}{\\partial \\varphi}-\\frac{a_{\\varphi}}{r^{2} \\sin ^{2} \\theta}+\\frac{2 \\sin }{r^{2} \\sin ^{2} \\theta} \\frac{\\partial \\varphi}{\\partial y}
\\end{aligned}
$$

\\section*{B. 4 Delta Function}
The delta function is defined as follows:

$$
\\begin{aligned}
& \\delta(x)=\\left\\{\\begin{array}{cc}
\\infty & (\\text { at } x=0) \\\\
0 & (\\text { at } x \\neq 0)
\\end{array}\\right. \\\\
& \\int_{-\\infty}^{\\infty} \\delta(x) \\mathrm{d} x=1 \\\\
& \\int_{-\\infty}^{\\infty} \\delta\\left(x-x_{0}\\right) f(x) \\mathrm{d} x=f\\left(x_{0}\\right) \\\\
& \\delta(x)=\\frac{1}{2 \\pi} \\int_{-\\infty}^{\\infty} \\exp (\\mathrm{ikx}) \\mathrm{d} k \\\\
& \\delta(x)=\\delta(-x) \\\\
& \\delta(\\operatorname{ax})=\\delta(x) \\frac{1}{|a|}(\\text { at } a \\neq 0)
\\end{aligned}
$$

$$
\\begin{aligned}
x \\delta(x) & =0 \\\\
f(x) \\delta(x-a) & =f(a) \\delta(x-a) \\\\
\\int_{-\\infty}^{\\infty} \\delta(x-a) \\delta(x-y) \\mathrm{d} x & =\\delta(y-a) \\\\
\\delta(f(x)) & =\\sum_{n} \\frac{1}{\\left|f_{n}^{\\prime}\\right|} \\delta\\left(x-x_{n}\\right)
\\end{aligned}
$$

(Here $\\sum_{n}$ is taken over all the solutions for $f(x)=0$. It means $f\\left(x_{n}\\right)=0$. In addition, $\\left.f_{n}^{\\prime}=f^{\\prime}\\left(x_{n}\\right) \\neq 0\\right)$.


\\begin{align*}
\\int_{-\\infty}^{\\infty} \\delta^{\\prime}(x) f(x) \\mathrm{d} x & =-f^{\\prime}(0) \\\\
\\delta^{\\prime}(x) & =-\\delta^{\\prime}(-x) \\\\
x \\delta^{\\prime}(x) & =-\\delta(x) \\\\
x^{2} \\delta^{\\prime}(x) & =0 \\\\
\\delta^{\\prime}(x) & =\\frac{i}{2 \\pi} \\int_{-\\infty}^{\\infty} k \\exp (\\mathrm{ikx}) \\mathrm{d} k \\\\
\\int_{-\\infty}^{\\infty} \\delta^{\\prime}(y-x) \\delta(x-a) \\mathrm{d} x & =\\delta^{\\prime}(y-a) \\tag{B.1}
\\end{align*}


\\section*{B. 5 Integral Formulae}
For $a>0$, the followings hold.

$$
\\begin{aligned}
& \\int_{0}^{\\infty} \\mathrm{e}^{-a x^{2}} \\mathrm{~d} x=\\frac{1}{2} \\sqrt{\\frac{\\pi}{a}} \\\\
& \\int_{0}^{\\infty} \\mathrm{e}^{-a x^{2}} x^{2 n} \\mathrm{~d} x=\\frac{(2 n-1) ! !}{2^{n+1}} \\sqrt{\\frac{\\pi}{a^{2 n+1}}} \\\\
& (\\text { Here }(2 n-1) ! !=(2 n-1)(2 n-3) \\ldots 3 \\cdot 1) \\\\
& \\int_{0}^{\\infty} \\mathrm{e}^{-a x^{2}} x^{2 n+1} \\mathrm{~d} x=\\frac{n !}{2 a^{n+1}}
\\end{aligned}
$$

\\section*{Appendix C 
 Complex Analysis: Summary}
\\section*{1 Cauchy's Integral Theorem}
In a region of $D$ in a complex plane in Fig. C.1, if a function $f(z)$ is differentiable, $f(z)$ is analytic or holomorphic. For the function $f(z)$, the following is found.

$$
\\oint_{C} f(z) \\mathrm{d} z=0
$$

Here $C$ is a simple closed path in $D$. This is the Cauchy's integral theorem. Here $f(z)$ is holomorphic in $D$.

$$
\\frac{1}{2 \\pi i} \\oint_{C} \\frac{f(t)}{t-z} \\mathrm{~d} t=f(z)
$$

This is the Cauchy's integral formula. Further, we obtain the derivative of $f(z)$ :

$$
\\frac{n !}{2 \\pi i} \\oint_{C} \\frac{f(t)}{(t-z)^{n+1}} \\mathrm{~d} t=\\frac{d^{n} f(z)}{d z^{n}}
$$

\\section*{2 Residue Theorem}
Here let us consider the following function $f(z)$ :

$$
f(z)=\\frac{A_{1}}{z-a}+\\frac{A_{2}}{(z-a)^{2}}+\\cdots+\\frac{A_{m}}{(z-a)^{m}}+g(z)
$$

Fig. C. 1 Complex plane. Here $C$ is a simple closed path in $D$

\\begin{center}
\\includegraphics[max width=\\textwidth]{2024_02_26_83e36543483eb7d284c1g-269(1)}
\\end{center}

Fig. C. 2 Residual theorem

\\begin{center}
\\includegraphics[max width=\\textwidth]{2024_02_26_83e36543483eb7d284c1g-269}
\\end{center}

Here a function $g(z)$ is holomorphic. In this case, $z=a$ is called as a pole of order $m$. After integrating the above function $f(z)$ along the path $C$ in Fig. C.2, we obtain the following result:

$$
\\oint_{C} f(z) \\mathrm{d} z=\\oint_{\\Gamma} f(z) \\mathrm{d} z=\\oint_{\\Gamma}\\left[\\frac{A_{1}}{z-a}+\\frac{A_{2}}{(z-a)^{2}}+\\cdots+\\frac{A_{m}}{(z-a)^{m}}\\right] \\mathrm{d} z .
$$

Here $\\Gamma$ is the path, which is a circle around $z=a$ with the radius of $r$.

Here $\\oint_{\\Gamma} g(z) \\mathrm{d} z=0$, because $g(z)$ is holomorphic. On $\\Gamma, z-a=r \\mathrm{e}^{i \\theta}$. Therefore, $d z=\\operatorname{ire}^{i \\theta} d \\theta$.

$$
\\begin{aligned}
\\oint_{\\Gamma} f(z) \\mathrm{d} z & =\\int_{0}^{2 \\pi} \\mathrm{d} \\theta \\operatorname{ire}^{i \\theta}\\left[\\frac{A_{1}}{r \\mathrm{e}^{i \\theta}}+\\cdots+\\frac{A_{m}}{r^{m} \\mathrm{e}^{i m \\theta}}\\right] \\\\
& =2 \\pi i A_{1}+i \\int_{0}^{2 \\pi} d \\theta\\left[\\frac{A_{2}}{r \\mathrm{e}^{i \\theta}}+\\cdots+\\frac{A_{m}}{r^{m-1} \\mathrm{e}^{i(m-1) \\theta}}\\right] \\\\
\\therefore \\oint_{C} f(z) \\mathrm{d} z & =2 \\pi i A_{1} .
\\end{aligned}
$$

Here we used $\\int_{0}^{2 \\pi} \\mathrm{d} \\theta \\mathrm{e}^{i l \\theta}=0 \\quad(l \\neq 0)$, and $A_{1}$ is called the residue.

When there are multiple singularities as shown in Fig.C.3, the following is obtained:

$$
\\oint_{C} f(z) \\mathrm{d} z=\\sum_{j} \\oint_{C} f(z) \\mathrm{d} z=2 \\pi i \\sum_{j} A_{j}
$$

The residue $A$ is obtained as follows:

$$
A=\\frac{1}{(m-1) !} \\lim _{z \\rightarrow a} \\frac{d^{m-1}}{\\mathrm{~d} z^{m-1}}\\left\\{(z-a)^{m} f(z)\\right\\}
$$

Here we consider the following example:

$$
\\oint_{C} f(z) \\mathrm{d} z=\\oint_{C} \\frac{1}{z^{2}+1} \\mathrm{~d} z=\\oint_{C} \\frac{1}{(z+i)(z-i)} \\mathrm{d} z
$$

The integral path in Fig. C. 4 is taken.

$$
\\oint_{C} \\frac{1}{(z+i)(z-i)} \\mathrm{d} z=2 \\pi i \\times(\\text { Residue })=2 \\pi i \\times \\frac{1}{2 i}=\\pi
$$

The following integration is also obtained by the path in Fig.C.4.

$$
\\oint_{C} \\frac{1}{\\left(z^{2}+1\\right)^{3}} \\mathrm{~d} z
$$

\\begin{center}
\\includegraphics[max width=\\textwidth]{2024_02_26_83e36543483eb7d284c1g-271(1)}
\\end{center}

Fig. C. 3 Residues

\\begin{center}
\\includegraphics[max width=\\textwidth]{2024_02_26_83e36543483eb7d284c1g-271}
\\end{center}

Fig. C. 4 Path for $\\int 1 /\\left(z^{2}+1\\right) d z$

Fig. C. 5 Integral path for $\\lim _{\\eta \\rightarrow 0} \\int_{-\\infty}^{\\infty} \\frac{1}{z-a-i \\eta} \\mathrm{d} z$

\\begin{center}
\\includegraphics[max width=\\textwidth]{2024_02_26_83e36543483eb7d284c1g-272}
\\end{center}

$$
\\begin{aligned}
\\oint_{C} \\frac{1}{\\left(z^{2}+1\\right)^{3}} \\mathrm{~d} z=\\oint_{C} \\frac{1}{(z+i)^{3}(z-i)^{3}} \\mathrm{~d} z & =2 \\pi i \\times \\frac{1}{2 !}\\left[\\frac{\\mathrm{d}^{2}}{\\mathrm{~d} z^{2}}\\left\\{\\frac{1}{(z+i)^{3}}\\right\\}\\right]_{z=i} \\\\
& =\\frac{3}{8} \\pi
\\end{aligned}
$$

Here the following integral is examined:

$$
\\lim _{\\eta \\rightarrow 0} \\int_{-\\infty}^{\\infty} \\frac{1}{z-a-i \\eta} \\mathrm{d} z
$$

For this integral, the path in Fig. C. 5 is employed. Here $\\eta$ is infinitesimally small, and the path circles the pole $a+i \\eta$ by a semicircle. Then we obtain the following result:

$$
\\begin{aligned}
& \\int_{-\\infty}^{\\infty} \\frac{1}{z-a} \\mathrm{~d} z+\\int_{-\\pi}^{0} \\frac{1}{r \\mathrm{e}^{i \\theta}} \\mathrm{ri}^{i \\theta} \\mathrm{d} \\theta \\\\
= & \\int_{-\\infty}^{\\infty} \\frac{1}{z-a} \\mathrm{~d} z+\\pi i \\\\
= & \\int_{-\\infty}^{\\infty}\\left[\\frac{1}{z-a}+\\pi i \\delta(z-a)\\right] \\mathrm{d} z
\\end{aligned}
$$

By the following relation, the above integral is simply expressed.

$$
\\frac{1}{z-a \\pm i \\eta}=\\frac{1}{z-a} \\mp \\pi i \\delta(z-a)
$$

This relation is called the Plemelj formula.

\\section*{3 Derivation of Eq. (6.90)}
By using Eqs. (6.87) and (6.89), Eq. (6.88) becomes as follows:

$$
\\begin{aligned}
\\varphi(\\mathbf{r}) & =\\frac{1}{(2 \\pi)^{3}} \\int \\frac{q / \\varepsilon_{0}}{k^{2}+1 / \\lambda_{D}^{2}} \\mathrm{e}^{-i \\mathbf{k} \\cdot \\mathbf{r}} \\mathrm{d} \\mathbf{k} \\\\
& =\\frac{q}{(2 \\pi)^{3} \\varepsilon_{0}} \\int \\frac{\\mathrm{e}^{-i k r \\cos \\theta}}{\\left(k+i / \\lambda_{D}\\right)\\left(k-i / \\lambda_{D}\\right)} 2 \\pi k \\sin \\theta \\mathrm{d} \\theta k \\mathrm{~d} k
\\end{aligned}
$$

Here the spherical coordinate is used, and $\\theta$ is the angle between $\\mathbf{r}$ and $\\mathbf{k}$. The absolute value of $\\mathbf{k}$ is denoted by $k$.

$$
\\begin{aligned}
\\varphi(\\mathbf{r}) & =\\frac{q}{4 \\pi^{2} \\varepsilon_{0}} \\int_{-1}^{1}-d(\\cos \\theta) \\int_{0}^{\\infty} \\mathrm{d} k \\frac{k^{2} \\mathrm{e}^{-i k r \\cos \\theta}}{\\left(k+i / \\lambda_{D}\\right)\\left(k-i / \\lambda_{D}\\right)} \\\\
& =\\frac{q}{4 \\pi^{2} \\varepsilon_{0}} \\int_{0}^{\\infty} d k \\frac{k^{2}}{\\left(k+i / \\lambda_{D}\\right)\\left(k-i / \\lambda_{D}\\right)} \\frac{-1}{i k r}\\left[\\mathrm{e}^{-i k r \\cos \\theta}\\right]_{\\cos \\theta=-1}^{\\cos \\theta=1} \\\\
& =\\frac{q}{4 \\pi^{2} i \\varepsilon_{0} r} \\int_{0}^{\\infty} d k c \\frac{k}{\\left(k+i / \\lambda_{D}\\right)\\left(k-i / \\lambda_{D}\\right)}(2 i \\sin (k r)) \\\\
& =\\frac{q}{2 \\pi^{2} \\varepsilon_{0} r} \\int_{0}^{\\infty} d k \\frac{k \\sin (k r)}{\\left(k+i / \\lambda_{D}\\right)\\left(k-i / \\lambda_{D}\\right)} .
\\end{aligned}
$$

The function is even with respect to $k$, and the integral range is modified below:


\\begin{align*}
\\varphi(\\mathbf{r}) & =\\frac{q}{8 \\pi^{2} i \\varepsilon_{0} r} \\int_{-\\infty}^{\\infty} d k \\frac{k\\left(\\mathrm{e}^{i k r}-\\mathrm{e}^{-i k r}\\right)}{\\left(k+i / \\lambda_{D}\\right)\\left(k-i / \\lambda_{D}\\right)} \\\\
& =\\frac{q}{8 \\pi^{2} i \\varepsilon_{0} r} 2 \\pi i\\left[\\frac{\\left(i / \\lambda_{D}\\right) \\mathrm{e}^{-r / \\lambda_{D}}}{2 i / \\lambda_{D}}-\\frac{-\\left(-i / \\lambda_{D}\\right) \\mathrm{e}^{-r / \\lambda_{D}}}{-2 i / \\lambda_{D}}\\right] \\\\
& =\\frac{q}{4 \\pi \\varepsilon_{0} r} \\mathrm{e}^{-r / \\lambda_{D}} \\tag{C.1}
\\end{align*}


Here we used the integral in the complex analysis shown in Sects. C. 1 and C.2. Now we reproduced the Debye shielding.

\\section*{Appendix D Derivation of Ponderomotive Force}
The ponderomotive force is a nonlinear force under an oscillating non-uniform field. Examples are shown in Figs. 3.14 and 3.15. When we have an oscillating force of $F=F_{0}(x) \\exp (-i \\omega t)$, the equation of motion for a particle is as follows:


\\begin{equation*}
m \\frac{\\mathrm{d} v_{x}}{\\mathrm{~d} t}=F_{0}(x) \\cos (\\omega t) \\tag{D.1}
\\end{equation*}


Here $m$ is the particle mass. When the amplitude of the oscillating force $F_{0}(x)$ varies slowly in the space $x, v_{x}$ and the position can be solved approximately around the oscillating center of $x=x_{0}: v_{x}(t) \\simeq F_{0}\\left(x_{0}\\right)\\left(\\frac{1}{m \\omega}\\right) \\sin (\\omega t)+\\delta v_{x}$ and $x(t) \\simeq$ $x_{0}+\\delta x \\simeq x_{0}-F_{0}\\left(x_{0}\\right)\\left(\\frac{1}{m \\omega^{2}}\\right) \\cos (\\omega t)$. Here $\\delta x=-F_{0}\\left(x_{0}\\right)\\left(\\frac{1}{m \\omega^{2}}\\right) \\cos (\\omega t)$ and $\\delta v_{x}$ are the perturbations from the zeroth-order quantities. The correction on the force $F_{0}\\left(x_{0}+\\delta x\\right)$ is obtained:


\\begin{align*}
F_{0}\\left(x_{0}+\\delta x\\right) & \\simeq F_{0}\\left(x_{0}\\right)+\\delta F_{0} \\simeq F_{0}\\left(x_{0}\\right)+\\delta x \\frac{\\mathrm{d} F_{0}\\left(x_{0}\\right)}{\\mathrm{d} x} \\\\
& \\simeq F_{0}\\left(x_{0}\\right)-F_{0}\\left(x_{0}\\right)\\left(\\frac{1}{m \\omega^{2}}\\right) \\cos (\\omega t) \\frac{\\mathrm{d} F_{0}\\left(x_{0}\\right)}{\\mathrm{d} x} \\tag{D.2}
\\end{align*}


Therefore, the nonlinear force $\\delta F$ is obtained:


\\begin{equation*}
\\delta F=-F_{0}\\left(x_{0}\\right)\\left(\\frac{1}{m \\omega^{2}}\\right) \\cos ^{2}(\\omega t) \\frac{\\mathrm{d} F_{0}\\left(x_{0}\\right)}{\\mathrm{d} x} \\tag{D.3}
\\end{equation*}


After the time average over one oscillation $\\left(<\\cos ^{2}(\\omega t)>=1 / 2\\right)$, the ponderomotive force is obtained:


\\begin{equation*}
<\\delta F>=-F_{0}\\left(x_{0}\\right)\\left(\\frac{1}{2 m \\omega^{2}}\\right) \\frac{\\mathrm{d} F_{0}\\left(x_{0}\\right)}{\\mathrm{d} x}=-\\left(\\frac{1}{4 m \\omega^{2}}\\right) \\frac{\\mathrm{d} F_{0}^{2}\\left(x_{0}\\right)}{\\mathrm{d} x} \\tag{D.4}
\\end{equation*}


When the particle is an electron under an oscillating electromagnetic field, the equation of motion is $m \\frac{\\mathrm{d} \\mathbf{v}}{\\mathrm{d} t}=-e(\\mathbf{E}+\\mathbf{v} \\times \\mathbf{B})$. Then we can obtain the following ponderomotive force:


\\begin{equation*}
-\\frac{\\mathrm{e}^{2}}{2 m \\omega^{2}} \\nabla<\\mathbf{E}^{2}> \\tag{D.5}
\\end{equation*}


\\section*{Appendix E 
 Parallel Computing by OpenMP}
In scientific computations frequently we need parallel computings. For the parallel computing, OpenMP (open multiprocessing) [1] and MPI (Message Passing Interface) [2] are widely used.

The MPI works on multiple computers with distributed memories. Massive parallel computings would employ MPI. The OpenMp works on a computer for multiple threads with shared memory. Here we focus on the simple OpenMp program as an introduction to parallel computing.

A simple following example shows a multi-thread computing, in which the value of $i$ is set to each array $a[i]$ :

\\begin{center}
\\includegraphics[max width=\\textwidth]{2024_02_26_83e36543483eb7d284c1g-277}
\\end{center}

(C) The Editor(s) (if applicable) and The Author(s), under exclusive license

In this program the message of "\\# pragma omp parallel for" indicates the parallelization for the for loop. The output is following:

\\begin{center}
\\includegraphics[max width=\\textwidth]{2024_02_26_83e36543483eb7d284c1g-278(1)}
\\end{center}

In this example 6 threads are used. Each thread accesses the same memory independently and uses the variable of Temp independently. At each execution, we have to worry if the value of Temp is correct for our purpose. However, in this example case we got the correct results.

As a special case please check the following program and the output:

\\begin{center}
\\includegraphics[max width=\\textwidth]{2024_02_26_83e36543483eb7d284c1g-278}
\\end{center}

\\begin{center}
\\includegraphics[max width=\\textwidth]{2024_02_26_83e36543483eb7d284c1g-279}
\\end{center}

In this simple program every array $a[i]$ is set to 0 . The second "for" loop is parallelized, but the loop should be successively done. Therefore, the following incorrect results are obtained. During the compilation and computation no error appears, and the results are incorrect. In OpenMP we have to remind again that each thread, used for parallelization, accesses the shared same memory independently.

\\begin{center}
\\includegraphics[max width=\\textwidth]{2024_02_26_83e36543483eb7d284c1g-279(1)}
\\end{center}

\\begin{center}
\\includegraphics[max width=\\textwidth]{2024_02_26_83e36543483eb7d284c1g-280}
\\end{center}

\\section*{References}
\\begin{enumerate}
  \\item OpenMP Architecture Review Board, \\href{https://www.openmp.org/}{https://www.openmp.org/}. Cited 28 August, 2021

  \\item Message Passing Interface Forum, \\href{https://www.mpi-forum.org/}{https://www.mpi-forum.org/}. Cited 28 August, 2021

\\end{enumerate}

\\section*{Appendix F}
\\section*{Example 3D Pure Euler Fluid Code Structure}
In the Appendix F, a structure of a simple pure 3D Euler fluid code is shown for readers' convenience. The following code is not fully shown to shorten the explanation. In the following code, $C$ language is used, and include sentences may need be added or modified. The data output functions would be also needed, and they are not shown here. A simple parallelization is implemented based on Open $M P$, shown in Appendix E.

The following example code is prepared to simulate a 3D shock wave generation in a plasma (fluid) by a supersonic plasma bullet, similar to Fig. 5.8, which shows the shock wave generation in 2D. When the 3D fluid code is executed with the initial setup shown below, the following 3D shock wave propagates as shown in Fig. F.1, in which the density contour surfaces are plotted.

The variable declarations are below with their explanations:

\\begin{center}
\\includegraphics[max width=\\textwidth]{2024_02_26_83e36543483eb7d284c1g-281}
\\end{center}

Fig. F. 1 A shock wave propagates in the $3 \\mathrm{D}$ space by the following example initial condition. The density contours are plotted. The boundary conditions in $x$ are cyclic. The boundary conditions in $y$ and $z$ are free ones. A plasma bullet near the left boundary moves to the + direction with a supersonic velocity, and a shock wave is generated. An expansion wave (rarefaction wave) appears behind the plasma bullet, and it appears from the right boundary by the cyclic boundary condition in $x$

\\begin{center}
\\includegraphics[max width=\\textwidth]{2024_02_26_83e36543483eb7d284c1g-282}
\\end{center}

\\begin{center}
\\includegraphics[max width=\\textwidth]{2024_02_26_83e36543483eb7d284c1g-282(1)}
\\end{center}

Now the main function is below:

\\begin{center}
\\includegraphics[max width=\\textwidth]{2024_02_26_83e36543483eb7d284c1g-283}
\\end{center}

\\begin{center}
\\includegraphics[max width=\\textwidth]{2024_02_26_83e36543483eb7d284c1g-284}
\\end{center}

\\section*{The main fluid solver is below:}
\\begin{center}
\\includegraphics[max width=\\textwidth]{2024_02_26_83e36543483eb7d284c1g-284(1)}
\\end{center}

\\begin{center}
\\includegraphics[max width=\\textwidth]{2024_02_26_83e36543483eb7d284c1g-285}
\\end{center}

\\begin{center}
\\includegraphics[max width=\\textwidth]{2024_02_26_83e36543483eb7d284c1g-286}
\\end{center}

\\begin{center}
\\includegraphics[max width=\\textwidth]{2024_02_26_83e36543483eb7d284c1g-287}
\\end{center}

\\begin{center}
\\includegraphics[max width=\\textwidth]{2024_02_26_83e36543483eb7d284c1g-288(1)}
\\end{center}

The below shows the boundary condition, which should be adjusted according to problems. The time step $\\mathrm{d} t$ control is very important for stable computations. The $\\mathrm{d} t$ control is also needed to be changed to problems concerned. The data copy is also shown, and new values are copied to arrays for old values. The initial setup is also presented for this specific problem of a 3D shock wave generation in a plasma (fluid) by a supersonic plasma bullet. The output function is not shown below, and it should be prepared to each problem.

\\begin{center}
\\includegraphics[max width=\\textwidth]{2024_02_26_83e36543483eb7d284c1g-288}
\\end{center}

\\begin{center}
\\includegraphics[max width=\\textwidth]{2024_02_26_83e36543483eb7d284c1g-289}
\\end{center}

\\begin{center}
\\includegraphics[max width=\\textwidth]{2024_02_26_83e36543483eb7d284c1g-290}
\\end{center}

\\begin{center}
\\includegraphics[max width=\\textwidth]{2024_02_26_83e36543483eb7d284c1g-291}
\\end{center}

\\begin{center}
\\includegraphics[max width=\\textwidth]{2024_02_26_83e36543483eb7d284c1g-292}
\\end{center}

\\section*{Appendix G 
 Derivation of $\\epsilon_{x x}$ in Eq. (6.151) for Magnetized Uniform Plasma}
In the Appendix G, a derivation for $\\epsilon_{x x}$ in Eq. (6.151) is summarized.


\\begin{align*}
& \\epsilon_{x x}= \\\\
& 1-\\frac{k_{D}^{2} \\Omega^{2}}{k_{\\perp}^{2} \\omega} \\sum_{n=-\\infty}^{+\\infty} \\frac{n^{2}}{\\omega-n \\Omega} I_{n}\\left(\\frac{k_{\\perp}^{2} T}{m \\Omega^{2}}\\right) \\exp \\left(-\\frac{k_{\\perp}^{2} T}{m \\Omega^{2}}\\right)\\left\\{1-W\\left(\\frac{\\omega-n \\Omega}{k_{\\|} \\sqrt{T / m}}\\right)\\right\\} \\tag{6.151}
\\end{align*}


From Eqs. (6.141) and (6.142), $\\epsilon_{x x}$ is shown below again for general distribution functions $f_{0}$ :

\\[
\\begin{array}{r}
\\epsilon_{x x}=\\left(1-\\frac{\\omega_{p}^{2}}{\\omega^{2}}\\right)-\\frac{\\omega_{p}^{2}}{\\omega^{2}} \\sum_{n=-\\infty}^{+\\infty} 2 \\pi \\int_{0}^{\\infty} d v_{\\perp} v_{\\perp} \\int_{-\\infty}^{\\infty} d v_{\\|} \\\\
\\left(\\frac{n \\Omega}{v_{\\perp}} \\frac{\\partial f_{0}}{\\partial v_{\\perp}}+k_{\\|} \\frac{\\partial f_{0}}{\\partial v_{\\|}}\\right) \\frac{\\left(\\frac{n \\Omega}{k_{\\perp}}\\right)^{2} J_{n}^{2}\\left(\\frac{k_{\\perp} v_{\\perp}}{\\Omega}\\right)}{n \\Omega+k_{\\|} v_{\\|}-\\omega} \\tag{G.1}
\\end{array}
\\]

Now $f_{0}$ is the Maxwell distribution in the non-relativistic case:


\\begin{align*}
f_{0}\\left(v_{\\perp}, v_{\\|}\\right) & =\\left(\\frac{m}{2 \\pi T}\\right)^{3 / 2} \\exp \\left(-\\frac{m v_{\\perp}^{2}}{2 T}-\\frac{m v_{\\|}^{2}}{2 T}\\right) \\\\
& =F_{0 \\perp} \\times F_{0 \\|} \\tag{G.2}
\\end{align*}


Here $F_{0 \\perp}=\\left(\\frac{m}{2 \\pi T}\\right) \\exp \\left(-\\frac{m v_{\\perp}^{2}}{2 T}\\right)$ and $F_{0 \\|}=\\left(\\frac{m}{2 \\pi T}\\right)^{1 / 2} \\exp \\left(-\\frac{m v_{\\|}^{2}}{2 T}\\right)$. Then $\\frac{n \\Omega}{v_{\\perp}} \\frac{\\partial f_{0}}{\\partial v_{\\perp}}=-\\frac{n \\Omega m}{T} f_{0}$ and $k_{\\|} \\frac{\\partial f_{0}}{\\partial v_{\\|}}=-\\frac{k_{\\|} m v_{\\|}}{T} f_{0}$.

Now we use the following remarkable relation in p. 395 in Ref. [1]. The relation can be also found in Refs. [2-5].


\\begin{equation*}
\\int_{0}^{\\infty} \\exp \\left(-\\mathrm{pt}^{2}\\right) J_{n}(\\mathrm{at}) J_{n}(\\mathrm{bt}) t \\mathrm{~d} t=\\frac{1}{2 p^{2}} \\exp \\left(-\\frac{a^{2}+b^{2}}{4 p^{2}}\\right) I_{n}\\left(\\frac{\\mathrm{ab}}{2 p^{2}}\\right) \\tag{G.3}
\\end{equation*}


Here $I_{n}(\\Psi)$ is the modified Bessel function of the $n$th order. Now we use the $W$ function in Eq. (6.69).


\\begin{equation*}
W(Z)=\\frac{1}{\\sqrt{2 \\pi}} \\int \\frac{\\xi}{\\xi-Z} \\exp \\left(-\\frac{\\xi^{2}}{2}\\right) d \\xi \\tag{6.69}
\\end{equation*}


During the calculations we need to use the relation of $\\sum_{n=-\\infty}^{+\\infty} n^{2} I_{n}(b)=\\mathrm{be}^{b}$. After a relatively long calculation, $\\epsilon_{x x}$ is obtained.

For the other components in the dielectric tensor $\\epsilon$ for plasmas in the Maxwell distribution, the following relations would be also needed:


\\begin{align*}
\\exp \\left(\\frac{z}{2}\\left(t-\\frac{1}{t}\\right)\\right) & =\\sum_{n=-\\infty}^{+\\infty} J_{n}(z) t^{n}  \\tag{G.4}\\\\
J_{n}(z) & =\\left(\\frac{z}{2}\\right)^{n} \\sum_{m=0}^{+\\infty} \\frac{(-1)^{m}(z / 2)^{2 m}}{m ! \\Gamma(n+m+1)}  \\tag{G.5}\\\\
J_{-n} & =(-1)^{n} J_{n}  \\tag{G.6}\\\\
\\sum_{n=-\\infty}^{+\\infty} J_{n}^{2} & =1, \\quad \\sum_{n=-\\infty}^{+\\infty} J_{n} J_{n}^{\\prime}=0  \\tag{G.7}\\\\
\\sum_{n=-\\infty}^{+\\infty} n J_{n}^{2} & =0  \\tag{G.8}\\\\
\\frac{d J_{n}(z)}{d z} & =J_{n}^{\\prime}(z)=\\frac{n}{z} J_{n}(z)-J_{n+1}(z) \\\\
& =\\frac{1}{2}\\left(J_{n-1}(z)-J_{n+1}(z)\\right)=J_{n-1}(z)-\\frac{n}{z} J_{n}(z)  \\tag{G.9}\\\\
J_{n-1}(z) & +J_{n+1}(z)=2 \\frac{n}{z} J_{n}(z) \\tag{G.10}
\\end{align*}



\\begin{align*}
& I_{n-1}(z)-I_{n+1}(z)=\\frac{2 n}{z} I_{n}(z)  \\tag{G.11}\\\\
& I_{n-1}(z)+I_{n+1}(z)=2 I_{n}^{\\prime}(z) \\tag{G.12}
\\end{align*}


Here $n$ and $m$ are integers, shows the gamma function [6], and now $(n+m+1)=$ $(n+m)$ !. The explicit expressions can be found in Refs. [2-4] for each component of the dielectric tensor for the Maxwell plasmas.

\\section*{References}
\\begin{enumerate}
  \\item G.N. Watson, in A Treatise on the Theory of Bessel Functions (Cambridge at the University Press, 1922)

  \\item S. Ichimaru, Chapter 4, in Statistical Plasma Physics, vol. 1: Basic Principles (CRC Press, Boca Raton, 2004)

  \\item K. Miyamoto, Chapter. 12, in Plasma Physics and Controlled Nuclear Fusion, Springer (Springer, Berlin, Heidelberg, 2013)

  \\item T.H. Stix, Chapter 10, in The Theory of Plasma Waves (MacGraw Hill, New York, 1962)

  \\item R.C. Davidson, Kinetic waves and instabilities in a uniform plasma, Sec. 3.3, in Handbook of Plasma Physics, vol. 1 Basic Plasma Physics, eds. by A.A. Galeev, R.N. Sudan (North Holland Pub., Amsterdam; New York, Oxford, 1983), pp. 519-586,

  \\item F.W.J. Olver, D.W. Lozier, R.F. Boisvert, C.W. Clark (ed.), Chapter 5, in NIST Handbook of Mathematical Functions (Cambridge University Press, Cambridge, 2010). \\href{https://dlmf.nist.gov/}{https://dlmf.nist.gov/}

\\end{enumerate}

\\section*{Index}
\\section*{A}
Alfvén speed, 107

Alfvén wave, 107, 142

Artificial viscosity, 84, 92

\\section*{B}
BBGKY hierarchy, 115

Bernstein wave, 143

Boltzmann equation, 119

Burgers equation, 186

\\section*{C}
Cauchy's integral theorem, 261

Cluster ion interaction with plasma, 216

Collective behavior, 12

Complex analysis, 261

Computer simulation methods for plasma, 15

Constant of motion, 118

Control of plasma instability, 218

Control of plasma instability, filamentation instability, 230

Control of plasma instability, instabilities of magnetized plasma column, 224

Control of plasma instability, KelvinHelmholtz instability, 233

Control of plasma instability, RayleighTaylor instability, 223

Control of plasma instability, tearing mode instability, 230

Control of plasma instability, two-stream instability, 235
Control of plasma non-uniformity, dynamic mitigation of fuel target implosion in heavy ion inertial fusion, 237

Coulomb collision, 36

Coulomb coupling parameter, 13

Courant-Friedrichs-Lewy (CFL) number, 78

Cryo fuel target, ICF, 212

Cyclotron motion, 44

\\section*{D}
Debye length, 9

Debye shielding, 7, 126, 267

Delta function, 258

Dielectric response function, unmagnetized uniform plasma, 122

Differential operators, 257

Discretization of equation, $66,78,86$

Discretization of space and time, 17, 79, 85

Dispersion relation, magnetized uniform plasma, 135

Distribution function, 29

Drift, diamagnetic drift, 48

Drift, ExB drift, 46

Drift, gradient B drift, 47

Drift motion, 45

Drift, polarization drift, 48

DT reaction, 200

Dynamic control of plasma instability, 220, 222

Dynamic mitigation of fuel target implosion in heavy ion inertial fusion, 237

Dynamic mitigation of plasma instabilities and non-uniformity by phase control, 221

(C) The Editor(s) (if applicable) and The Author(s), under exclusive license

Dynamic smoothing of plasma non-
uniformity, 220
Dynamic stabilization of instability by
adding strong oscillation, 220

\\section*{E}
Electromagnetic wave, fluid model, 99

Electron plasma wave, 128

Electron plasma wave, fluid model, 93

Electron plasma wave, Vlasov model, 129

Enstrophy, 190

Equation of motion, 43

Equilibrium solution, Vlasov equation, 118

Euler method, fluid simulation, 74

\\section*{F}
Faraday rotation, 142

Fast ignition, Inertial Confinement Fusion, 210

Field solver, PIC code, 65

Finite Difference Method (FDM), 75

Floating-point number, 20

Floating-point rounding method, 20

Fluid equations, 71

Fluid model, 71

Frozen magnetic flux, 104

Fuel target alignment in ICF, 212

\\section*{G}
Gamma function, 289

\\section*{$\\mathbf{H}$}
Hasegawa-Mima equation, 192

Heavy Ion Beam Inertial Confinement Fusion (HIF), 210

Helicon wave, 142

\\section*{I}
Implosion efficiency, 208, 211

Implosion, ICF, 209

Implosion non-uniformity, ICF, 210

Inertial Confinement Fusion (ICF), 205

Inertial range, 191

Instability, drift instability, 173

Instability, filamentation instability, 167, 230

Instability, interchange instability, 159

Instability, ion acoustic instability, 153

Instability, Kelvin-Helmholtz instability, 162,233
Instability, Kelvin-Helmholtz instability in $2 \\mathrm{D}, 82$

Instability, kink instability, 154, 224

Instability, parametric instability, 165

Instability, Rayleigh-Taylor instability, 159, 223

Instability, sausage instability, 154, 224

Instability, tearing mode instability, 170, 230

Instability, two-stream instability, 147, 235

Instability, Weibel instability, 166

Integral formulae, 259

Interaction between field and particles, PIC code, 66

Interval arithmetic, 24

Inverse scattering transform for $\\mathrm{KdV}$ equation, 185

Inverted pendulum, 220

Ion acoustic wave, fluid model, 97

ITER, MCF, 205

\\section*{K}
Kinetic particle simulation, 64

Kink instability, 156, 157

Klimontovich equation, 111

Kolmogorov energy spectrum, 191

Kolmogorov spectrum, 191

Korteweg-de Vries (KdV) equation, 179, 183

Krook model for collisional plasma, 120, 169

Kruskal-Shafranov limit, 156

\\section*{L}
Lagrange method, fluid simulation, 88

Landau damping, 128, 131

Langrange equation of motion, 119

Laser ion acceleration, 215

Laser particle acceleration, 213

Laser wake field electron acceleration, 214

Lawson criterion, 202

Liouville equation, 112

L wave, 106, 142

\\section*{M}
Machine epsilon, unit in the last place, 20

Magnetic Confinement Fusion (MCF), 203

Magnetic moment, 50

Magnetohydrodynamic (MHD) equation, 101

Magnetosonic wave, 107

Mathematical models for plasmas, 14

Maxwell distribution, 31

Maxwell's equations, 60

Moment equation, Vlasov equation and fluid model, 120

\\section*{$\\mathbf{N}$}
NIF, laser fusion, 209

Non-linear plasma physics, 179

Nuclear fusion, 200

Nuclear fusion reaction, 201

Numerical instability, 80,85

Numerical precision, 22

Numerical stability analysis, 81

Numerical stability, time step control, 80

\\section*{O}
O (ordinary) wave, 108, 143, 144

\\section*{$\\mathbf{P}$}
Parallel computing, OpenMP, 271

Particle-in-Cell (PIC) code, 65

Physical constants, 255

Physical constants and mathematical formulae, 255

Plasma density, 35

Plasma dispersion function, $\\mathscr{Z}(Z), 124$

Plasma echo, 188

Plasma etching, 197

Plasma frequency, electron, 12

Plasma in equilibrium, 29

Plasma instability, 147

Plasma jet, 199

Plasma oscillation, 10, 126

Plasma process, 197

Plasma temperature, 39

Poisson's equation, 59

Ponderomotive force, 57, 269

Potential, 63

\\section*{R}
Residue theorem, 261

Reynolds number, 190

Rounding error, 20

R wave, 106, 142

\\section*{$\\mathbf{S}$}
Sausage instability, 155,157

Simulation, charged particle confinement by mirror field, 55
Simulation, charged particle trajectory in a plane laser, 55

Simulation, cyclotron motion, 45

Simulation, Debye shielding, 25

Simulation, diamagnetic drift, 49

Simulation, electron motion in laser field, 55

Simulation, ExB drift, 45

Simulation, filamentation instability, 230

Simulation, fluid convection, 76

Simulation, fuel target implosion in heavy ion inertial fusion, 237

Simulation, gradient B drift, 47

Simulation, heat conduction, 85

Simulation, Kelvin-Helmholtz instability, 233

Simulation, kink instability, 157, 227

Simulation, laser interaction with neutral gas and plasma, 7

Simulation, laser interaction with plasma, 69

Simulation, laser propagation, 68

Simulation, neutral gas ball expansion, 4

Simulation, PIC, electron cloud with laser field, 68

Simulation, plasma ball expansion, 5

Simulation, polarization drift, 47

Simulation, ponderomotive force by a Gaussian laser, 57

Simulation, ponderomotive force by a longitudinal wave, 57

Simulation, ponderomotive force, longitudinal, 57

Simulation, ponderomotive force, transverse, 57

Simulation process, 74

Simulation, relativistic particle motion, 53

Simulation, relativistic particle motion by a particle pusher, 44

Simulation, sausage instability, 157, 225

Simulation, shock wave generation by $2 \\mathrm{D}$ Euler fluid code, 82

Simulation, shock wave generation by $3 \\mathrm{D}$ Euler fluid code, 276

Simulation, spherical shell implosion by $2 \\mathrm{D}$ Lagrange fluid code, 92

Simulation, tearing mode instability, 230

Simulation, temperature relaxation, 40

Simulation, 3D Rayleigh-Taylor instability, 223

Simulation, 2D Kelvin-Helmoltz instability, 82

Simulation, two-stream instability, 235

Single particle motion, 43

Single particle simulation, 52

Solitary wave, 179

Soliton, 179

\\section*{$\\mathbf{T}$}
Temperature measurement, single-probe method, 197

3D Euler fluid code, 275

Transverse electromagnetic wave, 133

Turbulence, 189

2D turbulence, 192

\\section*{V}
Vector formulae, 256

Vlasov equation, 115

Vorticity, 190

\\section*{W}
Wake field by ion, 216

Wake field by laser, 214

Waves in magnetized plasma, fluid model, 105

Waves in magnetized uniform plasma, Vlasov model, 141

W function, 124

$\\mathbf{X}$

X (extraordinary) wave, 108, 143

$\\mathbf{Z}$

Zonal flow, 193


\\end{document}